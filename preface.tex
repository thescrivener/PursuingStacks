%% Anti-Copyright 2015 - the scrivener

\chapter{Scrivener's preface}

I wanted to read \emph{Pursuing Stacks} (PS), and at the same time I
wanted to read it carefully enough to make sure I understood as much
as possible, and also make some notes and links to make
it easier to navigate. That was my primary motive for creating this
\LaTeX-typeset version of the work. I hope it may be of some use to
others, but it is not necessarily the best way to read PS.

One of the great pleasures of reading PS in the
scanned typescript with the handwritten margin notes is that you get a
sense of intimacy that is necessarily lost upon transcription.

However, I think it is a great shame that PS is not more widely read. This
situation is likely due to the obstacles involved with navigating and
parsing the typescript which only exists as \texttt{djvu}-file, accessible from
Maltsiniotis' web-page:
\[\text{\small\url{http://webusers.imj-prg.fr/~georges.maltsiniotis/ps.html}}\]
I am of course aware that Maltsiniotis himself is preparing an edition of the
first five chapters of PS, and with K\"unzer and To\"en an edition of
the last two chapters, both for the series \emph{Documents
  Math\'ematiques} of the French mathematical society. This has been
announced since 2010, while we now have 2015. Thus, I decided that it
might not be an entirely wasted effort to publish my own edition of
PS, knowing full well that it will be succeeded by presumably better
ones made by more knowledgeable and capable individuals. In any event,
should my files contain some small help (e.g., in the forms of diagrams)
towards the goals of Maltsiniotis et al.\ they are more than welcome
to it!

I'm also not including an index, which I don't consider myself capable of
compiling. At least modern \textsc{pdf}-viewers are capable of search,
and I provide the \TeX-files which can be used for \texttt{grep}-ing.

I have tried to stay true to Grothendieck's voice, but on some
matters of grammar and orthography, I have taken the liberty of
correcting slightly:
\begin{itemize}
\item some ``which''s to ``that''s
\item many ``it's''s to ``its''s
\item every ``standart'' to ``standard''
\end{itemize}
and so on. It is my hope that these tiny corrections make for a better
reading experience, and I have put references in the margins to the
exact pages in the original typescript for those who want to compare
any particular bit with what came from the horse's mouth.

For what it's worth, my ``philosophy'' when \TeX 'ing PS was to
imagine that I did it right as AG was typing (or perhaps even
dictating to me! That would certainly explain the variance in
punctuation.). Sometimes I would ask him whether something should be
an $E$ or an $e$, and thus catch little misprints. Unfortunately, I
wasn't there to ask about the mathematical content, so I'll leave that
as intact as I found it, for the reader to wrestle with. Among the
many remarkable qualities of PS is that it is a record of mathematics
\emph{as it is being made}, and not just a polished record for a
journal made after the fact.

Some people say that PS is mostly visions and loose ideas; while those
are certainly present, I hope perhaps to change this impression with
this edition where all the ``Lemmata'', ``Propositions'',
``Corollaries'', etc., are marked up in the usual way.

I hope you'll forgive me for deviating from the Computer Modern font
family and choosing Bitstream Charter with Pichaureau's
\texttt{mathdesign} package instead.\footnote{Though I swapped
  MathDesign's $\infty$-symbol for Euler Math's $\euinfty$-symbol.} I
find it reads well on the screen and the density mimics that of the
AG's typewriter. If you don't like it, you're welcome to compile your
own version with different settings!

I give my work to the public domain, and hope merely that others may
find it useful. Some people might ask me to identify myself, to which
I can only reply: ``I'd prefer not to.''

\bigskip
\noindent\hfill\emph{the scrivener}\par

%%% Local Variables:
%%% mode: latex
%%% TeX-master: "main.tex"
%%% End:
