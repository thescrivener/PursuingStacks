%% Anti-Copyright 2015 - the scrivener

\chapter{Homology and cohomology (abelianization of homotopy types)}
\label{ch:V}

\presectionfill\ondate{1.7.}\pspage{291}\par

% 87
\hangsection[Comments on Thomason's paper on closed model structure
\dots]{Comments on Thomason's paper on closed model structure on
  \texorpdfstring{\Cat}{(Cat)}.}\label{sec:87}%
I couldn't resist last night and had to look through Thomason's
preprint on the closed model structure of
\Cat.\scrcomment{\textcite{Thomason1980}} The paper is really pleasant
reading -- and it gives exactly what had been lacking me in my
reflections lately on the homotopy theory of \Cat{} -- namely, a class
of neat monomorphisms $Y\to X$ which all have the property that cobase
change by these preserves weak equivalence, the so-called \emph{Dwyer
  maps}.\scrcomment{These Dwyer maps are not closed under retracts, so
  \textcite{Cisinski1999} introduced the
  \href{http://ncatlab.org/nlab/show/Dwyer+map}{\emph{pseudo-Dwyer
      maps}} which are, and \emph{they} are now called Dwyer maps.} I
had hoped for a while that ``open immersions'' and their duals, the
``closed immersions'' (namely sieve and cosieve maps, in Thomason's
wording), have this property, and when it turned out they hadn't, I
had been at a loss of what stronger property to put instead, wide
enough however to allow for the standard factorization statements for
a map to go through. The definition of a Dwyer map is an extremely
pretty one -- it is an open immersion $Y\to X$ such that the induced
map from $Y$ into its closure $\overline Y$ should have a right
adjoint. Now this implies that $Y\to\overline Y$ is aspheric, and I
suspect that this extra condition on an open immersion $Y\to X$ should
be sufficient to imply that it has the ``cofibration property'' above
with respect to weak equivalence. It would mean in a sense that the
given open immersion is very close to being a closed immersion too,
without however being a direct summand necessarily. The dual notion is
that of a closed immersion such that the corresponding interior
$\mathring Y$ of $Y$ in $X$ gives a map $\mathring Y \to Y$ which has
a left adjoint, or only which is ``coaspheric''. By Quillen's duality
principle, if one notion works well for pushouts, so does the
other. -- With this notion in hands, it shouldn't be difficult now to
get a closed model structure on \Cat{} a lot simpler than
Thomason's. Visibly, he was hampered by the standing reflex: homotopy
= semi-simplicial algebra, which caused him to pass by the detour of
the category \Simplexhat{} of semi-simplicial complexes, rather than
just working in \Cat{} itself. I'll have to come back upon this in
part \ref{ch:V} of the notes, where I intend to investigate the
homotopy properties of \Cat{} and elementary modelizers \Ahat,
including the existence of closed model structures.

Some comments of Thomason's at the end of his preprint, about
application to algebraic K-theory, seem to indicate that the notion of
``integration'' and ``cointegration'' of homotopy types I have been
interested in, has been studied (under the name holim and hocolim) in
the context of closed model categories by Anderson (his paper appeared
in 1978).\scrcomment{\textcite{Anderson1978}} As I am going to develop
some ideas along these lines in part \ref{ch:VI} on derivators, I
should have a look at what Anderson does, notably\pspage{292} what his
assumptions on the indexing categories are. Thomason seems to believe
that the closed model structure of \Cat{} is essential for being able
to take homotopy limits -- whereas it is clear a priori to me that the
notion depends only on the notion of weak equivalence. Indeed, he
seems to consider the possibility of taking homotopy limits in \Cat{}
as the main application of his theorem, and in order to apply
Anderson's results, believes it is necessary to be able to give
concrete characterizations for ``fibrations'' and ``cofibrations'' of
his closed model structure. Now, it turns out that the case he is
interested in (for proving ``Lichtenbaum's
conjecture'')\scrcomment{aka the
  \href{https://en.wikipedia.org/wiki/Quillen-Lichtenbaum_conjecture}{Quillen–Lichtenbaum
    conjecture}} is a typical
case of \emph{direct} homotopy limits, namely ``integration'' -- which
can be described directly in \Cat{} in such an amazingly simple way
(as sketched in section \ref{sec:69}, p.\
\ref{p:198}--\ref{p:199}). Thus, I feel for this application, the
closed model structure is wholly irrelevant. As for cointegration, I
do not expect that there is a comparably simple construction of this
operation within the modelizer \Cat, but presumably there is when
taking \oo-Gr-stacks as models (as suggested by the ``geometric''
approach to cohomology invariants, via stacks, where the operation of
``direct image'', namely cohomology precisely, is the obvious one,
whereas inverse images are more delicate to define, by an adjunction
property with respect to direct images\ldots). When working in \Cat,
cointegration of homotopy types should be no more nor less involved
than in any closed model category say, and involve intensive recourse
to ``fibrations'' (in the sense of the closed model structure, or more
intrinsically, in the sense that base change by these should preserve
weak equivalence). Now the latter have become quite familiar to me
during my long scratchwork on cohomology properties of maps in \Cat,
and I'll have to try to put it down nicely in part \ref{ch:V} of the
notes.

% 88
\hangsection[Review of pending questions and topics (questions 1) to
\dots]{Review of pending questions and topics
  \texorpdfstring{\textup(questions \textup{1)} to \textup{5)},
    including characterizing canonical modelizers\textup)}{(questions
    1) to 5), including characterizing canonical
    modelizers)}.}\label{sec:88}%
The present ``part \ref{ch:IV}'' on asphericity structures (and their
relations to contractibility structures) turns out a lot longer than I
anticipated, and the end is not yet quite in sight! Therefore, before
pursuing, I would like to make a review of the questions along these
lines which seem to require elucidation, and then decide which I'm
going to deal with, before going over to part \ref{ch:V}.

\namedlabel{q:88.1}{1)}\enspace Whereas the relevant notions of
``morphisms'' and ``bimorphisms'' for contractibility structures seem
to me well understood, there remains\pspage{293} a certain feeling of
uneasiness with respect to asphericity structures, which haven't got
yet a reasonable notion of morphism. Thus, I have left unanswered the
questions raised in section \ref{sec:84} around the inclusion
condition
\[ f_!(M\suba) \subset M'\suba,\]
so it is not impossible that, while relying on the mere feeling that
the inclusion is just not reasonable and that the answer to the
specific questions are presumably negative, I am about to miss some
unexpected important fact! It seems I developed kind of a block
against checking -- maybe the answers are well-known and Tim Porter
will tell me\ldots Maybe I better leave the question for a later
moment, as it will ripen by itself if I leave it alone\ldots

\namedlabel{q:88.2}{2)}\enspace I should at last introduce
contractors, and morphisms of such. When I set out on part \ref{ch:IV}
of the notes, I expected that the notion of a contractor would be one
main notion, alongside with the notion of a canonical modelizer -- it
turns out that so far I didn't have much use yet for one or the
other. Contractors can be viewed as categories generating
contractibility structures, just as aspherators are there for
generating asphericity structures. However, whereas any small category
is an aspherator, the same is definitely not so for contractors, as we
demand that every object in $C$ should be contractible, for the
homotopy structure in \Chat{} generated by $C$ itself. If we except
the case of a contractor equivalent to the final category (a so-called
\emph{trivial} contractor), a contractor is a strict test category --
thus the notion appears somewhat as a hinge notion between the test
notions, and the ``pure'' homotopy notions and more specifically,
contractibility structures. Writing up some scratchnotes I got should
be a pure routine matter.

\namedlabel{q:88.3}{3)}\enspace A lot more interesting seems to me to
try and resolve a persistent feeling of uneasiness which has been
floating, throughout the long-winded reflections on homotopy and
asphericity structures in parts \ref{ch:III} and \ref{ch:IV} of the
notes. This is tied up with this fact, that my treatment of the main
notions, namely contractibility and asphericity, has been consistently
\emph{non-autodual}. More specifically, when a category $M$ is
endowed with either a contractibility or an asphericity structure, it
does \emph{not} follow that the opposite category $M\op$ is too in a
natural way. When defining homotopy relations and homotopism
structures (section \ref{sec:51}, \ref{subsec:51.A} and
\ref{subsec:51.B}), these were autodual notions, but the notion of a
homotopy interval structure, which we used in order to pass from a
contractibility\pspage{294} structure to the corresponding notions of
homotopy equivalence between maps and of homotopisms, is highly
non-autodual too. It breaks down altogether when $M$ is a ``pointed''
category, namely contains an object which is both initial and final --
in this case, for any homotopy interval structure on $M$, any two maps
in $M$ are homotopic, and any map is a homotopism, hence any object is
contractible!

Our initial motivation, namely understanding ``modelizers'' for
ordinary homotopy types, made it very natural to get involved in
non-autodual situations, as the homotopy category \Hot{} itself, and
the usual model categories for it, displays strongly non-autodual
features. (Thus, whereas the usual test categories all have a final
object, it is easy to see that a test category cannot possibly have an
initial object.) However, the homotopy and asphericity notions we then
came to develop make sense and are familiar indeed not only in the
``modelizing story'', but in any situation whatever which turns up,
giving rise to anything like a ``homology'' or a ``homotopy''
theory. To give one specific example, starting with an abelian
category \scrA, the corresponding homology theory is concerned with the
category of \scrA-valued complexes, say $\mathrm
K^\bullet(\scrA)$. The most basic notions here are the three homotopy
notions and two asphericity notions, namely: homotopy equivalence
between maps, homotopism, contractible objects, and quasi-isomorphisms
(= ``weak equivalences''), and acyclic (= ``aspheric'') objects. The
two first homotopy notions determine each other in the usual way, and
define the third, namely an object is contractible (or null-homotopic)
if{f} the map $0\to X$, or equivalently $X\to 0$, is a homotopism. On
the other hand, if we use the mapping-cylinder construction for a map,
the set of contractible objects determines the set of homotopisms, as
the maps whose mapping cylinder is contractible. Likewise, weak
equivalences determine aspheric objects, and conversely if the mapping
cylinder construction is given. The question that now comes to mind
immediately is whether the two sets of notions, the three pure
homotopy notions (determining each other), and the two ``asphericity
notions'' (determining each other too), mutually determine each
other, as in the non-commutative set-up we have worked in so far. We
can also remark that the functor
\[ \mathrm H^0 : \mathrm K^\bullet(\scrA)\to\scrA\]
visibly plays the part of the functor \piz{} in the non-commutative
set-up, it gives rise moreover to the $\mathrm H^i$ functors (any
integer $i$) by composing with the iterated shift functor (where the
shift of $X$ is just the mapping\pspage{295} cylinder of $0\to X$),
thus the set of aspheric objects formally from the functor $\mathrm
H^0$ (as the objects $X$ such that $\mathrm H^i(X)=0$ for all $i$),
much as in the non-commutative set-up the functor \piz{} for a
contractibility structure determines the latter, and hence also the
corresponding asphericity structure.

Thus, the question arises of formulating the basic structures, namely
contractibility and asphericity structures, in an autodual way,
applying both to the autodual situation just described, and to the
non-autodual one we have been working out in the notes -- and if
possible even, in all situations met with so far where a homology or
homotopy theory of some kind of other has turned up. Of course, it is
by no means sure a priori that we can do so, by keeping first nicely
apart the two sets of notions (contractibility and asphericity),
namely defining them separately, and then showing that a
contractibility structures determines an asphericity structures, and
is determined by the latter. Maybe we'll have to define from the
outset a richer kind of structure, where both ``pure'' homotopy
notions and asphericity notions are involved. Also, the familiar
generalization of mapping cylinders, namely homotopy fibers and
cofibers, and the corresponding long exact sequences, will evidently
play an important role in the structure to be described. Now, this
again ties in with the corresponding structure of a derivator, as
contemplated in section \ref{sec:69}, namely ``integration'' and
``cointegration'' of diagrams in a given category.

Definitely, this reflection is going to lead well beyond the scope of
the present part \ref{ch:IV}\scrcomment{I guess it is because this
  question is already growing in prominence that AG later decided to include
  the present section in part \ref{ch:V}!} -- it relates rather to
part \ref{it:71.C} of the working program envisioned by the end of May
(section \ref{sec:71}, p.\ \ref{p:207}--\ref{p:210}), and rather
belong to part \ref{ch:VI} of these notes, which presumably will
center around the notion of a derivator. There is however a more
technical question, and of more limited scope, which deserves some
thought and goes somewhat in the same direction, namely: how to define
(for a given contractibility or asphericity structure on $M$) an
\emph{induced} structure on a category $M_{/a}$, where $a$ is in $M$?
There is a little perplexity in my mind, even when $M$ is of the type
\Chat{} say, with $C$ a contractor and $a$ in $C$, taking the
canonical structure on \Chat{} -- because with the most evident choice
of an ``induced'' asphericity structure on $(\Chat)_{/a} \simeq
(C_{/a})\uphat$, namely the usual notion of aspheric objects, this
structure\pspage{296} will practically never be totally aspheric
(unless we take $a$ to be a final object of $C$), hence will not be
associated to a contractibility structure -- whereas we expect that
the contractibility structure of \Chat{} should induce one on
$\Chat_{/a}$. Presumably, the ``correct'' notions of induced
structure, in the case of asphericity structure and the corresponding
notion of weak equivalence, should be considerably stronger than the
one I just envisioned, and correspond to the intuition of
``\emph{fiberwise} homotopy types'' over the object $a$ (visualized as
a space-like object). A careful description of such induced structures
seems to me to be needed, and the natural place to be the present part
\ref{ch:IV} of the notes.

\namedlabel{q:88.4}{4)}\enspace A little reflection on semi-simplicial
homotopy notions (and their analogons when $\Simplex$ is replaced by a
general test category $\Delta$) seems needed, in order to situate the
following fact: ss homotopy notions, namely for ss objects in any
category $A$, behave well with respect to \emph{any} functor
\begin{equation}
  \label{eq:88.1}
  A\to B,\tag{1}
\end{equation}
without having to assume that this functor commutes with finite
products, whereas in the context of homotopy structures, when have a
functor between categories endowed say with homotopy interval
structures (for instance, with contractibility structures), such a
functor
\begin{equation}
  \label{eq:88.2}
  M\to N\tag{2}
\end{equation}
behaves well with respect to homotopy notions only, it would seem, if
we assume beforehand it commutes with finite products (plus, of
course, that it transforms a given generating family of homotopy
intervals of $M$ into homotopy intervals of $N$). In
case
\[ M = \bHom(\Delta\op,A), \quad N=\bHom(\Delta\op,B),\]
and \eqref{eq:88.2} comes from a functor \eqref{eq:88.1} which does
not commutes to finite products, neither does \eqref{eq:88.2} -- and
still \eqref{eq:88.2} is well-behaved with respect to ss homotopy
notions! It should be noted of course that the semi-simplicial
homotopy notions in $M$ can be defined, even without assuming that in
$A$ finite products exist, namely in situations when $M$ does not
admit finite products -- and hence, strictly speaking, the set-up of
section \ref{sec:51} does not apply. All this causes a slightly
awkward feeling, which I would like to clarify and see what's going
on. I suspect it should be simple enough to do it here and now.

First, assume that $A$ is stable under finite products, and under
direct sums (with small indexing set say -- for what we want to do
with $\Simplex$, finite direct sums even would be enough). We'll use
sums only with summands\pspage{297} equal to a chosen final object $e$
of $A$, in order to get a functor
\[ I\mapsto I_A : \Sets\to A,\]
where $I_A$ is the ``constant'' object of $A$ with value $I$, namely a
sum of $I$ copies of $e$ (sometimes also written $I\times e$). Using
this functor, we get a functor
\[\Delta\!\uphat = \bHom(\Delta\op,\Sets) \to M =
\bHom(\Delta\op,A),\]
which I denote by
\[K\mapsto K_A,\]
associating to any ss~set the corresponding ``constant'' (relative to
$A$) ss~object of $A$.\scrcomment{I think it's a bit unclear here
  whether the test category $\Delta$ is actually assumed to be
  $\Simplex$ after all\ldots} On the other hand, because $A$ admits
finite products, so does $M$, which enables us to make use of the
homotopy notions developed in sections \ref{sec:51} etc. Thus, if
\[\bI=(I,\delta_0,\delta_1)\]
is any interval in $\Delta\!\uphat$, considering the corresponding
``$A$-constant'' interval $\bI_A$ in $M$, we get homotopy notions in
$M$, which we may refer to as \emph{\bI-homotopy} (dropping the
subscript $A$). They can all be deduced from the elementary
\bI-homotopy between maps in $M$, which is expressed in the known way,
in terms of a map in $M$
\[ h : \bI_A \times X\to Y,\]
where $X$ and $Y$ are the source and target in $M$ of the two
considered maps, between which we want to find an elementary
\bI-homotopy $h$. This map $h$ decomposes componentwise into
\begin{equation}
  \label{eq:88.star}
  h_n : (I_n)_A \times X_n\to Y_n,\tag{*}
\end{equation}
and each $h_n$ can be interpreted, in view of the definition of $I_n$,
as a map
\begin{equation}
  \label{eq:88.starprime}
  h_n': I_n \to \Hom_A(X_n,Y_n),\tag{*'}
\end{equation}
provided we assume that taking products in $A$ is distributive with
respect to the sums we are taking, whence
\[ (I_n)_A \times X_n \simeq (I_n)_{X_n} = \text{direct sum of $I_n$
  copies of $X_n$.}\]
Now to give $h$, or equivalently a sequence of maps $h_n$ in
\eqref{eq:88.star} ``functorial in $n$ for variable $n$'', amounts to
giving a sequence of maps \eqref{eq:88.starprime}, satisfying a
corresponding compatibility relation for variable $n$. The point of
course is that (for any $I$ in $\Delta\!\uphat$ and $X,Y$ in $M$)
\emph{the set of data \eqref{eq:88.starprime} plus the compatibility
  condition make sense, formally, independently of any exactness
  assumptions on $A$}. Thus, it can be taken\pspage{298} as the formal
ingredient of a definition of ``elementary \bI-homotopy'' between two
maps in $M$, without any assumptions whatever on the category $A$ we
start with. The standard case is the one when $I=\Simplex_1$, the
``unit interval'', but never mind. The definition works just as well,
when $\Simplex$ is replaced by any (let's say small) category $\Delta$
whatever. On the other hand, it is immediate that for a functor $M\to
N$ as above, induced by a functor $A\to B$, for two maps in $M$, any
elementary homotopy between them gives rise to an elementary homotopy
of their images in $N$ -- which is just the well-known fact (in case
$\Delta=\Simplex$, $I=\Simplex_1$) that $M\to N$ is compatible with
simplicial homotopy notions.

In order to fit this into the general framework of section
\ref{sec:51}, let's remark that if $A$ is a full subcategory of a
category $A'$, then for a pair of maps in $M$, the elementary
\bI-homotopies between these are the same as when considering the
given maps as maps in $M'=\bHom(\Delta\op,A')$, in which $M$ is embedded
as a full subcategory. Now, any (small, say) category $A$ can be
embedded canonically into $A'=\Ahat$ as a full subcategory, and any
functor
\[f:A\to B\]
embeds in the corresponding functor
\[f_!:\Ahat\to\Bhat,\]
hence the functor
\[\varphi:M\to N\]
embeds in the corresponding functor
\[\varphi':M'\to N', \quad M'=\bHom(\Delta\op,\Ahat),
N'=\bHom(\Delta\op,\Bhat).\]
As \Ahat, \Bhat{} satisfy the required exactness properties, it
follows that the \bI-homotopy notions in $M',N'$ can be interpreted in
terms of the notions of section \ref{sec:51}, with respect to $\bI_A$
and $\bI_B$, defined now as (componentwise) constant objects of
\Ahat{} and \Bhat{} respectively. Still, $f_!$ commutes to finite
products only if $f$ does, so we are still left with explaining why
$M'\to N'$ is well-behaved with respect to \bI-homotopy
notions. Equivalently, we need only see this in the case of
\eqref{eq:88.2} $M\to N$, when $A$ and $B$ are supposed to have the
required exactness properties to allow for the interpretation given
above of the \bI-homotopy notions in terms of the formalism of section
\ref{sec:51}, and when moreover $f$ (as $f_!$ above)\pspage{299}
commutes with sums. This now is readily expressed by the relations
\begin{align*}
  &\varphi'(\bI_A) \simeq \bI_B , \\
  &\varphi'(\bI_A\times X) \tosim \varphi'(\bI_A)\times \varphi'(X),
\end{align*}
i.e., while $\varphi'$ does \emph{not} commute to finite products in
general, however it \emph{does} commute to the products which enter in
the description of elementary homotopies (as these products can be
expressed in terms of direct sums in $A,B$, and $f$ commutes to
these).

These reflections suggest that the notions of homotopy interval
structures and contractibility structures may be generalized, in a way
that the underlying category need no longer be stable under finite
products nor even admit a final object; and likewise, the notion of a
morphism of such structures may be generalized, without assuming that
the underlying functor should commute with finite products. The
thought that this kind of generalization may be needed had already
occurred before in these notes, in connection with the corresponding
situation about twenty five years ago, when the notion of a site was
developed. But at present, the extension doesn't seem urgent yet, and
I better stop here this long digression!

The remaining questions possibly to deal with in part \ref{ch:IV} are
all concerned with modelizers. I'll try to be brief!

\namedlabel{q:88.5}{5)}\enspace Consider an ``algebraic structure
type'', and the category $M$ of its set-theoretic realizations. I am
looking for a comprehensive set of sufficient conditions on $M$ to
ensure that $M$ is a ``canonical modelizer''. It seems natural to
assume beforehand that in $M$ (where at any rate small direct and
inverse limits must exist) internal $\bHom$'s exist, and more
generally, for $X,Y$ two objects over an object $S$ of $M$, that
$\bHom_S(X,Y)$ -- this implies that base change $S'\to S$ in $M$
commutes with small direct limits and a fortiori, that direct sums are
universal -- we may as well suppose them disjoint too. One feel quite
willing too to throw in the total $0$-connectedness assumption (cf.\
section \ref{sec:58}), and that every non-empty object has a section
over the final object. This preliminary set of conditions on an
algebraic structure species is of course highly unusual, however it is
satisfied for must ``elementary'' algebraic structures (by which I
mean $M\equeq \Ahat$ for some small category $A$), as well as for
$n$-stacks or \oo-Gr-stacks, for any $n$ between $0$ and \oo. The hope
now is, in terms of these assumptions, to give a necessary and
sufficient condition in order that a)\enspace the ``canonical''
homotopy structure on $M$ be a contractibility structure, and moreover
b)\enspace the latter structure be ``modelizing'', by which we mean
that the\pspage{300} associated \scrWoo-asphericity structure (\scrWoo{}
= usual weak equivalences) be modelizing, which will imply that for
\emph{any} basic localizer \scrW, the corresponding \scrW-asphericity
structure is modelizing.

Even if I don't look into this question now, it'll turn up soon enough
in a similar shape, when it comes to prove modelizing properties for
categories of stacks of various kinds. The best we could hope for
would be a statement in terms of the category structure of $M$ alone,
with no assumption that $M$ be defined in terms of an algebraic
structure type. If I try to formulate anything by way of wishful
testing conjecture, what comes to mind is: is it enough that there
should exist a separating contractible interval? So the first I would
try to get an idea, is to see how to make a counterexample to
this\ldots

\bigbreak
\presectionfill\ondate{3.7.}\par

% 89
\hangsection[Digression (continued) on left exactness properties of
$f_!$ \dots]{Digression
  \texorpdfstring{\textup(continued\textup)}{(continued)} on left
  exactness properties of \texorpdfstring{$f_!$}{f!}
  functors.}\label{sec:89}%
In connection with the left exactness properties of a $f_!$ functor,
considered three days ago (section \ref{sec:85}), I have been befallen
by some doubts whether any subcategory of \Cat{} containing the
subcategory of standard simplices is strictly generating. I wrote
there (p.\ \ref{p:284}) that as this is true for $\Simplex$ itself, it
``follows a fortiori'' for any subcategory $A$ of \Cat{} containing
$\Simplex$. Assuming $A$ to be full and denoting by $i$ the inclusion
functor, this is known to be equivalent (cf.\ remark \ref{rem:85.2}
same page) to $i^*:\Cat\to\Ahat$ being fully faithful, and in this
form, it doesn't look so obvious that when this is true for one full
subcategory, $A_0$ say, it should be true for any larger one $A$. This
thought had been lingering for a second while writing the ``a
fortiori'' and I then brushed it aside, because of the formulation of
being generating in terms of strict epimorphisms. Only the next day
did it occur to me that it is by no means clear that if a family of
maps $X_i\to X$ in a category $M$ is strictly epimorphic, any larger
family with same target $X$ should be ``a fortiori'' strictly
epimorphic too -- the ``a fortiori'' is known to apply only in the
case of the similar notions of epimorphic, or universally strictly
epimorphic, families of maps. After a little perplexity, I found the
situation was saved, in the case I was interested in, through the fact
that it was known from Giraud's article on
descent\scrcomment{\textcite{Giraud1964}} (Bull.\ Soc.\ Math.\ France,
Mémoire 2, 1964, prop.\ 2.5, p.\ 28) that $\Simplex$ and even the
smaller subcategory of simplices of dimension $\le 2$, is
even\pspage{301} generating by ``\emph{universally} strict
epimorphisms'', a notion which is stable under enlargement of the
family of maps, as recalled above. Thus, the statement made on p.\
\ref{p:284} does hold true. And I just checked today that, while this
stability property by enlargement is surely not always true for a
family of maps which is strictly epimorphic, however, it \emph{is}
true that if a full subcategory $A_0$ of a category $M$ is generating
by strict epimorphisms (or, as we'll say, is ``strictly generating''),
then so is any larger full subcategory $A$. This is seen by an easy
direct argument, in terms of the initial definition, as meaning that
for any object $X$ in $M$, the family of maps $a_i\to X$ with target
$X$ and source in the given subcategory ($A$ say) should be strictly
epimorphic. (For\scrcomment{\textcite{SGA4vol1}} the definition of
common variants of the notion of epimorphism, see the ``Glossaire'' at
the end of chapter 1, SGA~4, vol.~1.)

It occurred also to me that (as suspected in remark \ref{rem:85.3},
loc.\ cit.) the functor
\[i_!: \Simplexhat\to\Cat\]
coming from the inclusion functor $i:\Simplex\to\Cat$ is \emph{not}
left exact (for another reason though than first contemplated), namely
because \Cat{} is known \emph{not to be a topos} (for instance, an
epimorphism need not be \emph{strict} (or, what amounts here to the
same, \emph{effective}) -- as stated in the cited result of
Giraud). Indeed, we have the following
\begin{proposition}[which should belong to section \ref{sec:85}!]
  Let $M$ be a \scrU-category stable under small direct limits, $A$ a
  small full subcategory, $i:A\to M$ the inclusion functor, hence a
  functor
  \[i_!:\Ahat\to M.\]
  If $A$ is strictly generating \textup(i.e., $i^*:M\to\Ahat$ fully
  faithful\textup), then $i_!$ is left exact if{f} $M$ is a topos.
\end{proposition}

Indeed, the inclusion functor into \Ahat{} of $M'$, the essential
image of $i^*$ in \Ahat, admits a left adjoint ($i_!$ essentially). By
the criterion of Giraud, left exactness of this adjoint, or
equivalently of $i_!$, means that $M'$ is the category of sheaves on
$A$ for a suitable site structure on $A$, qed.
\begin{corollary}
  If $M$ is \emph{not} a topos, then $i_!$ does \emph{not} commute to
  fibered products in \Ahat{} of diagrams of the type
  \begin{equation}
    \label{eq:89.star}
    \begin{tabular}{@{}c@{}}
      \begin{tikzcd}[baseline=(O.base),column sep=tiny,row sep=small]
        b\ar[dr] & & c\ar[dl] \\ & |[alias=O]| F &
      \end{tikzcd},
    \end{tabular}\tag{*}
  \end{equation}
  with $b,c$ in $A$ and $F$ in \Ahat,\pspage{302} while it does
  commute to finite products, and to fibered products of any two
  objects of \Ahat{} over an object of $A$\kern1pt.
\end{corollary}

The ``while'' comes from prop.\ \ref{prop:85.1} and prop.\
\ref{prop:85.2} of section \ref{sec:85}, which imply too that, if $M$ is
strictly generating and whether or not $M$ is a topos, left exactness
of $i_!$ is equivalent with commutation to fibered products of the
diagrams \eqref{eq:89.star}. Hence the corollary.

This corollary answers also the perplexity raised in remark
\ref{rem:85.1} (p.\ \ref{p:283}), as to a hypothetical sharper version
of part \ref{it:85.prop1.b}, concerning fibered products. As
anticipated there, it turns out that this sharper version is not
valid, -- not without additional assumptions at any rate.

% 90
\hangsection[Review of questions (continued): 6) Existence of test
\dots]{Review of questions
  \texorpdfstring{\textup(continued\textup)}{(continued)}:
  \texorpdfstring{\textup{6)}}{6)} Existence of test functors and
  related questions. Digression on strictly generating
  subcategories.}\label{sec:90}%
After this digression on exactness properties of $f_!$ functors, let's
come back to the review of those questions not yet dealt with, which
seem more or less to belong to the present part \ref{ch:IV} of the
notes. We had stopped two days ago with the question \ref{q:88.5} of
finding some simple characterization of canonical modelizers,
comparable maybe in simplicity to the characterization we found for
test categories (in part \ref{ch:II}). This question may well turn out
to be related to the following one.

\namedlabel{q:90.6}{6)}\enspace This is the question of finding handy
existence theorems for test functors, whereas so far our attention to
test functors had been turned towards a thorough understanding of the
very notion of a test functor and its variants. I have the feeling
that, after the reflections of sections \ref{sec:78} and \ref{sec:86}
notably, the notion in itself is about understood now, so that time is
getting ripe for asking for existence theorems. As all modelizers we
have been meeting so far were associated to asphericity structures, it
seems reasonable to restrict to these, namely to the case of a given
modelizing asphericity structure
\[(M,M\suba),\]
and, if need be, even restrict to the case when this structure is
associated to a contractibility structure $M\subc$. We suppose given
moreover a test category $A$, which we may (if needed) assume to be
strict even, or even a contractor (i.e., the objects of $A$ in \Ahat{}
are moreover contractible, for the homotopy interval structure in
\Ahat{} defined by all intervals coming from $A$). The question then
is whether there exists a test functor
\[A\to M.\]
This (under the assumptions made) just reduces to the existence
of\pspage{303} a functor which be $M\suba$-\scrW-aspheric. Here,
\scrW{} is a given basic localizer, with respect to which we got an
asphericity structure. The most important case for us surely is the
one when $\scrW=\scrWoo$, namely usual weak equivalence. It is
immediate indeed that an $M\suba$-\scrW-aspheric functor is equally
aspheric for the corresponding $\scrW'$-asphericity structure of $M$,
for any basic localizer $\scrW'\supset\scrW$. Thus, if we get an
aspheric functor for \scrWoo, the finest basic localizer of all, we
get ipso facto an aspheric functor for any basic localizer
\scrW. (Note also that if an asphericity structure is modelizing for a
given \scrW, the corresponding $\scrW'$-structure is modelizing too,
for any $\scrW'\supset\scrW$; and the analogous fact holds for the
notion of a test category -- namely a \scrW-test category is also a
$\scrW'$-test category, and similarly for total asphericity of \Ahat{}
and hence for the condition of being a \emph{strict} test category.)

In case $M$ is even endowed with a contractibility structure, we will
be interested, more specifically still, in aspheric functors factoring
not only through $M\suba$, but even through $M\subc$:
\[ i:A\to M\subc,\]
while replacing the asphericity requirement on this functor, by the
stronger one that for any $x$ in $M\subc$, the object $i^*(x)$ in
\Ahat{} be \emph{contractible} (for the homotopy structure in \Ahat{}
defined by homotopy intervals coming from objects in $A$, say). In
other words, we are interested in the question of existence of
\emph{bimorphisms} of contractibility structures (in the sense of
section \ref{sec:86}) from $(M,M\subc)$ to $(\Ahat,\Ahatc)$. It may be
noted that in both cases (working with asphericity structures or with
the contractibility structures instead), in this existence question,
we may altogether forget $M$ itself, and consider it as an existence
question for functors from $A$ into either $M\suba$, or $M\subc$, with
the property that for any object $x$ in the target category $M\suba$
or $M\subc$, the object $i^*(x)$ in \Ahat{} be either aspheric, or
contractible. In the second case, we may even restrict $x$ to be in
any given subcategory $C$ of $M\subc$ generating the contractibility
structure -- and in the cases met with so far, we can find such a $C$
reduced to just one object $I$. In the case of asphericity structures,
the same holds when taking for $C$ a subcategory generating the
asphericity structure, provided however $C$ contains the image of $A$
by $i$ (which gives little hope to have $C$ restricted to just one
element!)

The interest of finding criteria for existence of \scrW-aspheric
or\pspage{304} or more stringently still of ``c-\emph{aspheric
  functor}'' (as we may call them) is rather evident, as it gives a
way, via $i^*$, for any homotopy type described in terms of a
``model'' $x$ in $M$, to find a corresponding model $i^*(x)$ in
\Ahat. The situation would be more satisfactory still if we could find
the test functor $i$ such that the corresponding functor
\begin{equation}
  \label{eq:90.1}
  i_!:\Ahat\to M\tag{1}
\end{equation}
be modelizing too (assuming $M$ to be stable under small direct
limits, so that $i_!$ is defined as the left adjoint of
\begin{equation}
  \label{eq:90.2}
  i^*:M\to\Ahat\quad\text{.)}\tag{2}
\end{equation}
In this case, for a homotopy type described by a model $K$ in \Ahat,
$i_!(K)$ gives a description of the same by a model in $M$.

Maybe we should remember though that even if we do not know about any
test functor from $A$ to $M$, still we always can find in three steps
a modelizing functor
\[M\to\Ahat,\]
namely a composition
\begin{equation}
  \label{eq:90.star}
  M \xrightarrow{j^*} \Bhat \xrightarrow{i_B}
  \Cat\xrightarrow{j_A=i_A^*} \Ahat,\tag{*}
\end{equation}
where $j:B\to M$ is an $M\suba$-aspheric functor from an auxiliary
small category, which we may assume to be a test category, by a mild
extra assumption on $M$ (cf.\ cor.\ \ref{cor:79.3} p.\
\ref{p:253}). The modelizing functor we thus get has the disadvantage
of not being left exact, whereas the looked-for functor $i^*$ commutes
to small inverse limits. Still, the composition \eqref{eq:90.star} is
pretty near to being left exact, it commutes to fibered products
(because $i_B$ does) which is the next best -- we can view it as a
left-exact functor from $M$ to $\Ahat_{/E}$, where $E$ is the image in
\Ahat{} of the final object of $M$ (assuming $e_M$ exists).

There is another advantage still of having a test functor $i:A\to M$,
rather than merely using \eqref{eq:90.star}, namely it allows us to
``enrich'' the category structure of $M$, in such a way as to get
``external $\Hom$'s'' of objects of $M$, with ``values in \Ahat'', by
defining, for $x,y$ in $M$, the object $\bHom(A)(x,y)$ of \Ahat{} as
\begin{equation}
  \label{eq:90.3}
  \bHom(A)(x,y) = \bigl\{ a\mapsto\Hom_M(i(a)\times x,y)\bigr\}.
  \tag{3}
\end{equation}
Such enriched structure, when $A=\Simplex$, plays an important part in
the second part of Quillen's treatment of homotopical algebra, under
the name of (semi-)simplicial categories, especially with the notion
of (semi-)simplicial model categories, which looks quite handy
indeed.\pspage{305} We should of course define composition of the
$\bHom(A)$'s, as required too in Quillen's set-up. This is done by
relating the $\bHom(A)$'s to the well-known internal $\bHom$'s in
$M\uphat$ -- which will show at the same time that for formula
\eqref{eq:90.3} to make sense, we do not really have to assume $M$ be
stable under binary products, as we can interpret the products
$i(a)\times x$ as being taken in $M\uphat$, as well as the $\Hom$, so
as to get
\[\Hom_{M\uphat}(i(a)\times x,y) \simeq \Hom_{M\uphat}(i(a),
\bHom_{M\uphat}(x,y)),\]
hence
\begin{equation}
  \label{eq:90.4}
  \bHom(A)(x,y)\simeq i^*(\bHom_{M\uphat}(x,y)),\tag{4}
\end{equation}
where $i^*$ in the right hand side is interpreted as a functor
\[ i^*:M\uphat\to\Ahat,\]
rather than $M\to\Ahat$. (I leave to the reader the task of enlarging
the basic universe, as need may be\ldots) As $i^*$ commutes to
products, the evident composition of the internal $\bHom$'s in
$M\uphat$ gives rise to the looked-for composition of the
$\bHom(A)$'s, with the required associativity properties. Of course,
in case $M$ is stable under binary products and $\bHom$'s, which
apparently is going to be the case in all modelizing situations, there
is no need in the interpretation \eqref{eq:90.4} to introduce the
prohibitively large $M\uphat$, and we can work in $M$ throughout.

There is an important relation though on the external $\bHom(A)$'s
which we would like to be true for a satisfactory formalism, namely
\begin{equation}
  \label{eq:90.5}
  \Gamma_\Ahat(\bHom(A)(x,y)) \fromsim \Hom_M(x,y),\tag{5}
\end{equation}
where $\Gamma_\Ahat$ just means $\Hom_\Ahat(e_\Ahat,\dots)$. This is
equivalent to the requirement
\begin{equation}
  \label{eq:90.6}
  i_!(e_\Ahat)\simeq e_M\tag{6}
\end{equation}
(assuming a final object $e_M$ in $M$ to exist), or equivalently
\begin{equation}
  \label{eq:90.7}
  i(e_A)\simeq e_M\tag{7}
\end{equation}
if we assume moreover $e_A$ to exist. Thus, it will be natural to ask
for test functors satisfying the extra condition \eqref{eq:90.5} or
\eqref{eq:90.6} -- and when trying to construct test functors in
various situations (even without being aware of constructing test
functors, as Mr~Jourdain\scrcomment{the reference is of course to
  Molière's \emph{comédie-ballet},
  \href{https://en.wikipedia.org/wiki/Le_Bourgeois_gentilhomme}{Le
    Bourgeois gentilhomme}} was ``doing prose without knowing
it''\ldots), the very first thing everybody has been doing
instinctively was to write down formula \eqref{eq:90.7},\pspage{306} I
would bet!

The motivation for wanting to find test functors being reasonably
clear by now, what kind of existence theorems may we hope for? When
$A$ is such a beautiful test category as $\Simplex$, $\Square$ or
$\Globe$, I would expect that for practically any $M$ endowed with a
modelizing contractibility structure say, under mild restrictions
(such as the exactness assumptions which are natural in the modelizing
story), there should exist a test functor indeed. What I feel less
definite about is whether it is reasonable to expect we can find $i$
even such that $i_!$ be modelizing too, in which case we would expect
of course that the pair of equivalences of categories
\begin{equation}
  \label{eq:90.8}
  \begin{tikzcd}[cramped,sep=small]
    \HotOf_A \ar[r,shift left] & \HotOf_M \ar[l,shift left]
  \end{tikzcd}\tag{8}
\end{equation}
defined in terms of $i_!$ and $i^*$ should be quasi-inverse to each
other, and the adjunction maps deduced from those between the functors
$i_!$ and $i^*$ themselves. This in turn is equivalent with the
adjunction morphism
\begin{equation}
  \label{eq:90.9}
  F\to i^*i_!(F)\tag{9}
\end{equation}
being a weak equivalence, for any object $F$ in \Ahat. A test functor
satisfying this exacting extra property merits a name of its own, we
may call it a \emph{perfect test functor} (or a \emph{perfect aspheric
  functor}, when not making any modelizing assumptions on $A$ or
$M$). Thus, the existence problem of finding test functors can be
sharpened to the one of finding perfect ones. Remember though that the
most familiar test functor of all (besides the geometric realization
functor $\Simplex\to\Spaces$, namely the inclusion
\begin{equation}
  \label{eq:90.10}
  i:\Simplex\to\Cat\tag{10}
\end{equation}
giving rise to the nerve functor (introduced for the first time, I
believe, in a Bourbaki talk of mine, on passage to quotient by a
preequivalence relation in the category of schemes\ldots), is
\emph{not} perfect. The most natural perfect test functor from
$\Simplex$ into \Cat, more generally from any weak test category $A$
into \Cat, is of course $i_A$ -- the functor indeed which has been
dominating the whole modelizing picture in our reflections from the
start. In the case of $A=\Simplex$, Thomason discovered another
perfect test functor, conceptually less simple surely, namely
$i_!\Sd^2$, where $\Sd$ is the ``barycentric subdivision functor''.
I suspect there must be an impressive bunch of perfect test functors
from $\Simplex$ with values in more or less any given modelizer, not
only the basic one -- and the question here is to get a clear picture
of\pspage{307} how to get them, and the same of course for test
functors which need not be perfect, including \eqref{eq:90.10}.

Next question then would be to see whether the existence theorems we
may get for $\Simplex$, or its siblings $\Square$ and $\Globe$ and the
like, still hold true for a more or less arbitrary test category, or
contractor. If so, this would be a very strong confirmation of the
feeling which has been prompting the reflections in part \ref{ch:II},
namely that for the purpose of having ``all-purpose''-models for
homotopy types (insofar as this is feasible), any strict test
category, or any contractor at any rate, is just as good as simplices
or cubes, which people have kept working with for the last twenty-five
years. If not, it will be quite interesting indeed to come a grasp of
what the relevant extra features of $\Simplex$ and the like are, and
how restrictive they are.

I doubt I will dive into these questions, still less come to a clear
picture, in the present part of the notes. Still, before leaving the
topic now, I would like to write down some hints I came upon while
doing my scratchwork on homotopy properties of \Ahat{}
categories. When looking for functors
\[A\to M\]
having some specified properties (such as being a test functor, or a
perfect one, etc.), we may view this question as meaning that we are
looking for an object with specified properties in the category
\[M^A =\bHom(A,M).\]
Presumably, this category is endowed with an asphericity or
contractibility structure if $M$ is (as we assume), presumably even a
modelizing one. This reminds me that as far as the notion of weak
equivalence goes, there may be even several non-equivalent ways of
finding such structure on $M^A$, one being modelizing, whereas
another, more useful one in some respects, is not. Thus, if $M$ is of
the type \Bhat, we may rewrite $M^A$ as
\[ M^A \equeq (A\op\times B)\uphat \equeq P\uphat;\]
hence we get the weak equivalence notion coming from $P\uphat$,
disregarding its product structure, which is modelizing indeed if $B$
is a test category and $A$ aspheric, hence $A\op\times B$ a test
category. The structure which should be of more relevance though for
our purpose should be a considerably finer one (namely with a smaller
set of weak equivalences), which we may\pspage{308} visualize best
maybe by writing
\[M^A=(\bHom(A\op,M\op))\op,\]
i.e., interpreting the dual of $M^A$ as the category of $A$-objects of
$M\op$, for instance (if $A=\Simplex$) as the dual of the category of
ss~objects of $M\op$. Now, Quillen has given handy conditions, in
terms of projectives of $M\op=N$, namely in terms of injectives of
$M$, for the category $\bHom(\Simplexop,N)$ of ss~objects of $N$ to
be a closed model category -- hence the dual category $M^A$ will turn
out as a closed model category too, under suitable conditions
involving existence of injective objects in $M$. These conditions are
satisfied for instance when $M$ is a topos, and notably when $M$ is of
the type \Bhat{} -- quite an interesting particular case indeed!
Assuming that $M$ is stable under both types of limits, so is $M^A$,
hence there is an initial and final object, and according to Quillen's
factorization axiom, the map from the former to the latter can be
factored through an object
\[\text{$F$ in $M^A$,}\quad\text{i.e.,}\quad F:A\to M,\]
which is \emph{cofibering}, and such that $F\to e$ is a \emph{trivial
  fibration}. The idea is that these conditions mean more or less, at
any rate imply, that $F$ is a test functor.

I hit upon this ``way out'' while trying to construct test functors
from $\Simplex$ to any elementary modelizer \Bhat, in order to try and
check that \Bhat{} is a (semi)simplicial model category in the sense
of Quillen. The intuitive idea of constructing inductively the
components $F_n$ of $F$ was simple enough, still I got stuck in some
messiness and did not try to push through this way, all the less as
this naive approach had no chance of generalizing to the case of a
more or less general test category $A$. Of course, for the time being
Quillen's theorems, about certain categories $\bHom(A\op,N)$ being
closed model categories, is equally restricted to the case when
$A=\Simplex$, which looks as usual like a rather arbitrary
assumption. Thus, to ``test'' whether the feeling about \emph{any}
test category more or less being ``just as good'' as $\Simplex$, a
second point would be to see whether Quillen's theorems extend, which
presumably is going to be very close to the first point I raised.

I take this occasion to raise a third point -- where there is no
reason to restrict to an $M$ which is modelizing (neither was there
such reason before, when phrasing everything in terms of aspheric or
c-aspheric functors, rather than test functors\ldots). Namely,
assuming as\pspage{309} above that for any test category or contractor
$A$, $M^A$ or $M^{A\op}$ can be endowed with an asphericity structure
or a closed model structure, or at any rate with a set of weak
equivalences, hence a localization or corresponding ``homotopy
category''. What one would expect now is that up to (canonical)
equivalence, the latter does not depend upon the choice of $A$, and
hence is the same for arbitrary $A$ as when using $\Simplex$, i.e.,
simplices. This should be true at any rate for $M^{A\op}$ and when $M$
is a topos -- which means that \emph{Illusie's derived category
  $\mathrm D_\bullet(X)$ of the category of semisimplicial sheaves on
  a topos $X$, could be constructed by using, instead of ss~objects,
  $A$-objects of the category of sheaves on $X$, where $A$ is any test
  category.} This should be one of the main points to settle in part
\ref{ch:VII} of the notes.

There is a slight discrepancy though between the first point, about
existence of test functors with values in $M$, depending on a given
asphericity or contractibility structure of $M$, and the second and
the third, which seem to depend only on the category structure of
$M$. This is further evidence that the set of questions raised here is
still far from being clear in my mind yet. Stating them now, however
confusingly, is a first step towards clarification!

\bigbreak
\presectionfill\ondate{4.7.}\par

Just still two comments about the existence questions for test
functors, before going over to the next questions in our present
review. One is that for given $A$, to prove that for rather general
modelizing $M$ there exists a test functor from $A$ to $M$, we are
reduced to the case when $M$ is of the type \Bhat, where $B$ is a test
category -- namely, it is enough to take a $B$ such that there exists
a test functor $B\to M$. If we can even find a perfect test functor
from some $B$ to $M$, then likewise the existence question for perfect
test functors from $A$ to $M$ is reduced to the case when
$M=\Bhat$. These comments may be useful for applying to the situation
Quillen's model theory, as envisioned on the previous page -- as his
criteria for $\bHom(\Simplexop,N)$ to be a closed model category
apply when $N$ is the dual of a topos, for instance the dual of
\Bhat. The second comment is about Thomason's result concerning the
standard inclusion
\[i:\Simplex\hookrightarrow \Cat,\]
which can be expressed by saying that, although $i$ itself is not a
\emph{perfect} test functor, however, for any integer $n\ge2$, the
composition\pspage{310}
\[i_n=i_!\Sd^n \alpha: \Simplex\to\Simplexhat\to\Simplexhat\to\Cat\]
is a perfect test functor, where $\alpha:\Simplex\to\Simplexhat$ is
the canonical inclusion. It is tempting to surmise that this result is
not special to $i$ alone, but that it holds for a large class, if not
all, test functors from $\Simplex$ to asphericity or contractibility
modelizers. Here, $\Sd^n$ denotes the $n$'th iterate of the
barycentric subdivision functor $\Sd$ (following now the notation in
Thomason's paper, which presumably is standard, while I have been
using ``Bar'' in part \ref{ch:II} of the notes). Presumably, functors
analogous to $\Sd$ can be defined in any elementary modelizer \Ahat,
as suggested by the natural constructions arising in connection with
the factorization property for a closed model structure on
\Ahat. Thus, possibly there is a general method in view for deducing
perfect test functors from ordinary ones. However, it definitely seems
to me that the natural place for these existence questions is in part
\ref{ch:V} of the notes, as they seem intimately related to an
understanding of the homotopy structures of elementary modelizers, and
more specifically to the closed model structures to which such
modelizers give rise in various ways.

\starsbreak

Before proceeding, I would like to state still another afterthought to
the reflections of section \ref{sec:89}, about strictly generating
subcategories of a category $M$. I recall that a family of arrows in
$M$ with same target $X$
\[u_i:X_i\to X\]
is called \emph{strictly epimorphic}, if for every object $Y$ of $M$,
the corresponding map
\begin{equation}
  \label{eq:90.starbis}
  \Hom(X,Y) \to \prod_i\Hom(X_i,Y)\tag{*}
\end{equation}
is injective (which is expressed by saying that the family $(u_i)$ is
\emph{epimorphic}), and if \emph{moreover} the following obviously
necessary condition for an element $(f_i)$ of the product set of
\eqref{eq:90.starbis} to be in the image of the map \eqref{eq:90.starbis},
is also sufficient:
\begin{description}
\item[\namedlabel{cond:90.Comp}{(Comp)}]
  For any two indices $i,j$ (possibly equal) and any commutative
  square
  \[\begin{tikzcd}[sep=tiny]
    & T\ar[dl,"v_i"']\ar[dr,"v_j"] & \\
    X_i\ar[dr,"u_i"'] & & X_j\ar[dl,"u_j"] \\
    & X &
  \end{tikzcd}\]
  in $M$, the relation $f_iv_i=f_jv_j$ holds.
\end{description}
It is immediate that the condition for $(u_i)$ to be strictly
epimorphic\pspage{311} depends only on the \emph{sieve} $X_0$ of $X$
in $M$ (namely, the subobject of $X$, viewed as an object of
$M\uphat$) generated by the $u_i$'s -- we'll say also that this
\emph{sieve is strictly epimorphic}. One should beware that this does
not mean of course that $X_0\to X$ is epimorphic in $M\uphat$ (which
would imply $X_0=X$, i.e., that one of the $u_i$'s admits a section,
i.e., a right inverse); nor is it true that if a sieve is strictly
epimorphic, a large one should be so too -- which means that when
adding more arrows to a strictly epimorphic family, the family need
not stay str.\ ep.

We'll say that the family $(u_i)$ is \emph{universally strictly
  epimorphic} if it is strictly epimorphic, i.e., the corresponding
sieve $X_0$ is, and if the latter remains so by arbitrary base change
$X'\to X$ in $M$, i.e., if the corresponding sieve $X'_0$ of $X'$ is
strictly epimorphic too. If the fibered products
\[X'_i = X_i\times_X X'\]
exist in $M$, this condition also means that the corresponding family
of maps
\[ u'_i:X'_i\to X'\]
is strictly epimorphic. The condition that $(u_i)$ be univ.\ str.\
epimorphic again depends only on the generated sieve, it is moreover
\emph{stable under base change}, and equally \emph{stable under adding
  new arrows}, i.e., replacing a sieve in $X$ by a larger one.

It should be noted that if the fibered products $X_i\times_X X_j$
exist in $M$, then the compatibility condition \ref{cond:90.Comp}
above is equivalent to the one obtained by restricting to
\[T = X_i\times_X X_j,\]
with $v_i$ and $v_j$ the two projections.

Assume the indices $i$ are objects of a category $I$, and the $X_i$
are the values of a functor
\[I\to M,\]
and that the family of arrows $(u_i)$ turns $X$ into the direct limit
in $M$ of the $X_i$:
\[X=\varinjlim_i X_i,\]
then it follows immediately that the family $(u_j)$ is strictly
epimorphic.

After these terminological preliminaries, we're ready to give
the\pspage{312} following useful statement, which is lacking in
SGA~4\scrcomment{\textcite{SGA4vol1}} 
Chap.~I (compare loc.\ cit.\ prop.\ 7.2, page 47, giving part of the
story):
\begin{proposition}
  Let $M$ be a \scrU-category, $A$ a small full subcategory, $i:A\to
  M$ the inclusion functor, hence a functor
  \[i^*:M\to \Ahat.\]
  For any object $X$ of $M$, we consider the family $F_X$ of all
  arrows in $M$ with target $X$, source in $A$\kern1pt. The following
  conditions are equivalent:
  \begin{enumerate}[label=(\roman*),font=\normalfont]
  \item\label{it:90.i}
    For any $X$ in $M$, the family $F_X$ is strictly epimorphic.
  \item\label{it:90.ii}
    For any $X$ in $M$, the family $F_X$ is universally strictly
    epimorphic.
  \item\label{it:90.iii}
    For any $X$ in $M$, $F_X$ turns $X$ into a direct limit of the
    composition functor $A_{/X}\to A\to M$, i.e.,
    \[X \fromsim \varinjlim_{A_{/X}} a.\]
  \item\label{it:90.iv}
    The functor $i^*$ is fully faithful.
  \end{enumerate}
\end{proposition}

Proof left to the reader (who may consult loc.\ cit.\ for
\ref{it:90.i} $\Rightarrow$ \ref{it:90.iii} $\Leftrightarrow$
\ref{it:90.iv}, so that only \ref{it:90.i} $\Rightarrow$
\ref{it:90.ii} is left to prove).
\begin{definition}
  When the equivalent conditions above are satisfied, we'll say that
  $A$ is a \emph{strictly generating} subcategory of $M$.
\end{definition}

NB\enspace The notion makes sense too without assuming $A$ to be
small, nor $M$ to be a \scrU-category, by passing to a larger universe
(it is immediate for \ref{it:90.i} or \ref{it:90.iii} that the
condition does not depend on the choice of the universe.)
\begin{corollary}
  If $A$ is strictly generating in $M$, then so is any larger full
  subcategory $B$.
\end{corollary}

This is clear by criterion \ref{it:90.ii} (whereas it isn't by any one
of the other two criteria!).

% 91
\hangsection[Review of questions (continued): 7) Homotopy types of
\dots]{Review of questions \texorpdfstring{\textup(continued\textup):
    \textup{7)}}{(continued): 7)} Homotopy types of finite type,
  \texorpdfstring{\textup{8)}}{8)} test categories with boundary
  operations, \texorpdfstring{\textup{9)}}{9)}
  miscellaneous.}\label{sec:91}%
I see three more questions to review -- presumably they will be a lot
shorter than the last!

\namedlabel{q:91.7}{7)}\enspace\textbf{Description of homotopy types
  ``of finite type'',} in terms of an elementary modelizer \Ahat. In
terms of the modelizer \Spaces, a natural finiteness condition on a
homotopy type is that it may be described (up\pspage{313} to
isomorphism) as the homotopy type of a \emph{space admitting a finite
  triangulation}. In terms of ss~sets, i.e., of the modelizer
\Simplexhat, the natural finiteness condition, suggested by the
algebraic formalism, is that the homotopy type be isomorphic to one
defined by an object \emph{``of finite presentation'' in} \Simplexhat,
namely one which is a direct limit of a \emph{finite} diagram in
\Simplexhat, made up with simplices, i.e., coming from a diagram in
$\Simplex$. It is clear that the first finiteness condition implies
the second, by using a total order on the set of vertices of the
triangulation. The converse shouldn't be hard either, using an
induction argument on the number of simplices occurring in the
diagram, and using the fact that any quotient object in \Simplexhat{}
of a simplex is again a simplex, hence also a subobject of a simplex
is a union of subsimplices; from this should follow by induction that
the geometric realization of a ss~set of finite presentation is
endowed with a natural compact \emph{piecewise linear structure}, and
hence can be finitely triangulated. Presumably, one can even find a
canonical triangulation, using twofold barycentric subdivision $\Sd^2$
(again!) on any simplex. All this is surely standard knowledge, and I
don't feel like diving into technicalities on this matter, unless I am
forced to.

If we start with an arbitrary test category $A$, the notion of an
object of finite presentation in \Ahat{} still makes sense. Indeed, in
any category $M$, stable under filtering small direct limits, we may
define objects of finite presentation as those for which the
corresponding \emph{covariant} functor
\[Y\mapsto \Hom(X,Y)\]
commutes to filtering direct limits. If $M$ is stable under some type
of finite direct limits, say under any finite direct limits, then so
is the full subcategory $M\subfp$ of objects of finite presentation of
$M$. In the case when $M=\Ahat$, $A$ any small category, it is obvious
that objects of $A$, and hence finite direct limits of such, are of
finite presentation, and it is not difficult to show that the converse
equally holds. As a matter of fact $\Ahatfp$ \emph{can be viewed as
  the solution of the $2$-universal problem of ``adding finite direct
  limits to $A$''}.

Coming back to the case when $A$ is a test category, and hence \Ahat{}
is modelizing, one may ask for conditions upon $A$ which ensure that
the homotopy types of finite presentation are exactly those isomorphic
to the homotopy types defined by objects of \Ahatfp. One expects that
some stringent\pspage{314} extra condition is needed on $A$ to ensure
this. To see this, let's take a finite group $G$, and an aspherical
topological space $E_G$ upon which $G$ operates freely, with quotient
$B_G$, a classifying space of $G$. If $G\ne1$, the homotopy type of
$B_G$ isn't of finite type, because $B_G$ has non-vanishing cohomology
groups in arbitrarily high dimensions, as well known. We could
transpose the following construction in either \Cat{} or \Simplexhat{}
say, but we may as well work in the modelizer $M=\Spaces$, and take
any small full subcategory $A$ of $M$, containing the unit interval
but not the empty space, and stable under finite products -- which
implies that $A$ is a strict test category. We'll take $A$ large
enough to contain $E_G$, and small enough to be made up with aspheric
spaces, hence the inclusion functor
\[i: A\to\Spaces\]
is a test functor. Now consider the quotient object \emph{in} \Ahat
\[F = E_G / G,\]
i.e., the presheaf on $A$
\[ F : T\mapsto \Hom(T,E_G)/G \simeq \Hom(T,B_G),\]
where the last isomorphism comes from the fact that the object $T$ of
$A$ is aspheric and hence $1$-connected. Thus, we get
\[F\simeq i^*(B_G),\]
hence the homotopy type defined by $F$ is the homotopy type of $B_G$,
which is not of finite type, despite the fact that $F$ is of finite
presentation.

In order to ensure that the homotopy type defined by any object in
\Ahatfp{} be of finite type, it may be useful perhaps to make on $A$
the assumption that any quotient in \Ahat{} of an object in $A$ is
isomorphic to an object in $A$, and that the set of all subobjects in
\Ahat{} of an object $a$ in $A$ (i.e., the set of all sieves on $a$)
is finite -- possibly too that for any two objects $a$ and $b$ of $A$,
$\Hom(a,b)$ is finite. As for the opposite inclusion, namely that any
homotopy type of finite type can be described by an object of \Ahatfp,
this would follow from the existence of a \emph{perfect} test functor
from $\Simplex$ into \Ahat, factoring through\pspage{315}
\Ahatfp. Thus, the present question about finiteness conditions, seems
to be related (possibly) to the previous one about existence of
various types of test functors.

The condition for a strict test category $A$ we are looking at is
surely satisfied, besides $\Simplex$, by the cubical and hemispherical
test categories $\Square$ and $\Globe$, and surely also by any finite
products of these. I add this comment, of course, in order to ``push
through'' the point that not any more with respect to finiteness
conditions on homotopy types, than (presumably at least, for the time
being)in any other essential respect concerning the ability for
expressing basic situations and facts in homotopy or cohomology
theory, the category of simplices stands singled out by itself from
all other test categories. Nor does it seem that the ``trinity''
\[ \Simplex, \Square, \Globe\]
has this property, with the only exception so far, possibly, of the
Dold-Puppe theorem (as no other test category except these is known to
me for which a Dold-Puppe theorem in its strict form holds true
(compare reflections section \ref{sec:71})).

\namedlabel{q:91.8}{8)}\enspace
I could make the same point in favor of more general test categories
than the trinity above, when it comes to the existence of an algorithm
for computing homology and cohomology groups, using suitable
\emph{boundary operators}. What is meant by these is clear of course
for the three types of complexes, but then it extends in an obvious
way to multicomplexes too -- which means that for the test categories
deduced from the trinity by taking finite products, we still get an
algorithm for cohomology via boundary operators. Of course, for any
test category $A$, using a test functor (if we can find one) from one
of the three above (say) into \Ahat{} will allow us to reduce
``computation'' of homology and cohomology invariants in terms of a
model in \Ahat, to the case of the corresponding type of complexes --
hence again an algorithm (similar to the familiar one of computing the
cohomology of an object of \Cat{} semisimplicially, via the
nerve). But this is cheating of course! The question I want to raise
here is about existence of ``boundary operations'' \emph{in} $A$,
similar to the familiar ones used for the three basic types of
complexes, and allowing to compute the homology and cohomology groups
of an object of \Ahat{} in the usual way, involving suitable
\emph{signs} $+$ or $-$ associated to the various boundary
operations.\pspage{316} It shouldn't be hard, I feel, to pin down
exactly what is needed for getting such a formalism. The intuitive
idea behind it (suggested by the example of standard complexes and
multicomplexes) is that \emph{such a formalism should be associated to
  cellular decompositions of $n$-cells for variable $n$}, such that
the interior of each $n$-cell should be an open cell of the
subdivision. There may of course be several $n$-cells for the same
$n$, which are not combinatorially isomorphic. When trying to express
this idea in a precise way, we are led to assume, as an extra
structure on the would-be test category $A$, a functor
\begin{equation}
  \label{eq:91.1}
  i:A\to\Ord\tag{1}
\end{equation}
of $A$ into the category of ordered sets, such that for any $a$ in
$A$, $i(a)$ be a \emph{finite} ordered set, whose geometric
realization (cf.\ section \ref{sec:22}) is an $n$-cell for suitable
$n\eqdef\dim(a)$. We assume moreover that $i(a)$ has a largest element
$e(a)$, and that the geometric realization of $i(a)\setminus\{e(a)\}$
is the bounding $(n-1)$-sphere of the $n$-cell $\abs{i(a)}$:
\begin{equation}
  \label{eq:91.2}
  \abs{i(a)^*} \simeq \mathrm S^{n-1}, \quad\text{where}\quad
  i(a)^* \eqdef i(a)\setminus\{e(a)\},\tag{2}
\end{equation}
which will imply the precedent condition, namely
\begin{equation}
  \label{eq:91.3}
  \abs{i(a)}\simeq \mathrm B^n,\tag{3}
\end{equation}
as $\mathrm B^n$ can be identified to the cone over $\mathrm
S^{n-1}$. As another condition, we need that
\medbreak
\noindent(\namedlabel{eq:91.4}{4})\hfill%
\parbox[t]{0.9\textwidth}{For any $a$ in $A$ and $x\in i(a)$, there
  exists $b$ in $A$ and an isomorphism
  \[i(b)\tosim i(a)_{/x} \eqdef\set[\big]{y\in i(a)}{y\le x}
  \hookrightarrow i(a),\]
  induced by a map
  \[b\to a\]
  in $A$.}\par
\medbreak
\noindent It is enough to make this assumption for $x$ of codimension
$1$ in $i(a)$, which will imply that it is true for any $x$. It seems
reasonable on the other hand to make the assumption that for a given
$x$, the object $b$ in \eqref{eq:91.4}, viewed as an object of
$A_{/a}$, is determined up to a unique isomorphism, we may call it
$a_x$, and $\partial_x$ the canonical map of $b$ into $a$
\begin{equation}
  \label{eq:91.5}
  \partial_x:a_x\to a \qquad
  \begin{tabular}[t]{@{}c@{}}
    ($x\in i(a)$, of codim.\ $1$ in $i(a)$, \\
    i.e., $\dim(x)=\dim(a)-1$).
  \end{tabular}
  \tag{5}
\end{equation}
As an extra structure, we need for any $a$ in $A$
\begin{equation}
  \label{eq:91.6}
  \omega_a,\quad\text{an \emph{orientation} of the $n$-cell
    $i(a)$}\quad(n=\dim(a)).\tag{6}
\end{equation}
This allows us, for any $x$ as in \eqref{eq:91.5}, to define a
signature
\begin{equation}
  \label{eq:91.7}
  \varepsilon(x)\quad\text{or}\quad \varepsilon_a(x)\in\{+1,-1\},\tag{7}
\end{equation}
which\pspage{317} will be $+1$ or $-1$, depending on whether in the
inclusion
\[\abs{\partial_x}:\abs{i(a_x)}\to\abs{i(a)},\]
the orientation $\omega(a_x)$ is induced ``à la Stokes'' by the
orientation $\omega(a)$ of the ambient $n$-cell, or not. Having the
boundary operations \eqref{eq:91.5} with their signatures
\eqref{eq:91.7}, and the decomposition
\begin{equation}
  \label{eq:91.8}
  A = \coprod_{n\ge0} A_n,\quad\text{where $A_n=\set[\big]{a\in\Ob A}{\dim
      a=n}$,}\tag{8}
\end{equation}
we get in the usual way, for any contravariant functor $K_\bullet$
from $A$ with values in an additive category, a corresponding chain
complex in this category, with components
\begin{equation}
  \label{eq:91.9}
  K_n = \coprod_{a\in A_n} K_\bullet(a),\tag{9}
\end{equation}
and boundary operators defined in the usual way via \eqref{eq:91.5}
and \eqref{eq:91.7}. Applying this to the case of the category \Ab{}
of abelian groups, or to its dual, and to the abelianization of an
object $X$ of \Ahat, we obtain a tentative way for computing the
homology and cohomology groups of the homotopy type of $A_{/X}$, and
similarly for any system of twisted coefficients on $X$. The question
which arises here is to write down a set of natural extra conditions
on the data, which will ensure that we do get a canonical isomorphism
between the ``homology'' and ``cohomology'' groups thus constructed,
and the usual homology and cohomology invariants of the object
$A_{/X}$ of \Cat. Moreover, we would like too to have conditions to
ensure that $A$ is a test category, or even a strict one.

One difficulty here, if one really wants a test category and not just
a weak one (which may not be without any problem either), is that
presumably for this, we'll need suitable \emph{degeneracy operations},
which may well turn out a very exacting condition indeed! The
skeptical reader may wonder, as I am just doing myself, whether there
will be any example within the set-up I propose, which does not reduce
to a finite product of test categories in our trinity.\pspage{318}

I just spent a while trying to find some convincing example, by using
a suitable full subcategory $A_0$ of \Ord, made up with finite sets
satisfying the assumption \eqref{eq:91.2} above for $i(a)$, under some
additional assumption on $A_0$ such as stability under finite products
and under passage from $a$ to an object $a_{/x}$, and that $A_0$
contain the ordered set
\[I =
\begin{tikzcd}[baseline=(O.base),cramped,row sep=-3pt,column sep=small]
  \bullet\ar[dr] & \\ & |[alias=O]| \bullet \\ \bullet\ar[ur] &
\end{tikzcd},\]
whose geometric realization is the segment $\mathrm B^1$ with its
usual cellular decomposition. In terms of $A_0$ and introducing
``orientations'' of object of $A_0$, the idea was to define another
category $A$ (of pairs $(a,\omega)$, with $a$ in $A_0$ and $\omega$ an
orientation of $\abs a$), \emph{stable under finite products} so that
\Ahat{} is totally aspheric, together with a functor
$i:A\to\Ord\hookrightarrow\Cat$ such that $i^*(I)$ should be
representable, and hence furnish the homotopy interval needed to
ensure that $A$ is a test category. The first idea that comes to mind,
namely define a map from $(a,\omega)$ to $(a',\omega')$ as merely a
map from $a$ to $a'$ in $A_0$, is nonsense unfortunately, as in the
data \eqref{eq:91.6}, the orientations will not be stable under
isomorphisms, a condition which I forgot to state before, and which is
visibly needed in order to be able to define the differential between
the $K_n$'s. If we try to define $A$ taking into account this
compatibility condition, we loose existence of products, anyhow
$i^*(I)$ isn't representable anymore, so why should it be aspheric
over the final object, so why should the functor $i$ be a test functor?

The difficulty I find in carrying through any explicit example for a
``test category with boundary operations'', except those which stem
from our trinity, is rather intriguing I feel. The question is whether
maybe in this direction, one might get at an intrinsic description of
the trinity, in terms of the rather natural structure species of a
``test category with boundary operations''. This is the second
instance where the thought arises that the three standard test
categories $\Simplex$, $\Square$ and $\Globe$ may be distinguished in
some respects -- the first instance was in relation to the Dold-Puppe
theorem.

\namedlabel{q:91.9}{9)}\enspace\textbf{Miscellaneous residual
  questions from part \ref{ch:II}.} One of these was about the category
$\Simplexf$ of simplices without degeneracies being a weak test
category (cf.\ section \ref{sec:43}) -- while it is definitely
\emph{not} a test category. It seems worth while to write down a proof
for this, maybe too for the analogous statements for
$\Square^{\mathrm f}$ and $\Globe^{\mathrm f}$. This reminds
me\pspage{319} too that I never got around to introducing formally the
hemispherical test category, which presumably will be very useful when
it comes to studying stacks -- this too could be done in part
\ref{ch:IV}, as well as proving of course that $\Globe$ is a strict
test category indeed, or better still, a contractor. It may be fun too
constructing test functors from any one of the three basic test
categories in the trinity, to the category of complexes defined by the
two others -- six cases altogether to consider! But as I am not in the
process of writing the\scrcomment{pity!} ``Elements d'Algèbre
Homotopique'', maybe I will skip this!

In the same section \ref{sec:43}, I raised the question as to whether
the ordered set of all non-empty finite subsets of a given infinite
set, viewed as a category in the usual way for an ordered set, was a
weak test category (on page \ref{p:78} it was seen not to be a test
category, and it is immediate then that it is not totally aspheric
either). One interesting application, as noticed there, would be to
the effect that \Ord, the subcategory of \Cat{} defined by ordered
sets, is a modelizer (for the induced notion of weak
equivalences). Now the question arises moreover whether this
modelizing structure comes from an asphericity structure, or even from
a contractibility structure -- and the same question arises in the
more general situation described in the proposition of page
\ref{p:74}.

A last question along these lines I would like to clear up, is the
relation of total \scrW-asphericity for an asphericity structure, for
variable \scrW, when $\scrW\subset\scrW'$. Assuming the localizers
satisfy the condition \ref{loc:4}, is it true that total
\scrW-asphericity is equivalent to total $\scrW'$-asphericity -- or
equivalently, is it equivalent to total $0$-connectedness?

\bigbreak

\presectionfill\ondate{5.7.}\pspage{320}\par

% 92
\hangsection[Short range working program, and an afterthought on
\dots]{Short range working program, and an afterthought on
  abelianization of homotopy types: a handful of questions around the
  Whitehead and Dold-Puppe theorems.}\label{sec:92}%
The review on ``pending questions and topics'' related to part
\ref{ch:IV} of the notes has taken pretty much longer than
expected. It was quite useful though, to get a clearer view of what
those questions are about, and to get a feeling for what to include
and develop, and where. As I do not intend to spend my life on the
task, not even one year, it is becoming clear that I am not going to
get the whole picture of all the questions touched -- and some
presumably I am going to leave just aside, as they do not seem
indispensable for a comprehensive overall picture of what I'm
after. This seems to me to be the case for the questions \ref{q:91.7}
and \ref{q:91.8}, concerned with finiteness conditions for homotopy
types in terms of models, and with test categories with boundary
operations. At the opposite side, it seems that the questions
\ref{q:88.2}, part of \ref{q:88.3}, and \ref{q:91.9}, about the notion
of contractor, induced asphericity and contractibility structures on a
category $M_{/a}$, and ``miscellaneous'' left-overs from part
\ref{ch:II}, should be dealt with in part \ref{ch:IV} -- whose end now
is in sight after all!  On the other hand, questions \ref{q:88.1},
another part of \ref{q:88.3}, \ref{q:88.5} and \ref{q:90.6}, about
morphisms of asphericity structures and related problems, about an
autodual treatment of asphericity and contractibility notions, about a
handy criterion for canonical modelizers, and about existence theorems
for various kinds of test functors or aspheric functors, while I feel
that I should come at least to a considerably clear understanding of
these matters than now, the adequate place for developing such
reflection is definitely \emph{not} in the present part \ref{ch:IV},
but belong to one or the other of the three parts still ahead in our
overall reflection on the modelizing story.

During our review, we came a number of times upon situations when the
question arose as to whether one point I like to make, namely that a
more or less arbitrary (strict) test category ``is just as good'' as
the sacrosanct test category $\Simplex$, or its twin brothers
$\Square$ and $\Globe$, is a valid one or not. I would like to list
here these situations, with a view of coming back to it later:
\begin{enumerate}[label=\alph*)]
\item\label{it:92.a}
  Existence theorems for test functors (cf.\ section \ref{sec:90}).
\item\label{it:92.b}
  \Ahat{} and various other categories constructed in terms of $A$,
  such as $\bHom(A\op,M)$, are closed model categories (under suitable
  assumptions\ldots).
\item\label{it:92.c}
  Independence of the derived category of $\bHom(A\op,M)$ on the
  choice of test category, notably when $M$ is a topos or the dual of
  a\pspage{321} topos (with suitable assumptions on $A,M$\ldots).
\item\label{it:92.d}
  Possibility of expressing finite type of a homotopy type in terms of
  \Ahatfp, for suitable test categories $A$.
\item\label{it:92.e}
  Possibility of defining boundary operations within a test category
  -- and/or getting Dold-Puppe type relations.
\end{enumerate}

\starsbreak

Before resuming more technical work with the matters left over for
part \ref{ch:IV}, I would like still to write down some afterthoughts,
concerning the question of boundary operations in a test category
(question \ref{q:91.8} in our review). It occurred to me that perhaps
it isn't such a good idea, to try at all costs to subordinate this
question to a question of cellular decompositions of spheres, however
natural this idea may be in view of the examples of the standard types
of complexes and multicomplexes. In this connection, I remember that
among my first thoughts when starting unwittingly on the modelizing
story, was that a ``test category'' $A$ (namely one such that \Ahat{}
should be ``modelizing'') should more or less correspond to such
decompositions. Soon after it came as a big surprise that so little
was needed in fact for $A$ to merit the name of a test category -- and
that the relevant conditions had nothing to do with cellular
decompositions of this or that. The same may well turn out, when
looking for a generalization of the standard simplicial, cubical or
hemispherical chain complexes, giving rise to the homology and
cohomology invariants of a given ``complex''. The kind of set-up I
proposed in yesterday's notes, for a formalism of boundary operations
in a test category $A$, now looks to me in some respects
somewhat\scrcomment{``étriqué'' can again be translated as
  ``narrow-minded''} ``étriqué'', and I'll try another start in a
different spirit.

In order not to get involved in irrelevant technicalities, I assume
that the basic localizer \scrW{} is $\scrWoo=$ usual weak
equivalence. It seems that one basic fact for writing down
a relationship between homotopy types and ``homology types'', is the
existence of a canonical ``abelianization functor''
\begin{equation}
  \label{eq:92.1}
  \Hot\to\D_\bullet(\Ab) \quad (\eqdef \HotabOf),\tag{1}
\end{equation}
where \Ab{} is the abelian category of abelian groups, and
$\D_\bullet$ designates the ``derived category'' of the category
$\Ch_\bullet(\Ab)$ of chain complexes of abelian groups, namely its
localization with respect to ``weak equivalences'', i.e.,
quasi-isomorphisms:\pspage{322}
\begin{equation}
  \label{eq:92.2}
  \D_\bullet\Ab = W^{-1}\Ch_\bullet\Ab,\tag{2}
\end{equation}
where $W$ means ``quasi-isomorphisms'', i.e., maps inducing
isomorphisms for all homology groups. The most common way for defining
the canonical functor \eqref{eq:92.1}, where as usual here \Hot{} is
defined as $\scrW^{-1}\Cat$, is via the test category $\Simplex$, as
the composition in the bottom row of
\begin{equation}
  \label{eq:92.3}
  \begin{tabular}{@{}c@{}}
    \begin{tikzcd}[baseline=(O.base)]
      \Cat \ar[r,"i^*"]\ar[d] &
      \Simplexhat \ar[r,"\Wh_{\Simplex}"]\ar[d] &
      \Simplexhatab \ar[r,"\equ\supDP"]\ar[d] &
      \Ch_\bullet\Ab \ar[d] \\
      \Hot \ar[r,"\equ"] &
      \HotOf_{\Simplex} \ar[r] &
      \HotabOf_{\Simplex} \ar[r,"\equ"] &
      |[alias=O]| \D_\bullet\Ab
    \end{tikzcd},
  \end{tabular}\tag{3}
\end{equation}
which is deduced from the top row by passing to the localized
categories. The subscript $\mathrm{ab}$ in \Simplexhatab{} denotes the
category of abelian group objects in \Simplexhat, the functor
\begin{equation}
  \label{eq:92.4}
  \Wh_{\Simplex} : \Simplexhat\to\Simplexhatab\tag{4}
\end{equation}
is the ``\emph{abelianization functor}'' obtained by composing a
presheaf $\Simplexop\to\Sets$ with the abelianization functor
\[\Sets\to\Ab, \quad X\mapsto\bZ^{(X)}.\]
We call this functor $\Wh_\Simplex$ also the ``Whitehead functor'', as
its main property is expressed in \emph{Whitehead's theorem}, namely
that it is compatible with weak equivalences (where weak equivalences
in \Simplexhatab{} are defined in terms of the underlying
semisimplicial sets, forgetting the addition laws). The localized
category of \Simplexhatab{} with respect to the latter notion of weak
equivalence is denoted by $\HotabOf_\Simplex$, the functor
\begin{equation}
  \label{eq:92.4prime}
  \HotOf_\Simplex \to \HotabOf_\Simplex\tag{4'}
\end{equation}
induced by $\Wh_\Simplex$ may be equally (and still more validly) be
designated by $\Wh_\Simplex$. The two subscripts $\Simplex$ (in $\Wh$
and in $\HotabOf$) refer to the fact that the notions make still a
sense when $\Simplex$ is replaced by an arbitrary small category $A$,
cf.\ below.

The functor $\DP$ in the top row is the well-known \emph{Dold-Puppe
  functor}, which is an equivalence of categories. As for $i^*$, it is
defined in terms of an arbitrary test functor
\[i:\Simplex\to\Cat,\]
which may be either the standard inclusion (which is the more commonly
used one) or the canonical functor $a\mapsto\Simplex_{/a}$, called
$i_\Simplex$. The functors $i^*$ corresponding to different choices of
$i$ are of course in general non-isomorphic, however (as follows from
section \ref{sec:77})\pspage{323} the corresponding functor
\[\overline{i^*}:\Hot\to\HotOf_\Simplex\]
is independent of such choice, up to canonical isomorphism.

Instead of $\Simplex$, we could have worked with $\Square$ or $\Globe$
instead, as these give rise to a Dold-Puppe functor (which is still an
equivalence), and (almost certainly, see below) to a corresponding
variant of the ``Whitehead theorem''. We thus get two other ways for
defining a canonical ``abelianization functor'' \eqref{eq:92.1} for
homotopy types, and it should be an easy and pleasant exercise to show
these three functors are canonically isomorphic, using the fact that a
product of test categories is again a test category.

\begin{remark}
  Of course, when concerned mainly with defining a functor
  \eqref{eq:92.1} we don't really need Whitehead and Dold-Puppe
  theorems -- indeed, instead of taking the functor
  \[\DP\circ\Wh_\Simplex:\Simplexhat\to\Ch_\bullet\Ab,\]
  we could have taken directly (using the standard boundary operations
  between the components of a semisimplicial abelian group) the
  standard chain complex structure of $\bZ^{(K_\bullet)}$ (for
  $K_\bullet$ in \Simplexhat), without taking the trouble and
  normalizing it à la Dold-Puppe -- and it is a lot more trivial than
  either Whitehead's or Dold-Puppe's theorem, that the latter functor
  transforms weak equivalences into quasi-isomorphisms; moreover, it
  gives rise to the same functor
  \begin{equation}
    \label{eq:92.5}
    \HotOf_\Simplex\to\D_\bullet\Ab\quad(=\HotabOf)\tag{5}
  \end{equation}
  as $\DP\circ\Wh_\Simplex$.
\end{remark}

Let now $A$ be \emph{any} small category, we are interested in the
functor
\begin{equation}
  \label{eq:92.6}
  \HotOf_A \to \D_\bullet\Ab\quad (=\HotabOf)\tag{6}
\end{equation}
obtained as the composition
\begin{equation}
  \label{eq:92.7}
  \HotOf_A\to\Hot \to\D_\bullet\Ab,\tag{7}
\end{equation}
where the second functor is the abelianization functor
\eqref{eq:92.1}, and the first is the canonical functor, deduced by
localization from
\begin{equation}
  \label{eq:92.8}
  i_A:\Ahat\to\Cat,\quad a\mapsto A_{/a}.\tag{8}
\end{equation}
We see immediately that for $A=\Simplex$, the functor \eqref{eq:92.6}
reduces to \eqref{eq:92.5} up to canonical isomorphism -- and the same
of course when\pspage{324} $A$ is either $\Square$ or $\Globe$. In
these three cases, the functor \eqref{eq:92.6} can be factorized in a
natural way through the category
\[\HotabOf_A = W^{-1}\Ahatab,\]
where now $W$ stands for the set of ``weak equivalences'' in \Ahatab,
defined in the same way as above in the case $A=\Simplex$. The
question then arises, for any given $A$, as to whether such a
factorization can be still obtained, and how exactly.

This formulation is inspired by the description of abelianization of
homotopy types via the (slightly sophisticated diagram)
\eqref{eq:92.3}. When following the more naive approach of the remark
above, this leads us to the closely related question of defining
\eqref{eq:92.6} via a composition
\begin{equation}
  \label{eq:92.9}
  \Ahat\xrightarrow{\Wh_A} \Ahatab \xrightarrow L \Ch_\bullet\Ab,\tag{9}
\end{equation}
(for a suitable functor $L$, cf.\ below), by passing to localizations.

Both approaches seem to me of interest. The first one, to make sense
at all as stated, relies on the existence of a canonical functor
\begin{equation}
  \label{eq:92.9prime}
  \HotOf_A\to\HotabOf_A,\tag{9'}
\end{equation}
induced by the abelianization functor
\[\Wh_A:\Ahat\to\Ahatab,\quad X\mapsto\bZ^{(X)},\]
i.e., on the \emph{validity of Whitehead's theorem, with $\Simplex$
  replaced by $A$}. This looks like an interesting question, whose
answer should be in the affirmative. At any rate, if we can find an
aspheric functor
\begin{equation}
  \label{eq:92.10}
  j:\Simplex\to\Ahat\tag{10}
\end{equation}
(with respect to the standard asphericity structure of $A$), then the
answer is affirmative, as we are immediately reduced to the known case
$A$ is replaced by $\Simplex$. Thus, the answer is possibly tied up
with the question of existence of test functors, which we'll deal with
presumably in part \ref{ch:V}. It should be noted though that if such
a functor \eqref{eq:92.10} exists, then necessarily $A$ is aspheric,
and even totally aspheric -- a substantial restriction indeed.

More generally, let
\begin{equation}
  \label{eq:92.10prime}
  j:B\to\Ahat\tag{10'}
\end{equation}
any aspheric functor with respect to the standard asphericity
structure of \Ahat, where $B$ is any small category. The corresponding
functor\pspage{325}
\begin{equation}
  \label{eq:92.11}
  j^*:\Ahat\to\Bhat\tag{11}
\end{equation}
then satisfies
\[(j^*)^{-1}(\scrWB)=\scrWA,\]
and the corresponding functor for the localizations gives rise to a
commutative diagram
\[\begin{tikzcd}[baseline=(O.base),column sep=tiny]
  \HotOf_A \ar[rr,"\overline{j^*}"]\ar[dr,"\overline{i_A}"'] & &
  \HotOf_B \ar[dl,"\overline{i_B}"] \\
  & |[alias=O]| \Hot &
\end{tikzcd},\]
and hence the corresponding diagram
\begin{equation}
  \label{eq:92.12}
  \begin{tikzcd}[column sep=tiny]
    \HotOf_A \ar[rr,"\overline{j^*}"]\ar[dr] & &
    \HotOf_B \ar[dl] \\
    & \D_\bullet\Ab=\HotabOf &
  \end{tikzcd}
  \tag{12}
\end{equation}
is commutative, where the vertical arrows are the canonical functors
\eqref{eq:92.6}. Coming back to the base $B=\Simplex$ say, this shows
that \eqref{eq:92.6} can be viewed as the composition
\[ \HotOf_A \xrightarrow{\overline{j^*}} \HotOf_\Simplex
\xrightarrow{\overline{\Wh_\Simplex}} \HotabOf_\Simplex
\xrightarrow[\equ]{\overline{\DP}} \HotabOf,\]
and hence it can be inserted in the commutative diagram
\begin{equation}
  \label{eq:92.13}
  \begin{tabular}{@{}c@{}}
    \begin{tikzcd}[baseline=(O.base)]
      \HotOf_A\ar[r]\ar[d] & \HotOf_\Simplex\ar[d] & \\
      \HotabOf_A\ar[r] & \HotabOf_\Simplex \ar[r,"\equ"] &
      |[alias=O]| \HotabOf
    \end{tikzcd},
  \end{tabular}\tag{13}
\end{equation}
where the functor
\[\HotabOf_A \to \HotabOf_\Simplex\]
is induced by $j^*\subab$. Thus, we get the wished for factorization
of \eqref{eq:92.6} via $\HotabOf_A$, \emph{provided we can find an
  aspheric functor} \eqref{eq:92.10}. It should not be hard moreover
to see that the factorizing functor obtained from \eqref{eq:92.13},
namely
\begin{equation}
  \label{eq:92.14}
  \HotabOf_A\to\HotabOf,\tag{14}
\end{equation}
does not depend up to canonical isomorphism on the choice of $j$, at
least in the case when $A$ is a contractor, using the end remarks of
section \ref{sec:82} (p.\ \ref{p:272}) concerning products of aspheric
functors.

The question remains whether we can define \eqref{eq:92.14} for (more
or less) any small category $A$, without having to rely upon the
existence of\pspage{326} an aspheric functor \eqref{eq:92.10}, in such
a way that it factors the canonical functor \eqref{eq:92.6} (granting
Whitehead's theorem holds for $A$), and moreover that for an aspheric
functor \eqref{eq:92.10prime} $j:B\to\Ahat$, giving rise to
\[\overline{j^*\subab}:\HotabOf_A\to\HotabOf_B,\]
then corresponding diagram
\begin{equation}
  \label{eq:92.15}
  \begin{tikzcd}[column sep=tiny]
    \HotabOf_A\ar[rr]\ar[dr] & & \HotabOf_B\ar[dl] \\ & \HotabOf &
  \end{tikzcd}\tag{15}
\end{equation}
should commute, where the vertical arrows are the functors
\eqref{eq:92.14}.

For defining \eqref{eq:92.14} in this general case, recalling that
\[\HotabOf \fromequ \HotabOf_\Simplex,\]
we can't help it and have to use diagram \eqref{eq:92.3} and the
functor
\[i_A:\Ahat\to\Cat,\]
or rather (\Cat{} serving only as an intermediary) the functor
\[u \eqdef i^*i_A : \Ahat\to\Simplexhat,\]
so that \eqref{eq:92.6} can be viewed as deduced by localization of
\Ahat{} from the composition
\begin{equation}
  \label{eq:92.star}
  \Ahat\xrightarrow
  u\Simplexhat\xrightarrow{\Wh_\Simplex}\Simplexhatab\to\HotabOf_\Simplex
  \quad (\toequ\HotabOf).\tag{*}
\end{equation}
One difficulty here is that $u$ does not commute to finite products,
and hence doesn't induce a functor
\[\Ahatab\to\Simplexhatab,\]
it would seem. Now this difficulty, I just noticed, can be overcome,
using the fact that $i_A$ and hence also $u$ \emph{commutes to fibered
  products}, or, what amounts to the same, induces an \emph{exact}
functor
\[u_0 : \Ahat\to\Simplexhat_{/F}\toequ(\Simplex_{/F})\uphat,\]
where
\[F=u(e_\Ahat)=i^*(A)\]
is a suitable object in \Simplexhat. A fortiori, $u_0$ commutes to
finite products, hence transforms abelian group objects into same,
i.e., induces
\[u_{0\mathrm{ab}} : \Ahatab\to(\Simplex_{/F})\uphat\subab,\]
on\pspage{327} the other hand, we do have too a natural functor
\begin{equation}
  \label{eq:92.16}
  \alpha^{\mathrm{ab}}_!:(\Simplex_{/F})\uphat\subab\to\Simplexhatab,\tag{16}
\end{equation}
defined as the left adjoint of the evident functor
\[\alpha^*\subab:\Simplexhatab\to(\Simplex_{/F})\uphat\subab\]
induced by the left exact functor $\alpha^*$, where $\alpha$ is the
``localization morphism of topoi''
\[\alpha:\Simplex_{/F}\to \Simplex_F\]
defined by the object $F$ of \Simplexhat. We thus get a diagram
\begin{equation}
  \label{eq:92.17}
  \begin{tabular}{@{}c@{}}
    \begin{tikzcd}[baseline=(O.base)]
      \Ahat\ar[r,"\Wh_A"] \ar[d,"u_0"'] \ar[dd,bend right=50,"u"'] &
      \Ahatab\ar[d,"u_{0\mathrm{ab}}"]  & & \\
      (\Simplex_{/F})\uphat \ar[r,"\Wh_{\Simplex_{/F}}"]
      \ar[d,"\alpha_!"'] &
      (\Simplex_{/F})\uphat\subab \ar[d,"\alpha_!^{\mathrm{ab}}"] & & \\
      \Simplexhat\ar[r,"\Wh_\Simplex"] & \Simplexhatab \ar[r] &
      \HotabOf_\Simplex \ar[r] & |[alias=O]| \HotabOf
    \end{tikzcd},
  \end{tabular}
  \tag{17}
\end{equation}
containing \eqref{eq:92.star} above as the composition of maps in the
left-hand vertical column and in the bottom row. The lower square in
this diagram commutes (up to natural isomorphism) -- which is a
general fact surely for morphisms of topoi $f:X\to Y$ such that $f_!$
exists, which allows to define too a functor $f_!^{\mathrm{ab}}$ as
\eqref{eq:92.16} above. The upper square though doesn't look at all
commutative, too bad! The only hope left now is that the natural
compatibility arrow for this square (there is bound to be one, isn't
there!), when composed with the lower square (so as to give a
compatibility map for the composite rectangle) \emph{should give rise
  to a weak equivalence in} \Simplexhatab, for any choice of an object
$X$ in \Ahat.

It would seem to me that a reasonable functor \eqref{eq:92.14} will
exist, without an existence assumption of a test functor
\eqref{eq:92.10}, exactly in those cases when the composite rectangle
in \eqref{eq:92.17} is ``commutative up to weak equivalence''. I have
no idea whether or not this is true for any small category $A$, not
even (I confess) when there \emph{is} an aspheric functor
\eqref{eq:92.10} -- as a matter of fact, I don't feel like going any
further now in this direction, and trying to check anything
whatsoever.

I was a little rash in the definite statement I made about the
``exact'' assumption to make for a ``reasonable'' functor
\eqref{eq:92.14} to exist; another seems needed still, namely that the
functor\pspage{328}
\begin{equation}
  \label{eq:92.18}
  \Ahatab\to\Simplexhatab\tag{18}
\end{equation}
we obtain by composing the two arrows in the right hand vertical
column, transforms weak equivalences into same, which is needed in
order to deduce \eqref{eq:92.14} by passing to the localized
categories. When the two assumptions are satisfied, then the functor
\eqref{eq:92.14} obtained from \eqref{eq:92.18} does factorize
\eqref{eq:92.6} as required, and it should be clear too that it
satisfies the compatibility \eqref{eq:92.15}.

Thus, it seems there are good prospects for getting canonical functors
\[\HotOf_A \to \HotabOf_A \to \HotabOf,\]
whose composition is \eqref{eq:92.6}, i.e., inserting into the
commutative diagram
\begin{equation}
  \label{eq:92.19}
  \begin{tabular}{@{}c@{}}
    \begin{tikzcd}[baseline=(O.base)]
      \HotOf_A\ar[r]\ar[d] & \HotabOf_A\ar[d] \\
      \HotOf \ar[r] & |[alias=O]| \HotabOf
    \end{tikzcd}.
  \end{tabular}
  \tag{19}
\end{equation}
The next question then which arises is \emph{whether the second
  vertical arrow} (namely \eqref{eq:92.14}) \emph{is an equivalence,
  whenever the first one is}, i.e., when $A$ is a pseudo-test
category, or whether this is true if we make some familiar extra
assumption on $A$, such as being an actual test category say.

I feel I am getting gradually back into thin air conjecturing, I
wouldn't go on too long this way! This whole $\HotabOf_A$ business was
just a digression, which then took me longer than expected, it doesn't
seem to have much to do with what I have been out for in this section,
namely afterthoughts about \emph{boundary operations in a test
  category}, which are designed to gave a ``computational''
description of the canonical functor \eqref{eq:92.6}, the latter being
defined without any restriction nor difficulty for \emph{any} small
category $A$. More accurately still, we want to describe the
composition of \eqref{eq:92.6} with the canonical functor
$\Ahat\to\HotOf_A$, namely $\Ahat\to\HotabOf$, via a suitable functor
$\Ahat\to\Ch_\bullet\Ab$, with the expectation that the latter should
factor through \Ahatab{} via the abelianization functor $\Wh_A$. In
other words, we are looking for commutative diagrams
\begin{equation}
  \label{eq:92.20}
  \begin{tabular}{@{}c@{}}
    \begin{tikzcd}[baseline=(O.base),sep=tiny]
      & \Ahat\ar[dl]\ar[rr]\ar[dd,"K"] & & \HotOf\ar[dd] \\
      \Ahatab\ar[dr,"L"'] & & & \\
      & \Ch_\bullet\Ab\ar[rr] & & |[alias=O]| \HotabOf
    \end{tikzcd},
  \end{tabular}
  \tag{20}
\end{equation}
where\pspage{329} all functors in the diagram, except for $K$ and $L$,
are the canonical ones familiar to us. The question then is how to
define a suitable $L$, such that the corresponding square (where
$K=L\circ\Wh_A$) should commute up to (canonical?) isomorphism.

There \emph{is} such an $L$, whenever we have an aspheric functor
\eqref{eq:92.10} (where $\Simplex$ may be replaced by one of its
twins), using corresponding semisimplicial chain complexes -- but, as
already remarked yesterday, taking things this way is ``cheating''!
Conceivably too, there are quite general theorems asserting that a
functor from a category \Ahat{} say to a derived additive category
such as $\HotabOf$ can be lifted to the category of models (here
abelian chain complexes) it comes from, and possibly even in a way
factoring through \Ahatab? I don't intend to dive into these questions
either, but rather, make a comment on a general method for
constructing certain functors
\begin{equation}
  \label{eq:92.21}
  L:\Ahatab\to\Ch_\bullet\Ab,\tag{21}
\end{equation}
(maybe not in a way to give rise to a commutative diagram
\eqref{eq:92.20}), as suggested by the standard chain complexes
associated to the three types of complexes, using simplices, cubes or
hemispheres, or multicomplexes (using products of the standard test
categories). Writing
\[\Ahatab\simeq \bHom(A\op,\Ab),\]
we remark that the standard constructions of chain complexes
associated to semisimplicial (say) complexes of abelian groups, makes
sense not only for such abelian complexes, but more generally for
\emph{complexes with values in \emph{any} additive category}, $M$
say. This induces us to look more generally, for any such $M$, for a
functor
\begin{equation}
  \label{eq:92.22}
  L_M:\bHom(A\op,M)\to\Ch_\bullet(M),\tag{22}
\end{equation}
in a way \emph{compatible with additive functors}
\[M\to M'\]
(in the obvious sense of the word).

Now, for any category $B$ (here $A\op$) one can define an
``\emph{enveloping additive category}'' $\Add(B)$, together with a
canonical functor
\begin{equation}
  \label{eq:92.23}
  B\to \Add(B),\tag{23}
\end{equation}
which is ``$2$-universal'' for all possible functors of $B$ into any
additive category $M$. More specifically, for any such $M$, the
corresponding\pspage{330} functor ``composition with
\eqref{eq:92.23}'' is an \emph{equivalence}
\begin{equation}
  \label{eq:92.24}
  \mathbf{Homadd}(\Add(B),M) \toequ \bHom(B,M).\tag{24}
\end{equation}
This condition defines \eqref{eq:92.22} ``up to canonical equivalence''
-- but we'll give an explicit description in a minute. Before doing
so, let's just remark that the universal property of \eqref{eq:92.23}
implies that to give a system of functors $L_M$ as above, ``amounts to
the same'' as giving a chain complex $L_\bullet$ in $\Add(B)$. More
accurately, the category of all systems $L_M$ (where maps are defined
in an evident way) is \emph{equivalent} to the category
$\Ch_\bullet(\Add(B))$, where $B=A\op$. The functors $L$
\eqref{eq:92.21} we are specifically interested in, are those which
are associated to some chain complex in $\Add(A\op)$,
\begin{equation}
  \label{eq:92.25}
  L_\bullet\in\Ob(\Ch_\bullet(\Add(A\op))),\tag{25}
\end{equation}
by the formula
\begin{equation}
  \label{eq:92.26}
  L(X) = \widetilde X(L_\bullet)\quad\text{for any $X$ in
    $\Ahatab\simeq\bHom(A\op,\mathrm{Ab})$}\tag{26}
\end{equation}
where
\[\widetilde X: \Add(A\op)\to\Ab\]
is the \emph{additive} functor corresponding to $X$.

We are thus led to the question: if $A$ is any small category, does
there exist a chain complex $L_\bullet$ in $\Add(A\op)$, the additive
envelope of $A\op$, giving rise to a functor \eqref{eq:92.21} via
\eqref{eq:92.26} and hence to a diagram \eqref{eq:92.20}, such that
the square in \eqref{eq:92.20} commutes up to isomorphism? And when
this is so, what kind of unicity statement, if any, can be made for
$L_\bullet$ (such as being unique up to chain homotopy say), and what
about the structure of the category of all pairs
$(L_\bullet,\lambda)$, where $\lambda$ is a compatibility isomorphism
making the square in \eqref{eq:92.20} commute?

We are far here from the rather narrow set-up in yesterday's notes,
and as far as existence goes, if no extra conditions are put upon
$L_\bullet$, it seems likely that for a rather large class of small
categories $A$ (if not all) it should hold true. At any rate, the
functor $L$ obtained from a functor \eqref{eq:92.10}, i.e., ``by
cheating'', is visibly associated to an $L_\bullet$. Sorry, we have to
assume that the functor $j$ factors even through $A$ itself, i.e., is
just an aspherical functor between the small categories $\Simplex$ and
$A$, a much more stringent condition on $A$ to be sure -- and which
implies that there is an induced\pspage{331} functor
\begin{equation}
  \label{eq:92.27}
  \Add(j\op):\Add(\Simplexop)\to\Add(A\op),\tag{27}
\end{equation}
hence we get an $L_\bullet$ as the image of the canonical chain
complex $L_\bullet^\Simplex$ we got in $\Add(\Simplexop)$.

\bigbreak
\presectionfill\ondate{6.7.}\par

% 93
\hangsection[The afterthought continued: abelianizators, and
\dots]{The afterthought continued: abelianizators, and ``standard''
  abelianizators for categories with boundary
  operators.}\label{sec:93}%
It is time to give the promised construction of $\Add(B)$, the
additive envelope of $B$, for any given category $B$. The obvious idea
is to enlarge the sets $\Hom(a,b)$, for $a$ and $b$ in $B$, by taking
linear combinations with coefficients in \bZ, i.e., writing
\begin{equation}
  \label{eq:93.1}
  \Hom_{\Add(B)}(a,b) = \bZ^{(\Hom(a,b))},\tag{1}
\end{equation}
and composing these $\Hom_{\Add}$ in the obvious way. This is not
quite enough though, as we still have to add new objects, namely
direct sums of objects in $B$. The most convenient way for doing so
seems by defining an object of $\Add(B)$ to be defined by a finite set
$I$ (in the given universe), namely the indexing set for taking the
direct sum, and a map
\[ I\to\Ob B,\]
in other words, the new objects are just \emph{families} of objects of
$B$
\[(b_i)_{i\in I},\]
indexed by finite sets. We'll however denote by
\begin{equation}
  \label{eq:93.2}
  \bigoplus_{i\in I}b_i\tag{2}
\end{equation}
the corresponding object of $\Add(B)$, as this will turn out to be the
direct sum indeed of the images of the $b_i$'s in $\Add(B)$ -- but of
course we'll ignore the possible existence of direct sums in $B$
itself, when they exist, and not confuse \eqref{eq:93.2} with a direct
sum taken in $B$. Writing $\Homadd$ instead of $\Hom_{\Add(B)}$ for
the sake of abbreviation, the maps between objects \eqref{eq:93.2} are
defined by matrices in the obvious way
\begin{multline}
  \label{eq:93.3}
  \Homadd((a_i)_{i\in I},(b_j)_{j\in J}) = \\
  \set[\big]{(u_{ij})_{(i,j)\in I\times
      J}}{u_{ij}\in\Homadd(a_i,b_j)=\bZ^{\Hom(a_i,b_j)}},\tag{3}
\end{multline}
while composition of maps is defined by the composition of
matrices. We thus get a new category $\Add(B)$ and a functor
\begin{equation}
  \label{eq:93.4}
  B\to\Add(B),\tag{4}
\end{equation}
it\pspage{332} is immediately checked that $\Add(B)$ is an additive
category and that the functor \eqref{eq:93.4} has the $2$-universal
property for functors from $B$ into any additive category, stated in
yesterday's notes (p.\ \ref{p:330}).

\begin{remark}
  The same construction essentially applies when considering the
  universal problem of mapping $B$ into any $k$-\emph{additive}
  category $M$ (where $k$ is any commutative ring with unit), i.e.,
  an additive category $M$ endowed with a ring homomorphism
  \[k\to\End(\id_M),\]
  replacing $\bZ$ by $k$ in formulas \eqref{eq:93.1} and
  \eqref{eq:93.3}. The ``abelianization'' questions touched at in
  yesterday's notes still make sense in terms of ``$k$-linearization''
  -- a notion much in the spirit of our introduction of a general
  basic localizer \scrW, as the very notion of $k$-linearization will
  give rise to a corresponding basic localizer $\scrW_k$\ldots
\end{remark}

Let's come back to the case when $B=A\op$, $A$ being a small category,
and to our question about chain complexes
\[L_\bullet \quad\text{in}\quad \Ch_\bullet(\Add(A\op))\]
giving rise to a commutative diagram \eqref{eq:92.20} (p.\
\ref{p:328}), up to isomorphism. A pair
\[(L_\bullet,\lambda),\]
where $L_\bullet$ is a chain complex as above, and $\lambda$ a
compatibility isomorphism for the square in \eqref{eq:92.20}, could be
suggestively called an \emph{abelianizator} for the small category
$A$. The question of existence, and uniqueness up to homotopy say, of
an abelianizator for $A$ seems especially relevant when $A$ is a test
category say, and hence \Ahat{} modelizes homotopy types. In any case,
in terms of an abelianizator we get an additive functor
\begin{equation}
  \label{eq:93.5}
  L:\Ahatab\to\Ch_\bullet\Ab,\tag{5}
\end{equation}
and the question arises whether this is compatible with weak
equivalences and quasi-isomorphisms; maybe even if this is not
automatic, we should insist it holds when defining the notion of an
abelianizator. When this is OK, then by passing to localizations we
deduce from \eqref{eq:93.5} a functor
\begin{equation}
  \label{eq:93.6}
  \HotabOf_A\to\HotabOf,\tag{6}
\end{equation}
i.e., a functor \eqref{eq:92.14} as looked for in yesterday's notes,
giving rise to the commutative diagram \eqref{eq:92.19} (p.\
\ref{p:328}) -- whereas commutativity of diagrams of the type
\eqref{eq:92.15} (p.\ \ref{p:326}) looks less obvious.

When\pspage{333} $A$ is a finite product of copies taken from among
the three standard test categories $\Simplex$, $\Square$, $\Globe$,
the standard chain complex structure on multicomplexes does furnish us
with a \emph{``canonical'' abelianizator} for $A$, which we may denote
by $L_\bullet^A$ (as we did yesterday for $A=\Simplex$). This
``standard'' abelianizator has some very remarkable extra features
which I would like to pin down, which had caused our rather narrow
focus in the notes of two days ago (section \ref{sec:91}).

\namedlabel{cond:93.a}{a)}\enspace
There is a ``dimension map''
\begin{equation}
  \label{eq:93.7}
  \dim:\Ob A\to \bN.\tag{7}
\end{equation}
It can be described (in the particular case above at any rate) in
terms of the intrinsic category structure of $A$, by associating to
every $a$ in $A$ the ordered set
\begin{equation}
  \label{eq:93.8}
  i(a)=\text{set of subobjects of $a$ in $A$.}\tag{8}
\end{equation}
(NB\enspace not to be confused with subobjects of $a$ in \Ahat, namely
sieves in $a$). This is an ordered set with a largest object (namely
$a$ itself), and which turns out to be finite (in the particular cases
considered), hence of finite combinatorial dimension (equal to the
dimension of the geometrical realization $\abs{i(a)}$), and we have
\begin{equation}
  \label{eq:93.9}
  \dim(a)=\dim i(a).\tag{9}
\end{equation}

\namedlabel{cond:93.b}{b)}\enspace
The $n$'th component $L_n$ of $L_\bullet^A=L_\bullet$ is given by
\begin{equation}
  \label{eq:93.10}
  L_n = \bigoplus_{\dim(a)=n} a\tag{10}
\end{equation}
(where the direct sum of course is taken in $\Add(A\op)$ as in
\eqref{eq:93.2} above), which makes sense when we assume (as is the
case in our example) that the map \eqref{eq:93.7} is ``finite'', i.e.,
has finite fibers. For instance, in all our ``standard'' cases, there
is just \emph{one} object of $A$ which is of dimension $0$, and this
is also the final object.

\namedlabel{cond:93.c}{c)}\enspace
The differential operator
\begin{equation}
  \label{eq:93.11}
  d_n:L_n\to L_{n-1}\tag{11}
\end{equation}
can be obtained in the following way. We have only to describe $d_n$
on each summand $a$ of $L_n$, i.e., by \eqref{eq:93.10} on each $a$ in
\begin{equation}
  \label{eq:93.12}
  A_n=\set[\big]{a\in\Ob A}{\dim a=n}.\tag{12}
\end{equation}
In view of \eqref{eq:93.3}, this restriction $d_n\mid a$ can be
described as a linear combination of elements in the disjoint sum of
the sets
\[\Hom(b,a),\quad\text{with}\quad b\in A_{n-1}\quad (\text{$a\in A_n$
  fixed}).\]
This\pspage{334} being clear, the non-zero coefficients which occur in
this linear combination are all $\pm1$, and moreover the maps which
target $a$
\[b\to a\quad(b\in A_{n-1})\]
which occur with non-zero coefficient are exactly \emph{all
  monomorphisms} from objects $b$ in $A_{n-1}$ into $a$. Thus, the
differential operators are known, when we know, for all monomorphisms
in $A$
\begin{equation}
  \label{eq:93.13}
  \partial:b\to a,\quad\text{with}\quad \dim a=\dim b+1\tag{13}
\end{equation}
(the so-called \emph{``boundary maps''}), the corresponding
coefficients
\begin{equation}
  \label{eq:93.14}
  \varepsilon_{\partial}\in\{\pm1\}.\tag{14}
\end{equation}
Instinct tells us, at this point, that we may get into trouble, when
trying to define (in a more or less general case) boundary operations
in such a way, because of the ambiguity in the definition of
subobjects, namely, because of possible \emph{existence of
  isomorphisms which may not be identities}. But precisely, in the
standard cases we are copying from, \emph{any isomorphism is an
  identity}!

\namedlabel{cond:93.d}{d)}\enspace
For describing the ``signatures'' \eqref{eq:93.14}, in one of the
``standard'' cases, we still need to remark that for any $a$ in $A$,
we have
\begin{equation}
  \label{eq:93.15}
  \text{$i(a)$ is an $n$-cell, with $n=\dim(a)$,}\tag{15}
\end{equation}
and the choice of the signatures will be determined by a choice of
orientations
\begin{equation}
  \label{eq:93.16}
  \text{$\omega_a$ an \emph{orientation} of the $n$-cell
    $\abs{i(a)}$,}\tag{16}
\end{equation}
(a notion which could be given a purely combinatorial definition, by
induction on the dimension of a given ordered set whose geometrical
realization is a variety\ldots). We then get a the \emph{``Stokes
  rule''}
\begin{flalign}\label{eq:93.17}
  &&&\parbox[t]{0.9\textwidth}{For a boundary map $\partial:b\to a$,
  $\varepsilon_{\partial}=+1$ if{f} $\omega_b$ is ``induced'' à la
  Stokes by $\omega_a$, via the induced orientation on the boundary of
  $i(a)$ (which is the union of the images of all $i(b)$'s, for all
  boundary operations with target $a$).}
  \tag{17}
\end{flalign}

Whether or not we are in a ``standard'' case, if $A$ is any category
such that for any object $a$ of $A$, the ordered set $i(a)$ of its
subobjects in $A$ is finite, and its geometrical realization is an
$n$-cell (call $n$ the ``dimension'' of $a$), and if moreover for a
given $n$, the set $A_n$ of objects with dimension $n$ is finite, and
also (to be on the safe side!) assuming that all isomorphisms are
identities, then for \emph{any} choice of orientations
\eqref{eq:93.16}, giving rise to a system of\pspage{335} signatures
\eqref{eq:93.13} by the ``Stokes rule'' \eqref{eq:93.17}, the
corresponding operators \eqref{eq:93.11} do turn the family $(L_n)$
into a chain complex, namely we have the relations
\begin{equation}
  \label{eq:93.18}
  d_{n-1}d_n=0.\tag{18}
\end{equation}
This follows immediately from the well-known anti-commutativity
property of (twofold) induction of orientation on boundaries.

Things are a little more delicate if we don't assume that isomorphisms
are identities, even if (by compensation) we should insist that two
distinct objects are never isomorphic. To define $d_n$, we then must
\emph{choose}, for any subobject $b$ of dimension $n-1$ of an object
$a$ of dimension $n$, just \emph{one} representative monomorphism
\eqref{eq:93.13} of $b$. The coherence condition then needed in order
to get \eqref{eq:93.18} is that any square diagram
\begin{equation}
  \label{eq:93.19}
  \begin{tikzcd}[sep=tiny]
    & c\ar[dl]\ar[dr] & \\
    b\ar[dr] && b'\ar[dl] \\
    & a &
  \end{tikzcd}
  \tag{19}
\end{equation}
made up with such restricted boundary maps, should commute -- a
somewhat delicate condition, presumably hard to ensure, for the
choices involved for defining the ``strict'' boundary maps in $A$.

In one case as in the other, we are very close of course to the set-up
envisioned in section \ref{sec:91} -- it wouldn't be hard even to fit
the case considered here into this set-up, if we make the slight extra
assumption that any map in $A$ factors into an
epimorphism-with-section, followed by a monomorphism (which is true
indeed in the ``standard'' cases), which will ensure that for varying
$a$, $i(a)$ is indeed a functor with values in \Ord, as stated in
loc.\ cit. But from the point of view of construction of
abelianizators, it would seem that the existence of the functor
\begin{equation}
  \label{eq:93.20}
  i:A\to\Ord\tag{20}
\end{equation}
is irrelevant.

Our main question now, of course, is about \emph{the chain complex
  $L_\bullet$ being an abelianizator or not}. The question is
interesting even in the standard cases, by choosing the orientations
\eqref{eq:93.16} in a way different from the standard one. Are the
corresponding chain complexes in $\Add(A\op)$ necessarily
chain-homotopic?

It just occurs to me that indeed, between the chain
complexes\pspage{336}
\begin{equation}
  \label{eq:93.22}
  L_\bullet^\omega = ((L_n)_{n\in\bN},(d_n^\omega)_{n\in\bN})\tag{22}
\end{equation}
associated to all possible systems of orientations
\begin{equation}
  \label{eq:93.23}
  \omega=(\omega_a)_{a\in\Ob(A)}\tag{23}
\end{equation}
of the various cells $i(a)$, there is a canonical \emph{transitive
  system} of isomorphisms, by defining the isomorphism
\begin{equation}
  \label{eq:93.24}
  u_{\omega,\omega'}:L_\bullet \tosim L_\bullet'\tag{24}
\end{equation}
for two different choices $\omega,\omega'$ of systems of orientations,
by
\begin{equation}
  \label{eq:93.25}
  u_{\omega,\omega'} \mid a=\varepsilon_a^{\omega,\omega'}\,
  \id_a,\quad \varepsilon_a^{\omega,\omega'}\in\{\pm1\},\tag{25}
\end{equation}
where the sign $\varepsilon_a^{\omega,\omega'}$ is equal to $+1$ or
$-!$, \emph{according to whether $\omega_a$ and $\omega_a'$ are equal
  or not}. It is immediate that \eqref{eq:93.24} then is an
isomorphism componentwise, respecting degrees, and commuting to the
respective differential operators. Transitivity of the isomorphisms
\eqref{eq:93.24} for a triple $(\omega,\omega',\omega'')$ is equally
immediate. This implies that by this transitive system of
isomorphisms, \emph{we may identify all the chain complexes}
$L_\bullet^\omega$ in $\Add(A\op)$ to a single chain complex,
canonically isomorphic to each $L_\bullet^\omega$, and which we may
just designate by $L_\bullet$. This chain complex now is defined
intrinsically in terms of the category structure of $A$ (up to
canonical isomorphism), in the ``safe'' case at any rate when every
isomorphism in $A$ is an identity, so that in the construction of
$L_\bullet^\omega$ there enters no other choice besides
$\omega$. Otherwise as seen above (precedent page), we must still
suitably choose the so-called ``strict'' boundary operators
\eqref{eq:93.13}, among all monomorphisms $b\hookrightarrow a$ in $A$
such that $\dim a=\dim b+1$.

In the first case say (isomorphisms being identities), all conditions
considered for $A$ are stable under finite products, that's why in
terms of the three standard cases of $\Simplex$, $\Square$ and
$\Globe$, we could construct others by taking finite products. The
three standard test categories may be viewed as particularly
``economic'' of skillful ways of ``cutting out'' a suitable bunch of
cellular decompositions, and of eliminating automorphisms (by total
ordering of vertices and the like\ldots), so as to ensure: a)\enspace
that isomorphisms in $A$ are identities, b)\enspace the canonical
chain complex $L_\bullet$ in $\Add(A\op)$ is an abelianizator, and
c)\enspace $A$ moreover is a strict test category, and even a
contractor. On the other hand, as all these conditions (plus the
condition \eqref{eq:93.15} of course about the $i(a)$'s representing
$n$-cells)\pspage{337} are stable under taking products (of finite
non-empty families of categories $A_i$), hence in terms of the three
standard cases, the possibility of satisfying them too by the
``multistandard'' test categories. I wonder if there are any other
ways (up to equivalence). If we take categories such as
$\widetilde\Simplex$ (non-ordered simplices), we still get
contractors, but objects have non-trivial automorphisms, and if we
take categories such as \Simplexf{} (ordered simplices without
degeneracy operations, only boundary maps), it is true that
isomorphisms are identities, but the category is no longer a test
category but only a weak one.

If we do not insist on the rigidity assumption (isomorphisms are
identities), but on suitable choice of so-called ``\emph{strict}
boundary operations'' within $A$, then it would seem after all that we
do have a lot more elbow freedom than it seemed by the end of our
reflections on that matter two days ago (cf.\ p.\ \ref{p:318}), where
the picture of the relevant data and corresponding construction of
chain complexes was still a little confused. Let now $A$ be the
category called $A_0$ in loc.\ cit.  We don't have to modify it in
order to introduce orientations of cells $i(a)$ as extra structure and
take account of this in defining a new notion of maps. Therefore, it
is clear that $A$ \emph{just as it is, is a strict test category}
(presumably \emph{not} a contractor though). There is problem of
course of isomorphisms which are not identities, and particularly of
non-trivial automorphisms -- for instance the object $I$ (playing the
part of the unit segment) has a non-trivial automorphism, the
elimination of which does not look so trivial! However, there
\emph{is} a rather evident way of cutting out \emph{strict} boundary
maps, in a way as to satisfy the transitivity condition of p.\
\ref{p:335} -- namely by taking boundary maps \eqref{eq:93.13}
$\partial:b\to a$ which are \emph{inclusions} in the strict sense,
namely the inclusion map of a \emph{subset} of $a$, endowed with the
induced order relation.

Thus, there are many other cases still than just multi-standard test
categories for getting a canonical chain complex $L_\bullet$ in
$\Add(A\op)$, and for which now the question makes sense as to whether
$L_\bullet$ is an abelianizator. In the construction above, we were
careful to assume that the full subcategory $A$ of \Ord, besides
containing $I$, was stable under finite products, so as to make sure
it comes out as a test category. The silly thing is that this
condition is not satisfied by any one among the standard test
categories -- thus, it seems reasonable to try and\pspage{338} replace
it by a suitable substitute, such as the existence, of any two objects
$a$ and $b$ in $A$, of a cellular subdivision of
$\abs{i(a)}\times\abs{i(b)}$, made up with cells of the type
$\abs{i(c)}$, and inducing on the latter the given cellular structure
of $\abs{i(c)}$. The problem is now (besides getting or not an
abelianizator $L_\bullet$) \emph{whether $A$ is at any rate a weak
  test category} (in view of the example \Simplexf, we can't expect
now of course to get an actual test category). Maybe I'll come back to
this later, when writing down a proof for \Simplexf{} being a weak
test category, i.e., a more general result along these lines should
come out alongside.

\begin{remarks}
  I feel the canonical chain complex $L_\bullet$ in $\Add(A\op)$
  constructed in this section, under suitable assumptions on the small
  category $A$, merits a name of its own. We may call it the
  \emph{standard abelianizator} of $A$ -- but this is reasonable only
  if it turns out that in all cases when it can be constructed, it is
  an abelianizator indeed. Another convenient name may be the
  \emph{Dold-Puppe chain complex}, as in the three standard cases, the
  standard Dold-Puppe construction of the ``normalization'' of an
  abelian complex (ss say) can be viewed as being performed in the
  ``universal'' case, namely for $A\op\to\Add(A\op)$, and the
  corresponding ``full'' chain complex, namely $L_\bullet$ -- with
  this grain of salt though that we still have to enlarge $\Add(A\op)$
  slightly, so as to make stable under taking direct summands
  corresponding to projectors. But then it occurs to me that the name
  of Dold-Puppe chain complex is much more suitable for the
  \emph{result of normalization} applied to $L_\bullet$, which (if I
  got it right) is the ``new'' complex discovered by Dold-Puppe,
  together with the inverse construction, whereas $L_\bullet$ had
  already been known for ages (even if not under its universal
  disguise\ldots).
\end{remarks}

% 94
\hangsection[Afterthought (continued): retrospective on the ``De Rham
\dots]{Afterthought
  \texorpdfstring{\textup(continued\textup)}{(continued)}:
  retrospective on the ``De Rham complex with divided powers'' and on
  some wishful thinking about linearization of homotopy types and
  arbitrary ground-ring extension in homotopy types.}\label{sec:94}%
In the last section, as in the two preceding days, our emphasis with
abelianization of homotopy types has been to look at it in terms of
more or less arbitrary test categories and the corresponding
elementary modelizers, and even in terms of arbitrary small
categories. This has causes as spinning a kind of dream for a while,
with the Whitehead and Dold-Puppe theorems and generalized boundary
maps as our main thread. Now this reminds me of a rather different
line of thoughts tied up with abelianization, quite independently of
playing around with variable modelizers -- a question which has been
intriguing me for a very long time now, ever since I got acquainted a
little with the very notion of homotopy types, and the corresponding
homology and\pspage{339} cohomology invariants. This is the question
of \emph{how far a homotopy type can be expressed in terms of homology
  or cohomology invariants} (or both together), plus \emph{some
  relevant extra structure}, the most important surely being
cup-products in cohomology (or, dually, ``interior'' operation of
cohomology on homology).\scrcomment{I wonder whether AG knew of the
  Steenrod algebra\ldots yes, see below} Once the notion of derived
categories of various kinds had become familiar, in the early sixties,
the question would appear as expressing, or recovering, a homotopy
type, namely an object in the (highly non-abelian) ``derived
category'' \Hot, in terms of its abelianization in
$\HotabOf=\D_\bullet\Ab$, \emph{endowed with suitable extra
  structure}. It was about clear that this extra structure had to
include, as its main non-commutative item, the fundamental group
$\pi$, so as to allow for description of homology and cohomology
invariants with twisted coefficients. The most natural candidate for
expressing this would be the chain complex associated to the universal
covering, viewed as an object in the derived category
\begin{equation}
  \label{eq:94.26}
  \D_\bullet(\bZ(\pi))\tag{26}
\end{equation}
of chain complexes of modules over the group ring $\bZ(\pi)$. Another
important structural item, giving rise to all cup-products with
non-twisted coefficients, is the diagonal map for the abelianization
\[L_\bullet\quad\text{in}\quad \HotabOf=\D_\bullet\Ab,\]
namely a map
\begin{equation}
  \label{eq:94.27}
  L_\bullet \to L_\bullet \Lotimes L_\bullet,\tag{27}
\end{equation}
where $\Lotimes$\scrcomment{aka the $\Tor$ functor} is the ``total''
left derived functor of tensor product. This map is subjected to
suitable conditions, concerning mainly commutativity and
associativity. In case of a non-$1$-connected space, i.e., $\pi\ne1$,
it shouldn't be hard combining the two structural items so as to get a
structure embodying at any rate cup-products with arbitrary twisted
coefficients. One key question in my mind, which I never really looked
into, was whether these two structures were enough in order to
reconstruct entirely (up to canonical isomorphism) the (pointed,
$0$-connected) homotopy type giving rise to it, and hence also any
other homotopy invariants, such as ``operations'' on cohomology and
the like, K-invariants, etc.

If I got it right, it has been known now for quite a while that even
for a $1$-connected homotopy type, so that the relevant structure
reduces to \eqref{eq:94.27}, that this is \emph{not} quite enough for
recovering the homotopy type, maybe not even the rational homotopy
type. I believe I first got this from Sullivan, namely
that\pspage{340} what was needed for recovering a $1$-connected
rational homotopy type was not merely \eqref{eq:94.27} (where now
$L_\bullet$ is an object of $\D_\bullet(\bQ)$ rather than of
$\D_\bullet(\bZ)=\D_\bullet\Ab$), which reduces more or less (under
suitable finiteness assumption) to knowing the rational cohomology
ring, but an anti-commutative and associative \emph{differential graded
algebra} over $\bQ$ (giving rise to \eqref{eq:94.27} by
duality). Thus, $1$-connected rational homotopy types are expressible
as objects of the derived category defined in terms of such algebras,
and the obvious notion of quasi-isomorphism for these. To any space or
ss~set, Sullivan associates a corresponding \emph{``De Rham complex''}
with rational coefficients, in order to get a functor from rational
homotopy types to the derived category obtained from those algebras --
and (if I remember it right) this is an equivalence of categories,
provided one restricts to $1$-connected homotopy types, and
correspondingly to $1$-connected algebras. Probably somebody must have
explained to me by then (it was in 1976 more or less) why not every
eligible differential algebra could be recovered (up to isomorphism in
the derived category) by the corresponding cohomology algebra, namely
why it was not necessarily isomorphic to the latter, endowed with zero
differential operator; I am afraid I forgot it since! Also, it was
well-known by the informed people (as I was told too) that there where
obstructions against expressing the multiplicative structure in
cohomology with (say) integer coefficients, in terms of an
anti-commutative differential graded $\bZ$-algebra; so there was no
hope, I was informed, for defining something like a ``De Rham complex
with integer coefficients'' for an arbitrary topological space.

All this was very interesting indeed -- still, I found it hard to
believe that, while succeeding in constructing De Rham complexes with
rational coefficients for arbitrary spaces, by looking at the
algebraic De Rham complex on the enveloping affine space for the
various singular simplices of a space, that the same could not be
achieved with integral coefficients. Of course, the basic Poincaré
lemma for algebraic differential forms was no longer true, however
this reminded me strongly of a similar difficulty met with in
algebraic geometry, and which is overcome by working with suitable
``divided power structures'' -- as Poincaré's lemma becomes valid when
replacing usual polynomials (as coefficients for differential forms)
by ``polynomials with divided powers''. Then I got quite excited and
involved in a formalism of De Rham complexes with divided powers
for\pspage{341} arbitrary semisimplicial sets, which took me a few
weeks to work out and alongside getting back into homotopy and
cohomology formalism again. I had the feeling that this structure, or
the technically more adequate dual ``coalgebra'' structure, might well
turn out to be the more refined version of \eqref{eq:94.27} needed for
recovering homotopy types -- or at any rate $1$-connected ones. I gave
a talk about the matter at IHES while things were still hot in my mind
-- but it doesn't seem it went really through. It doesn't seem this
structure (which was worked out independently by someone else too, I
understand) has become a familiar notion to
topologists.\scrcomment{I'm guessing the reference is to \textcite{Cartan1976} and
  \textcite{Miller1978}\ldots anyhow, there has been many developments since
  concerning Witt vectors, crystalline cohomology, etc.\ldots} Maybe
one reason is that most topologists and homotopy theorists never
really got acquainted with the formalism of derived categories -- and
it seems that moreover, by the mid-seventies, it had even become
altogether unfashionable and ``mal vu'' to make any mention of them,
let alone work with them, also among some of the people who during
some time had been helping develop it. Now one of the main points I
was making in that talk was a somewhat delicate property of derived
categories of abelian categories, with respect to binomial
coefficients -- too bad!

I have not heard since about any work done in this direction I am
reflecting about now (somewhat retrospectively) -- namely recovering
homotopy types from their abelianization, \emph{plus} extra
structure. For all I know, \emph{the} relevant structure may well be
the differential algebra with divided powers structure embodied by the
De Rham complex (with a bigraduation however instead of just a
graduation), or its coalgebra version -- viewed as defining an object
of a suitable derived category. (Of course, when there is a
fundamental group $\pi$ around, one will have to look at a slightly
more complex structure still, involving operations of $\pi$, by
looking at the De Rham complex of the universal covering.) If it is
just the matter of describing homotopy types in terms of other models
than semisimplicial complexes, it must be admitted that the new models
are of incomparably more intricate description than the complexes!
There \emph{is} however one feature of it which greatly struck me by
that time, and still seems to me quite intriguing, namely \emph{that
  this structure, although definitely not ``abelian'' anymore} (due to
multiplication as well as to divided power structure), \emph{makes a
  sense over any commutative ground ring} (or even scheme, etc.). When
this ring is $\bQ$, the ``models'' we get modelize rational homotopy
types, which was the starting point of my reflections\pspage{342}
about seven years ago. Replacing $\bQ$ by a more general ring, this
suggests that \emph{there might exist a notion of ``homotopy types''
  over any ground ring $k$} -- and a corresponding notion of ground
ring extension for homotopy types. For abelianizations of homotopy
types, this is particularly ``obvious'', as being just the functorial
dependence of the derived category $\D_\bullet(k)$ with respect to the
ground ring $k$, corresponding to ring extension in a chain
complex. For a week or two I played around with this idea, which on
the semisimplicial level tied in with expressing homotopy types of
some simple spaces (such as standard $K(\pi,n)$ spaces and fibrations
between these) in terms of some simple \emph{semisimplicial schemes}
(affine and of finite type over $\Spec(\bZ)$), by taking $\bZ$-valued
points of these; ring extension $\bZ\to k$ was interpreted in the
scheme-theoretic sense.

I didn't go on very long, as soon after I was taken by personal
matters and never took up the matter later -- and maybe it was an
altogether unrealistic or silly attempt. If I remember it right, the
idea lurking was something of this kind, that \emph{there was a
  functor from \Hot{} to} (if not an equivalence of \Hot{} with\ldots)
\emph{a suitable derived category of some category of semisimplicial
  schemes over} $\Spec(\bZ)$, and that the base change intuition, as
suggested by the abelianized theory or by the subtler ``divided power
De Rham theory'', would reflect in naive base change $\bZ\to k$ for
schemes.

I was then looking mainly at $1$-connected structures, but there was
an idea too that nilpotent fundamental groups might fit into the
picture, with the hope that such a group (under suitable restrictions,
finite presentation and torsion freeness say) could be expressed in a
canonical way in terms of an affine nilpotent group scheme of finite
type over $\Spec(\bZ)$, by taking the integral points of the
latter. It seems (if I remember right) that this is not quite true
though -- that one couldn't hope for much better than getting a
nilpotent algebraic group scheme \emph{over $\bQ$} -- and that one
would recover the discrete group one started with only ``up to
commensurability''. Possibly, there may be \emph{an equivalence
  between localization of the category of nilpotent groups as above
  \textup(with respect to monomorphisms with image of finite
  index\textup) and affine nilpotent connected algebraic group schemes
  over $\bQ$}, or equivalently, group schemes whose underlying scheme
is isomorphic to standard affine space.

\bigbreak

\presectionfill\ondate{7.7.}\pspage{343}\par

% 95
\hangsection{Contractors}\label{sec:95}%
After this cascade of ``afterthoughts'' on abelianization of homotopy
types, it is time now to resume some more technical work, and get
through with this unending part \ref{ch:IV}, in accordance with the
short range working program I had come to four days ago (section
\ref{sec:92}, p.\ \ref{p:320}). I'll take up the three topics stated
there -- namely contractors, induced structures, and ``miscellaneous''
-- in that order, as reviewed previously. Thus, we'll start with
contractors. I have in mind now mainly the definition of contractors,
and a few basic facts following easily from what is already known to
us.

The first thought that comes to my mind is to define a contractor as a
category $A$ such that the set $\Ob(A)$ of all objects of $A$ is a
contractibility structure on $A$, i.e., that there exists a
contractibility structure on $A$ for which every object in $A$ is
contractible. The trouble with this definition is that it makes the
implicit assumption that $A$ is stable under finite products -- as the
notion of a contractibility structure was defined only in a category
satisfying this extra assumption (cf.\ section \ref{sec:51},
\ref{subsec:51.D}). Now, this assumption is \emph{not} satisfied by
the three standard test categories, including $\Simplex$, which surely
we do want to consider as contractors! The next thought then,
suggested by this reflection, is to embed $A$ into \Ahat{} to supply
the products which may be lacking in $A$, and demand there exist a
contractibility structure on \Ahat, such that the objects in $A$ be
contractible and moreover generate; or, what amounts to the same, that
for the homotopy interval structure on \Ahat{} defined by intervals
coming from $A$ as a generating family, the objects of $A$ are
contractible (which implies that this structure ``is'' indeed a
contractibility structure). This condition (in the more general case,
when $A$ appears as a full subcategory of any larger category $M$) has
been restated in wholly explicit terms as the \emph{``basic
  assumption''} \ref{cond:51.Bas.4} on a set of objects, in order that
it generate a contractibility structure (section \ref{sec:51}, p.\
\ref{p:118}). It is immediate that in the case when $A$ itself is
stable under finite products in the ambient category, that this
condition is intrinsic to $A$ and just amounts to the first definition
we had in mind.

Still,\pspage{344} we will call a category $A$ satisfying the
condition \ref{cond:51.Bas.4} with respect to the embedding
\begin{equation}
  \label{eq:95.1}
  A\hookrightarrow\Ahat
  \tag{1}
\end{equation}
a \emph{precontractor}, as we'll expect something more still from a
contractor, which will be automatically satisfied in the particular
case when $A$ is stable under finite products. Roughly speaking, we
want to have a satisfactory relation between contractibility and
asphericity in \Ahat{} -- we'll make this more precise below. For the
time being, let's dwell just a little more on the notion of a
precontractor.

A second thought about contractors, coming alongside with the first,
is that for any full embedding of $A$ into a larger category
\begin{equation}
  \label{eq:95.2}
  f:A\to M, \quad\text{$M$ stable under finite products,}\tag{2}
\end{equation}
$f(A)$ should generate in $M$ a contractibility structure. In the
particular case when $A$ is stable under finite products (and hence
the notion of a precontractor, already defined, coincides with the
notion of a contractor), this is indeed so provided $A$ is a
(pre)contractor, and moreover $f$ commutes to finite products. When
$A$ is just assumed to be a precontractor (without an assumption about
stability of $A$ under products), we'll assume in compensation that
$M$ is stable under small direct limits, which allows to take the
canonical extension $f_!$ of $f$ to \Ahat{} in a way commuting to
direct limits
\begin{equation}
  \label{eq:95.3}
  f_!:\Ahat\to M,\tag{3}
\end{equation}
and we can now state: \emph{if $f_!$ commutes to finite products}
(cf.\ prop.\ \ref{prop:85.1}, \ref{it:85.prop1.a}, p.\ \ref{p:281}),
\emph{then $f(A)$ generates a contractibility structure in $M$.} This
statement is true even without assuming that the functor $f$ is fully
faithful (and follows immediately from the criterion
\ref{cond:51.Bas.4} of p.\ \ref{p:118}); however, in the particular
case when $f$ is fully faithful, we have a handy criterion (prop.\
\ref{prop:85.2}, p.\ \ref{p:283}) for $f_!$ to commute to finite
products, namely that $f(A)$ be a strictly generating subcategory of
$M$, or equivalently, that the functor
\begin{equation}
  \label{eq:95.4}
  f^*:M\to\Ahat\tag{4}
\end{equation}
(right adjoint to $f_!$) be fully faithful. In this case, we may
identify $M$ (up to equivalence) to a full subcategory of \Ahat{}
containing $A$, and the fact that $f(A)$ generates a contractibility
structure in $M$ follows immediately directly (without having to rely
on existence of direct limits in $M$, nor even existence of $f_!$). To
sum up:
\begin{propositionnum}\label{prop:95.1}
  Let\pspage{345} $A$ be a small category, $M$ a category stable under
  finite products, $f:A\to M$ a functor, we assume $A$ is a
  \emph{precontractor}. Then $f(A)$ generates a contractibility
  structure in $M$ in each of the following three cases:
  \begin{enumerate}[label=\alph*),font=\normalfont]
  \item\label{it:95.prop1.a}
    $A$ stable under finite products, and $f$ commutes to these.
  \item\label{it:95.prop1.b}
    There exists a functor $f_!:\Ahat\to M$ extending $f$, and
    commuting to final object and binary products in \Ahat{} of
    objects in $A$\kern1pt.
  \item\label{it:95.prop1.c}
    The functor $f$ is fully faithful and strictly generating.
  \end{enumerate}
\end{propositionnum}

Of course, the validity of the conclusion in either case
\ref{it:95.prop1.b} or \ref{it:95.prop1.c}, for fixed $A$ and variable
$M$ and $f$, \emph{characterizes} the property for $A$ of being a
precontractor, and the same for \ref{it:95.prop1.a} if we assume
beforehand that $A$ is stable under finite products. Thus, we may view
the proposition \ref{prop:95.1} as the most comprehensive statement of
the meaning of this property.
\begin{propositionnum}\label{prop:95.2}
  Let $A$ be a precontractor. Let \Ahatc{} be the set of contractible
  objects in \Ahat{} for the contractibility structure generated by
  the subcategory $A$\kern1pt, \Ahatas{} \textup(resp.\ \Ahatlocas\textup) the
  set of aspheric \textup(resp.\ locally aspheric -- cf.\ p.\
  \ref{p:250}\textup) objects of \Ahat, $h$ the homotopy structure on
  \Ahat{} associated to the contractibility structure \Ahatc, i.e.,
  generated by the intervals in \Ahat{} coming from $A$\kern1pt. As usual,
  \scrWA{} denotes the set of weak equivalences in \Ahat{} -- it is
  understood here that the basic localizer $\scrW\subset\Fl\Cat$ is
  $\scrWoo$ = usual weak equivalence. The following conditions on $A$
  are equivalent:
  \begin{description}
  \item[\namedlabel{it:95.prop2.i}{(i)}]
    \Ahat{} is totally aspheric \textup(i.e., $\Ob A\subset\Ahatlocas$\textup).
  \item[\namedlabel{it:95.prop2.ii}{(ii)}]
    The asphericity structure \Ahatas{} on \Ahat{} is generated by the
    contractibility structure \Ahatc.
  \item[\namedlabel{it:95.prop2.iiprime}{(ii')}]
    $\Ahatc\subset\Ahatas$.
  \item[\namedlabel{it:95.prop2.iidblprime}{(ii'')}]
    $\Ahatc\subset\Ahatlocas$.
  \item[\namedlabel{it:95.prop2.iii}{(iii)}]
    Any $h$-homotopism is in \scrWA{} \textup(i.e., \scrWA{} is
    ``strictly compatible'' with the homotopy structure $h$,
    \textup(cf.\ section \ref{sec:54}\textup), i.e., $h\le
    h'=h_{\scrWA}$\textup).
  \item[\namedlabel{it:95.prop2.iv}{(iv)}]
    The homotopy structure $h$ is equal to the homotopy structure
    $h'=h_W$ associated to $W=\scrWA$ \textup(cf.\ section
    \ref{sec:54}\textup). 
  \end{description}
\end{propositionnum}
\begin{proof}[Proof of proposition]
  Immediate from what is known to us, via
  \[\text{\ref{it:95.prop2.i}} \Rightarrow
  \text{\ref{it:95.prop2.ii}} \Rightarrow
  \text{\ref{it:95.prop2.iiprime}} \Leftrightarrow
  \text{\ref{it:95.prop2.iidblprime}} \Rightarrow
  \text{\ref{it:95.prop2.i}}
  \quad\text{and}\quad
  \text{\ref{it:95.prop2.ii}} \Rightarrow
  \text{\ref{it:95.prop2.iv}} \Rightarrow
  \text{\ref{it:95.prop2.iii}} \Rightarrow
  \text{\ref{it:95.prop2.i}.}\]
\end{proof}
\begin{definitionnum}\label{def:95.1}
  A\pspage{346} small category $A$ is called a \emph{contractor} if it
  is a precontractor, and if moreover it is totally aspheric or,
  equivalently, satisfies one of the equivalent condition
  \ref{it:95.prop2.i} to \ref{it:95.prop2.iv} of prop.\
  \ref{prop:95.2}.
\end{definitionnum}

Equivalently, this also means that
\begin{equation}
  \label{eq:95.5}
  \Ob A \subset \Ahatc\sand \Ahatlocas,\tag{5}
\end{equation}
i.e., every object in $A$ is contractible and locally aspheric, where
the set \Ahatc{} of ``contractible'' objects of \Ahat{} is defined in
terms of the homotopy interval structure $h$ generated by all
intervals in \Ahat{} coming from objects in $A$. (Thus structure is
not necessarily a contractibility structure, but it is when $A$ is a
precontractor, namely $\Ob A\subset\Ahatc$.)

The most trivial example of a contractor is the final category
$\Simplex_0$, and more generally, any category equivalent to it
(NB\enspace a category equivalent to a precontractor resp.\ to a
contractor is again a precontractor resp.\ a contractor). Such a
contractor will be called \emph{trivial}. For a trivial contractor
$A$, we get an equivalence
\[\Ahat\equeq\Sets.\]

If
\[ (M,M\subc)\]
is a contractibility structure, and $A\subset M$ any small full
subcategory of $M$ generating the contractibility structure, then $A$
is a precontractor, hence a contractor if{f} $A$ is totally aspheric,
which will be the case if $A$ is stable in $M$ under binary products,
a fortiori if it is stable under finite products, i.e., contains
moreover a final object of $M$. Thus, \emph{the contractibility
  structure of $M$ can always be generated by a full subcategory $A$
  of $M$ which is a contractor}.

Apart from these two examples, the most interesting examples of
contractors are of course the three standard test categories
$\Simplex$, $\Square$ and $\Globe$, and also their finite
products. Note that the notion of a precontractor or of a contractor
is clearly stable under finite products.
\begin{propositionnum}\label{prop:95.3}
  Let $A$ be a precontractor, assume $A$ non-trivial, i.e.,
  non-equivalent to the final category. Then $A$ contains a separating
  interval, and hence it is a \textup(strict\textup) test category if
  $A$ is tot.\ asph., i.e., is a contractor.
\end{propositionnum}
\begin{proof}
  As any object of $A$ has a section (over $e_\Ahat$), it follows
  immediately that any non-empty object of \Ahat{} has a section too,
  hence \emph{any non-empty subobject of $e_\Ahat$ is equal to
    $e_\Ahat$}. This implies that for any\pspage{347} interval
  \[\bI=(I,\delta_0,\delta_1)\]
  in \Ahat, either \bI{} is separating, i.e.,
  $\Ker(\delta_0,\delta_1)$ is the empty object of \Ahat, or
  $\delta_0=\delta_1$. If no interval coming from $A$ was separating,
  this then would just mean that any two sections (over $e_\Ahat$) of
  an object $I$ of $A$ are equal. By the definition of the homotopy
  structure in \Ahat{} generated by these intervals, this would imply
  that any homotopism in \Ahat{} is an isomorphism, and hence that any
  contractible object for this structure is isomorphic to the final
  object $e_\Ahat$. As by assumption on $A$ all objects of $A$ are
  contractible, this would mean that $A$ is trivial, which is against
  our assumptions, qed.
\end{proof}
\begin{corollarynum}\label{cor:95.1}
  Let $(M,M\subc)$ be a contractibility structure.
  \begin{enumerate}[label=\alph*),font=\normalfont]
  \item\label{it:95.cor1.a}
    The following conditions are equivalent \textup(and will be
    expressed by saying that this given contractibility structure is
    \emph{trivial}\textup):
    \begin{enumerate}[label=(\roman*),font=\normalfont]
    \item\label{it:95.cor1.a.i}
      Two maps in $M$ which are homotopic are equal.
    \item\label{it:95.cor1.a.ii}
      Any homotopism in $M$ is an isomorphism.
    \item\label{it:95.cor1.a.iii}
      Any homotopy interval in $M$ is ``trivial'', i.e., any two
      homotopic sections of an object of $M$ are equal.
    \item\label{it:95.cor1.a.iv}
      Any contractible object of $M$ is a final object, i.e., $M\subc$
      is just the set of all final objects of $M$.
    \item\label{it:95.cor1.a.v}
      Any two sections of a contractible object are equal.
    \end{enumerate}
  \item\label{it:95.cor1.b}
    Assume\pspage{348} the contractibility structure $M\subc$
    non-trivial, i.e., there exists an interval
    \[\bI=(I,\delta_0,\delta_1)\quad\text{with}\quad I\in M\subc,
    \delta_0\ne\delta_1.\]
    Then for any small category $A$ and any functor
    \[i:A\to M\]
    factoring through $M\subc$, the interval $i^*(\bI)$ in \Ahat{} is
    separating. Hence if \textup(for a given basic localizer
    \scrW\textup) $i$ is totally \scrW-aspheric \textup(cf.\ theorem
    \ref{thm:79.1} cor.\ \ref{cor:79.1} p.\ \ref{p:252}\textup), hence
    $i^*(I)$ is totally aspheric in \Ahat, then $A$ is a \scrW-test
    category. In particular, if $A$ is totally \scrW-aspheric and $i$
    is $M\suba$-\scrW-aspheric, then $A$ is a strict \scrW-test category.
  \end{enumerate}
\end{corollarynum}
\begin{proof}
  Part \ref{it:95.cor1.a} is a tautology in terms of section
  \ref{sec:51}. For part \ref{it:95.cor1.b}, to prove that $i^*(\bI)$
  is separating, we only have to check that for any $a$ in $A$, the
  two compositions
  \[\begin{tikzcd}[cramped,sep=huge]
    a \to i^*(e_M) \simeq e_\Ahat
    \ar[r,shift left=1pt,"\ensuremath{i^*(\delta_0),i^*(\delta_1)}"]\ar[r,shift right=2pt] &
    i^*(I)
  \end{tikzcd}\]
  are distinct, or what amounts to the same by the definition of
  $i^*$, that the compositions
  \[\begin{tikzcd}[cramped,sep=small]
    i_!(a) \eqdef x \to e_M
    \ar[r,shift left=2pt]\ar[r,shift right=2pt] &
    I
  \end{tikzcd}\]
  are distinct. As $x$ is in $M\subc$, it has a section over $e_M$, so
  it is enough to check that the compositions with $e_M\to x$ are
  distinct, which just means that $\delta_0\ne\delta_1$, qed.
\end{proof}
\begin{remarks}
  \namedlabel{rem:95.1}{1)}\enspace Part \ref{it:95.cor1.b} of the
  corollary replaces cor.\ \ref{cor:79.3} on page \ref{p:253}, which
  is a little monster of incongruity (as I just discovered) -- namely,
  two of the assumption on \bI{} made there (namely that \bI{} be a
  multiplicative interval, and $I\in\Ob C$) are useless if we assume
  just $I$ contractible, moreover the awkward separation assumption
  made there just reduces by the trivial argument above to the
  assumption $\delta_0\ne\delta_1$!

  \namedlabel{rem:95.2}{2)}\enspace It should be noted that the
  homotopy structure on \Ahat{} envisioned in th.\ \ref{thm:79.1} of
  section \ref{sec:79} (p.\ \ref{p:252}) is \emph{not} defined as in
  the present section, in terms of intervals in \Ahat{} coming from
  $A$ (call this structure $h$), but as
  \[ h' = h_\scrWA\]
  defined in terms of intervals in $\Ahatlocas$; this depends a priori
  on the choice of \scrW, as it has to because th.\ \ref{thm:79.1}
  gives a criterion for the functor $i$ to be $M\suba$-\scrW-aspheric
  which does depend on \scrW. (It surely won't be the same if we take
  $\scrW=\scrWoo=$ usual weak equivalence, or $\scrW=\Fl(\Cat)$ hence
  $\scrWA=\Fl(A)$ and\pspage{349} the condition that $i$ be
  $M\suba$-\scrW-aspheric is always satisfied!) However, let's assume
  $A$ to be totally \scrW-aspheric and every object of $A$ has a
  section (over the final object $e_\Ahat$ of \Ahat) or, what amounts
  to the same, every ``non-empty'' object of \Ahat{} has a section --
  we'll say in this case $A$ is \emph{``strictly totally
    \scrW-aspheric''} (compare section \ref{sec:60}, p.\ \ref{p:149},
  in the particular case $\scrW=\scrWoo$, and with \Ahat{} replaced by
  an arbitrary topos). Let's assume moreover that \scrW{} satisfied
  \ref{loc:4}. In this case, the homotopy structure $h'=h_\scrWA$ does
  not depend on the choice of \scrW, namely it is the so-called
  ``canonical homotopy structure''
  \[h'' = h_\Ahat\]
  of the (strictly totally $0$-connected) category \Ahat{} (cf.\
  section \ref{sec:57}), which in the special case of a category
  \Ahat{} can also be defined as the homotopy structure $h_\scrWz$
  associated to ${\scrWz}_A$, where \scrWz{} is the coarsest basic
  localizer satisfying \ref{loc:4}, i.e.,
  \[\scrWz=\set[\big]{f\in\Fl\Cat}{\text{$\piz(f)$ bijective}}.\]
  The proof of this fact $h'=h''$, i.e.,
  \[h_\scrWA = h_\Ahat \quad ({}=h_{\scrWz})\]
  is essentially the same as for the similar prop.\ (section
  \ref{sec:60}, p.\ \ref{p:149}). The condition \ref{loc:4} on \scrW,
  i.e., $\scrW\subset\scrWz$ clearly implies
  \[h_\scrWA\subset h_{{\scrWz}_A},\]
  and to get the opposite inequality, for which we'll use the
  assumption on $A$, we only have to prove that for any $0$-connected
  object $K$ of \Ahat, any two sections are ($h'=h_\scrWA$)-homotopic,
  a fortiori (as $h\le h'$ by the assumption of total
  \scrW-asphericity of $A$) it is enough to prove they are
  $h$-homotopic. Now this follows from lemma \ref{lem:82.2}, p.\
  \ref{p:268}, applied to $\scrC=\Ahat$, $C=A$.
\end{remarks}

The\pspage{350} preceding reflections thus prove the following
afterthought to theorem \ref{thm:79.1} of section \ref{sec:79}:
\begin{propositionnum}\label{prop:95.4}
  Let $(M,M\subc)$ be a contractibility structure, $A$ a small
  category, $i:A\to M$ a functor factoring through $M\subc$. Let
  moreover \scrW{} be a basic localizer satisfying
  \textup{\ref{loc:4}}. We assume $A$ \emph{strictly totally
    \scrW-aspheric}, i.e., totally \scrW-aspheric and moreover any
  object of $A$ has a section \textup(over $e_\Ahat$\textup).
  \begin{enumerate}[label=\alph*),font=\normalfont]
  \item\label{it:95.prop4.a}
    The homotopy structure $h_\scrWA$ on \Ahat{} is equal to the
    canonical homotopy structure $h_\Ahat$ defined by $0$-connected
    intervals, and equal also to the homotopy structure $h$ defined by
    intervals coming from $A$:
    \begin{equation}
      \label{eq:95.6}
      h=h_\scrWA=h_\Ahat.\tag{6}
    \end{equation}
  \end{enumerate}
  In what follows, we assume \Ahat{} endowed with this homotopy
  structure, and denote by \Ahatc{} the set of all contractible object
  in \Ahat. We equally endow \Ahat{} with its canonical
  \scrW-asphericity structure, and $M$ with the \scrW-asphericity
  structure associated to its contractibility structure $M\subc$. With
  these conventions:
  \begin{enumerate}[label=\alph*),font=\normalfont,start=2]
  \item\label{it:95.prop4.b}
    The following conditions on $i$ are equivalent, where
    \[i^*:M\to\Ahat\]
    is the functor defined as usual in terms of $i$:
    \begin{enumerate}[label=(\roman*),font=\normalfont]
    \item\label{it:95.prop4.b.i}
      $i^*$ is compatible with the homotopy structures \textup(cf.\
      criteria on pages \ref{p:251}--\ref{p:252}\textup), which can be
      expressed also by
      \[i^*(M\subc)\subset\Ahatc\]
      \textup(a condition independent from \scrW, in view of
      \textup{\ref{it:95.prop4.a})}.
    \item\label{it:95.prop4.b.ii}
      $i$ is \scrW-aspheric, i.e.,
      \[M_\scrW \subset (i^*)^{-1}(A\uphat_\scrW)\]
      \textup(where $M_\scrW$ and $A\uphat_\scrW$ are the sets of
      \scrW-aspheric objects in $M$ and in \Ahat\textup).
    \item\label{it:95.prop4.b.iii}
      \textup(For a given full subcategory $C$ of $M$ generating the
      contractibility structure $M\subc$\textup):
      \[i^*(C)\subset A\uphat_\scrWz = \text{set of $0$-connected
        objects of \Ahat.}\]
    \end{enumerate}
  \item\label{it:95.prop4.c}
    Assume these conditions hold, and moreover that the
    contractibility structure of $M$ is non-trivial. Then $A$ is a
    strict \scrW-test category.
  \end{enumerate}
\end{propositionnum}
\begin{proof}
  Part\pspage{351} \ref{it:95.prop4.a} has been proved in remark
  \ref{rem:95.2} above, and in view of th.\ \ref{thm:79.1}, p.\
  \ref{p:252}, the equivalence of \ref{it:95.prop4.b.i} and
  \ref{it:95.prop4.b.ii} is clear, hence also the equivalence with
  \ref{it:95.prop4.b.iii} by applying loc.\ cit.\ to \scrWz{} instead
  of \scrW. Part \ref{it:95.prop4.c} now follows from prop.\
  \ref{prop:95.3} cor.\ \ref{cor:95.1} \ref{it:95.cor1.b}.
\end{proof}
\begin{corollary}
  Under the conditions of \textup{\ref{it:95.prop4.c}} above, if
  $M\subc$ is \scrW-modelizing, then $i$ is a \scrW-test functor, and
  induces an \emph{equivalence}
  \[\HotOf_{M,\scrW} \eqdef \scrW_M^{-1} M \tosimeq \HotOf_{A,\scrW}
  \eqdef \scrWA^{-1}\Ahat.\]
\end{corollary}

\bigbreak

\presectionfill\ondate{8.7.}\par

% 96
\hangsection{Vertical and horizontal topoi\dots
  \texorpdfstring{\textup(afterthought on
    terminology\textup)}{(afterthought on
    terminology)}.}\label{sec:96}%
Yesterday's notes have proceeded very falteringly, to my surprise,
while everything seemed ready for smooth sailing. A number of times,
after going on for a page or two ``following my nose'' (as they say in
German),\scrcomment{I'm pretty sure that ``to follow one's nose'' is an
  English idiom in the sense of following one's instinct, while the
  meaning of going straight ahead is shared with the German ``der Nase nach
  gehen''} or for half a page, it turned out it just wasn't right that
way and I would feel quite stupid and put the silly pages away as
scratchpaper and have another start. There wouldn't have been any
point dragging the poor reader (if there is still one left\ldots)
along on my stumbling path, where it was a matter merely of getting
some technical adjustments right. Maybe it is just that attention was
distracted, perhaps precisely through this (partly mistaken, and
anyhow not too inspiring) feeling that everything was kind of cooked
already, and what was left to do was just swallow! What came out in
the process was that finally things were not so clear yet in my mind
as I thought they were. It is a frequent experience that whenever one
wants to go ahead too quickly, one finds oneself dispersing stupidly a
hell of a lot of energy\ldots

There occurred to me some inadequacies with terminology. One is about
the property of certain categories (contractor or precontractors for
instance) that every object of $A$ has a section (over $e_\Ahat$),
which can be viewed also as a property of the topos $\Ahat=\scrA$,
namely that any ``non-empty'' object of the topos has a section. This
is immediately seen (for any given topos \scrA) to imply the property
that the final object $e_\scrA$ has only the two trivial subobjects,
the ``empty'' and the ``full'' one -- or equivalently, that any
subtopos of the topos is either the empty of the full one, -- a
property, too of obvious geometric significance. In case of a topos of
the type \Ahat, one immediately sees the two properties are equivalent
-- but this is not\pspage{352} true for an arbitrary topos: for
instance the classifying topos $B_G$ of a discrete group $G$ has the
second property, but visibly not the first unless $G$ is the unit
group. (I recall that the category of sheaves on $B_G$ is the category
\Gsets{} of sets on which $G$ operates.) I feel both properties for a
topos merit a name. The first (every sheaf has a section) can be
viewed as the strongest conceivable (I would think) global asphericity
property for a topos, as far as $\mathrm H^1$ goes at any rate, as the
$\mathrm H^1$ of $X$ with coefficients in \emph{any} group object will
be zero. (But I confess I didn't try and look if any precontractor,
say, is aspheric\ldots) The second property (every subtopos is
trivial) comes with a rather different flavor, it suggests the image
of just one ``point'' -- and as a matter of fact, the étale topos of a
scheme, say has this property if{f} it is reduced to a point. Such a
topos may called ``punctual'' (not to be confused though with some
other meanings suggested by this word, such as being equivalent to the
topos defined by a one-point topological space, namely \scrA{} being
equivalent to \Sets) or ``atomic'' (which has rather unpleasant
connotations though nowadays!), or maybe ``vertical'' (this image is
suggested by the $B_G$ above) -- the ``base'', i.e., the final object
of \scrA{} being very ``small'' (in terms of harboring subobjects), so
the inner structure is expressed like a kind of tower, related (in the
case of $B_G$) to the ``Galois tower'' of subgroups of $G$\ldots The
corresponding notion of a ``horizontal'' topos is visibly the one when
\scrA{} admits the subobjects of the final object as a generating
family. In terms of these definitions, a topos is horizontal and
vertical if{f} it is either the ``empty'' or the ``final'' (or
``one-point'') topos. This brings to mind that in the notion of
verticality, we should exclude the ``empty'' topos (which formally
satisfies the condition -- every sheaf has a section). This brings to
my attention too that I certainly do not want to consider an empty
category $A$ (defining the ``empty'' topos \Ahat) as a precontractor,
although formally (in terms of yesterday's definition) it is. Thus, I
suggest I'll introduce the following
\begin{definition}
  A topos is called \emph{vertical} if it is not an ``empty'' topos
  (i.e., the category of sheaves on it is not equivalent to the final
  category $\Simplex_0$), and if moreover any open subtopos is either
  the ``empty'' or the ``full'' one (hence the same for any subtopos,
  whether open or not). A small category $A$ is called vertical, if
  the associated topos (with category of sheaves \Ahat)
  is,\pspage{353} or equivalently, if $A$ is non-empty and any object
  of $A$ has a section (over $e_\Ahat$). A topos is called
  \emph{horizontal} if the family of all subobjects of the final
  object in the category of sheaves \scrA{} is generating.
\end{definition}

For instance, the topos associated to a topological space is
horizontal -- in particular, an ``empty'' topos is horizontal. A topos
is both horizontal and vertical if{f} it is a ($2$-)final topos, i.e.,
equivalent to the topos defined by a one-point topological space
(i.e., the category of sheaves is equivalent to \Sets).

The property of verticality, I feel, is of interest in its own right,
as exemplified notably by lemma \ref{lem:82.2} p.\ \ref{p:268} (which
we used yesterday), and the related proposition of section
\ref{sec:60} (p.\ \ref{p:149}). It does not seem at all subordinated
to notions such as total asphericity or total $0$-connectedness, and
goes in an entirely different direction -- thus, the terminology
``\emph{strictly} totally aspheric'' (or totally $0$-connected), which
I still used yesterday (hesitatingly, I should say), is definitely
inadequate. I would rather say ``totally aspheric (or totally
$0$-connected) \emph{and} vertical''.

Another point is about the terminology of \emph{totally aspheric} and
\emph{locally aspheric} objects in a category \Ahat{} (with respect to
a given basic localizer \scrW), introduced in section \ref{sec:79}
(p.\ \ref{p:250}), and still used yesterday. This terminology does not
seem inadequate by itself, I introduced it because it struck me as
suggestive (and the notions it refers to do deserve a name, in order
to be at ease). The trouble here is that it conflicts with another
possible meaning, in accordance with the principle insisted upon
forcefully in the reflections of section \ref{sec:66} -- namely that
for objects or arrows in \Cat, or within a category \Ahat, the
terminology used for naming properties for these should be in
accordance with the terminology used for the corresponding topoi or
maps of topoi. Now, we do have already the notions of a locally
aspheric and totally aspheric topos, which therefore should imply
automatically the meaning of these notions for an object of \Cat{}
(which was done satisfactorily months ago), or for an object of a
category \Ahat. But in the latter case, there is definitely conflict
with the terminology introduced on p.\ \ref{p:252}. This conflict has
not manifested itself yet in any concrete situation, while the
unorthodox terminology has been used quite satisfactorily a number of
times. Therefore, I would like to keep it, as long as I am not forced
otherwise.

\starsbreak

We\pspage{354} were faced yesterday with three different homotopy
structures $h,h',h''$ on a category \Ahat, for a given small category
$A$, which make sense for any $A$, and which in case $A$ is a
contractor all coincide. The exact relationship between these
structures in more general cases has remained somewhat confused, and
in order to dispel the resulting feeling of uneasiness, I took finally
the trouble today to write it out with some case. One of these
structures, $h'$, depends on the choice of a basic localizer \scrW,
whereas the two others don't\ldots

\bigbreak
\presectionfill\ondate{12.7.}\par

% 97
\hangsection[``Projective'' topoi. Morphisms and bimorphisms of
\dots]{``Projective'' topoi. Morphisms and bimorphisms of
  contractors.}\label{sec:97}%
I was interrupted in my notes by visiting friends arriving in close
succession -- then since yesterday I have been busy mainly with letter
writing. Now, I am ready to take up the thread where I left it --
namely some afterthoughts to the reflections of section \ref{sec:95}
on contractors.

First an afterthought to the afterthoughts! I had introduced the name
``\emph{vertical} topos'' for a topos admitting only the two trivial
open subtopoi (page \ref{p:352}), whereas the stronger property that
every ``non-empty'' sheaf has a section remained unnamed (which is no
real drawback as long as we are restricting to topoi of the type
\Ahat, where indeed the two notions coincide). Now, the latter
property can be viewed as the property that every sheaf $F$ \emph{such
  that $F\to e$ be epimorphic}, should admit a section. It is this
last property which does merit to ``be viewed as the strongest
conceivable asphericity property for a topos'' as I commented on it
last Friday (p.\ \ref{p:352}). After I had written this down as a kind
of selfevidence, a doubt turned up though and I qualified the comment
by added ``as far as $\mathrm H^1$ goes at any rate, as the $\mathrm
H^1$ with coefficients in \emph{any} group object will be zero''. I
didn't pause then to see if the doubt was founded -- quite evidently
it isn't, except for $\mathrm H^0$, as it is clear by the usual shift
argument, using embedding of an abelian sheaf into an injective one,
that if for given $k$ (here $k=1$) $\mathrm H^k(X,F)=0$ for any
abelian sheaf $F$, then the same holds for $\mathrm H^n$ with any
$n\ge k$ -- i.e., the global cohomological dimension of $X$ is
$<k$. This implies that any small vertical category (a fortiori any
\emph{precontractor}) is aspheric, provided it is $0$-connected,
indeed its cohomology variants with values in \emph{any} sheaf of
coefficients (not necessarily commutative as far as $\mathrm H^1$
goes) are trivial.

The\pspage{355} property for a topos $X$, with category of sheaves
\scrA, that any object $F$ in \scrA{} covering the final object
$e_\scrA$ should have a section, can be expressed by saying that the
latter is a \emph{projective} object in the category \scrA. Following
the principle to use the same names for properties of a topos, and
corresponding properties of the final sheaf on it, we may call a topos
with the above property a \emph{projective topos}. Thus, the
``non-empty'' topoi such that every ``non-empty'' sheaf has a section,
are exactly the topoi which are both vertical and projective.

\starsbreak

Here is the promised ``exact relationship'' between the three standard
homotopy structures $h,h',h''$ on \Ahat, where $A$ is any small
category (cf.\ end of section \ref{sec:96}, p.\ \ref{p:354}):
\begin{equation}
  \label{eq:97.star}
  \begin{tabular}{@{}c@{}}
    \begin{tikzcd}[baseline=(O.base),row sep=small,column sep=large]
      h \ar[d,equal]\ar[r,phantom,"\le"{description}]
      \ar[r,invisible,"(\text{$A$ tot. \scrW-asph.})"{inner sep=1.2ex}]
      &
      h' \ar[d,equal]\ar[r,phantom,"\le"{description}]
      \ar[r,invisible,"(\scrW\subset\scrWz)"{inner sep=1.2ex}]
      &
      h'' \ar[d,equal]\ar[r,phantom,"\le"{description}]
      \ar[r,invisible,"(\text{$A$ vertical})"{inner sep=1.2ex}]
      & h \\
      |[alias=O]| h_{\Ahat\!,\;A} & h_\scrWA & h_{{\scrWz}_A}=h_\Ahat &
    \end{tikzcd},
  \end{tabular}\tag{*}
\end{equation}
where \scrW{} is a given basic localizer. Above each one of the three
conditional inequalities between homotopy structures $h,h',h''$ I
wrote the natural assumption on $A$ or \scrW{} validating it, and in
the diagram I have recalled the definition of the three homotopy
structures. Apropos the description $h=h_{\Ahat\!,\;A}$, the notation
used here is $h_{M,A}$ when $M$ is a category stable under finite
products and $A$ a full subcategory, for designating the homotopy
structure on $M$ generated by intervals in $M$ coming from
$A$. Apropos $h''=h_\Ahat$, I recall the notation $h_M$ for
designating the canonical homotopy structure on a category $M$
satisfying suitable conditions (section \ref{sec:57}). Also, I recall
\[\scrWz =\set[\big]{f\in\Fl\Cat}{\text{$\piz(f)$ bijective}}.\]
The diagram implies that if $\scrW\subset\scrWz$, i.e., \scrW{}
satisfies \ref{loc:4}, and if moreover $A$ is vertical and totally
\scrW-aspheric, then all three homotopy structures coincide. Also,
taking $\scrW=\scrWz$, we see that $h=h''$ if $A$ is vertical and
totally $0$-connected, which is lemma \ref{lem:82.2} of p.\
\ref{p:268} for $\Ahat$, $A$.

From \eqref{eq:97.star} it follows of course that if $A$ is a
contractor, then (for any \scrW{} satisfying \ref{loc:4}) the three
homotopy structures $h, h', h''$ on $\Ahat$ coincide. In case the
contractor $A$ is not trivial, hence $A$ is\pspage{356} a strict test
category and \Ahat{} is \scrW-modelizing, it follows that \Ahat{} is
even a \emph{canonical modelizer} (with respect to \scrW), i.e.,
defined in terms of the \scrW-asphericity structure associated to the
``canonical'' homotopy structure $h''=h_\Ahat$ on \Ahat{} (cf.\ prop.\
\ref{prop:95.2} \ref{it:95.prop2.ii} p.\ \ref{p:345}). These, for the
time being, together with the modelizers \Cat{} and \Spaces, are the
main examples we got of canonical modelizers. Presumably, stacks
should give another sizable bunch of canonical modelizers, not of the
type \Ahat.

\starsbreak

We still have to say a word about morphisms between contractors $A,
B$. The first thing that comes to my mind is that this should be a
functor
\begin{equation}
  \label{eq:97.0}
  f:A\to B\tag{0}
\end{equation}
such that the corresponding functor
\begin{equation}
  \label{eq:97.1}
  f^*:\Bhat\to\Ahat\tag{1}
\end{equation}
should be compatible with the homotopy structures, which can be
expressed, as we know, in manifold ways, the most natural one here
being the following two
\begin{equation}
  \label{eq:97.2}
  f^*(B) \subset \Ahatc\tag{2}
\end{equation}
or
\begin{equation}
  \label{eq:97.3}
  f^*(\Bhatc) \subset \Ahatc,\tag{3}
\end{equation}
which are both implied by the apparently weaker one
\begin{equation}
  \label{eq:97.4}
  f^*(B) \subset A\uphat_{\scrWz} \eqdef \text{set of $0$-connected
    objects of \Ahat,}\tag{4}
\end{equation}
and equivalently still, as $\Ahatc\subset A\uphat_\scrW\subset
A\uphat_\scrWz$ (where \scrW{} is a basic localizer satisfying
\ref{loc:4}), to the condition
\begin{equation}
  \label{eq:97.5}
  f^*(B)\subset A\uphat_\scrW \quad ( \eqdef \text{set of
    \scrW-aspheric objects of \Ahat}).\tag{5}
\end{equation}
Thus, the condition for $f$ to be a ``morphism of contractors'' just
boils down to the long familiar \emph{\scrW-asphericity} of $f$, and
implies the following relation, apparently stronger than
\eqref{eq:97.5}:
\begin{equation}
  \label{eq:97.5prime}
  B\uphat_\scrW=(f^*)^{-1}(A\uphat_\scrW).\tag{5'}
\end{equation}
This comes almost as a surprise (after a four day interruption in
contact with the stuff!) -- but it occurs to me now that we got
already a more general statement with prop.\ \ref{prop:95.4} of
section \ref{sec:95} (p.\ \ref{p:350}), which includes the
situation\pspage{357} when instead of $f:A\to B$, we got a functor
\begin{equation}
  \label{eq:97.6}
  f:A\to \Bhat, \quad\text{factoring through \Bhatc}\tag{6}
\end{equation}
(which need not factor through $B$), or equivalently a functor
\begin{equation}
  \label{eq:97.7}
  f_!:\Ahat\to\Bhat\tag{7}
\end{equation}
commuting with small direct limits, or equivalently still, a functor
$f^*$ in opposite direction, commuting with small inverse limits, but
in the last two cases with the extra condition that $f_!(A)\subset
\Bhatc$. We may want to extend the notion of morphism of contractors
to include this situation, hence expressed by the two following
conditions on a functor $f$ \eqref{eq:97.6} or $f_!$ \eqref{eq:97.7},
or on the pair $(f_!,f^*)$ of adjoint functors
\begin{equation}
  \label{eq:97.8}
  \begin{cases}
    f_!(A)\subset\Bhatc &\\
    f^*(B)\subset\Ahatc &\quad.
  \end{cases}\tag{8}
\end{equation}
However, in order for this notion to be stable under composition, we
should strengthen the first of the relations \eqref{eq:97.8} into
\begin{equation}
  \label{eq:97.9}
  f_!(\Ahatc)\subset\Bhatc,\tag{9}
\end{equation}
which follows automatically whenever $f_!$ commutes to finite products
(cf.\ section \ref{sec:85}), but may not follow from \eqref{eq:97.8}
in general, even in the case when $f$ factors through $B$, i.e., in
the case we start with a functor \eqref{eq:97.0} $f:A\to B$. Thus, we
get \emph{two} plausible notions of a morphism of contractors, neither
of which implies the other, and I feel unable to predict which one
will prove the more useful. As far as terminology goes, it seems
reasonable to reserve the name ``morphism of contractors'' to the
first notion, as the second is adequately characterized as a
\emph{bimorphism} between the contractibility structures
$(\Ahat,\Ahatc)$ and $(\Bhat,\Bhatc)$ defined by the contractors $A$
and $B$ (cf.\ section \ref{sec:86}).

\bigbreak
\presectionfill\ondate{16.7.}\par

% 98
\hangsection[Sketch of proof of \protect\smashSimplexf{} being a weak test
category -- and \dots]{Sketch of proof of
  \texorpdfstring{\protect\Simplexf}{Delta-f} being a weak test
  category -- and perplexities about its being
  aspheric!}\label{sec:98}%
Next point on my provisional program is induced structures
(asphericity of contractibility structures) on a category $M_{/a}$,
when one is given on $M$ -- but finally I decided to skip this, as
there was no urgent need for clarifying this and I am not writing a
treatise, thanks Gods! I felt more interested writing down the proof
that the category \Simplexf{} of standard ordered simplices without
degeneracies is a weak test category, as announced months ago, in
section \ref{sec:43}. It then\pspage{358} seemed to come out rather
simply, but I didn't keep notes of the proof I thought I found, which
caused me spending now a day or two feeling a little stupid, as the
stuff was resisting while I felt it shouldn't! It did come out in the
end I guess -- and still I feel a little stupid, with the impression
of having bypassed definitely some very simple argument which had
presented itself as a matter of evidence by the end of March. On the
other hand, I was led to reflect on some other noteworthy features of
the situation, so I don't feel I altogether have been loosing my time.

There are four main variants of categories of standard simplices,
inserting into a diagram
\begin{equation}
  \label{eq:98.1}
  \begin{tabular}{@{}c@{}}
    \begin{tikzcd}[baseline=(O.base)]
      \Simplex\ar[r,"\beta"] & \Simplextilde \\
      \Simplexf\ar[u,"\alpha"]\ar[r,"\beta^{\mathrm f}"] &
      |[alias=O]| \Simplextildef\ar[u,"\widetilde\alpha"] 
    \end{tikzcd},
  \end{tabular}\tag{1}
\end{equation}
where \Simplextilde{} denotes the category of non-ordered standard
simplices $\Simplex_n$, and where the exponent $f$ in \Simplexf{} and
\Simplextildef{} denotes restriction to maps which are injective,
namely compositions of boundary maps (plus symmetric in the
non-ordered case). I recall that $\Simplex$ and \Simplextilde{} are
contractors, wheres \Simplexf{} and \Simplextildef{} are not even test
categories. We'll see however that \Simplexf{} is a \emph{weak} test
category, and presumably the same kind of argument should apply to
prove that \Simplextildef{} is a weak test category too. On the other
hand, from the point of view of the modelizing story, the main common
property of the four functors in \eqref{eq:98.1} should be
asphericity. However, I checked this for $\alpha$ and $\beta$ only, as
this was all I needed for getting the desired result on \Simplexf. As
a matter of fact, $\beta$ is even better than being aspheric, it is a
\emph{morphism of contractors}; more precisely still, for any object
$E$ in \Simplextilde, namely essentially a finite non-empty set,
choosing one point $a$ in $E$, one easily constructs an elementary
homotopy for $\beta^*(E)$ from the identity map to the constant map
defined by $\beta^*(a)$. I do not know on the other hand whether
$\beta$ defines a bimorphism between the canonical contractibility
structures on \Simplexhat{} and \Simplextildehat, namely whether
$\beta_!$ transforms contractible elements into contractible ones
(which would be clear if we knew that $\beta_!$ commutes to finite
products).

We'll come back upon proof of asphericity of $\alpha$, and show at
once how this implies that \Simplexf{} is a weak test category, or
equivalently, that the canonical functor\pspage{359}
\[i_\Simplexf : \Simplexf\to\Cat\]
is aspheric, for the canonical asphericity structure on \Cat. (I
should have noted that the asphericity statements are meant here in
the strongest possible sense, namely with respect to $\scrW=\scrWoo=$
usual weak equivalence.) Now, for any $\Simplex_n$ in \Simplexf, the
category
\[ i_\Simplexf(\Simplex_n) \eqdef \Simplexf_{/\Simplex_n}\]
is canonically isomorphic to the category associated to the ordered
set of all non-empty subsets of $\Simplex_n$, hence we get a canonical
isomorphism
\begin{equation}
  \label{eq:98.2}
  i_\Simplexf \simeq \widetilde i(\beta\alpha),\tag{2}
\end{equation}
where
\begin{equation}
  \label{eq:98.3}
  \widetilde i:\Simplextilde\to\Cat\tag{3}
\end{equation}
is the standard test functor, associating to any non-ordered simplex
$E$ the category associated to the ordered set of all non-empty
subsets of $E$. We know already (section \ref{sec:34}) that
$\widetilde i$ is aspheric, i.e., $\widetilde i^*$ transforms aspheric
objects of \Cat{} into aspheric ones, hence the same holds for its
composition with the aspheric functor $\beta\alpha$, hence also for
$i_\Simplexf$, qed.

Thus, we are left with proving that $\alpha$ is aspheric, i.e., that
the categories
\begin{equation}
  \label{eq:98.star}
  \Simplexf_{/\alpha^*(\Simplex_n)}\tag{*}
\end{equation}
are aspheric. Now, let's denote by \Simplexprimef{} the category
deduced from \Simplexf{} by adding an initial object $\varnothing$
(which we may view as being the empty simplex), which defines an
``open subcategory'' $U$, namely as sieve in \Simplexprimef, in such a
way that \Simplexf{} appears as the ``closed subcategory'', i.e.,
cosieve in \Simplexprimef{} complementary to $U$. One immediately
checks that the category \eqref{eq:98.star} is canonically isomorphic
to
\begin{equation}
  \label{eq:98.starstar}
  (\Simplexprimef)^{n+1} \setminus U^{n+1},\tag{**}
\end{equation}
where $U^{n+1}$ is the open subcategory defined by the initial object
of the ambient category $(\Simplexprimef)^{n+1}$. Now, asphericity of
\eqref{eq:98.starstar} and hence of \eqref{eq:98.star} follows from
the following two lemmas:
\begin{lemmanum}\label{lem:98.1}
  The category \Simplexf{} is aspheric.
\end{lemmanum}
\begin{lemmanum}\label{lem:98.2}
  Let $(X_i,U_i)_{i\in I}$ be a finite non-empty family of pairs
  $(X_i,U_i)$, where $X_i$ is a small category, $U_i$ an open
  subcategory. We assume that for any $i$ in $I$, $X_i$ and the closed
  complement $Y_i$ of $U_i$ in $X_i$ are aspheric. Let $X$ be the
  product of the $X_i$'s, $U$ the products of the $U_i$'s,\pspage{360}
  then \textup($X$ and\textup) $X\setminus U=Y$ are aspheric too.
\end{lemmanum}
\noindent\emph{Proof of lemma \ref{lem:98.2}:} by an immediate
induction, we are reduced to the case when $I$ has just two elements,
$I=\{1,2\}$, but then $X\setminus U$ can be viewed as the union of the
two closed subcategories $X_1\times Y_2$ and $X_2\times Y_1$, whose
intersection is $Y_1\times Y_2$. As all three categories are aspheric
(being products of aspheric categories), it follows by the well-known
Mayer-Vietoris argument that so is $X\times U$, qed.

Thus, we are left with proving that \Simplexf{} is aspheric. Somewhat
surprisingly, that's where I spent a number of hours not getting
anywhere and feeling foolish! There is a very simple heuristic
argument though involving the standard calculation of the cohomology
invariants of any semisimplicial ``complex'', in terms of the standard
boundary operations: if we admit that the same calculations are valid
when working with semisimplicial ``face complexes'', i.e., objects of
\Simplexfhat, then it is enough to apply this to the final object of
\Simplexfhat{} (including for computation of the non-commutative
$\mathrm H^1$ with constant coefficients) to get asphericity of
\Simplexf. As a matter of fact, this argument would give directly
asphericity of $\alpha$, bypassing altogether the categories
\eqref{eq:98.starstar} and lemmas \ref{lem:98.1} and \ref{lem:98.2}.
Apparently, I got a block against the down-to-earth computational
approach to cohomology via semisimplicial calculations, and have been
trying to bypass it at all price -- and not succeeding! Then,
curiously enough, when finding no other way out than look at those
boundary operations and try to understand what they meant (something I
remember vaguely have been doing once ages ago!), this brought me back
again to the abelianization story of sections \ref{sec:92} and
\ref{sec:93}, and to a more comprehensive way for looking at
``abelianizators'', and get an existence and unicity statement for
these. (At any rate, for a suitably strengthened version of these.)
This seems to me of independent interest, and worth being written down
with some care.

% 99
\hangsection{Afterthoughts on abelianization IV: Integrators.}\label{sec:99}%
When writing down (in sections \ref{sec:92} and \ref{sec:93}) some
rambling reflections about ``abelianization'' and ``abelianizators'',
there has been a persistent feeling of uneasiness, which I kept
pushing aside, as I didn't want to spend too much thought on this
``digression''. This uneasiness had surely something to do with the
way abelianization (of an object $X$ say of an elementary modelizer
\Ahat) was handled, so that it was designed in a more or less
exclusive way for embodying information about the ``homology
structure'' of the homotopy type modelized by $X$, or\pspage{361}
equivalently, to describe its cohomology invariants with arbitrary
\emph{constant} coefficients. Now, among the strongest reflexes I
acquired in the past while working with cohomology, was systematically
to look at coefficients which are arbitrary sheaves (abelian say), and
to view constant or locally constant coefficients as being just
particular cases. This reflex has been remaining idle, not to say
repressed, during nearly all of the reflections of the last four
months, due to the fact that in the whole modelizing story woven
around weak equivalence, there was a rather exclusive emphasis on
constant and locally constant coefficients, disregarding any other
coefficients throughout. Probably, while reflecting on abelianization,
a more or less underground reminiscence must have been around of the
semisimplicial boundary operators having a meaning for computing
cohomology of ``something'', with coefficients in arbitrary sheaves --
and also that to get it straight, one had to be careful not to get
mixed up in the variances. But I just didn't want to dive into all
this again if I could help it -- and now it is getting clear, after a
day or two of feeling silly, that it can't be helped, and I'll have to
write things down at last, however ``well-known'' they may be.

Let $A$ be a small category. In section \ref{sec:93} we defined an
``abelianizator'' for $A$ to be a chain complex $L_\bullet$ in the
additive envelope $\Add(A\op)$ of the category $A\op$ opposite to $A$,
satisfying a suitable condition of commutativity (in the diagram
\eqref{eq:92.20} of p.\ \ref{p:328}), and endowed with a mild extra
structure $\lambda$, expressing this commutativity. The function of an
abelianizator in loc.\ cit.\ was essentially to allow for a
simultaneous ``computational'' description of the homology structure
of the homotopy types stemming from a variable object $X$ in \Ahat, or
equivalently, to describe cohomology of such $X$ (as an object of a
suitable derived category say, to get it at strongest) with
coefficients in any (\emph{constant}) ring or abelian
group. Introducing by an independent symbol the opposite category
\begin{equation}
  \label{eq:99.1}
  B=A\op,\tag{1}
\end{equation}
I want now to establish a relationship between this property or
function of a chain complex $L_\bullet$ in $\Add(B)$, involving
objects in \Ahat{} and their abelianizations in \Ahatab, with an
apparently different one, in terms of a variable object of \Bhatab{}
(\emph{not} \Ahatab{} this time!), namely expressing \emph{cohomology
  of $B$} (i.e., of the topos \Bhat{} defined by $B$) with
coefficients in an \emph{arbitrary} abelian presheaf $F$ on $B$, i.e.,
an arbitrary object\pspage{362} of \Bhatab. I will first describe this
property of (possible) function of a chain complex in $\Add(B)$,
forgetting for the time being the category $A=B\op$ and the homotopy
types defined by objects $X$ in \Ahat. Once this property is well
understood, it will be time to show it implies the previous one
relative to $A$ and objects of \Ahat, and presumably is even
equivalent with it.

First, we'll have to interpret the category $\Add(B)$, which was
constructed somewhat ``abstractly'' in section \ref{sec:93} (as the
solution of a universal problem stated in section \ref{sec:92}), as a
full subcategory of the category \Bhatab{} of abelian presheaves on
$B$. It will be useful to keep in mind the following diagram of
canonical functors
\begin{equation}
  \label{eq:99.2}
  \begin{tabular}{@{}c@{}}
    \begin{tikzcd}[baseline=(O.base)]
      B\ar[r,"\alpha_B"] \ar[d]\ar[dr,"\beta_B"] & \Bhat
      \ar[d,"\Wh_B"] \\
      \Add(B) \ar[r,"\gamma_B"'] & |[alias=O]| \Bhatab
    \end{tikzcd},
  \end{tabular}\tag{2}
\end{equation}
where $\alpha_B$ is the canonical inclusion, $\Wh_B$ is the
abelianization functor, $\beta_B$ the composition of the two, and
$\gamma_B$ the \emph{additive} functor factoring $\beta_B$, in virtue
of the universal property of $\Add(B)$. This functor is defined up to
canonical isomorphism, and the lower triangle of \eqref{eq:99.2} is
commutative, up to a given commutativity isomorphism. Also, we'll use
the composition of the following sequence of canonical equivalences of
categories:
\begin{multline*}
  \Bhatab = \bHom(B\op,\Ab) \tosim \bHom(B,\Ab\op)\op \\
  \tosim \bHomadd(\Add(B),\Ab\op)\op
  \tosim \bHomadd(\Add(B)\op,\Ab),
\end{multline*}
i.e., a canonical equivalence of category
\begin{equation}
  \label{eq:99.3}
  \Bhatab\tosim\bHomadd(\Add(B)\op,\Ab),\quad
  F\mapsto\widetilde F,\tag{3}
\end{equation}
which is a particular case of
\begin{equation}
  \label{eq:99.3prime}
  \bHom(B\op,M)\tosim \bHom(\Add(B)\op,M),\tag{3'}
\end{equation}
where $M$ is any additive category. If $F$ is an abelian presheaf on
$B$, i.e., an object in the left-hand side of \eqref{eq:99.3}, we'll
denote by
\begin{equation}
  \label{eq:99.4}
  \widetilde F: \Add(B)\op \to \Ab\tag{4}
\end{equation}
the corresponding additive functor. Now, this functor can be
interpreted very nicely in terms of the functor $\gamma_B$ in
\eqref{eq:99.2}, by the canonical isomorphism of abelian groups
\begin{equation}
  \label{eq:99.5}
  \widetilde F(L)\tosim \Hom_\Bhat(\gamma_B(L),F),\tag{5}
\end{equation}
functorial\pspage{363} with respect to the pair $(F,L)$ in
$\Bhatab\times\Add(B)\op$. This formula in turn implies easily that
the functor $\gamma_B$ is \emph{fully faithful}. Thus, we can
interpret $\Add(B)$ as the full subcategory of \Bhatab{} whose objects
are all finite direct sums (in \Bhatab) of objects of the type
$\Wh_B(b)$, with $b$ in $B$. In terms of this interpretation,
$\gamma_B$ is just an inclusion functor, and on the other hand, for
$F$ in the ambient category \Bhatab, $\widetilde F$ is just the
restriction to the subcategory $\Add(B)$ of the contravariant functor
on \Bhatab{} represented by $F$.

This situation is the exact ``additive'' analogon of the situation of
$B$ embedded in \Bhat{} as a full subcategory, the functor on $B$
defined by an object $F$ of \Bhat{} being the restriction to $B$ of
the contravariant functor on \Bhat{} represented by $F$, i.e., an
object of $F(b)$ or $\widetilde F(b)$ can (often advantageously) be
interpreted as a map in \Bhat, $b\mapsto F$. Moreover, the fact that
\[\alpha_B^*: F\mapsto \widetilde F: \Bhat \tosim \bHom(b\op,\Sets)\]
is an equivalence (in fact, an isomorphism even), is paralleled by the
equivalence \eqref{eq:99.3}, which can likewise be interpreted as
$\gamma_B^*$, or more accurately as the canonical factorization of the
purely set-theoretic $\gamma_B^*:\Bhatab\to\bHom(\Add(B)\op,\Sets)$
through $\bHomadd(\Add(B)\op,\Ab)$\ldots

The objects $L$ of the full subcategory $\Add(B)$ of \Bhatab{} have a
very strong common property, namely they are \emph{projectives}, and
they are of \emph{finite presentation} (``small'' in Quillen's
terminology), namely for variable $F$ in \Bhatab, the functor
\[F\mapsto \Hom_\Bhatab(L,F)\]
commutes with filtering direct limits. Both properties are immediate,
and they nearly characterize the objects in $\Add(B)$ -- more
accurately, it is immediately checked that the projectives of finite
presentation in \Bhatab{} are exactly those which are isomorphic to
\emph{direct factors} of objects in $\Add(B)$. It shouldn't be hard to
check that the full subcategory of \Bhatab{} made up with the
projectives of finite presentation can be identified up to equivalence
to the ``Karoubi envelope'' of the category $\Add(B)$ (obtained by
formally adding images of projectors), or equivalently, can be
described as the solution of the $2$-universal problem defined by
sending $B$ into categories which are, not only additive, but moreover
stable under taking images of projectors (i.e., endomorphisms $u$ of
objects, such that $u^2=u$).

We'll\pspage{364} henceforth identify $\Add(B)$ to a full subcategory
of \Bhatab{} (by replacing the solution of the universal problem,
constructed in section \ref{sec:95}, by the essential image in
\Bhatab{} say), and rewrite \eqref{eq:99.5} simply as
\begin{equation}
  \label{eq:99.5prime}
  \widetilde F(L)\simeq \Hom(L,F),\tag{5'}
\end{equation}
the $\Hom$ being taken in \Bhatab, category of abelian presheaves on
$B$. Accordingly, if $L_\bullet$ is a chain complex in $\Add(B)$,
hence in \Bhatab, the corresponding \emph{cochain} complex $\widetilde
F(L_\bullet)$ in \Ab{} can be interpreted as
\begin{equation}
  \label{eq:99.6}
  \widetilde F(L_\bullet) \simeq \Hom^\bullet(L_\bullet,F),\tag{6}
\end{equation}
where the symbol $\Hom^\bullet$ means taking $\Hom$'s componentwise.

What we're after here is to find a \emph{fixed} chain complex
$L_\bullet$ in $\Add(B)$, such that for \emph{any} abelian presheaf
$F$ on $B$, the cochain complex \eqref{eq:99.6} in \Ab{} should be
isomorphic (in the derived category $\D^\bullet\Ab$ of cochain
complexes in \Ab{} with respect to quasi-isomorphisms) to the
``integration'' of $F$ over the topos \Bhat, i.e., to $\mathrm
R\Gamma_B(F)$:
\begin{equation}
  \label{eq:99.star}
  \Hom^\bullet(L_\bullet,F)\simeq \mathrm R\Gamma_B(F)\text{
    ?}\quad(\text{isom.\ in $\D^\bullet\Ab$}),\tag{*}
\end{equation}
namely to the total right derived functor $\mathrm R\Gamma_B$ (taken
for the argument $F$) of the ``sections'' functor
\begin{equation}
  \label{eq:99.7}
  \Gamma_B(F) \eqdef \varprojlim_{B\op} F.\tag{7}
\end{equation}
Now, using the fact that the components of the chain complex
$L_\bullet$ are projective, hence $\Ext^i(L_n,F)=0$ for $i>0$ (any
$n$, any $F$), we get at any rate a canonical isomorphism in
$\D^\bullet\Ab$, or in $\D\Ab$:
\begin{equation}
  \label{eq:99.8}
  \Hom^\bullet(L_\bullet,F)\simeq \mathrm R\Hom(L_\bullet,F),\tag{8}
\end{equation}
i.e., an interpretation of \eqref{eq:99.6} as a ``hyperext''. Now,
let's remember that $\mathrm R\Gamma_B(F)$ (as on any topos) can be
interpreted equally as
\begin{equation}
  \label{eq:99.9}
  \mathrm R\Gamma_B(F)\simeq\mathrm R\Hom(\bZ_B,F),\tag{9}
\end{equation}
where $\bZ_B$ denotes the constant presheaf on $B$ with value
$\bZ$. Thus, the wished-for isomorphism \eqref{eq:99.star} will follow
most readily from a corresponding isomorphism in $\D_\bullet(\Bhatab)$
between $L_\bullet$ and $\bZ_B$. But using again the fact that the
components of $L_\bullet$ are projective, we see that to give a map in
the derived category of $L_\bullet$ into $\bZ_B$ amounts to the same
as to give an \emph{augmentation}
\begin{equation}
  \label{eq:99.10}
  L_\bullet\to\bZ_B,\tag{10}
\end{equation}
and the map is an isomorphism in $\D_\bullet(\Bhatab)$
if{f}\pspage{365} the augmentation \eqref{eq:99.10} turns $L_\bullet$
into a (projective) \emph{resolution} of $\bZ_B$.

We now begin to feel in known territory again! Let's call
``\emph{integrator}'' on $B$ any projective resolution of $\bZ_B$, and
let's call the integrator ``\emph{special}'' (by lack of a more
suggestive name) if its components $L_n$ are in $\Add(B)$, or what
amounts to the same, if it can be viewed as a chain complex in
$\Add(B)$, endowed with the extra structure \eqref{eq:99.10}. Of
course, $\bZ_B$ is no longer in $\Add(B)$ in general, and therefore
the data \eqref{eq:99.10} has to be interpreted as a map $L_0\to\bZ_B$
external to $\Add(B)$, or equivalently (via \eqref{eq:99.5prime}) as
an object
\begin{equation}
  \label{eq:99.11}
  \lambda\quad\text{in}\quad \widetilde{\bZ}_B(L_0) = \bZ^{(I_0)},\tag{11}
\end{equation}
where $I_0$ is the set of indices used in order to express $L_0$ as
the direct sum in \Bhatab{} of elements of $B$. We know, by the
general principles of homological algebra, that any two integrators
must be chain homotopic, hence, if they are special, as $\Add(B)$ is a
full additive subcategory of \Bhatab, they must be chain homotopic in
$\Add(B)$.

As for existence of integrators, it follows equally from general
principles, as we know that \Bhatab{} has ``enough projectives''
(which is a very special feature indeed of \Bhatab, coming from the
fact that the topos \Bhat{} has enough projectives, namely the objects
of $B$\ldots). It isn't clear though that there exists a special
integrator, because when trying inductively to construct the
resolution $L_\bullet$ of $\bZ_B$ with components in $\Add(B)$, it
isn't clear that the kernel of $L_n\to L_{n-1}$ is ``of finite type'',
namely is isomorphic to a quotient of an object of $\Add(B)$ (or,
equivalently, is a quotient of a projective of finite
presentation). If we take for instance $B$ to be the one-object
groupoid defined by a group $G$, an integrator on $B$ is just a
resolution of the constant $G$-module $\bZ$ by projective
$\bZ[G]$-modules, and the integrator is special of{f} the components
are even free modules of finite type -- I doubt such a resolution
exists unless $G$ itself is finite. This example seems to indicate
that the existence of a special integrator for $B$ is a very strong
condition on $B$, of the nature of a (homological) finiteness
condition. Maybe this condition, more than most others, singles out
the three standard test categories and their finite products, from
arbitrary test categories (even strict ones and contractors\ldots).

Even in case a strict integrator doesn't exist for $B$, there is
a rather evident way out to get ``the next best'' in terms of
computations, namely replacing the very much finitely restricted
category $\Add(B)$ by a larger category\pspage{366}
\begin{equation}
  \label{eq:99.12}
  \Addinf(B)\hookrightarrow \Bhatab\tag{12}
\end{equation}
deduced from $B$ by adding, not merely finite direct sums (and linear
combinations of maps), but equally infinite ones. The construction can
be given ``formally'' as in section \ref{sec:93}, and it can be
checked that this category satisfies the obvious $2$-universal
property with respect to functors
\[f:B\to M\]
from $B$ to infinitely additive categories $M$ (namely additive
categories stable under direct sums), and functors $M\to M'$ which are
not merely additive, but commute to small direct sums. Moreover, it is
checked that the category $\Addinf(B)$ thus constructed embeds by a
fully faithful functor into \Bhatab{} as indicated in
\eqref{eq:99.12}, and hence can be identified up to equivalence to a
full subcategory of \Bhatab. The formulas \eqref{eq:99.5} and
\eqref{eq:99.5prime} are still valid, when $L$ is in $\Addinf(B)$ only
instead of $\Add(B)$. The objects of $\Addinf(B)$ in \Bhatab{} are
still projective (as direct sums of projectives), but of course no
longer of finite presentation. In compensation, any element in
\Bhatab{} is now a quotient of an object in $\Addinf(B)$. As a
consequence, the projectives in \Bhatab{} can be characterized as the
direct factor of objects of $\Addinf(B)$, and presumably the full
subcategory of \Bhatab{} made up with all projectives can again be
described (up to equivalence) as the Karoubi envelope of $\Addinf(B)$,
or equivalently, as the solution of the $2$-universal problem of
sending $B$ into infinitely additive karoubian categories (karoubian =
every projective has an image, i.e., corresponds to a direct sum
decomposition). We may call an integrator $L_\bullet$ for $B$ with
components in $\Addinf(B)$ ``\emph{quasi-special}''. We did just what
was needed in order to be sure now that there always exist
quasi-special integrators; moreover, these integrators are unique up
to chain homotopy in $\Addinf(B)$. The interpretation \eqref{eq:99.11}
of the augmentation structure \eqref{eq:99.10} on $L_\bullet$ is still
valid in the quasi-special case, with the only difference that now the
indexing set $I_0$ need not be finite anymore.

\bigbreak

\presectionfill\ondate{17.6.}\pspage{367}\par

% 100
\hangsection{Abelianization V: Homology versus cohomology.}\label{sec:100}%
Yesterday I introduced the notion of an \emph{integrator} for any
small category $B$, to be just a projective resolution of $\bZ_B$ in
the category \Bhatab{} of all abelian presheaves on $B$, where $\bZ_B$
denotes the constant presheaf with value $\bZ$. Such an object in
$\Ch_\bullet(\Bhatab)$ exists, due to the existence of sufficiently
many projectives in \Bhatab, and it is unique up to homotopism of
augmented chain complexes, which encourages us to denote it by a
canonizing symbol, namely
\begin{equation}
  \label{eq:100.1}
  L_\bullet^B\to \bZ_B.\tag{1}
\end{equation}
As will become clear in the sequel, $L_\bullet^B$ can be viewed as
embodying \emph{homology properties} of $B$, i.e., of the topos
associated to $B$ (whose category of sheaves of sets is \Bhat). The
way we hit upon it though was in order to obtain a ``computational''
way for computing \emph{cohomology} of $B$ (i.e., of the associated
topos) with coefficients in any abelian presheaf $F$ in \Bhatab, by a
canonical isomorphism
\begin{equation}
  \label{eq:100.2}
  \mathrm R \Gamma_B(F) \tosim \Hom^\bullet(L_\bullet^B,F)\tag{2}
\end{equation}
in the derived category $\D^\bullet\Ab$, where $\Hom^\bullet$ denotes
the cochain complex obtained by applying $\Hom$ componentwise. Passing
to the cohomology groups of both members, this gives rise to
\begin{equation}
  \label{eq:100.3}
  \mathrm H^i(B,F) \simeq \mathrm H^i \Hom^\bullet(L_\bullet^B,F).\tag{3}
\end{equation}
The designation ``computational'' takes a rather concrete meaning,
when we choose $L_\bullet^B$ to have its components in the infinitely
additive envelope $\Addinf(B)$ of $B$, which (as we saw yesterday) can
be viewed as a full subcategory of \Bhatab, made up with projectives,
and such that any object in \Bhat{} is quotient of an object coming
from $\Addinf(B)$; this ensures that there exist indeed integrators
which are ``\emph{quasi-special}'', i.e., are made up with objects of
$\Addinf(B)$, and hence can be interpreted as chain complexes of this
additive category. Thus, any component $L_n$ can now be written, in an
essentially canonical way, as
\begin{equation}
  \label{eq:100.4}
  L_n = \bigoplus_{i\in I_n} \bZ^{(b_i)} ,\tag{4}
\end{equation}
where
\begin{equation}
  \label{eq:100.5}
  (b_i)_{i\in I_n} \tag{5}
\end{equation}
is a family of objects of $B$ indexed by $I_n$ (NB\enspace for
simplicity of notations, we assume the $I_n$'s mutually disjoint,
otherwise we should write the\pspage{368} general object in the family
\eqref{eq:100.5} $b_i^n$ rather than $b_i$). Thus, the $n$'th
component of the cochain complex of the second member of
\eqref{eq:100.2} can be explicitly written as
\begin{equation}
  \label{eq:100.6}
  \Hom^n(L_\bullet^B,F) = \Hom(L_n^B,F) \simeq \bigoplus_{i\in I_n} F(b_i),\tag{6}
\end{equation}
and the coboundary operators between these components can be made
explicit in a similar way, by means of (possibly infinite) matrices,
whose entries are $\bZ$-linear combinations of maps from some $b_i^n$
to some $b_j^{n-1}$ ($i\in I_n$, $j\in I_{n-1}$). We feel a little
happier still when the direct sums \eqref{eq:100.4} yielding the
components $L_n$ are finite, i.e., the sets $I_n$ are finite, which
also means that $L_\bullet^B$ can be interpreted as a chain complex in
the additive envelope $\Add(B)$ of $B$, as contemplated in the first
place -- in which case the integrator will be called
``\emph{special}''.

The formula \eqref{eq:100.2} immediately generalizes when $F$ is
replaced by a complex of presheaves $F^\bullet$, with degrees bounded
from below (NB\enspace as the notation indicates, the differential
operator is of degree $+1$), to
\begin{equation}
  \label{eq:100.7}
  \mathrm R\Gamma_B(F^\bullet) \tosim \Hom^{\bullet\bullet}(L_\bullet^B,F^\bullet),\tag{7}
\end{equation}
where now the left-hand side designates hypercohomology of $B$ (i.e.,
of the corresponding topos), viewed as an objects of the right derived
category $\D^+\Ab$ of the category of abelian groups, and where
$\Hom^{\bullet\bullet}$ designates the double complex obtained by
taking $\Hom$'s componentwise, or more accurately, the object in
$\D^+\Ab$ defined by the associated simple complex.

An interesting special case of \eqref{eq:100.7} is obtained when
starting with a complex of abelian groups $K^\bullet$ bounded from
below, i.e., defining an object of the right derived category
$\D^+\Ab$, and taking
\begin{equation}
  \label{eq:100.8}
  F^\bullet=K_B^\bullet=p_B^*(K^\bullet),\tag{8}
\end{equation}
the corresponding \emph{constant complex of presheaves} on $B$, which
may be viewed equally as the inverse image of $K^\bullet$ by the
projection
\begin{equation}
  \label{eq:100.9}
  p_B:B\to\Simplex_0\quad(\text{the final category}),\tag{9}
\end{equation}
which geometrically interprets as the canonical morphism of the topos
associated to $B$ to the final (or ``one-point'') topos. The second
member of \eqref{eq:100.7} can be rewritten componentwise, using the
adjunction formula for the pair $(p_!\supab,p^*)$ (where the
qualifying $B$ is omitted now in the notation $p$):
\[\Hom(L_n,p^*(K^m))\simeq\Hom(p_!\supab(L_n),K^M),\]
so that \eqref{eq:100.7} can be rewritten as\pspage{369}
\begin{equation}
  \label{eq:100.10}
  \mathrm R \Gamma_B(K_B^\bullet) \simeq
  \Hom^{\bullet\bullet}(p_{B!}\supab(L_\bullet^B),K^\bullet),\tag{10}
\end{equation}
where this time the $\Hom$'s in the right-hand side of
\eqref{eq:100.10} are taken in \Ab, not in \Bhatab.

This formula very strongly suggests to view the chain complex of
abelian groups
\begin{equation}
  \label{eq:100.11}
  p_{B!}\supab(L_\bullet^B),\tag{11}
\end{equation}
which is in fact a complex of projective (hence free) abelian groups
defined up to chain homotopy, as embodying the global homology
structure of $B$ (or of the corresponding topos), more accurately
still, as embodying the homology structure of the corresponding
homotopy type. It is easily seen that the corresponding object of
$\D_\bullet\Ab$ depends covariantly on $B$ when $B$ varies in the
category \Cat, so that we get a functor
\[\Cat\to \D_\bullet\Ab \eqdef \Hotab,\]
which in view of \eqref{eq:100.10} (an isomorphism functorial not only
with respect to $K^\bullet$, but equally with respect to $B$) factors
through the localization \Hot{} of \Cat, thus yielding a canonical
functor
\begin{equation}
  \label{eq:100.12}
  \Hot\to\Hotab,\tag{12}
\end{equation}
which deserves to be called the \emph{abelianization functor}, from
homotopy types to ``abelian homotopy types''. This cannot be of course
anything else (up to canonical isomorphism) but the functor
\eqref{eq:92.1} of section \ref{sec:92} (p.\ \ref{p:321}), but
obtained here in a wholly ``intrinsic'' way, without having to pass
through the particular properties of a particular test category such
as $\Simplex$ or one of its twins. One possible way to check this
identity would be by proving that an isomorphism \eqref{eq:100.10} is
valid when replacing (for a given $B$ in \Cat{} and $K^\bullet$ in
$\D^+\Ab$) the chain complex \eqref{eq:100.11} by the corresponding
one deduced from the map \eqref{eq:92.1} defined p.\ \ref{p:321} (via
the diagram \eqref{eq:92.3} on p.\ \ref{p:322}), and checking moreover
that an object $\ell_\bullet$ of $\D^-\Ab$ is known up to canonical
isomorphism, when we know the corresponding functor
\begin{equation}
  \label{eq:100.13}
  K^\bullet\mapsto \Hom_{\D\Ab}(\ell_\bullet,K^\bullet)\tag{13}
\end{equation}
on $\D^+\Ab$. Presumably, this latter statement holds when replacing
\Ab{} by any abelian category, but I confess I didn't sit down to
check it, nor do I remember having seen it stated somewhere -- as I
don't remember either having seen anywhere a comprehensive treatment
about the relationship between homology and cohomology. So maybe my
present reflections do fill a gap, or at any rate give some
indications as to how to fill it\ldots

I played around some yesterday and today with the formalism of
integrators, notably with respect to maps
\[ f:B'\to B\]
between small categories, and the corresponding \emph{integration
  functor}
\[ f_!\supab: {B'}\subab\uphat \to \Bhatab,\]
and its left derived functor $\mathrm Lf_!\supab$. Thus, the chain
complex in \Bhatab
\begin{equation}
  \label{eq:100.14}
  L_\bullet^{B'/B}\quad\text{or}\quad
  L_\bullet^f \eqdef f_!\supab(L_\bullet^{B'}),\tag{14}
\end{equation}
which has projective components (and even is a chain complex in
$\Addinf(B)$ resp.\ in $\Add(B)$, if $L_\bullet^{B'}$ is quasi-special
resp.\ is special), and is defined up to chain homotopism, embodies
the relative homology properties of $B'$ over $B$, i.e., of $f$, in
much the same way as \eqref{eq:100.11} embodies the global homology
properties of $B$ (i.e., of $B$ over one point). When the functor $f$
is ``coaspheric'', i.e., the functor
\[f\op:{B'}\op \to B\op\]
between the opposite categories is aspheric, then $L_\bullet^{B'/B}$
is again an integrator on $B$, and the converse should hold too
provided we take the meaning of ``coaspheric'' and ``aspheric'' with
respect to a suitable basic localizer $\scrW=\scrW_\oo^{\bZ}$ --
presumably, we'll come back upon this in part \ref{ch:V} or part
\ref{ch:VI} of the notes. For the time being, it seems more
interesting to give now the precise relationship between the notion of
an \emph{integrator} for $B$, and the notion of an
\emph{abelianizator} for the dual category $A=B\op$, introduced in
section \ref{sec:93}.

\medbreak

\noindent\textbf{Remarks.} \namedlabel{rem:100.1}{1})\enspace It is a familiar fact that when
working in \v Cech-flavored contexts, such as general topoi, or étale
topoi for schemes and the like, one has throughout and from the start
a good hold upon \emph{cohomology} notions, whereas it is a lot more
subtle to squeeze out adequate homology notions, which (to my
knowledge) can be carried through only indirectly via cohomology, and
using suitable finiteness and duality statements within the cohomology
formalism. Historically however, homology was introduced before
cohomology via cellular decompositions of spaces, with a more direct
appeal to geometric intuition. This preference for homology rather
than cohomology seems to be still prevalent among most homotopy
theorists, who have a tendency to view a topological space (however
wild it may be) as being no more no less than its singular complex. A
comprehensive statement establishing, in a suitable wide enough
context, essential equivalence between the two viewpoints, seems to be
still lacking, as far as I\pspage{371} know -- although a fair number
of partly overlapping results in this direction are known, among the
oldest being the relevant ``universal coefficients formulæ'' relating
homology and cohomology (reducing all to a formula of the type
\eqref{eq:100.2} or \eqref{eq:100.10} above), or Cartan's old seminar
on Leray's sheaf theory, introducing singular homology with
coefficients in a sheaf and proving that on a topological variety,
this was (up to dimension shift and twist by the twisted integers)
essentially the same as singular cohomology (with coefficients in
sheaves too). It is not sure that an all-inclusive statement of
equivalence between homology and cohomology (in those situations when
such equivalence is felt hold indeed) does at all exist -- at any
rate, according to what kind of coefficients one wants to consider,
and what kind of extra structures one is interested in when dealing
with homology and cohomology invariants, it seems that each of the two
points of view has an originality and advantages of its own and cannot
be entirely superseded by the other. From the contexts I have been
mainly working in, there definitely was no choice, namely cohomology
(including non-commutative one) was the basic data, while sheaves and
their generalizations (such complexes of sheaves, or stacks) were the
coefficients. I don't remember of any moment where I would have paused
and asked myself \emph{why} in most contexts where I was working in
(whose common denominator was topoi), there wasn't any direct hold on
anything like homology invariants. The reason for this inertness of
mine, probably, is that the cohomology formalisms I hit upon were
self-contained enough, so as to leave no regret for the absence of a
homology formalism, or at any rate of a more or less direct
description of it independently of cohomology. Another reason, surely,
is that I didn't have too much contact with topologists and
homotopists and their everyday tools, such as Steenrod operations,
homology of the symmetric group, and the like. This question of ``why
this reluctance of homology to show forth'' has finally surfaced only
during these very last days, when the answer for it (or one possible
answer at any rate) is becoming evident: namely, that \emph{for a
  general topos}, embodied by a category of sheaves (of sets) \scrA,
\emph{there are not enough projectives in \scrA, and not even enough
  projectives in $\scrAab$}, the category of abelian sheaves. It
is becoming apparent (what surely everybody has known ages) that in
technical terms, \emph{doing ``homology'' is working with projectives,
  while doing ``cohomology'' is working with injectives}. As there are
enough injectives in $\scrAab$ but not enough projectives,
cohomology is around and homology not, period!

There is however a rather interesting class of topoi admitting
sufficiently\pspage{372} many projective sheaves of sets, and hence
sufficiently many projective abelian sheaves -- namely the topoi
\Bhat{} defined in terms of small categories $B$. They include the
topoi which can be described in terms of semisimplicial complexes and
the like, and can be viewed equally as the topoi which are ``closest
to algebra'' or ``purely algebraic'' in a suitable sense -- for
instance, definable directly in terms of arbitrary presheaves, without
any reference to the notion of site and of localization. (The
intuition of localization remaining however and indispensable guide
even in the so-called ``algebraic'' set-up.) Moreover, the morphisms
which arise most naturally among such topoi, namely those associated
to maps
\[f:B'\to B\]
in \Cat, besides the traditional adjoint pair $(f^*,f_*)$ of functors
between sheaves of sets, gives rise equally to a functor
\begin{equation}
  \label{eq:100.15}
  f_!:{B'}\uphat\to\Bhat\tag{15}
\end{equation}
left adjoint to $f^*$ (i.e., $f^*$ commutes to small inverse limits,
not only to small direct limits and to finite inverse limits),
inserting in a triple of mutually adjoint functors (from left to
right)
\begin{equation}
  \label{eq:100.16}
  (f_!,f^*,f_*).\tag{16}
\end{equation}
The functors $f^*$ and $f_*$ induce corresponding adjoint functors on
abelian sheaves (due to the fact that they commute to finite
products), $f^*\subab$ and $f_*\supab$, whereas $f_!$ does not in
general transform group objects into group objects; however, as
$f^*\subab$ commutes to small inverse limits, it does admit again a
left adjoint $f_!\supab$, so as to give again a triple
\begin{equation}
  \label{eq:100.17}
  (f_!\supab,f^*\subab,f_*\supab)\tag{17}
\end{equation}
of mutually adjoint functors. Now, whereas the derived functors
\[\text{$f^*$ or $\mathrm L f^*\subab$,} \quad
\text{$\mathrm R f_*$ or $\mathrm R f_*\supab$}\]
of $f^*\subab$ and $f_*\supab$ have been extensively used in the
every-day cohomology formalism of topoi, the existence in certain
cases (such as the one we are interested in here) of a functor
$f_!\supab$ and of its left derived functor
\begin{equation}
  \label{eq:100.18}
  \text{$\mathrm L f_!$ or $\mathrm Lf_!\supab: \D^-({B'}\uphat\subab)
    \to \D^-(\Bhatab),$}\tag{18}
\end{equation}
seems to me to have been widely overlooked so far, except in extremely
particular cases such as inclusion of an open subtopos; at any rate, I
have been overlooking it till lately, when it came to my attention
through\pspage{373} the writing of these notes. (Namely, first in
connection with my reflections on derivators (cf.\ section
\ref{sec:69}), and now in connection with the reflections on
abelianization.) In view of my reflections on derivators, I would like
to view the functor \eqref{eq:100.18} as an operation of
``integration'', whereas the traditional functor
\begin{equation}
  \label{eq:100.19}
  \mathrm Rf_* : \D^+({B'}\uphat\subab) \to \D^+(\Bhatab)\tag{19}
\end{equation}
is viewed as ``cointegration'' (which I prefer to my former way of
calling it an ``integration''). The first should be viewed as
expressing \emph{homology} properties of the map $f$ in \Cat{} (or
between the corresponding topoi), just as the latter expresses
\emph{cohomology} properties of $f$. This does check with the
corresponding qualifications ``integration'' -- ``cointegration'' --
as well as with the intuition, when $B$ is reduced to a point,
identifying the first to a kind of direct sum (= integration), whereas
the latter is viewed as a kind of direct product (=
\emph{co}integration). The idea behind the terminology will go through
maybe when looking at the particular case when $B'$ is a sum of copies
of $B$, namely a product of $B$ by a discrete category $I$, and
\[f : B'=B\times I\to B\]
the projection.

The point I want to make here, mainly to myself, is that in the
present context \emph{when \eqref{eq:100.18}, namely integration,
  exists, this operation presumably is by no means less meaningful and
  important than the familiar $\mathrm Rf_*$ or cointegration} -- or
\emph{equivalently stated, that the homology properties of $f$ are
  just as meaningful and deserving close attention, as the cohomology
  properties}, which so far have been the only ones I have been
looking at. Presumably, when following this recommendation, a few
unexpected facts and relationships should come out, such as various
``duality'' relationships between homology properties of $f$, and
cohomology properties of the map $f\op$ between the opposite
categories. (This is suggested by some of the scratchwork I made on
derivators and cohomology properties of maps in \Cat.) The only
trouble is that such change or broadening of emphasis as I am now
suggesting will require a certain amount of extra attention, which I
am not too sure to be willing to invest in the subject, namely
algebraic topology. Thus presumably, my main emphasis will remain with
cohomology, rather than homology. I am no longer convinced though that
this point of view is technically more adequate than the dual one.

\namedlabel{rem:100.2}{2})\enspace All the reflections of yesterday's
notes as well as today's can be extended, when replacing throughout
abelian presheaves by presheaves of\pspage{374} $k$-modules, and
additive envelopes by $k$-linear ones, where $k$ is any given
commutative ring. Of course, the category \Ab{} and its various
derived categories will have to be replaced accordingly by the
category \kMod{} of $k$-modules etc. The same holds for the
relationship I am going to write down between integrators for $B$ and
abelianizators for $A=B\op$. For simplicity of notations, I am going
to keep the exposition in the \Ab-framework I have started with, and
leave the necessary adjustments to the reader.

\bigbreak
\presectionfill\ondate{18.7.}\par

% 101
\hangsection[Abelianization VI: The abelian integration operation
\dots]{Abelianization VI: The abelian integration operation
  \texorpdfstring{$\mathrm Lf_!\supab$}{Lf!ab} defined by a map
  \texorpdfstring{$f$ in \Cat{} \textup(}{f in (Cat) (}versus abelian
  cointegration
  \texorpdfstring{$\mathrm Rf_*$\textup)}{Rf*)}.}\label{sec:101}%
Finally with yesterday's non-technical reflections on homology versus
cohomology, it was getting prohibitively late, and there could be no
question to deal with the relationship between integrators (for $B$)
and abelianizators (for $B\op=A$). Also, I feel I should give some
``computational'' details about the functor $f_!\supab$ associated to
a map in \Cat
\[f:B'\to B,\]
namely
\[f_!\supab:{B'}\uphat\subab \to \Bhatab,\]
which is a lot less familiar to me than its right adjoint and
biadjoint $f^*$ and $f_*$. One way to get a ``computational hold''
upon it is by noting that $f_!\supab$ commuting to small direct limits
and a fortiori being right exact, and moreover any object $F'$ in
${B'}\uphat\subab$ being a cokernel of a map between ``\emph{special
  projectives}'' in ${B'}\uphat\subab$, i.e., between objects in
$\Addinf(B')$, namely inserting into an exact sequence
\[L_1' \xrightarrow d L_0' \to F' \to 0 \quad\text{with $L_0',L_1'$ in
  $\Addinf(B')$,}\]
the functor $f_!\supab$ (via its values on any $F'$ say) is
essentially known, when we know its restriction to the subcategory
$\Addinf(B')$, as we'll get a corresponding exact sequence in \Bhatab
\[f_!\supab(L_1') \to f_!\supab(L_0') \to f_!\supab(F') \to 0,\]
describing $f_!\supab(F')$ as a cokernel of a map $f_!\supab(d)$
corresponding to a map in $\Addinf(B')$. The relevant fact now is that
we have a commutative diagram of functors (up to can.\ isomorphism as
usual)
\begin{equation}
  \label{eq:101.1}
  \begin{tabular}{@{}c@{}}
    \begin{tikzcd}[baseline=(O.base)]
      \Addinf(B')\ar[r]\ar[d,"\Addinf(f)"'] &
      {B'}\uphat\subab\ar[d,"f_!\supab"] \\
      \Addinf(B)\ar[r] & |[alias=O]| \Bhatab
    \end{tikzcd},
  \end{tabular}\tag{1}
\end{equation}
where\pspage{375} the horizontal arrows are the canonical inclusion
functors, and $\Addinf(f)$ is the ``tautological'' extension of $f:B'
\to B$ to the infinitely additive envelopes, defined computationally
as
\begin{equation}
  \label{eq:101.2}
  \Addinf(f)(L') \simeq \bigoplus_{i\in I}\bZ^{f(b_i')}\tag{2}
\end{equation}
for an object of $\Addinf(B')$ written canonically as
\begin{equation}
  \label{eq:101.3}
  L'=\bigoplus_{i\in I}\bZ^{(b_i')}.\tag{3}
\end{equation}
Here, for an object $b$ in a small category $B$, we denote by the more
suggestive symbol $\bZ^{(F)}$ the abelianization $\Wh_B(F)$ of an
object $F$ of \Bhat, and accordingly of $F$ is an object $b$ in
$B$. The fact that \eqref{eq:101.2} is equally an expression for
$f_!\supab$ follows immediately from commutation of $f_!\supab$ to
small direct sums, and from the canonical isomorphism
\begin{equation}
  \label{eq:101.4}
  f_!\supab(\bZ^{F'}) \simeq \bZ^{(f_!(F'))},\tag{4}
\end{equation}
i.e., commutation up to canonical isomorphism of the diagram
\begin{equation}
  \label{eq:101.5}
  \begin{tabular}{@{}c@{}}
    \begin{tikzcd}[baseline=(O.base)]
      {B'}\uphat \ar[r,"\Wh_{B'}"] \ar[d,"f_!"'] &
      {B'}\uphat\subab \ar[d,"f_!\supab"] \\
      \Bhat\ar[r,"Wh_B"] & |[alias=O]| \Bhatab
    \end{tikzcd},
  \end{tabular}\tag{5}
\end{equation}
the verification of which is immediate. (For a generalization to
sheaves endowed with arbitrary ``algebraic structures'' and taking
free objects, see\scrcomment{\textcite{SGA4vol1}} SGA~4 I~5.8.3,
p.~30.)

Of course, \eqref{eq:101.1} and \eqref{eq:101.2} imply that
$f_!\supab$ maps $\Add(B')$ into $\Add(B)$, and induces the
tautological extension $\Add(f)$ of $f$ to the additive
envelopes. Thus, \eqref{eq:101.1} and \eqref{eq:101.5} can be inserted
into a beautiful commutative diagram (up to canonical isomorphism)
\begin{equation}
  \label{eq:101.6}
  \begin{tabular}{@{}c@{}}
    \begin{tikzcd}[baseline=(O.base),column sep=small]
      B' \ar[rr,hook] \ar[d,"f"'] \ar[ddrrr,hook,%
      dash pattern=on 33 pt off 30pt on 60pt] & &
      \Add(B')\ar[rr,hook] \ar[d,"\Add(f)"] & &
      \Addinf(B')\ar[rr,hook] \ar[d,"\Addinf(f)"'] & &
      {B'}\uphat\subab \ar[d,"f_!\supab"] \\
      B \ar[rr,hook] \ar[ddrrr,hook] & &
      \Add(B) \ar[rr,hook] & &
      \Addinf(B)\ar[rr,hook] & &
      \Bhatab \\
      & & & {B'}\uphat \ar[d,"f_!"] \ar[uurrr,"\Wh_{B'}" pos=0.15,%
      dash pattern=on 40pt off 30pt on 60pt] & & & \\
      & & & |[alias=O]| \Bhat \ar[uurrr,"\Wh_B" pos=0.4] & & &
    \end{tikzcd}.
  \end{tabular}\tag{6}
\end{equation}

The formula \eqref{eq:101.1} (or equivalently, \eqref{eq:101.2}) can
be viewed as giving a computational description of the left derived
functor
\begin{equation}
  \label{eq:101.7}
  \mathrm Lf_!\supab:\D^-({B'}\uphat\subab)\to \D^-(\Bhatab).\tag{7}
\end{equation}
Indeed,\pspage{376} by general principles of homological algebra, for
any small category $B$, from the fact that $\Addinf(B)$ is made up
with projective objects of \Bhatab{} and that any object in \Bhatab{}
is isomorphic to a quotient of an object in this subcategory, it
follows that
\begin{equation}
  \label{eq:101.8}
  \D^-(\Bhatab)\equeq W_B^{-1}\Comp^-(\Addinf(B)),\tag{8}
\end{equation}
i.e., the left derived category $\D^-(\Bhatab)$ is equivalent with the
category obtained by localizing, with respect to the set $W_B$ of
homotopy equivalences, the category $\Comp^-(\ldots)$ of differential
complexes in the additive category $\Addinf(B)$, with degrees bounded
from above (the differential operator being of degree $+1$, according
to my preference for cohomology notation, sorry!). An object of
$\D^-(\Bhatab)$ may thus be viewed as being essentially the same as a
differential complex in $\Addinf(B)$ with degrees bounded from above,
and given ``up to homotopism''.  The similar description holds for
$\D^-({B'}\uphat\subab)$, and in terms of these descriptions, the
``integration functor'' (in the abelian context) \eqref{eq:101.7} can
be described by
\begin{equation}
  \label{eq:101.9}
  \mathrm Lf_!\supab(L_\bullet') = \Addinf(L_\bullet'),\tag{9}
\end{equation}
i.e., by applying componentwise the tautological extension
$\Addinf(f)$ of $f$ to the differential complexes in
$\Addinf(B')$. This very concrete description applies notably to the
complex
\begin{equation}
  \label{eq:101.10}
  L_\bullet^{B'/B}\quad\text{or}\quad L_\bullet^f\eqdef f_!\supab(L_\bullet^{B'})\tag{10}
\end{equation}
introduced yesterday, whenever a (quasi-special) integrator
$L_\bullet^{B'}$ for $B'$ has been chosen. Applying this to the case
of the map
\[p_B:B\to\Simplex_0,\]
we get (for a given integrator $L_\bullet^B$ for $B$) an explicit
description of the abelianization of the homotopy type of $B$ in terms
of the chain complex $p_{B!}\supab(L_\bullet^B)$ in \Ab, with the
$n$'th component given by
\begin{equation}
  \label{eq:101.11}
  \bigl(L_\bullet^{B/\mathrm{pt}}\bigr)_n = \bZ^{(I_n)},\tag{11}
\end{equation}
where $I_n$ is the set of indices used for describing $L_n^B$ as the
direct sum of objects of the type $\bZ^{(b_i)}$.

Returning to the case of a general map $f:B'\to B$, maybe I should
still write down the formula generalizing \eqref{eq:100.2} or
\eqref{eq:100.10} of yesterday's notes (pages \ref{p:367} and
\ref{p:369}), relating $L_\bullet^{B'/B}$ to the cohomology properties
of the map $f$, i.e., to cointegration relative to $f$. The
formula\pspage{377} expresses cointegration $\mathrm Rf_*$ with
coefficients coming from downstairs, namely $f^*(K^\bullet)$, where
$K^\bullet$ is any differential complex in \Bhatab{} with degrees
bounded from below (thus defining an object in $\D^+(\Bhatab)$). The
relevant formula is\scrcomment{see section~\ref{sec:139}, bottom of
  p.~\ref{p:588}, for corrections to this formula and
  \eqref{eq:101.12prime}, \eqref{eq:101.13} below\dots}
\begin{equation}
  \label{eq:101.12}
  \mathrm Rf_*(f^*(K^\bullet)) \simeq \bHom^{\bullet\bullet}(L_\bullet^{B'/B},K^\bullet),\tag{12}
\end{equation}
an isomorphism in $\D^+(\Bhatab)$, where $\bHom^{\bullet\bullet}$
designates the double complex in \Bhatab{} obtained by applying
$\bHom$ componentwise, more accurately the associated simple complex,
and where $\bHom$ is the internal $\bHom$ in the category \Bhatab,
namely the (pre)sheaf of additive homomorphisms of a given abelian
(pre)sheaf ($L_n$ say) into another ($K^m$ say). The proof of
\eqref{eq:101.12} is essentially trivial, it is just the computational
interpretation, in terms of using projective resolutions, of the
adjunction formula ``localized on $B$
\begin{equation}
  \label{eq:101.12prime}
  \mathrm Rf_*(\mathrm Lf^*(K^\bullet)) \simeq \mathrm R\bHom(\mathrm
  Lf_!\supab(\bZ_{B'}), K^\bullet),\tag{12'}
\end{equation}
which is a particular case of the more general ``adjunction formula''
\begin{equation}
  \label{eq:101.13}
  \mathrm Rf_*(\mathrm R\bHom(F_\bullet', \mathrm Lf^*(K^\bullet))
  \simeq \mathrm R\bHom(\mathrm Lf_!\supab(F'_\bullet),K^\bullet),\tag{13}
\end{equation}
valid for
\[ \text{$F_\bullet'$ in $\D^-({B'}\uphat\subab)$,}\quad
\text{$K^\bullet$ in $\D^+(\Bhatab)$,}\]
\eqref{eq:101.12prime} following from \eqref{eq:101.13} by taking
$F_\bullet'=\bZ_{B'}$.

\begin{remarks}
  We may view \eqref{eq:101.13}, and its particular case
  \eqref{eq:101.12} or \eqref{eq:101.12prime}, as the main formula
  relating the \emph{homology} and \emph{cohomology} invariants for a
  map $f$ in \Cat, or equivalently, the (abelian) \emph{integration}
  and \emph{cointegration} operations defined by $f$. It now occurs to
  me that this formula, and the variance formalism in which it
  inserts, is valid more generally whenever we have a map $f$ between
  two ringed topoi, such that $f_!$ exists for sheaves of sets, hence
  there exists too a corresponding functor $f_!^{\mathrm{mod}}$ for
  sheaves of modules. The fact that we have been restricting to the
  case of the constant sheaves of rings defined by $\bZ$ isn't
  relevant, and (in the case of topoi defined by objects in \Cat,
  hence with sufficiently many projective sheaves of sets) the
  formalism of the subcategories $\Add(B)$ and $\Addinf(B)$ in
  \Bhatab{} can be generalized equally to arbitrary sheaves of rings
  on $B$. At present, I don't see though any striking particular case
  where this generalization would seem useful.\pspage{378}
\end{remarks}

% 102
\hangsection[Abelianization VII: Integrators (for $A\op$) are
\dots]{Abelianization VII: Integrators
  \texorpdfstring{\textup(\kernifitalic{2pt}}{(}for
  \texorpdfstring{$A\op$\textup)}{Aop)} are abelianizators
  \texorpdfstring{\textup(\kernifitalic{2pt}}{(}for
  \texorpdfstring{$A$\textup)}{A)}.}\label{sec:102}%
We now focus attention upon the pair of mutually dual small categories
\begin{equation}
  \label{eq:102.1}
  (A,B), \quad\text{with $B=A\op$, i.e., $A=B\op$,}\tag{1}
\end{equation}
and recall the equivalence of section \ref{sec:93} following from the
universal property of $\Add(B)$
\begin{equation}
  \label{eq:102.2}
  \Ahatab = \bHom(A\op,\Ab) \equeq \bHomadd(\Add(A\op),\Ab),\tag{2}
\end{equation}
which we parallel with the formula \eqref{eq:99.3} of section
\ref{sec:99} (p.\ \ref{p:362}), which reads when replacing in it $B$
by $A$
\[\Ahatab\simeq\bHomadd(\Add(A)\op,\Ab)\text{;}\]
this immediately suggests a canonical equivalence of categories
\begin{equation}
  \label{eq:102.3}
  \Add(A\op)\equeq \Add(A)\op,\tag{3}
\end{equation}
following immediately indeed from the $2$-universal properties of
these categories. We complement \eqref{eq:102.2} by the similar
formula
\begin{equation}
  \label{eq:102.4}
  F\mapsto \widetilde F: \quad
  \Ahatab\toequ\bHomaddinf(\Addinf(B),\Ab), \quad B=A\op,\tag{4}
\end{equation}
where $\bHomaddinf$ denotes the category of infinitely additive
functors between two infinitely additive categories. In view of the
emphasis lately on chain complexes in $\Addinf(B)$ rather than in
$\Add(B)$, in order to reconstruct say the derived category
$\D_\bullet(\Bhatab)$ of chain complexes in \Bhatab, and get existence
of ``integrators'' with components in $\Addinf(B)$ (whereas there may
be none with components in $\Add(B)$), it is formula \eqref{eq:102.4}
rather than \eqref{eq:102.2} which is going to be relevant for our
homology formalism. Using \eqref{eq:102.4}, we get a canonical
biadditive pairing
\begin{equation}
  \label{eq:102.star}
  \Ahatab \times \Addinf(B) \to \Ab,\tag{*}
\end{equation}
which visibly is exact with respect to the first factor, and which we
may equally interpret as a functor
\begin{equation}
  \label{eq:102.5}
  L\mapsto \widetilde L: \Addinf(B) \to \bHomex(\Ahatab,\Ab),\tag{5}
\end{equation}
where $\bHomex$ denotes the category of \emph{exact} (hence additive)
functors from an abelian category to another one.

It can be shown that the pairing \eqref{eq:102.star} can be extended
canonically to a pairing
\begin{equation}
  \label{eq:102.6}
  \Ahatab \times \Bhatab \to \Ab\tag{6}
\end{equation}
commuting to small direct limits in each variable, and identifying
(up\pspage{379} to equivalence) each left hand factor to the category
of functors from the other functor to \Ab{} which commute with small
direct limits (much in the same way as the corresponding relationship
between \Ahat{} and \Bhat, with \Ab{} being replaced by \Sets), and
the accordingly the functor \eqref{eq:102.5} is equally fully
faithful, and extends to a fully faithful functor from \Bhatab{} to
$\bHomadd(\Ahatab,\Ab)$, inducing in fact an equivalence between
\Bhatab{} and the full subcategory $\bHom_!(\Ahatab,\Ab)$ of
$\bHom(\Ahatab,\Ab)$ made up by all functors $\Ahatab\to\Ab$ which
commute to small direct limits. But for what we have in mind at
present, these niceties are not too relevant yet it seems -- all what
matters is that an object $L$ of $\Addinf(B)$ defines an exact functor
\[\widetilde L :\Ahatab\to\Ab,\]
depending functorially on $L$, in an infinitely additive way. Thus, as
noted in section \ref{sec:92} (but where $\Addinf(B)$ was replaced by
the smaller category $\Add(B)$, which has turned out insufficient for
our purposes), whenever we have a chain complex $L_\bullet$ in
$\Addinf(B)$, we get a corresponding functor
\begin{equation}
  \label{eq:102.7}
  \widetilde L_\bullet:\Ahatab\to\Ch_\bullet\Ab\tag{7}
\end{equation}
from \Ahatab{} to the category of chain complexes of \Ab, which is
moreover an \emph{exact} functor. Generalizing slightly the
terminology introduced in section \ref{sec:93}, where we restricted to
chain complexes with components in $\Add(B)$ rather than in $\Add(B)$,
we'll say that $\widetilde L_\bullet$ is an \emph{abelianizator for
  $A$}, if the following diagram commutes up to isomorphism:
\begin{equation}
  \label{eq:102.8}
  \begin{tabular}{@{}c@{}}
    \begin{tikzcd}[baseline=(O.base)]
      \Ahat\ar[r]\ar[d,"{%
        \begin{array}{@{}c@{}}
          \Wh_A \\ \text{(abelianization)}
        \end{array}}"'] & \HotOf_A \ar[r] & \Hot \ar[d,"{%
        \begin{array}{@{}c@{}}
          \text{``absolute''} \\ \text{abelianization} \\ \text{functor}
        \end{array}}" inner sep=1em] \\
      \Ahatab\ar[r,"\widetilde L_\bullet"] &
      \Ch_\bullet\Ab\ar[r] & |[alias=O]| \Hotab
    \end{tikzcd}.
  \end{tabular}\tag{8}
\end{equation}
More accurately, an abelianizator is a pair $(\widetilde L_\bullet,\lambda)$,
where $\lambda$ is an isomorphism of functors $\Ahat\to\Hotab$ making
the diagram commute. Here, I like to view the abelianization functor
\begin{equation}
  \label{eq:102.9}
  \Hot\to\Hotab\eqdef\D_\bullet\Ab\tag{9}
\end{equation}
as the one described directly in section \ref{sec:100} via integrators
of arbitrary modelizing objects in \Cat, without any reference to an
auxiliary test category such as $\Simplex$ or the like.

The point of \eqref{eq:102.8} is that via an ``abelianizator'' for
$A$, we want to be able to give a)\enspace a \emph{simultaneous} handy
expression, in terms of ``computable'' chain complexes in \Ab,
of\pspage{380} abelianization of homotopy types modelized by a
variable object $X$ in \Ahat, and b)\enspace we want that the chain
complex in \Ab{} expressing abelianization of $X$, should be
expressible in terms of the ``tautological abelianization'' $\Wh_A(X)
= \bZ^{(X)}$ of $X$ itself, by a formula moreover which should make
sense functorially with respect to an arbitrary abelian presheaf,
i.e., an object $F$ in \Ahatab.

The main fact I have in view here is that whenever the chain complex
$L_\bullet$ in $\Addinf(B)$ is endowed with an augmentation
\begin{equation}
  \label{eq:102.10}
  L_\bullet\to\bZ_B\tag{10}
\end{equation}
turning it into a resolution of $\bZ_B$, i.e., into a (quasi-special)
\emph{integrator} for $B$, then ipso facto $L_\bullet$ is an
abelianizator for $A$, the commutation isomorphism $\lambda$ being
canonically defined by the augmentation \eqref{eq:102.10}.

Some comments, before proceeding to a proof. Presumably, the converse
of our statement holds too -- namely that the natural functor we'll
get from quasi-special integrators for $B$ to abelianizators
$(L_\bullet,\lambda)$ for $B$ is an equivalence (even an isomorphism!)
between the relevant categories. I don't feel like pursuing this --
the more relevant fact here, whether or not a converse as contemplated
holds, is that we can pin down at any rate a special class of
abelianizators for $A$, namely those which come from (quasi-special)
integrators for $B$, and these abelianizators are defined up to chain
homotopism in $\Addinf(B)$. In this sense, \emph{we get an existence
  and unicity statement for abelianizators in $A$}, as strong as we
possibly could hope for. In practical terms, it would seem, \emph{an
  abelianizator for $A$ will be no more no less than just a
  \textup(quasi-special\textup) integrator for $B$, namely a
  projective resolution of $\bZ_B$ in \Bhatab, whose components
  satisfy a mild extra assumption besides being projective}.

Here, I am struck by a slight discrepancy in terminology, as we would
rather have a correspondence
\[
\begin{cases}
  &\text{integrators for $B$ $\to$ abelianizators for $A$} \\
  &\text{quasi-special int.s for $B$ $\to$ quasi-special abelian.s for $A$,}
\end{cases}\]
and the same for ``special'' integrators and abelianizators. As I
still feel that the general appellation of an ``integrator'' for
\emph{any} projective resolution of $\bZ_B$ is adequate (without
insisting that the components should be in $\Addinf(B)$), this kind of
forces us to extend accordingly still the notion of an abelianizator
for $A$. This does make sense, using the pairing \eqref{eq:102.6}
(which we had dismissed as an ``irrelevant nicety for the time
being''!), and the corresponding equivalence\pspage{381}
\begin{equation}
  \label{eq:102.11}
  \Bhatab \toequ \bHom_!(\Ahatab, \Ab),\tag{11}
\end{equation}
where the index $!$ denotes the full subcategory of $\bHom$ made up
with functors commuting to small direct limits. It is immediate that
projective objects in \Bhatab{} give rise to objects in $\bHom_!$
which are \emph{exact} functors from \Ahatab{} to \Ab, and I'll have to
check that the converse also holds. If so, a chain complex in
\Bhatab{} with projective components can be interpreted as being just
an \emph{arbitrary exact functor commuting to small sums}
\[ \Ahatab \to \Ch_\bullet\Ab,\]
(never minding whether or not it can be described ``computationally''
in terms of objects in $\Add(B)$ or in $\Addinf(B)$!) -- which is all
that is needed in order to complete the diagram \eqref{eq:102.8}, and
wonder if it commutes up to isomorphism! And the most natural
statement here is that this is indeed so whenever this functor, viewed
as a chain complex in the abelian category
\begin{equation}
  \label{eq:102.12}
  \bHom_!(\Ahatab,\Ab),\tag{12}
\end{equation}
is a (projective) resolution of the canonical object $\widetilde\bZ_B$
of the category \eqref{eq:102.12}, coming from the object $\bZ_B$ of
the left-hand side of \eqref{eq:102.11}. Now, this functor is just the
familiar ``direct limit'' functor
\begin{equation}
  \label{eq:102.13}
  \widetilde\bZ_B \simeq \varinjlim_B : \Ahatab \eqdef \bHom(B,\Ab)
  \to \Ab,\tag{13}
\end{equation}
which can be equally interpreted as
\begin{equation}
  \label{eq:102.14}
  \widetilde \bZ_B \simeq p_{A!}\supab : \Ahatab \to \Ab,\tag{14}
\end{equation}
namely (abelian) ``integration'' with respect to the map in \Cat
\[p_A:A\to\Simplex_0.\]
Thus, ultimately, \emph{abelianizators for $A$} (or what we may call
``standard abelianizators'', if there should turn out to be any
others, and that they are worth looking at) \emph{turn out to be no
  more, no less than just a projective resolution, in the category
  \eqref{eq:102.12} of functors from \Ahatab{} to \Ab{} commuting with
  small direct limits, of the most interesting object in the category,
  namely the functor}
\begin{equation}
  \label{eq:102.15}
  p_{A!}\supab \simeq \varinjlim_B : \Ahatab\to\Ab.\tag{15}
\end{equation}
We are far indeed from the faltering reflections of section
\ref{sec:91}, about computing homology and cohomology of homotopy
models described in terms of test categories deduced some way or other
from cellular decompositions of spheres!

\bigbreak

\presectionfill\ondate{11.8.}\pspage{382}\par

% 103
\hangsection{Integrators versus cointegrators.}\label{sec:103}%
It has been over three weeks now I haven't been working on the
notes. Most part of this time was spent wandering in the Pyrenees with
some friends (a kind of thing I hadn't been doing since I was a boy),
and touring some other friends living the simple life around there, in
the mountains. I was glad to meet them and happy to wander and breathe
the fresher air of the mountains -- and very happy too after two weeks
to be back in the familiar surroundings of my home amidst the gentle
hills covered with vineyards\ldots Yesterday I resumed mathematical
work -- I had to spend the day doing scratchwork in order to get back
into it, now I feel ready to go on with the notes. I'll have to finish
in the long last with that abelianization story I got into
unpremeditatedly -- which turns out to be essentially the same thing
as some systematics about (commutative) cohomology and homology, in
the context of ``models'' in \Cat, or in a category \Ahat{} (with $A$
is \Cat). We were out for proving a statement about ``integrators''
for a small category $B$ being ``abelianizators'' for the dual
category $A=B\op$. The proof I had in mind for this is somewhat
indirect via cohomology, and follows the proof I gave myself a very
long time ago (in case $A\Simplex$), that the usual semi-simplicial
boundary operations do give the correct (topos-theoretic) cohomology
invariants for any object $X$ in \Ahat{} (i.e., any semisimplicial
set), for any locally constant coefficients on $A_{/X}$. The idea was
to replace $A_{/X}$ by the dual category
$(A_{/X})\op=\preslice{A\op}X$ (which, according to a nice result of
Quillen, has a homotopy type canonically isomorphic to the one defined
by $A_{/X}$), and use the canonical functor
\[f = (p_X)\op : (A_{/X})\op \to A\op\eqdef B,\]
which is a cofibration with discrete fibers, and hence gives rise, for
any abelian presheaf $F$ on the category $C$ upstairs, to an
isomorphism
\[\mathrm R \Gamma(C,F) \simeq \mathrm R\Gamma(B,f_*(F))\]
(due to $\mathrm Rf_*(F) \fromsim f_*(F)$, as $f_*$ is exact,
due to the fact that $f$ is a cofibration with discrete fibers). We'll
get Quillen's result about the isomorphism $C\simeq C\op$ in \Hot, for
any object $C$ in \Cat, very smoothly in part \ref{ch:VI}, as a result
of the asphericity story of part \ref{ch:IV}. However, I now realize
that the proof of the fact about abelianizators via Quillen's result
and cohomology is rather awkward, as what we're after now is typically
a result on homology, not cohomology -- and I was really turning it
upside down in order to fit it at all costs into the\pspage{383} more
familiar (to me) cohomology pot! Therefore, I'm not going to write out
this proof, as ``the'' natural proof is going to come out by itself,
once we got a good conceptual understanding of homology, cohomology
and abelianization, in the context of ``spaces'' embodies by objects
of \Cat. Thus, I feel what is mainly needed now is an overall review
of the relevant notions and facts along these lines -- most of which
we've come in touch with before, be it only ``en passant''.

Before starting, just an afterthought on terminology. It occurred to
me that the name of an ``integrator'' (for $A$), for a projective
resolution of the constant abelian presheaf $\bZ_A$ in \Ahatab, in
inaccurate -- as it was meant to suggest that its main use is for
allowing computation, for an arbitrary abelian presheaf (or complex of
such presheaves) $F$ on $A$, of $\mathrm R\Gamma(A,F)$, which we were
thinking of by that time as the ``integration'' of $F$ over $A$ (or
over the associated topos). But it has turned out that for the sake of
coherence with a broader use of the notions of ``integration'' and
``cointegration'' (compare section \ref{sec:69}), the appropriate
designation of $\mathrm R\Gamma(A,F)$ is ``\emph{co}integration'' of
$F$ over $A$, not integration. Therefore, the appropriate designation
for a projective resolution $L_\bullet^A$ of $\bZ_A$, allowing
computational expression of cointegration, is ``cointegrator'' (for
$A$) rather than ``integrator''. On the other hand, in terms of the
dual category $B=A\op$, it turns out that such $L_\bullet^A$ allows
computational expression of \emph{integration} (i.e., homology) over
$B$, and therefore it seems adequate to call $L_\bullet^A$ also an
\emph{``integrator'' for $B$}. Moreover, it turns out that such an
integrator for $B$ is equally an ``abelianizator'' for $B$, i.e., it
allows simultaneous computation of the ``abelianizations'' of the
homotopy types defined by arbitrary objects $X$ in \Ahat, in terms of
the abelianization $\Wh_A(X)=\bZ^{(X)}$ of $X$ (cf.\ sections
\ref{sec:93} and \ref{sec:102}) -- and possibly the converse holds
too. Whether this is so or not, there doesn't seem at present much
sense to bother about abelianizators which do not come from
integrators, while the latter have the invaluable advantage (besides
mere existence) of being unique up to homotopism. Thus, in practical
terms, it would seem that abelianizators (for a given small category
$B$) are no more no less than just integrators (for the same $B$,
i.e., cointegrators for $A=B\op$) -- and I would therefore suggest to
simply drop the designation ``abelianizator'' for the benefit of the
synonym ``integrator'', which fits more suggestively into the pair of
dual notions integrator---cointegrator.\pspage{384}

% 104
\hangsection{Overall review on abelianization
  \texorpdfstring{\textup{(1)}}{(1)}: Case of
  pseudo-topoi.}\label{sec:104}%
I'll have after all to give a certain amount of functorial ``general
non-sense'' which I've tried to bypass so far.

\subsection{Pseudo-topoi and adjunction equivalences.}
\label{subsec:104.A}
In what follows, ordinary capital letters as $A,B,\ldots$ will
generally denote small categories (mostly objects in \Cat), whereas
round capital letters $\scrA,\scrB,\scrM$ will denote \scrU-categories
which may be ``large'', for instance $\scrA=\Ahat$, $\scrB=\Bhat$,
etc.  For two such categories $\scrA,\scrB$, we denote by
\begin{equation}
  \label{eq:104.1}
  \bHom_!(\scrA,\scrB), \quad \bHom^!(\scrA,\scrB)\tag{1}
\end{equation}
the full subcategories of the functor category $\bHom(\scrA,\scrB)$,
made up with all functors which commute with small direct or inverse
limits respectively. This notation is useful mainly in case \scrA{}
and \scrB{} are stable under small direct resp.\ inverse limits, in
which case the same holds true for the corresponding category
\eqref{eq:104.1}, because as a full subcategory of
$\bHom(\scrA,\scrB)$ (where direct resp.\ inverse limits exist and are
computed componentwise) it is stable under direct resp.\ inverse
limits. Thus, the inclusion functors
\begin{equation}
  \label{eq:104.2}
  \bHom_!(\scrA,\scrB) \to \bHom(\scrA,\scrB),\quad
  \bHom^!(\scrA,\scrB) \to \bHom(\scrA,\scrB)\tag{2}
\end{equation}
commute with direct resp.\ inverse limits, i.e., those limits in the
categories \eqref{eq:104.1} are computed equally componentwise.

The canonical inclusion
\[\scrA \hookrightarrow \bHom(\scrA\op,\Sets)\]
factors into a fully faithful inclusion functor
\begin{equation}
  \label{eq:104.3}
  \scrA \hookrightarrow \bHom^!(\scrA\op,\Sets).\tag{3}
\end{equation}
Let's recall the non-trivial useful result:
\begin{propositionnum}\label{prop:104.1}
  Assume the \scrU-category \scrA{} is stable under small direct
  limits, and admits a small full subcategory $A$ which is
  ``generating for monomorphisms'', i.e., any monomorphism $i:X\to Y$
  in \scrA{} such that $\Hom(Z,i):\Hom(Z,X)\to\Hom(Z,Y)$ is bijective
  for any $Z$ in $C$, is an isomorphism. Then the fully faithful
  functor \eqref{eq:104.3} is an equivalence, i.e., any functor
  \[\scrA\op\to\Sets\]
  that commutes with small inverse limits is representable.
\end{propositionnum}

For a proof,\scrcomment{\textcite{SGA4vol1}} see SGA~4~I~8.12.7.

\begin{corollarynum}\label{cor:104.prop1.1}
  If \scrA{} satisfies the assumptions above, then \scrA{} is equally
  stable under small \emph{inverse} limits.
\end{corollarynum}

For\pspage{385} the sake of brevity, we'll say that a \scrU-category
satisfying the assumptions of prop.\ \ref{prop:104.1} is a
\emph{pseudo-topos} (as these conditions are satisfied for any
topos). We get at once the
\begin{corollarynum}\label{cor:104.prop1.2}
  Let $\scrA,\scrB$ be two pseudo-topoi. Then a functor from $\scrA$
  to $\scrB\op$ \textup(resp.\ from $\scrB\op$ to \scrA\textup) has a
  right adjoint \textup(resp.\ a left adjoint\textup) if{f} it
  commutes to small direct limits \textup(resp.\ to small inverse
  limits\textup). Thus, taking right and left adjoints we get two
  equivalences of categories, quasi-inverse to each other
  \begin{equation}
    \label{eq:104.4}
    \bHom_!(\scrA,\scrB\op)\leftrightarrows\bHom^!(\scrB\op,\scrA),\tag{4}
  \end{equation}
  and the two members of \eqref{eq:104.4} are canonically equivalent
  to the category
  \begin{equation}
    \label{eq:104.5}
    \bHom^{!!}(\scrA\op,\scrB\op; \Sets)\tag{5}
  \end{equation}
  of functors
  \[\scrA\op\times\scrB\op\to\Sets\]
  which commute with small inverse limits with respect to either
  variable \textup(the other being fixed\kern1pt\textup), \eqref{eq:104.5}
  being viewed as a full subcategory of $\bHom(\scrA\op\times\scrB\op,
  \Sets)$.
\end{corollarynum}
\begin{remarks}
  \namedlabel{rem:104.1}{1})\enspace The first-hand side of
  \eqref{eq:104.4} is tautologically isomorphic to the category
  $\bHom^!(\scrA\op,\scrB)$ (as for any two \scrU-categories we have
  the tautological isomorphism
  \begin{equation}
    \label{eq:104.6}
    (\bHom_!(\mathscr P,\mathscr Q))\op \simeq \bHom^!(\mathscr P\op,
    \mathscr Q\op)\quad\text{),}\tag{6}
  \end{equation}
  thus, the equivalence \eqref{eq:104.4} can be seen more
  symmetrically as an equivalence
  \begin{equation}
    \label{eq:104.4prime}
    \bHom^!(\scrA\op,\scrB)\simeq\bHom^!(\scrB\op,scrA),\tag{4'}
  \end{equation}
  both categories being equivalent to \eqref{eq:104.5} using the
  equivalence \eqref{eq:104.3} for the second, and the corresponding
  equivalence for \scrB{} for the first, plus the tautological
  isomorphism
  \begin{equation}
    \label{eq:104.7}
    \bHom^!(\scrP, \bHom^!(\scrQ,\scrM)) \simeq
    \bHom^{!!}(\scrP, \scrQ; \scrM),\tag{7}
  \end{equation}
  for any three \scrU-categories $\scrP,\scrQ,\scrM$.

  \namedlabel{rem:104.2}{2})\enspace When \scrA{} is a topos, \scrB{}
  any \scrU-category stable under small inverse limits, then we may
  interpret the category $\bHom^!(\scrA\op,\scrB)$ as the category of
  \emph{\scrB-valued sheaves} on the topos (defined by) \scrA. When
  \scrA{} and \scrB{} are both topoi, then the equivalence
  \eqref{eq:104.4} states that \scrB-valued sheaves on \scrA{} can be
  identified with \scrA-valued sheaves on \scrB, and both may be
  identified with set-valued ``bi-sheaves'' on $\scrA\times\scrB$. In
  case the topoi \scrA, \scrB{} are defined respectively by
  \scrU-sites $A$, $B$ (not necessarily small ones),\pspage{386} these
  bisheaves can be interpreted in a rather evident way as bisheaves on
  $A\times B$, namely functors
  \[A\op\times B\op\to\Sets\]
  which are sheaves with respect to each variable (the other being
  fixed). It is easy to check that the category of all such bisheaves
  is again a topos, and that the latter is a $2$-product of the two
  topoi \scrA, \scrB{} in the $2$-category of all topoi -- it plays
  exactly the same geometrical role as the usual product for two
  topological spaces\ldots
\end{remarks}

\subsection{Abelianization of a pseudo-topos.}
\label{subsec:104.B}
Let \scrA{} be a pseudo-topos, and let's denote by
\begin{equation}
  \label{eq:104.8}
  \scrAab\tag{8}
\end{equation}
the category of abelian group-objects in \scrA. It is immediate that
the forgetful functor
\begin{equation}
  \label{eq:104.9}
  \scrAab\to\scrA\tag{9}
\end{equation}
commutes with small direct limits (and that such limits exist in
$\scrAab$, whereas they exist in \scrA{} by cor.\ \ref{cor:104.prop1.1}
above) -- thus, we may expect that this functor admits a left
adjoint. When so, this will be denoted by
\begin{equation}
  \label{eq:104.10}
  \Wh_\scrA : \scrA\to\scrAab,\tag{10}
\end{equation}
we'll write also
\[\Wh(X)=\bZ^{(X)}\]
when no confusion may arise. The abelianization functor exists for
instance when \scrA{} is a topos, in this case it is well-known that
$\scrAab$ is not only an additive category, but an \emph{abelian}
category with small filtering direct limits which are exact, and a
small generating subcategory. This in turn ensures, as well-known too,
that any object of $\scrAab$ can be embedded into an injective one,
and from this follows (cf.\ SGA~4
I~7.12)\scrcomment{\textcite{SGA4vol1}} that $\scrAab$ admits also a
small full subcategory which is \emph{co}generating with respect to
epimorphisms, in other words that $\scrAab$ is not only an abelian
pseudo-topos, but that the dual category $(\scrAab)\op$ is a
pseudo-topos too. Conversely (kind of), without assuming \scrA{} to be
a topos, if we know some way or other (but this may be hard to check
directly\ldots) that $(\scrAab)\op$ is a pseudo-topos, then it follows
from cor.\ \ref{cor:104.prop1.2} above that the abelianization functor
$\Wh_\scrA$ exists, and this in turn implies that $\scrAab$ is a
pseudo-topos, i.e., admits a small full subcategory which is
generating with respect to monomorphisms (as we see by taking such a
full subcategory\pspage{387} $A$ in \scrA, and the full subcategory in
$\scrAab$ generated by $\Wh_\scrA(A)$).

Let now \scrM{} be an \emph{additive} \scrU-category, which is
moreover a \emph{pseudo-cotopos}, i.e., the dual category $\scrM\op$
is a pseudo-topos. Using twice the corollary \ref{cor:104.prop1.2} above,
for the pair of pseudotopoi $(\scrA,\scrM\op)$ and
$(\scrAab,\scrM\op)$, we get the sequence of equivalences of
categories
\[\bHom_!(\scrA,\scrM)\equeq\bHom^!(\scrM,\scrA)\op \fromequ
\bHom^!(\scrM,\scrAab)\op \equeq \bHom_!(\scrAab,\scrM),\]
where the second equivalence of categories comes from the fact that
any functor
\[f:\scrM\to\scrA\]
from an \emph{additive} category \scrM{} to a category \scrA, which
commutes with finite products, factors canonically through
$\scrAab\to\scrA$. We are interested now in the composite
equivalence
\begin{equation}
  \label{eq:104.11}
  \bHom_!(\scrA,\scrM)\equeq\bHom_!(\scrAab,\scrM),\tag{11}
\end{equation}
defined under the only assumption that \scrM{} is additive and the
categories \scrA, $\scrAab$ and $\scrM\op$ are pseudotopoi
(without having to assume the existence of the abelianization functor
$\Wh_\scrA$). This equivalence is functorial for variable additive
pseudo-cotopos \scrM, when we take as ``maps'' $\scrM\to\scrM'$
functors which commute to small direct limits (a fortiori, these are
right exact and hence additive). In case $\scrAab$ itself is among
the eligible \scrM's, i.e., is a pseudo-cotopos (not only
pseudotopos), we may say that \emph{$\scrAab$ $2$-represents the
  $2$-functor $\scrM\mapsto\bHom_!(\scrA,\scrM)$} on the $2$-category
of all additive pseudo-cotopoi and functors between these commuting to
small direct limits. As we noticed above, the assumption just made
implies that $\Wh_\scrA$ exists. On the other hand, assuming merely
existence of $\Wh_\scrA$ (besides \scrA{} being a pseudo-topos), which
implies that $\scrAab$ is equally a pseudotopos as we say above,
it is readily checked that the equivalence \eqref{eq:104.11} can be
described as
\begin{equation}
  \label{eq:104.12}
  F\mapsto F\circ\Wh_\scrA : \bHom_!(\scrAab,\scrM) \toequ
  \bHom_!(\scrA,\scrM).\tag{12} 
\end{equation}
Thus, we get the
\begin{propositionnum}\label{prop:104.2}
  Let \scrA{} be a pseudotopos such that the abelianization functor
  \eqref{eq:104.10} exists \textup(\kern2pt for instance \scrA{} a
  topos\textup). Then $\scrAab$ is a pseudotopos and an additive
  category. Moreover, for any additive category \scrM{} which is a
  pseudo-cotopos, the functor \eqref{eq:104.12} is an equivalence of
  categories. 
\end{propositionnum}
\setcounter{corollarynum}{0}
\begin{corollarynum}\label{cor:104.prop2.1}
  Let\pspage{388} \scrA{} be a pseudotopos such that $\scrAab$ is
  a pseudotopos \textup(\kern2pt for instance, \scrA{} is a
  topos\textup). Then the abelianization functor $\Wh_\scrA$ exists,
  and this functor is $2$-universal for functors from \scrA{} into
  \scrU-categories which are both additive and are pseudotopoi
  \textup(maps between these being functors which commute with small
  direct limits\textup).
\end{corollarynum}

By duality, using \eqref{eq:104.6}, we can restate the equivalence
\eqref{eq:104.12} as
\begin{equation}
  \label{eq:104.13}
  F\mapsto F\circ\Wh\op : \bHom^!(\scrAab\op,\scrN) \toequ
  \bHom^!(\scrA\op,\scrN),\tag{13}
\end{equation}
valid provided \scrA{} is a pseudo-topos, $\Wh_\scrA$ exists, and
\scrN{} is an additive category which is moreover a pseudotopos.

Take for instance $\scrN=\Ab$, the category of abelian groups, i.e.,
\[\scrN=\scrB\subab, \quad\text{where $\scrB=\Sets$,}\]
then as already noticed above the left hand side of \eqref{eq:104.13}
is canonically equivalent with
$\bHom^!(\scrAab\op,\scrB)=\bHom^!(\scrAab\op,\Sets)$, as
$\scrAab\op$ is additive; on the other hand, by prop.\
\ref{prop:104.1} applied to the pseudotopos $\scrAab$ we get
\[\scrAab\toequ \bHom^!(\scrAab\op,\Sets) \quad\bigl(\fromequ
\bHom^!(\scrAab\op,\Ab)\bigr),\]
and hence an equivalence
\begin{equation}
  \label{eq:104.14}
  \scrAab \toequ\bHom^!(\scrA\op,\Ab), \quad
  F\mapsto\bigl(X\mapsto\Hom(X,F)\bigr),\tag{14}
\end{equation}
valid whenever \scrA{} is a pseudotopos such that $\Wh_\scrA$
exists. When \scrA{} is a topos, this corresponds to the familiar fact
that an abelian group object in the category of sheaves (of sets) on a
topos, can be interpreted equally as a sheaf on the topos with values
in the category \Ab{} of abelian groups.

\subsection{Interior and exterior operations
  \texorpdfstring{$\otimes_{\bZ}$ and $\Hom_{\bZ}$}{tensorZ and
    HomZ}.}\label{subsec:104.C}
In the first place, I want to emphasize the basic tensor product
operation
\begin{equation}
  \label{eq:104.15}
  (F,G)\mapsto F\otimes_{\bZ} G : \scrAab\times\scrAab \to \scrAab\tag{15}
\end{equation}
between abelian group objects of the pseudotopos \scrA, defined as
usual argumentwise as the solution of the universal problem, expressed
by the ``Cartan isomorphism''
\begin{equation}
  \label{eq:104.16}
  \Hom_\scrAab(F\otimes G, H) \simeq \Bil_{\bZ}(F,G ; H),\tag{16}
\end{equation}
where we dropped the subscript $\bZ$ in the tensor product, and where
$\Bil_{\bZ}$ or simply $\Bil$ denotes the set of maps $F\times G\to H$
which are ``biadditive'' in the usual sense of the word. Thus, the
existence of \eqref{eq:104.15}\pspage{389} just means, by definition,
that for any pair $(F,G)$ of objects in \scrAab, the functor in
$H$
\[H \mapsto \Bil(F,G;H)\]
is representable. It is clear that this functor commutes with small
inverse limits, hence by prop.\ \ref{prop:104.1} it is representable,
provided we know that \scrAab{} is a pseudo-cotopos (for instance,
when \scrA{} is a topos, in which case the existence of tensor
products is anyhow a familiar fact). The familiar Bourbaki
construction of a tensor product amounts on the other hand to viewing
$F\otimes G$ as a quotient of $\Wh_\scrA(F\times G)=\bZ^{(F\times G)}$ by
suitable ``relations'', i.e., as the cokernel of a map in
\scrAab
\[L_1\to L_0=\Wh_\scrA(F\times G),\]
where, as a matter fact, $L_1$ can be described as
\[L_1=\Wh_\scrA(F\times F\times G) \times \Wh_\scrA(F\times G\times
G).\]
Thus, if we know beforehand that cokernels exist in \scrAab{}
(which would indeed follow from \scrAab{} being a pseudotopos, but
may be checked more readily in terms of suitable exactness properties
of \scrA{} directly), plus the existence of course of $\Wh_\scrA$, the
tensor product functor \eqref{eq:104.15} exists. (On the other hand,
no use is made here of the assumption that \scrA{} be a pseudotopos.)

Let's assume the tensor product functor \eqref{eq:104.15} exists. Then
it is readily checked it is associative and commutative up to
canonical isomorphisms, giving rise to the usual compatibilities. If
moreover $\Wh_\scrA$ exists, we readily get the canonical isomorphism
\begin{equation}
  \label{eq:104.17}
  \Wh_\scrA(X\times Y)\eqdef\bZ^{(X\times Y)} \fromsim
  \Wh_\scrA(X)\otimes \Wh_\scrA(Y)\quad \bigl(\eqdef
  \bZ^{(X)}\otimes\bZ^{(Y)}\bigr),\tag{17} 
\end{equation}
compatible of course with the commutativity and associativity
isomorphisms for the operations $\times$ and $\otimes$. If on the
other hand \scrA{} admits moreover a final object (as it does if
\scrA{} is a pseudotopos and hence stable under small direct limits),
then
\begin{equation}
  \label{eq:104.18}
  \bZ_\scrA \eqdef \Wh_\scrA(e) = \bZ^{(e)}\tag{18}
\end{equation}
is a two-sided unit for the tensor product operation.

In what follows, we are interested in categories of the type
\begin{equation}
  \label{eq:104.19}
  \scrA^\scrM \eqdef\bHom_!(\scrA,\scrM), \quad\scrA_\scrN
  \eqdef\bHom^!(\scrA\op,\scrN)\tag{19} 
\end{equation}
where now \scrA{} is assumed to be a fixed pseudotopos, and \scrM{}
and \scrN{} are\pspage{390} \emph{additive} categories, \scrM{} being
moreover a pseudo-cotopos, \scrN{} a pseudotopos. We assume moreover
that $\Wh_A$ exists, and hence the categories \eqref{eq:104.19} can be
interpreted up to equivalence, via \eqref{eq:104.12} and
\eqref{eq:104.13}, as
\begin{equation}
  \label{eq:104.19prime}
  \bHom_!(\scrAab, \scrM), \quad \bHom^!(\scrAab\op,\scrN).\tag{19'}
\end{equation}
Let's remark that the dual of a category of one of the types
\eqref{eq:104.19} (or equivalently, \eqref{eq:104.19prime}) is
isomorphic to a category of the other type, more accurately, by
\eqref{eq:104.6} we get
\begin{equation}
  \label{eq:104.20}
  \bHom_!(\scrA,\scrM)\op\tosim\bHom^!(\scrA\op,\scrN),\tag{20}
\end{equation}
i.e., $(\scrA^\scrM)\op\tosim\scrA_\scrN$, with $\scrN=\scrM\op$. In
case \scrA{} is a topos, the objects of the second category
$\scrA_\scrN$ in \eqref{eq:104.19} (or equivalently, in
\eqref{eq:104.19prime}) can be interpreted as \emph{\scrN-valued
  sheaves} on the topos \scrA, whereas the object of the first,
$\scrA^\scrM$, may be called, correspondingly, \emph{cosheaves on
  \scrA{} with values in \scrM}. Thus, in virtue of \eqref{eq:104.20},
\scrM-valued cosheaves on \scrA{} can be interpreted as
$scrN=\scrM\op$-valued sheaves on \scrA, the corresponding categories
of cosheaves and sheaves being however \emph{dual} to each other. In
the next subsection \ref{subsec:105.D}, when $\scrA=\Ahat$, we'll
interpret moreover \scrM-valued cosheaves on \scrA{} (or on $A$, as
we'll call them equivalently) as \scrM-valued \emph{sheaves} on the
(topos associated to the) dual category $B=A\op$, and in this context
the difference between the categories of cosheaves and of sheaves
(which for the time being appear as categories \emph{dual} to each
other) will disappear altogether, provided we allow the ground topos
\scrA{} to change (from $\scrA=\Ahat$ to the ``dual'' topos
$\scrB=\Bhat$).

In terms of the expressions \eqref{eq:104.19prime} of the category of
``cosheaves'' and ``sheaves'' we are interested in, we want now to
define an external operation of the fixed category \scrAab{} on those
categories, using the tensor product operation \eqref{eq:104.15} in
\scrAab. What is needed visibly for this end is that for fixed $F$,
the functor
\begin{equation}
  \label{eq:104.21}
  G\mapsto F\otimes G\tag{21}
\end{equation}
from \scrAab{} to itself should commute with small direct limits --
hence composing it with a functor $L$ in $\bHom_!(\scrAab,\scrM)$ will
yield a functor in the same category, which will be the looked-for
external tensor product $F\oast L$, i.e.,
\begin{equation}
  \label{eq:104.22}
  \mathop{F\oast L}(G) = L(F\otimes G),\tag{22}
\end{equation}
which we may write more suggestively as
\begin{equation}
  \label{eq:104.22prime}
  G * (F \oast L) = (G \otimes F) * L,\tag{22'}
\end{equation}
with the notation
\begin{equation}
  \label{eq:104.23}
  H * L \eqdef L(H), \quad\text{for $H$ in \scrAab, $L$ in
    $\scrA^\scrM=\bHom_!(\scrA,\scrM)$,}\tag{23}
\end{equation}
which\pspage{391} will be convenient mainly in the context of the next
subsection (when \scrA{} is of the type \Ahat).

The exactness property needed for the functor \eqref{eq:104.21} is
equivalent with the property that for any object $H$ in \scrAab, the
functor
\[G\mapsto \Hom(F\otimes G,H) \simeq \Bil(F,G;H)\]
from $\scrAab\op$ to \Sets{} commute with small inverse limits. As
\scrAab{} is a pseudotopos (prop.\ \ref{prop:104.2}), this is
equivalent by prop.\ \ref{prop:104.1} with this functor being
representable. By definition of $\Bil$, we get
\[\Bil(F,G;H)\simeq\Hom_{\scrA\subab\uphat}(G, \bHom_{\bZ}(F,H)),\]
where the object
\begin{equation}
  \label{eq:104.24}
  \bHom_{\bZ}(F,H)\tag{24}
\end{equation}
is taken in the category of presheaves $\scrA\subab\uphat$ (cheating a
little with universes here\ldots). To sum up, the condition we want
for \eqref{eq:104.21} just boils down to the representability of the
abelian group objects \eqref{eq:104.24} in $\scrA\subab\uphat$, for
any two objects $F,G$ in \scrAab, i.e., essentially to the existence
of ``internal $\bHom$'s'' in the category \scrAab{} (endowed with the
tensor product $\otimes_{\bZ}$), satisfying the familiar Cartan
isomorphism formula
\begin{equation}
  \label{eq:104.25}
  \Hom(F\otimes G,H) \simeq\Hom(G,\bHom(F,H)).\tag{25}
\end{equation}

To sum up, what is needed for a nice formalism of interior and
exterior tensor products and $\Hom$'s for ``sheaves'' and
``presheaves'' on the pseudotopos \scrA, are the following assumptions
on \scrA:
\begin{enumerate}[label=\arabic*)]
\item\label{it:104.C.1}
  Tensor products and corresponding internal $\bHom$'s exist in
  \scrAab, and
\item\label{it:104.C.2}
  the abelianization functor $\Wh_\scrA$ \eqref{eq:104.15} exists,
\end{enumerate}
the latter assumption being needed in order to feel at ease with the
equivalence between the categories \eqref{eq:104.19} of sheaves and
cosheaves, and their ``abelianized'' interpretations
\eqref{eq:104.19prime}.

Under these assumptions, we define the exterior tensor product
operation of \scrAab{} upon a category of cosheaves $\scrA^\scrM$
\begin{equation}
  \label{eq:104.26}
  \scrAab\times\scrA^\scrM \to \scrA^\scrM, \quad
  (F,L)\mapsto F\oast L,\tag{26}
\end{equation}
by formula \eqref{eq:104.22} (which may be written also under the form
\eqref{eq:104.22prime}). This operation has the obvious associativity
property
\begin{equation}
  \label{eq:104.27}
  F\oast(G\oast L)\simeq (F\otimes G)\oast L,\tag{27}
\end{equation}
and moreover the unit $\bZ_\scrA$ for the internal tensor product in
\scrAab{} operators on $\scrA^\scrM$ as the identity functors,
i.e.,\pspage{392}
\begin{equation}
  \label{eq:104.28}
  \bZ_\scrA\oast L\simeq L.\tag{28}
\end{equation}

Using the tautological duality relation \eqref{eq:104.20} between
categories of cosheaves (with values in \scrM) and categories of
sheaves (with values in $\scrN=\scrM\op$), we deduce accordingly an
associative and unitary operation of \scrAab{} on any category of the
type $\scrA_\scrN$, namely \scrN-valued sheaves on \scrA. This
operation is most conveniently denoted by the $\bHom$ symbol
\begin{equation}
  \label{eq:104.29}
  (F,K)\mapsto \bHom(F,K): \scrAab\op\times\scrA_\scrN\to\scrA_\scrN,\tag{29}
\end{equation}
its explicit description in terms of $K$, viewed as a functor
\[K:\scrAab\op\to\scrN\]
is by
\begin{equation}
  \label{eq:104.30}
  \bHom(F,L)(G)=K(G\otimes F)\tag{30}
\end{equation}
for $K$ in $\scrA_\scrN$, $F$ and $G$ in \scrAab. This may be written
more suggestively as
\begin{equation}
  \label{eq:104.30prime}
  \Hom(G, \bHom(F,L)) \simeq \Hom(G\otimes F, K),\tag{30'}
\end{equation}
similar to \eqref{eq:104.22prime}, where we use the notation $\Hom$
(non-bold!)\scrcomment{In the typescript this parenthetical remark
  says ``non-underlined!'', as internal $\bHom$'s there are underlined
  rather than in boldface.} in analogy to \eqref{eq:104.23} for
denoting $K(H)$, namely
\begin{equation}
  \label{eq:104.31}
  \Hom(H,K)\eqdef K(H).\tag{31}
\end{equation}
If confusion is feared, we may put a subscript $\bZ$ in all $\Hom$'s
and $\bHom$'s just introduced, as well as in the internal and external
tensor product operations $\otimes$, $*$, $\oast$.

\begin{comments}
  Take for instance
  \[\scrN=\Ab,\quad\text{hence $\scrA_\scrN\simeq\scrAab$}\]
  by prop.\ \ref{prop:104.1} (compare \eqref{eq:104.14}). By this
  equivalence, the $\Hom$ in \eqref{eq:104.31} is just the usual
  $\Hom$ set corresponding to the category structure of \scrAab{} (the
  $\Hom$ endowed moreover with the structure of abelian group, coming
  from the fact that \scrAab{} is an additive category), whereas
  formula \eqref{eq:104.30prime} shows that the ``external'' $\bHom$
  in this case is nothing but the usual internal $\Hom$ as
  contemplated in \eqref{eq:104.24}. This I hope will convince the
  reader of the adequacy of the notation used (in case of exterior
  operation of abelian sheaves on \scrN-valued sheaves), and of the
  convention \eqref{eq:104.31}. We would like to give a similar
  justification for the notations \eqref{eq:104.22} \eqref{eq:104.23}
  used in connection with operation of abelian sheaves on \scrM-valued
  emph{co}sheaves, by interpreting this as the internal tensor product
  operation in \scrAab, for suitable choice of \scrN. This I am afraid
  cannot be done for\pspage{393} an arbitrary \scrA{} satisfying our
  assumptions, even when \scrA{} is a topos and even when it is of the
  special type \Ahat, as I do not know of any \scrM{} such that
  \begin{equation}
    \label{eq:104.star}
    \scrA^\scrM\simeq\scrAab.\tag{*}
  \end{equation}
  However, in case $\scrA=\Ahat$, introducing the dual category
  $B=A\op$ and $\scrB=\Bhat$, we get (see \ref{subsec:105.D} below)
  \begin{equation}
    \label{eq:104.starstar}
    \scrA^\scrM \equeq \scrB_\scrM \equeq
    \scrB\subab\quad\text{if}\quad \scrM=\Ab,\tag{**}
  \end{equation}
  hence we do get a canonical isomorphism \eqref{eq:104.star} provided
  $A=B$ say and hence $\scrA=\scrB$. Let's look at any rate at the
  simplest case, namely when $A$ is a final object in \Cat, hence
  \scrA{} can be identified with \Sets, and \scrAab{} with \Ab, the
  identification between the categories \Ab{} and $\bHom_!(\Ab,\Ab)$
  being obtained (we hope!) by associating to every object $L$ in \Ab,
  the functor
  \[F\mapsto L\otimes F: \Ab\to\Ab.\]
  This being so, the external operation \eqref{eq:104.22} of \Ab{} on
  $\bHom_!\equeq\Ab$ can be interpreted (using this identification) as
  the interior tensor product operation in \Ab. On the other hand, the
  operation $*$ of \eqref{eq:104.23} is equally interpreted as nothing
  but the tensor product in \Ab, which justifies the notation
  suggesting a tensor product. It would be nice checking corresponding
  compatibilities for a general object $A$ in \Cat{} satisfying
  $A=A\op$, namely a direct sum of one-object categories $A_i$ defined
  each in terms of a \emph{commutative} monoid $M_i$ -- I didn't work
  it out myself, sorry!

  It should be noted that the relationship between the two exterior
  $\Hom$'s in \eqref{eq:104.29} and \eqref{eq:104.31} is essentially
  the same as between the two exterior tensor-type operations
  \eqref{eq:104.26} and \eqref{eq:104.23}, $\Hom(F,K)$ designating an
  object in \scrN{} and $\bHom(F,K)$ an \scrN-valued sheaf on \scrA,
  just as $F*L$ designates an object in \scrM{} and $F\oast L$ an
  \scrM-valued cosheaf on \scrA; the graphical device of
  \emph{bold-facing}\scrcomment{in the typescript: \emph{underlining}}
  the symbol $\Hom$ (used for sheaves) corresponds to the device of
  \emph{circling} the symbol $*$ (used for cosheaves). With this
  luxury of explanations, I hope the notations introduced here are
  getting through\ldots
\end{comments}

\bigbreak

\presectionfill\ondate{12.8.}\pspage{394} and \ondate{13.8.}\par

% 105
\hangsection[Review (2): duality equivalences for ``algebraic'' topoi
and \dots]{Review \texorpdfstring{\textup{(2)}}{(2)}: duality
  equivalences for ``algebraic'' topoi and abelian topoi.}\label{sec:105}%
Let's go on with the overall review of abelianization.
\addtocounter{subsection}{3}

\subsection{Duality for topoi of the type
  \texorpdfstring{\Ahat}{Ahat}, and tentative generalizations.}
\label{subsec:105.D}
The main fact, it seems, which will give rise to duality statements
for topoi of the type \Ahat{} is the following, rather familiar one:
\addtocounter{propositionnum}{2}
\begin{propositionnum}\label{prop:105.3}
  Let $A$ be a small category, \scrM{} a \scrU-category stable under
  small direct limits,
  \begin{equation}
    \label{eq:105.32}
    \varepsilon_A:A\hookrightarrow\Ahat\tag{32}
  \end{equation}
  the canonical inclusion functor. Then the following functor is an
  equivalence of categories:
  \begin{equation}
    \label{eq:105.33}
    F\mapsto F\circ\varepsilon_A: \bHom_!(\Ahat,\scrM) \to \bHom(A,\scrM).\tag{33}
  \end{equation}
\end{propositionnum}

As I am at a loss to give a reference for this standard fact of
category theory, I'll give in guise of a proof the indication that a
quasi-inverse functor for \eqref{eq:105.33} is given by the familiar
construction
\begin{equation}
  \label{eq:105.34}
  i\mapsto i_! : \bHom(A,\scrM)\to\bHom_!(\Ahat,\scrM),\tag{34}
\end{equation}
where for any functor
\[i:A\to\scrM,\]
the functor
\[i_!:\Ahat\to\scrM\]
is defined by the formula
\begin{equation}
  \label{eq:105.35}
  i_!(F) = \varinjlim_{\text{$a$ in $A_{/F}$}} i(a).\tag{35}
\end{equation}
It is readily checked (and we have already used a number of times)
that this functor admits a right adjoint
\begin{equation}
  \label{eq:105.36}
  i^*:\scrM\to\Ahat, \quad i^*(x)=(a \mapsto \Hom(i(a),x)),\tag{36}
\end{equation}
and hence $i_!$ commutes to small direct limits, i.e., is in
$\bHom_!(\Ahat,\scrM)$, hence \eqref{eq:105.34}.
\begin{remark}
  If \scrM{} is a pseudotopos, then by prop.\ \ref{prop:105.3} above
  and by corollary \ref{cor:104.prop1.2} of prop.\ \ref{prop:104.1}
  (p.\ \ref{p:385}) the functor
  \begin{equation}
    \label{eq:105.37}
    i\mapsto i^*: \bHom(A,\scrM) \to \bHom^!(\scrM,\Ahat)\tag{37}
  \end{equation}
  (which in any case is fully faithful) is equally an equivalence of
  categories. 
\end{remark}

The\pspage{395} equivalence \eqref{eq:105.33} of prop.\
\ref{prop:105.3} can be interpreted by saying that the canonical
functor \eqref{eq:105.32} from $A$ to \Ahat{} is \emph{$2$-universal},
for functors of $A$ into \scrU-categories stable under small direct
limits, taking as ``maps'' between such categories functors which
commute to small direct limits. Thus, the \scrU-category \Ahat{} may
be viewed as ``the'' category deduced from $A$ by adding arbitrary
direct limits (disregarding the direct limits which may perchance
already exist in $A$\ldots).

Taking the duals of the two members of \eqref{eq:105.33}, we get an
equivalent statement of prop.\ \ref{prop:105.3}:
\setcounter{corollarynum}{0}
\begin{corollarynum}\label{cor:105.prop3.1}
  Let \scrN{} be a \scrU-category stable under small inverse
  limits. Then the functor
  \begin{equation}
    \label{eq:105.38}
    F\mapsto \varepsilon_A\op : \bHom^!(\Ahat\op,\scrN)\to \bHom(A\op,\scrN)\tag{38}
  \end{equation}
  is an equivalence of categories.
\end{corollarynum}

In terms of the topos
\[\scrA=\Ahat,\]
we may interpret the left-hand side of \eqref{eq:105.33} as the
category of \scrM-valued \emph{cosheaves} on this topos, which by
prop.\ \ref{prop:105.3} can be interpreted (up to equivalence) as the
category of functors from $A$ to \scrM. Dually, the left-hand side of
\eqref{eq:105.38} can be viewed as the category of \scrN-valued
\emph{sheaves} on the topos \scrA, which (via the right-hand side) can
be interpreted up to equivalence as the category of functors
$A\op\to\scrN$, i.e., as the category of \scrN-valued
\emph{presheaves} on $A$. As $A$ endowed with the coarsest
(``chaotic'') site structure is a generating site for the topos \scrA,
the equivalence \eqref{eq:105.38} may be viewed as a particular case
of the familiar fact, according to which (up to equivalence) the
category of \scrN-valued sheaves on a topos can be constructed in
terms of \scrN-valued sheaves on any \scrU-site defining this
topos. (When the site structure is chaotic, then those sheaves are
just arbitrary \scrN-valued presheaves.)

Assume now that the \scrU-category \scrM{} is stable under \emph{both}
types of small limits (direct and inverse). Then applying
\eqref{eq:105.33} for $(A,\scrM)$ and \eqref{eq:105.38} for
$(B,\scrM)$ where $B=A\op$, we get an equivalence
\begin{equation}
  \label{eq:105.39}
  \delta_A^\scrM:\bHom_!(\Ahat,\scrM) \toequ\bHom^!(\Bhat\op,\scrM),\tag{39}
\end{equation}
i.e.\ (as announced in yesterday's notes, p.\ \ref{p:390}), we get:
\begin{corollarynum}\label{cor:105.prop3.2}
  Let $A$ be a small category, $B=A\op$ the dual category, \scrM{} any
  \scrU-category stable under small direct and inverse limits. Then
  the category of \scrM-valued \emph{cosheaves} on the topos \Ahat{}
  is equivalent to\pspage{396} the category of \scrM-valued
  \emph{sheaves} on the topos \Bhat.
\end{corollarynum}

This equivalence, defined up to unique isomorphism, is deduced from
the diagram of canonical equivalences
\begin{equation}
  \label{eq:105.40}
  \begin{tabular}{@{}c@{}} % another non-portable diagram, sorry!
    \begin{tikzpicture}[commutative diagrams/every diagram,baseline=(O.base)]
      \node (A) at (-3.5cm,0) {\makebox[6ex][l]{$\bHom_!(\Ahat,\scrM)\eqdef(\Ahat)^\scrM$}};
      \node (B) at (3.5cm,0) {\makebox[6ex][r]{$(\Bhat)_\scrM\eqdef\bHom^!(\Bhat\op,\scrM)$}};
      \node (O) at (-1.2cm,-1.2cm) {$\bHom(A,\scrM)$};
      \node (P) at (1.2cm,-1.2cm) {$\bHom(B\op,\scrM)$};

      \path[commutative diagrams/.cd, every arrow, every label]
      (A) edge node {$\equ$} (O)
      (B) edge node[swap] {$\equ$} (P);
      \path[commutative diagrams/.cd, every arrow, every label, equal]
      (O) to (P);
    \end{tikzpicture},
  \end{tabular}\tag{40}
\end{equation}
and depends upon the choice of a quasi-inverse of the second vertical
equivalence in \eqref{eq:105.40} (which choice can be made, via the
dual of \eqref{eq:105.35}, via the choice of small inverse limits in
\scrM).

\begin{remark}
  It is felt that the way we got the equivalence \eqref{eq:105.39} via
  \eqref{eq:105.40}, the role of $A$ and $B$ in it is symmetric. To
  give a more precise statement, consider the equivalence deduced from
  \eqref{eq:105.39} by passing to the dual categories of the two
  members -- using the tautological isomorphisms \eqref{eq:104.6} of
  page \ref{p:385}, we get an equivalence
  \begin{equation}
    \label{eq:105.39prime}
    (\delta_A^\scrM)' : \bHom^!((\Ahat)\op,\scrN) \toequ
    \bHom_!(\Bhat,\scrN), \quad \text{where
      $\scrN=\scrM\op$;}\tag{39'}
  \end{equation}
  this equivalence is \emph{canonically quasi-inverse to the
    equivalence $\delta_B^\scrN$ in opposite direction}, associated to
  the pair $(B,\scrN=\scrM\op)$ instead of $(A,\scrM)$.
\end{remark}

In the rest of this subsection \ref{subsec:105.D}, we'll elaborate on
some particular cases of the equivalence \eqref{eq:105.39} between
cosheaves and sheaves.

\textbf{Case \namedlabel{case:105.1}{1\textsuperscript{\b o})}.}%
\enspace Assume $\scrM=\Sets$,
then by prop.\ \ref{prop:104.1} (p.\ \ref{p:384}) the right-hand side
of \eqref{eq:105.39} is canonically equivalent to \Bhat{} itself,
hence we get an equivalence
\begin{equation}
  \label{eq:105.41}
  \Bhat \toequ \bHom_!(\Ahat,\Sets).\tag{41}
\end{equation}
If we want to keep track of the symmetry aspect described in the
remark above, we may consider the functor \eqref{eq:105.41} as being
deduced from a canonical ``pairing'' between the categories \Ahat{}
and \Bhat
\begin{equation}
  \label{eq:105.42}
  \delta_A : \Ahat\times\Bhat\to\Sets,\tag{42}
\end{equation}
which is an object in
\[\bHom_{!!}(\Ahat,\Bhat; \Sets),\]
i.e., which commutes to small direct limits in each variable. (For
this interpretation, compare the dual statement contained in formula
\eqref{eq:104.7} of page \ref{p:385} -- and note that
\eqref{eq:105.41}, being an equivalence, commutes to small direct
limits, i.e., is in a category $\bHom_!(\Bhat, \bHom_!(\Ahat,\Sets))$.)
This\pspage{397} pairing gives rise, in a symmetric way, to the
functor \eqref{eq:105.41} (which is an equivalence) \emph{and} to a
functor
\begin{equation}
  \label{eq:105.41prime}
  \Ahat\toequ \bHom_!(\Bhat,\Sets)\tag{41'}
\end{equation}
which (it turns out) is none other (up to canonical isomorphism) than
\eqref{eq:105.41} with $B$ replaced by $A$ (and hence $A$ replaced by
$B$), and therefore is equally and equivalence. \emph{Thus, the
  pairing \eqref{eq:105.42} between the two topoi \Ahat{} and \Bhat{}
  has the remarkable property that it defines an equivalence of each
  of these topoi with the category of \textup(set-valued\textup)
  cosheaves on the other.} I do not know of any other example of a
pair of topoi related in such a remarkable way, which we may express
by saying that the two topoi are ``\emph{dual}'' to each other.

We still have to give an explicit expression for the pairing
\eqref{eq:105.42}, plus a convenient notation. I'll write
\begin{equation}
  \label{eq:105.43}
  \delta_A(F,G) \eqdef F * G \quad \text{for $F$ in \Ahat, $G$ in \Bhat,}\tag{43}
\end{equation}
and I'll use the canonical equivalence (valid for any pair of small
categories $A,B$ -- not necessarily dual to each other -- and any
\scrU-category stable under direct limits), deduced by twofold
application of prop.\ \ref{prop:105.3}, plus the tautological
isomorphism similar to \eqref{eq:104.7} p.\ \ref{p:385}:
\begin{equation}
  \label{eq:105.44}
  \bHom_{!!}(\Ahat,\Bhat;\scrM) \toequ \bHom(A\times B,\scrM), \quad
  F\mapsto F\circ(\varepsilon_A\times\varepsilon_B),\tag{44}
\end{equation}
which shows that \eqref{eq:105.42} is known up to canonical
isomorphism when we know its restriction to the full subcategory
$A\times B=A\times A\op$, identifying as usual an object $a$ of $A$ to
its image in \Ahat, and similarly for $B$. If $b$ is an object of $A$,
we'll denote by $b\op$ the same object viewed as an object of
$A\op=B$. With these conventions (including \eqref{eq:105.43}) we get
the nice formula
\begin{equation}
  \label{eq:105.45}
  a * b\op = \Hom_A(b,a) \quad(\; = \Hom_B(a\op,b\op)), \quad
  \text{for $a,b$ in $A$,}\tag{45}
\end{equation}
which has the required symmetry property -- which, for general objects
$F$ in \Ahat{} and $G$ in \Bhat, can be stated as a bifunctorial
isomorphism
\begin{equation}
  \label{eq:105.46}
  F * G \simeq G * F,\tag{46}
\end{equation}
where the operation $*$ in the first member refers to the pair $(A,B)$,
and in the second to the pair $(B,A)$.

From \eqref{eq:105.45} we easily deduce the more general formula for
$F*G$, when either $F$ or $G$ is in $A$ resp.\ $B$, namely\pspage{398}
\begin{equation}
  \label{eq:105.47}
  a * G \simeq G(a), \quad F * b\op \simeq F(b).\tag{47}
\end{equation}
\begin{remarks}
  We are mainly interested here in abelianization and (commutative)
  homology and cohomology, and hence in sheaves and cosheaves with
  values in additive (even abelian) categories, we are not going to
  use for the time being the relationship between \Ahat{} and \Bhat{}
  just touched upon. We could elaborate a great deal more on it, for
  instance introducing a canonical pairing (more accurately, a
  bi-sheaf) with opposite variance to \eqref{eq:105.42}
  \begin{equation}
    \label{eq:105.48}
    \Ahat\op\times\Bhat\op\to\Sets\tag{48}
  \end{equation}
  (or what amounts to the same, canonical functors adjoint to each
  other
  \[\Ahat\op\to\Bhat,\quad\Bhat\op\to\Ahat\quad),\]
  deduced (via the equivalence dual to \eqref{eq:105.44}
  \[\bHom^{!!}(\Ahat\op,\Bhat\op;\scrN) \toequ \bHom(A\op\times
  B\op,\scrN)\quad)\]
  from the co-pairing
  \[A\op\times B\op=B\times A\to\Sets\]
  given by
  \[(b\op,a)\mapsto\Hom(b,a)\quad(\text{for $a,b$ in $A$}).\]
  The two pairings \eqref{eq:105.42} and \eqref{eq:105.48} can be
  given a common interpretation as $\Hom$-sets in a suitable category
  \[\scrE=\scrE(A),\]
  which is the union of the two full subcategories \Ahat{} and
  $\Bhat\op$ (which may be interpreted as deduced from $A$ by
  adjoining respectively small direct and small inverse limits to it)
  intersecting in the common subcategory $A$, the $\Hom$-sets in the
  two directions between an object $F$ of \Ahat{} and an object $G\op$
  of $\Bhat\op$ (corresponding to an object $G$ of \Bhat) being given
  respectively by the pairings \eqref{eq:105.48} and
  \eqref{eq:105.42}.\footnote{\scrcommentinline{unreadable footnote}}
  The full relationship between these pairings is most conveniently
  expressed, it seems, by the composition law of maps in $\scrE(A)$,
  and associativity for this law. The symmetry of the situation with
  respect to the pair $(A,B)$ is expressed by the canonical
  isomorphism of categories
  \[\scrE(A)\op\simeq\scrE(B)\quad (\text{where $B=A\op$}).\]
\end{remarks}

\textbf{Case \namedlabel{case:105.2}{2\textsuperscript{\b o})}.}%
\enspace Of direct relevance for the abelianization story
is the particular case of the equivalence \eqref{eq:105.39}, obtained
by taking
\[\scrM=\Ab.\]
Using formula \eqref{eq:104.12} (page \ref{p:387}) for the pair
$(\Ahat,\Ab)$ and formula \eqref{eq:104.14} for \Bhat{} in guise of
\scrA, we get the canonical equivalence\pspage{399}
\begin{equation}
  \label{eq:105.49}
  \Bhatab \toequ \bHom_!(\Ahatab,\Ab),\tag{49}
\end{equation}
which should be viewed as the ``abelian'' analogon of the equivalence
\eqref{eq:105.41} above (corresponding to the case
$\scrM=\Sets$). This equivalence again may be viewed as described (in
analogy to \eqref{eq:105.45}) by a canonical pairing
\begin{equation}
  \label{eq:105.50}
  \Ahatab\times\Bhatab\to\Ab,\tag{50}
\end{equation}
which commutes to small direct limits in each variable, i.e., can be
viewed as an object in the category of ``abelian bi-cosheaves''
\begin{equation}
  \label{eq:105.star}
  \bHom_{!!}(\Ahatab,\Bhatab;\Ab),\tag{*}
\end{equation}
and gives rise simultaneously to the equivalence \eqref{eq:105.49},
and to the symmetric equivalence
\begin{equation}
  \label{eq:105.49prime}
  \Ahatab\toequ\bHom_!(\Bhatab,\Ab)\tag{49'}
\end{equation}
of the category of abelian sheaves on \Ahat{} with the category of
abelian cosheaves on \Bhat{} (which is just \eqref{eq:105.49} with $A$
replaced by $B$, up to canonical isomorphism at any rate).

Using the equivalence
\begin{equation}
  \label{eq:105.51}
  \bHom_{!!}(\Ahatab,\Bhatab;\Ab) \toequ \bHom(A\times B,\Ab)\tag{51}
\end{equation}
(which is a particular case of the evident abelian analogon of the
equivalence \eqref{eq:105.44}), we see that the pairing
\eqref{eq:105.50} is described, up to canonical isomorphism, by its
composition with
\[(a,b)\mapsto (\Wh_\Ahat(a),\Wh_\Bhat(b)): A\times B\to
\Ahatab\times\Bhatab,\]
and the latter, as is readily checked, is given by
\begin{equation}
  \label{eq:105.52}
  \Wh_A(a) *_{\bZ} \Wh_B(b\op) \simeq \bZ^{(\Hom(b,a))} \quad\text{for
    $a,b$ in $A$,}\tag{52}
\end{equation}
where we have written $\Wh_A$ instead of $\Wh_\Ahat$ for brevity and
accordingly for $B$, and where the pairing \eqref{eq:105.50} is
denoted by the symbol $*_{\bZ}$, in analogy with the notation $*$ in
\eqref{eq:105.43}, the index $\bZ$ being added in order to avoid
confusion with the non-abelian case \eqref{eq:105.43} (and the index
being dropped when no such confusion is to be feared). The formula
\eqref{eq:105.52} can be written, with different notations
\begin{equation}
  \label{eq:105.52prime}
  \bZ^{(a)} *_{\bZ} \bZ^{(b\op)} \simeq \bZ^{(\Hom(b,a))}
  \quad\text{for $a,b$ in $A$.}\tag{52'}
\end{equation}
Comparing with the similar formula \eqref{eq:105.45}, this suggests
the generalization
\begin{equation}
  \label{eq:105.53}
  \Wh_A(F) *_{\bZ} \Wh_B(G) \simeq \bZ^{(F*G)},\tag{53}
\end{equation}
or with the exponential notation
\begin{equation}
  \label{eq:105.53prime}
  \bZ^{(F)} *_{\bZ} \bZ^{(G)} \simeq \bZ^{(F*G)},\tag{53'}
\end{equation}
valid\pspage{400} for $F$ in \Ahat{} and $G$ in \Bhat. As both members
of \eqref{eq:105.53} commute with small direct limits in each
variable, the formula \eqref{eq:105.53} follows from the particular
case \eqref{eq:105.52}, in view of the equivalence of categories
\eqref{eq:105.51}.

\medbreak

\noindent\textbf{Remarks.}\enspace\namedlabel{rem:105.1}{1})%
\enspace In order to appreciate the significance of the pairing
\eqref{eq:105.50}, we may forget altogether about the non-additive
categories \Ahat{} and \Bhat, and view \eqref{eq:105.50} as a
remarkable ``duality'' relationship between two additive
\scrU-categories, stable under small direct limits, say \scrP{} and
\scrQ, endowed with a ``pairing''
\[\scrP\times\scrQ\to\Ab\quad\text{in
  $\bHom_{!!}(\scrP,\scrQ;\Ab)$,}\]
giving rise to two functors which are \emph{equivalences}
\[ \scrQ\toequ \bHom_!(\scrP,\Ab),\quad \scrP\toequ
\bHom_!(\scrQ,\Ab),\]
identifying each of \scrP, \scrQ{} to the category of ``abelian
cosheaves'' on the other. In the particular case \eqref{eq:105.50},
$\scrP=\Ahatab$ and $\scrQ=\Bhatab$ with $B=A\op$, each of these
categories is even an ``abelian topos'' by which I mean an abelian
category \scrP{} stable under small filtering direct limits, with the
latter being \emph{exact}, and moreover \scrP{} admitting a small
generating subcategory. (These categories are sometimes called,
somewhat misleadingly, ``Grothendieck categories''. Of course, an
``abelian topos'' is by no means a category which is a topos, besides
being abelian!) There are many other examples of dual pairs of abelian
topoi. One evident generalization is by taking
\[\scrP=A_k\uphat,\quad\scrQ=B_k\uphat,\quad\text{where again
  $B=A\op$,}\]
where $k$ is any commutative ring, and $A_k\uphat\eqdef A_\kMod\uphat$
is the category of presheaves of $k$-modules on $A$, or equivalently,
of objects in \Ahat{} endowed with a structure of a $k$-module -- and
accordingly for the notation $B_k\uphat$. Indeed, the generalities
\ref{subsec:104.B} and \ref{subsec:104.C} in yesterday's notes about
abelian sheaves and cosheaves, as well as today's, could be developed
replacing throughout abelian group objects and additive categories by
$k$-module objects and $k$-linear categories. In case $k$ is not
supposed commutative, one still should get a duality pairing
\[A_{k\op}\uphat\times B_k\uphat\to\Ab,\]
where $k\op$ denotes the ring opposite to $k$ (i.e., a duality pairing
between presheaves on $A$ of right $k$-modules, and copresheaves on
$A$ of left $k$-modules), given in terms of $*_{\bZ}$ in
\eqref{eq:105.50} by the formula\pspage{401}
\[M *_{k} N = (M*_{\bZ} N)^\natural,\]
where in the right-hand side $P=M*_{\bZ}N$ is viewed as a
bi-$k$-module via the right and left $k$-module structures on $M$ and
$N$ respectively and bifunctoriality of $*_{\bZ}$, and where for any
bimodule $P$, we write
\[P^\natural \eqdef P /
\parbox[t]{0.65\textwidth}{sub-$\bZ$-module generated by elements of
  the type $s.x-x.s$ for $x$ in $P$ and $s$ in
  $k$.}\]
I confess I didn't do the checking that this does give rise indeed to
a duality pairing as desired. When $A=B=$ final category, then the
pairing above is just the pairing given by usual tensor product
\[(M,N)\mapsto M\otimes_k N\]
between right and left $k$-modules, which is immediately checked to be
dualizing indeed. More generally, posing
\[\scrP=(k\op\textup{-Mod}),\quad Q=\kMod,\]
it is immediately checked that for any additive category \scrM{}
stable under small direct limits, we get a canonical
equivalence\scrcomment{Clearly, the ``d'' in $k\subd$ is for
  ``dexter'' (on the right), and the ``s'' below is for ``sinister''
  (on the left).}
\[\bHom_!(\scrP,\scrM) \toequ (k\textup{-}\scrM), \quad
F\mapsto F(k\subd),\]
where $(k\textup{-}\scrM)$ denotes the category of objects $L$ of
\scrM{} endowed with a structure of a ``left $k$-module in \scrM'',
i.e., a ring homomorphism $k\to\End_\scrM(L)$, $k\subd$ denotes $k$
viewed as a right $k$-module, and $F(k\subd)$ is viewed as an object
of $(k\textup{-}\scrM)$ via the operations of $k$ on it coming from
left multiplication of $k$ upon $k\subd$. A quasi-inverse equivalence
is obtained by associating to an object $L$ in $(k\textup{-}\scrM)$
the functor
\[M\mapsto M\otimes_k L: (k\op\textup{-Mod})\to\scrM.\]
Dually, we get an equivalence (if \scrN{} stable under small inv.\
limits)
\[\bHom^!(\scrQ,\scrN) \toequ (k\textup{-}\scrN), \quad
F\mapsto F(k\subs),\]
where $k\subs$ denotes $k$ viewed as a left $k$-module, so that the
\emph{contravariant} functor $F$ transforms the endomorphisms of
$k\subs$ (obtained by \emph{right} operation of $k$ on $k\subs$ via
right multiplication) into a \emph{left} operation of $k$ on
$F(k\subs)$; the quasi-inverse is given by the familiar $\Hom_k$
operation, it associates to the left $k$-module $L$ in \scrM{} the
functor
\[M\mapsto \Hom_k(M,L): \kMod\to\scrN.\]

Comparing the two pairs of equivalences, we get the
equivalence\pspage{402}
\begin{equation}
  \label{eq:105.starbis}
  \bHom_!(scrP,\scrM) \equeq \bHom^!(\scrQ\op,\scrM),\tag{*}
\end{equation}
valid when \scrM{} is stable under both (small) direct and inverse
limits, and which should be viewed as an abelian analogon of the
equivalence \eqref{eq:105.39}.

\namedlabel{rem:105.2}{2})%
\enspace It is well-known that an abelian topos \scrP{} is
equivalent to a category \kMod, for a suitable ring $k$ (not
necessarily commutative) if{f} it admits an object $L$ which is
\namedlabel{it:105.rem2.a}{a)}\enspace generating, and
\namedlabel{it:105.rem2.b}{b)}\enspace ``\emph{ultraprojective}'',
i.e., the functor $X\mapsto\Hom(L,X)$ commutes to small direct
limits. The condition \ref{it:105.rem2.b} (for an object of an
\emph{abelian} category stable under small direct limits or
equivalently, under small direct sums) is equivalent with $L$ being:
\namedlabel{it:105.rem2.b1}{b\textsubscript{1})}\enspace projective,
and \namedlabel{it:105.rem2.b2}{b\textsubscript{2})}\enspace of
``\emph{finite presentation}'', i.e., the functor $X\mapsto\Hom(L,X)$
commutes with small filtering direct limits. This observation suggests
one common feature of all the examples of abelian duality pairings
considered so far, namely that the abelian topoi under consideration
in the pairing \emph{have a small set of ultraprojective
  generators}. I don't know if a structure theory of such categories
(which are the abelian analogons for topoi equivalent to topoi of the
special type \Ahat, with $A$ in \Cat) has been worked out yet. I
didn't do it at any rate -- but the natural thing to expect is that
these abelian topoi \scrP{} (which we may call ``\emph{algebraic}''
ones, just as an ordinary topos equivalent to one of the type \Ahat{}
may be called ``algebraic'', which equally means that the set of
ultraprojective objects in it is generating\ldots) are exactly those
equivalent to a category of the type
\[\Homadd(P\op,\Ab),\]
where $P$ is any \emph{small additive} category, and where $\Homadd$
denotes the category of additive functors from one additive category
to another. Instead of assuming $P$ small, we may as well take $P$
merely ``essentially small'', i.e., equivalent to a small category,
with the benefit that for a given \scrP, there is a canonical choice
of an additive category $P$ together with an equivalence
\[\scrP\toequ\Homadd(P\op,\Ab),\]
namely by taking
\[P =
\begin{tabular}[t]{@{}l@{}}
  full subcategory of \scrP{} made up with all
  ultraprojective\\objects in \scrP.
\end{tabular}\]
As we saw earlier, in case $\scrP=\Ahatab$, $P$ is nothing but the
abelian Karoubi envelope of the category $A$, namely the Karoubi
envelope of the additive category $\Add(A)$ (cf.\ sections
\ref{sec:93} and \ref{sec:99}). Another choice for $P$ in this case
would be just $\Add(A)$ itself, whose objects are more amenable to
computations.\pspage{403}

Associating to any small \scrcommentinline{additive} category $P$ the
algebraic abelian topos $\bHomadd(P\op,\Ab)$ should be viewed of
course as the abelian analogon of $A\mapsto\Ahat$, associating to a
small category $A$ the corresponding algebraic topos. It merits a
notation of its own, say
\[P\supamp\eqdef \bHomadd(P\op,\Ab),\]
and as in the non-additive case, we get a canonical inclusion functor
\[\varepsilon_P:P\to P\supamp\]
which is additive. (Its composition with the canonical functor
$P\supamp\to P\uphat$ is the canonical inclusion
previously denoted by $\varepsilon_P$ too from $P$ to $P\uphat$.) Next
thing we'll expect, in analogy to prop.\ \ref{prop:105.3}, is that for
any additive category \scrM{} stable under small direct limits, the
following canonical functor is an equivalence of categories:
\begin{equation}
  \label{eq:105.starstar}
  \bHom_!(P\supamp,\scrM) \toequ \bHomadd(P,\scrM), \quad
  F\mapsto F\circ\varepsilon_P.\tag{**}
\end{equation}
The proof, via construction of a quasi-inverse functor, should be
about the same as for prop.\ \ref{prop:105.3}, which should go through
once we get the abelian analogon of the well-known fact in \Ahat, that
any object $F$ in \Ahat{} can be recovered as a direct limit in
\Ahat{} of objects of $A$, according to $A_{/F}$ as an indexing
category -- which makes us expect that we get too:
\[F \fromsim \varinjlim_{\text{$a$ in $P\supamp_{/F}$}} a\quad
(\text{direct limit in $P\supamp$}).\]
From \eqref{eq:105.starstar} we get as in cor.\ \ref{cor:105.prop3.1},
passing to the dual categories, the dual equivalence
\[\bHom^!({P\supamp}\op,\scrN) \toequ \bHomadd(P\op,\scrN),\]
valid if \scrN{} is an additive category stable under small inverse
limits. Hence, if \scrM{} is additive and stable under both types of
limits, the equivalence
\begin{equation}
  \label{eq:105.starstarstar}
  \bHom_!(P\supamp,\scrM)\toequ \bHom^!({Q\supamp}\op,\scrM), \quad
  \text{with $Q=P\op$,}\tag{***}
\end{equation}
between abelian cosheaves on $P\supamp$ and abelian sheaves in
$Q\supamp$, with values in the same additive category (in analogy to
\eqref{eq:105.39}). In the particular case $\scrM=\Ab$, this then
gives rise to the equivalence
\[Q\supamp \equeq \bHom_!(P\supamp,\Ab)\]
and to the corresponding pairing
\[P\supamp\times Q\supamp\to\Ab\]
which is a duality, namely induces an equivalence between each abelian
topos\pspage{404} $P\supamp$, $Q\supamp$ and the category of abelian
cosheaves on the other.

We may call an abelian topos ``\emph{reflexive}'' if it can be
inserted in a pair $(\scrP,\scrQ)$ of dually paired abelian topoi --
where \scrQ, or the ``dual'' of \scrP, is defined up to equivalence in
terms of \scrP{} as $\bHom_!(\scrP,\Ab)$, the category of cosheaves on
\scrP{} with values in \Ab. Thus, it seems that a sufficient condition
for reflexivity is ``algebraicity'' of \scrP, namely the existence of
a small generating family made up with ultraprojective
objects. (NB\enspace In the non-abelian case, it is well-known that a
topos \scrA{} is ``algebraic'', i.e., equivalent to a topos \Ahat,
if{f} it admits such a generating family -- and as we saw in
\ref{case:105.1}, as a consequence of \eqref{eq:105.39} for
$\scrM=\Sets$, such a topos is indeed ``reflexive''.) I wouldn't be
too surprised if this sufficient condition for reflexivity turned out
to be necessary too, at any rate if we want a property stronger still
than reflexivity, namely validity of a duality equivalence
\eqref{eq:105.starstarstar} for sheaves and cosheaves with values in
an arbitrary additive \scrU-category stable under small direct and
inverse limits, satisfying (for varying \scrM) suitable compatibility
assumptions.

\namedlabel{rem:105.3}{3})%
\enspace With respect to this duality equivalence
\eqref{eq:105.starstarstar}, I am a little unhappy still, as I do not
see how to get (for a general dual pair \scrP, \scrQ{} of abelian
topoi) a functor i one direction or the other between the two
categories
\[\bHom_!(\scrP,\scrM),\quad\bHom^!(\scrQ\op,\scrM),\]
in terms of just the duality pairing. The same perplexity holds in the
non-abelian case. This is one of the reasons that make me feel that I
haven't yet a thorough understanding of the duality formalism I am
developing here, except in the ``algebraic'' case (granting for the
latter that the tentative theory just outlined for algebraic
\emph{abelian} topoi is indeed correct).

\namedlabel{rem:105.4}{4})%
\enspace To finish with the comments on the (pre-homological)
duality formalism for algebraic topoi and algebraic abelian topoi, I
still would like to add that the category $\scrE(A)$ (union of \Ahat{}
and $\Bhat\op$, with $B=A\op$) introduced in \ref{case:105.1} (cf.\
remark on page \ref{p:398}) admits also an abelian analogon. In the
non-abelian case still, the simplest way to construct the category
$\scrE(A)$, is via an equivalent category canonically embedded in the
category $\bHom(\Bhat,\Sets)$ as a strictly full subcategory
(``strict'' referring to the fact that with any object it contains all
isomorphic ones), namely the union $\overline{\scrE}(A)$ of the
(strictly full) subcategories\pspage{405} $\bHom_!(\Bhat,\Sets)$
(equivalent to \Ahat) and the subcategory of representable functors
(equivalent to $\Bhat\op$).\scrcomment{There seems to be another
  unreadable footnote here\ldots} The intersection of these two categories
contains of course $A$ (embedded in $\bHom(\Bhat,\Sets)$ by
associating to $a$ in $A$ the functor $G\mapsto G(a)$ from
$\Bhat=\bHom(A,\Sets)$ to \Sets), but in general need not be quite
equivalent to $A$ -- it turns out to be the ``Karoubi envelope'' of
$A$, obtained by adjoining to $A$ formally images (or equivalently,
coimages) of projectors in $A$. The more immediate interpretation of
this intersection, is that it is equivalent to the dual category of
the category of ultraprojective objects in \Bhat{} (and the latter can
be viewed as $\Kar(B)$, but formation of the Karoubi envelope up to
equivalence commutes to taking dual categories\ldots). All these
constructions immediately extend to the abelian set-up, starting with
a small \emph{additive} category $P$, instead of $A$.

After this endless procession of remarks, which are really digressions
for what we're after (namely abelianization and duality in the context
of small categories as homotopy models), it is time to resume our main
line of thought in this subsection, namely looking at interesting
particular cases for the general duality relation \eqref{eq:105.39}.

\textbf{Case \namedlabel{case:105.3}{3\textsuperscript{\b o})}.}%
\enspace This is the case when \scrM{} is an \emph{additive} category,
stable under both types of small limits. If we assume moreover that
\scrM{} and $\scrM\op$ are both pseudotopoi, using the equivalences
\eqref{eq:104.12} and \eqref{eq:104.13} (p.\ \ref{p:387}),
\eqref{eq:105.39} may be interpreted as an equivalence
\begin{equation}
  \label{eq:105.54}
  \bHom_!(\Ahatab,\scrM)\equeq \bHom^!(\Bhatab,\scrM),\tag{54}
\end{equation}
interpreting \scrM-valued cosheaves on the abelian topos \Ahatab{} in
terms of \scrM-valued sheaves on the dual abelian topos \Bhatab, as
anticipated in a more general situation in the remark above (cf.\
formula \eqref{eq:105.starstarstar} on page \ref{p:403}). There,
however, the assumption that \scrM{} and/or $\scrM\op$ should be
pseudotopoi didn't seem to come in at all, so we expect this condition
to be irrelevant indeed. This will of course follow, if the same holds
for \eqref{eq:104.12} (hence by duality for \eqref{eq:104.13}), namely
that the canonical functor
\begin{equation}
  \label{eq:105.55}
  \bHom_!(\Ahatab,\scrM)\toequ\bHom_!(\Ahat,\scrM), \quad
  F\mapsto F\circ\Wh_A,\tag{55}
\end{equation}
is an equivalence, under the only assumption that the additive
category \scrM{} is stable under small direct limits (without assuming
that \scrM{} be a pseudotopos). The line of thought of the remark
\ref{rem:105.2} above suggests a way for proving this, via an
equivalence\pspage{406}
\begin{equation}
  \label{eq:105.56}
  \bHom_!(\Ahatab,\scrM)\toequ\bHomadd(\Add(A),\scrM), \quad
  F\mapsto F\circ j_A,\tag{56}
\end{equation}
where
\[j_A: \Add(A)\to\Ahatab\]
is the canonical inclusion functor (cf.\ section \ref{sec:97}). This,
and the dual equivalence (deduced from \eqref{eq:105.55}, taking
$\scrN=\scrM\op$)
\begin{equation}
  \label{eq:105.57}
  \bHom^!(\Ahatab\op,\scrN)
  \toequ\bHomadd(\Add(A)\op,\scrN),\tag{57}
\end{equation}
valid for any additive category stable under small inverse limits,
will immediately imply an equivalence \eqref{eq:105.54} by a direct
argument as in the remark above, without passing through the
non-abelian case \eqref{eq:105.39}. At any rate, \eqref{eq:105.56}
implies that \eqref{eq:105.55} is an equivalence, as is seem by
looking at the commutative diagram
\begin{equation}
  \label{eq:105.58}
  \begin{tabular}{@{}c@{}}
    \begin{tikzcd}[baseline=(O.base),row sep=small,column sep=-1em]
      & \bHom_!(\Ahat,\scrM)\ar[dr] & \\
      \bHom_!(\Ahatab,\scrM)\ar[dr]\ar[ur] & &
      \bHom(A,\scrM) \\
      & |[alias=O]| \bHomadd(\Add(A),\scrM)\ar[ur] &
    \end{tikzcd},
  \end{tabular}\tag{58}
\end{equation}
where the two right-hand arrows are equivalences, which implies that
one of the two left-hand arrows is an equivalence if{f} the other is.

Thus, for getting \eqref{eq:105.54} and \eqref{eq:105.55} without
extraneous assumptions on \scrM, we are left with proving
\eqref{eq:105.56}. Now, writing
\[\scrP=\Ahatab,\quad P=\Add(A),\]
we do have indeed an equivalence
\[\scrP\equeq P\supamp \eqdef \bHomadd(P\op,\Ab),\]
as seen from \eqref{eq:105.57} taking $\scrN=\Ab$ (which satisfies the
extra assumptions). So we may as well prove \eqref{eq:105.56} in the
more general case when $P$ is any small additive category and \scrP{}
is defined as $P\supamp$, namely prove the equivalence
\eqref{eq:105.starstar} of page \ref{p:403} above, as I don't expect
the particular case at hand here to be any simpler. The suggestion for
a proof there there seems convincing, I guess I should check it works
during some in-between scratchwork\ldots

\bigbreak

\presectionfill\ondate{14.8.}\pspage{407} and \ondate{15.8}\par

% 106
\hangsection[Review (3): A formulaire for the basic integration and
\dots]{Review \texorpdfstring{\textup{(3)}}{(3)}: A formulaire for the
  basic integration and cointegration operations \texorpdfstring{$*$
    and $\Hom$}{* and Hom}.}\label{sec:106}%
\phantomsection\addcontentsline{toc}{subsection}{\numberline {E)}A
  formulaire around the basic operations \texorpdfstring{$*$ and
    $\Hom$}{* and Hom}.}%
\textbf{\namedlabel{subsec:106.E}{E)}\enspace A formulaire around the
  basic operations $*$ and $\Hom$.}\enspace I would like to dwell a
little more still on the duality formalism weaving around formula
\eqref{eq:105.39} (p.\ \ref{p:395}), stating that for two given small
categories $A$ and $B$ dual to each other
\[B=A\op, \quad A=B\op,\]
and any \scrU-category \scrM{} stable under both types of small
limits, \scrM-valued \emph{cosheaves} on \Ahat{} may be interpreted as
\scrM-valued \emph{sheaves} on the dual topos \Bhat. This
identification preserves variance (i.e., \eqref{eq:105.39} is an
\emph{equivalence} of categories, not an antiequivalence:
\begin{equation}
  \label{eq:106.59}
  (\Ahat)^\scrM = \bHom_!(\Ahat,\scrM) \toequ
  \bHom^!((\Bhat)\op,\scrM) = \BhatM\quad\text{).}\tag{59}
\end{equation}
It should not be confused with the tautological interpretation of
\scrM-valued cosheaves on \Ahat{} as $\scrM\op$-valued sheaves on the
\emph{same} topos, an identification \emph{reversing variance}, as
expressed by the canonical antiequivalence between the corresponding
categories -- an anti-isomorphism even (reflecting its tautological
nature):
\begin{equation}
  \label{eq:106.60}
  \bHom_!(\Ahat,\scrM)\op \tosim \bHom^!((\Ahat)\op,\scrM\op),
  \text{ i.e., }((\Ahat)^\scrM)\op \tosim \AhatM.\tag{60}
\end{equation}
In the latter formula, the basic topos \Ahat{} remains the same in
both sides, it is the category of values that changes from \scrM{} to
the dual one $\scrM\op$, whereas in formula \eqref{eq:106.59} =
\eqref{eq:105.39}, it is the opposite. In terms of the tautological
formula \eqref{eq:106.60} (a particular case of formula
\eqref{eq:104.6} p.\ \ref{p:385}), the not-so-tautological formula
relating cosheaves and sheaves can be reformulated as a formula in
terms of sheaves only (due to our preference for sheaves rather than
cosheaves\ldots):
\begin{equation}
  \label{eq:106.61}
  \bHom^!(\Ahat,\scrM)\op \equeq \bHom^!(\Bhat,\scrM\op),\quad
  \text{i.e.,}\quad (\AhatM)\op \equeq B\uphat_{\scrM\op},\tag{61} 
\end{equation}
namely \scrM-valued sheaves on the topos \Ahat{} can be interpreted as
sheaves on the dual topos with values in the dual category $\scrM\op$,
this interpretation \emph{reversing variances}. In the homology and
cohomology formalism which is to follow, due to habits of long
standing, I prefer systematically to take as coefficients
\emph{sheaves} rather than cosheaves -- hence rule out cosheaves in
favor of sheaves via \eqref{eq:106.60}. From this point of view the
relevant basic duality statement is \eqref{eq:106.61} rather than
\eqref{eq:106.59}.

On the \emph{cosheaves} side, yesterday's diagram \eqref{eq:105.58} of
equivalences gives a fourfold description of cosheaves on the topos
\Ahat{} with values in an \emph{abelian} category \scrM{} stable under
small direct limits. We could still enlarge this diagram, by including
in it a fifth category equivalent\pspage{408} to the four others,
namely
\[\bHomaddinf(\Addinf(A),\scrM),\]
the category of infinitely additive functors from the infinitely
additive envelope of $A$ into $M$ (cf.\ section \ref{sec:99} p.\
\ref{p:366} for description of the category $\Addinf(A)$). Rather than
writing down the larger diagram here, I'll write down the dual
enlarged one, for the dual topos \Bhat{} and for various expressions
of the category of \emph{sheaves} on this topos, with values in a
category \scrM{} stable this time under small inverse limits:
\begin{equation}
  \label{eq:106.62}
  \begin{tabular}{@{}c@{}}
    % again somewhat fragile
    \begin{tikzpicture}[commutative diagrams/every diagram,baseline=(E.base)]
      \node (A) at (0,1.2cm) {$\bHom^!((\Bhat)\op,\scrM)$};
      \node (B) at (-3.2cm,0) {$\bHom^!((\Bhatab)\op,\scrM)$};
      \node (C) at (2.2cm,0) {$\bHom(B\op,\scrM)$};
      \node (D) at (-3.2cm,-1.2cm) {$\bHommultinf(\Addinf(B)\op,\scrM)$};
      \node (E) at (1.5cm,-1.2cm) {$\bHomadd(\Add(B)\op,\scrM)$};

      \path[commutative diagrams/.cd, every arrow, every label]
      (A) edge node {$\equ$} (C)
      (B) edge node {$\equ$} (A)
      (B) edge node[swap] {$\equ$} (D)
      (D) edge node {$\equ$} (E)
      (E) edge node {$\equ$} (C);
    \end{tikzpicture},
  \end{tabular}\tag{62}
\end{equation}
where $\bHommultinf$ denotes the category of ``infinitely
multiplicative'' functors from one additive category stable under
infinite products to another. Recalling for the extreme right term of
\eqref{eq:106.62} that $B\op=A$, we see that this term is
\emph{identical} to the corresponding term in the diagram (even the
enlarged one) \eqref{eq:105.58} -- hence, if \scrM{} is stable under
both types of limits, the ten categories occurring altogether in the
two diagrams are mutually equivalent (as a matter of fact, there are
nine only which are mutually different). It may be noted that there is
still another pair of corresponding terms in the two diagrams for
which the equivalence between them may be viewed as tautological,
namely
\[\bHomadd(\Add(A),\scrM) \equeq \bHomadd(\Add(B)\op,\scrM),\]
due to the tautological equivalence of categories
\[\Add(B)\op \equeq \Add(B\op) = \Add(A).\]

As emphasized in yesterday's notes, the canonical pairing (deduced
from \eqref{eq:106.59} by taking $\scrM=\Ab$)
\begin{equation}
  \label{eq:106.63}
  \Ahatab\times\Bhatab\to\Ab,\quad
  (F,G)\mapsto F*_{\bZ} G,\tag{63}
\end{equation}
deserves special attention, giving rise to an equivalence between each
of the mutually dual abelian topoi \Ahatab, \Bhatab{} with the
category of abelian cosheaves on the other
\begin{equation}
  \label{eq:106.64}
  \Bhatab \toequ \bHom_!(\Ahatab,\Ab), \quad
  \Ahatab \toequ \bHom_!(\Bhatab,\Ab)\tag{64}
\end{equation}
(cf.\ \eqref{eq:105.49} and \eqref{eq:105.49prime} page
\ref{p:399}). It should be kept in mind that besides this duality
pairing between two abelian topoi, there is important
extra\pspage{409} structure in this abelianized duality context,
embodied by the \emph{tensor product structure} on both abelian topoi
\Ahatab{} and \Bhatab, as contemplated in section \ref{sec:104}
\ref{subsec:104.C}, in a somewhat more general context. Corresponding
to this extra structure on \Ahatab{} say, we saw that this category of
abelian sheaves ``operates'' covariantly (by an operation denoted by
$\oast_{\bZ}$ or simply $\oast$) on any category of \scrM-valued
cosheaves on \Ahatab, and contravariantly (by an operation denoted by
$\bHom_{\bZ}$ or simply $\bHom$, if no confusion may arise) on any
category of \scrN-valued sheaves on \Ahatab, where \scrM, \scrN{} are
additive categories, stable under small direct resp.\ inverse
limits. The latter operation (cf.\ page \ref{p:392})
\begin{equation}
  \label{eq:106.65}
  (L,K) \mapsto \bHom_{\bZ}(L,K) : (\Ahatab)\op \times \AhatN \to
  \AhatN,\tag{65}
\end{equation}
involving sheaves, will be used in the sequel ``tel
quel'',\scrcomment{``tel quel'' = ``as is''} its
definition is of a tautological character, independent of duality. As
for the former operation involving cosheaves, we may view it via the
duality relation \eqref{eq:106.59} as an operation of \Ahatab{} on
\scrM-valued \emph{sheaves} on the dual abelian topos \Bhatab, and
this operation will be denoted by the same symbol $\oast_{\bZ}$:
\begin{equation}
  \label{eq:106.66}
  (L,M') \mapsto L\oast_{\bZ} M' : \Ahatab \times \BhatM
  \to \BhatM.\tag{66} 
\end{equation}
Replacing $A$ by $B$ in \eqref{eq:106.66}, we get an operation of
\Bhatab{} upon \AhatM,
\begin{equation}
  \label{eq:106.67}
  (M,L') \mapsto M \oast_{\bZ} L' :
  \AhatM\times\Bhatab\to\AhatM.\tag{67}
\end{equation}
Whenever convenient, we'll allow ourselves to write $L'\oast M$
instead of $M\oast L'$ (which doesn't seem to lead to any trouble),
and will henceforth (unless special need should arise) drop the
subscripts $\bZ$.

Thus, for a given category $\AhatM$ of \scrM-valued sheaves on
\Ahatab, there is a twofold operation on this category, namely
\Ahatab{} itself operates (the operation defined by $L$ in \Ahatab{}
depending contravariantly on $L$) as well as the dual abelian topos
\Bhatab{} (the operation defined by $L'$ in \Bhatab{} depending
covariantly on $L'$), \scrM{} being any additive category stable under
small inverse and direct limits (in order to ensure existence of both
types of operations). I would like to dwell a little more on this
twofold structure, as I don't feel to have understood it thoroughly
yet. It is this second operation mainly which hasn't become really
familiar yet, still less its relationship to the first, more
familiar\pspage{410} operation is understood. I'll have to play around
a little more with it for being really at ease. It's worth the while,
as the $\bHom$ and corresponding $\Hom$ operation is the key operation
for expressing cohomology of \Ahat{} (with coefficients in $K$), where the
$\oast$ and corresponding $*$ operation is the key for expressing
homology of \Ahat{} (with coefficients in $M$, where $K$ and $M$ are
the sheaves occurring in \eqref{eq:106.65} and \eqref{eq:106.67}
respectively).

A typical special case is the one when $A$ is the final category,
hence $B=A$ and $\AhatM\simeq \scrM$, in which case
\eqref{eq:106.65} and \eqref{eq:106.67} are the two familiar exterior
operations of \Ab{} on any additive category stable under the two
types of small limits
\[ (L,X)\mapsto \bHom_{\bZ}(L,X): \Ab\op\times\scrM\to\scrM,\]
and
\[ (X,L)\mapsto X\otimes_{\bZ} L:\Ab\times\scrM\to\scrM,\]
each one of these two operations being deducible from the other by the
usual device of replacing \scrM{} by the dual category
$\scrM\op$. When $\scrM=\Ab$, these are just the usual internal
$\Hom=\bHom$ and tensor product operations. This very particular case
shows at once that we shouldn't expect in general the operations
$\bHom_{\bZ}(L,{-})$ of \Ahatab{} and ${-} * L'$ of \Bhatab{} upon
\AhatM{} to commute up to isomorphism -- we shouldn't expect, for a
given $L$ in \Ahatab{} or a given $L'$ in \Bhatab{} the commutation
relation to hold, except when this object is ``projective of finite
presentation'', i.e., is a direct factor of an object of $\Add(A)$
resp.\ of $\Add(B)\equeq\Add(A)\op$. Another fact becoming evident by
this particular case, is that whereas it is true that in the
equivalence \eqref{eq:106.64}
\[\Bhatab \equeq \bHom_!(\Ahatab,\Ab)\]
a projective object $L'$ in \Bhatab{} gives rise to a functor
$\Ahatab\to\Ab$ which is \emph{exact} (besides commuting to small
direct limits) -- and even to a functor commuting to small inverse
limits if $L'$ is ultraprojective, i.e., projective and of finite
presentation -- the converse to this (as contemplated on page
\ref{p:381}) does \emph{not} hold true. Indeed, in the particular case
$A=\Simplex_0$, when $L'$ is just an object in \Ab, the exactness
property envisioned, i.e, exactness of the functor $M\mapsto
M\otimes_{\bZ} L'$ from \Ab{} to itself, only means that $L'$ is a
\emph{flat} $\bZ$-module (i.e., torsion-free), which does \emph{not}
imply that it is projective (i.e., free). 

The\pspage{411} feeling I had earlier today, that the familiar looking
operation \eqref{eq:106.65} $\bHom(L,K)$ of abelian sheaves on \Ahat{}
upon \scrM-valued ones (\scrM{} an additive category stable under
small inverse limits) was well-understood, whereas the less familiar
one $M*L'$ in \eqref{eq:106.67} was not, turns out to be mistaken. In
computational terms, and writing $F$ for $K$ in \eqref{eq:106.65} and
$M$ in \eqref{eq:106.67}, the three basic data
\[\text{$F$ in \AhatM,} \quad
\text{$L$ in \Ahatab,} \quad
\text{$L'$ in \Bhatab}\]
should surely be interpreted as just functor
\begin{equation}
  \label{eq:106.68}
  F:A\op\to\scrM, \quad
  L:A\op\to\Ab,\quad
  L':B\op=A\to\Ab,\tag{68}
\end{equation}
and the practical question of ``computing'' $\bHom(L,F)$ or $F\oast
L'$ thus amounts to describing, directly in terms of these data, the
corresponding objects in \AhatM{} as again a functor
\[\bHom(L,F)\quad\text{or}\quad F\oast L':A\op\to\scrM.\]
It would seem that neither of the two can be expressed in simplistic
computational terms, via the data \eqref{eq:106.68}. I feel I have to
come to terms with this fact and get as close as I can to an explicit
expression of both. The point I want to make first, is that this
question of expressing $\bHom(L,F)$, or of expressing the operation
$F\oast L'$, is essentially the same, via replacement of \scrM{} by
$\scrM\op$, and mere interchange of $A$ and $B$. More accurately,
passing from $F$ to the corresponding functor $F\op$ between the dual
categories, we may view the data \eqref{eq:106.68} as being functors
\begin{multline}
  \label{eq:106.68prime}
  F\op:B\op\to\scrN,\quad
  L':B\op\to\Ab,\quad
  L:A\op\to\Ab\\
  \text{ (where $\scrN=\scrM\op$),}\tag{68'}
\end{multline}
i.e., a set of data like \eqref{eq:106.68}, with $(A,\scrM)$ replaced
by the dual pair $(B,\scrN)$. This being understood, we have the
tautological isomorphisms
\begin{equation}
  \label{eq:106.69}
  \begin{cases}
    \bHom(L,F)\op \simeq F\op \oast L & \\
    (F\oast L')\op \simeq \bHom(L',F\op) &\text{,}
  \end{cases}\tag{69}
\end{equation}
where the first members involve operations relative to the pair
$(A,\scrM)$, the second members operations relative to the dual pair
$(B,\scrN)$. This makes very clear, it seems to me, that the
operations $\bHom$ \eqref{eq:106.65} and $\oast$ \eqref{eq:106.67} may
be viewed as the same type of operation, simply viewed with two
different pairs of spectacles -- one being $(A,\scrM)$, the other the
dual pair $(B,\scrN)$. Thus, if we got a good understanding of one of
the two operations, embodied by a comprehensive
formulaire\scrcomment{I'm leaving in ``formulaire'' (form), even
  though ``formula'' seems to work better\ldots} for it, by just
dualizing we should get just as good a formulaire and corresponding
comprehension for the dual operation.

Now, it is clear indeed that it is the operations
\eqref{eq:106.65}\pspage{412} which is closer to my experience, it
makes sense however, independently of any duality statements, in the
vastly more general context of topoi \scrA{} (or even only pseudotopoi
satisfying some mild extra conditions, cf.\ section \ref{sec:104}
\ref{subsec:104.C}) instead of just \Ahat, provided we make on \scrM{}
the mild extra assumption of being a pseudotopos, needed in this more
general context in order to ensure equivalence between the category
$\scrA_\scrM$ of \scrM-valued sheaves, and the category
$\bHom^!({\scrA\subab}\op,\scrM)$ (compare \eqref{eq:104.13} p.\
\ref{p:388}). What I should do then is, first to write down a basic
``formulaire'' for the $\bHom$ operation in this general and familiar
context, then see how it can be used for clarifying the computational
puzzle raised on the previous page, in the case of the $\bHom$
operation, and finally dualize the formulaire and computational
insight, for getting a hold on the dual operation $\oast$.

We'll need too the $\Hom$ operation (non-bold-face) of
\eqref{eq:104.31} p.\ \ref{p:392}
\begin{equation}
  \label{eq:106.70}
  (L,F)\mapsto \Hom(L,F) : {\scrA\subab}\op \times \scrA_\scrM \to
  \scrM,\tag{70} 
\end{equation}
with values in \scrM, not $\scrA_\scrM$, where $\Hom(L,F)$ denotes the
value on $L$ of the functor
\[\widetilde F : {\scrA\subab}\op\to\scrM\]
defined by $F$, $F$ being viewed for the time being as an object in
$\bHom^!(\scrA\op,\scrM)$:
\begin{equation}
  \label{eq:106.71}
  F:\scrA\op\to\scrM.\tag{71}
\end{equation}
(In case $\scrA=\Ahat$, we get the description \eqref{eq:106.68} of
$F$ by taking the restriction of \eqref{eq:106.71} to the subcategory
$A$ of \Ahat.)

In the following formulaire, $L$, $L'$ are objects in $\scrA\subab$,
$F$ is an object in $\scrA_\scrM$, where \scrM{} is an additive
category stable under small inverse limits, which moreover is assumed
to be a pseudotopos (i.e., admits a small set of objects generating
with respect to monomorphisms, and is stable under small direct
limits) in case the topos \scrA{} is not equivalent to a category
\Ahat. We denote by
\[X\mapsto\bZ^{(X)}, \quad \scrA\to\scrA\subab\]
the abelianization functor, which in the case $\scrA=\Ahat$ is just
``componentwise abelianization'', i.e.,
\[\bZ^{(X)}(a)=\bZ^{(X(a))}\quad\text{for $a$ in $A$.}\]
We recall that the constant sheaf $\bZ_\scrA$ on \scrA{} with value
$\bZ$ can be also described as
\begin{equation}
  \label{eq:106.72}
  \bZ_\scrA=\bZ^{(e)},\tag{72}
\end{equation}
where $e$ is the final object of \scrA, and that the \emph{sections}
functor on \scrA\pspage{413} is defined as
\begin{equation}
  \label{eq:106.73}
  \Gamma_\scrA(F) \eqdef F(e);\tag{73}
\end{equation}
in case $\scrA=\Ahat$, this can equally be interpreted as the inverse
limit functor for the functor $A\op\to\scrM$ defined by $F$:
\begin{equation}
  \label{eq:106.74}
  \Gamma_\scrA(F) \simeq \varprojlim_{A\op} F(a).\tag{74}
\end{equation}
We are now ready to give a basic formulaire for the operations $\bHom$
and $\Hom$, and their relations to the abelianization functor and to
the sections functor (i.e., to inverse limits, in case $\scrA=\Ahat$).
\begin{equation}
  \label{eq:106.75}
  \left\{%
    \renewcommand*{\arraystretch}{1.1}%
    \begin{array}{@{}r@{}rl@{}}
      \left\{\rule{0pt}{3ex}\right. &
      \begin{tabular}{@{}r@{}}
        \textcircled{a} \\ 
        a')
      \end{tabular} &
      \begin{array}{@{}l@{}}
        \Hom(\bZ^{(X)},F)\simeq F(X) \\
        \bHom(\bZ^{(X)},F)\simeq (Y\mapsto F(X\times Y):\scrA\op\to\scrM)
      \end{array} \\
      \left\{\rule{0pt}{3ex}\right. &
      \begin{tabular}{@{}r@{}}
        \textcircled{b} \\
        b')
      \end{tabular} &
      \begin{array}{@{}l@{}}
        \Hom(L',\bHom(L,F)) \simeq \Hom(L'\otimes L,F) \\
        \bHom(L',\bHom(L,F)) \simeq \bHom(L'\otimes L,F)
      \end{array} \\
      \left\{\rule{0pt}{3ex}\right. &
      \begin{tabular}{@{}r@{}}
        c) \\
        c')
      \end{tabular} &
      \begin{array}{@{}l@{}}
        \Hom(\bZ_\scrA,F)\simeq\Gamma_\scrA(F) \\
        \bHom(\bZ_\scrA,F)\simeq F
      \end{array} \\
      & \text{d)} & \Hom(L,F)\simeq\Gamma_\scrA\bHom(L,F) \\
      \left\{\rule{0pt}{6ex}\right. &
      \begin{tabular}{@{}r@{}}
        \textcircled{e} \\
        $\phantom{()}$ \\
        e') \\
        $\phantom{()}$
      \end{tabular} &
      \begin{tabular}{@{}l@{}}
        The functor $L\mapsto\Hom(L,F):{\scrA\subab}\op\to\scrM$ \\
        commutes to small inverse limits \\
        Similar statement as e) for \\
        $L\mapsto\bHom(L,F):{\scrA\subab}\op\to\scrA_\scrM$.
      \end{tabular}
    \end{array}
  \right.\tag{75}
\end{equation}
\textbf{Comments on the formulaire \eqref{eq:106.75}.}\enspace I have
limited myself to chose canonical isomorphisms
(\hyperref[eq:106.75]{a)} to \hyperref[eq:106.75]{d)}) and exactness
properties (\hyperref[eq:106.75]{e)} and \hyperref[eq:106.75]{e')})
which seem to me the most relevant for what follows. Other exactness
and variance properties are commutation of the functors
\[F\mapsto\Hom(L,F):\scrA_\scrM\to\scrM\quad\text{and}\quad
F\mapsto\bHom(L,F):\scrA_\scrM\to\scrA_\scrM\]
to small inverse limits, and compatibility of formation of $\Hom(L,F)$
and $\bHom(L,F)$ with functors
\[u:\scrM\to\scrM'\]
commuting to small inverse limits. As for formulæ for varying topos
\scrA, corresponding to a morphism of topoi, we'll come back upon this
in a later section, in relation with the homology and cohomology
invariants of maps in \Cat. Also, I am completely disregarding here
compatibilities between canonical isomorphisms (surely the reader
won't complain about this). All this as far as omissions are
concerned.

As\pspage{414} for the formulas included in \eqref{eq:106.75}, the
three basic ones, including all others in a more or less formal way,
are the circled ones \hyperref[eq:106.75]{a)},
\hyperref[eq:106.75]{e)} and \hyperref[eq:106.75]{b)}. The properties
\hyperref[eq:106.75]{a)} and \hyperref[eq:106.75]{e)} jointly can be
viewed as the characterization up to canonical isomorphism, for fixed
$F$, of the operation $\Hom(L,F)$, i.e., of the functor
\[\widetilde F: L\mapsto\Hom(L,F): {\scrA\subab}\op\to\scrM,\]
factoring the functor
\[F:\scrA\op\to\scrM\]
via the abelianization functor $\Wh_A:X\mapsto \bZ^{(X)}$. In terms of
a), the formula b) can be viewed as essentially the definition of
$\bHom(L,F)$ via $\Hom({-},F)$, more specifically we get
\begin{equation}
  \label{eq:106.76}
  \bHom(L,F)(X) \simeq \Hom(\bZ^{(X)},\bHom(L,F)) \simeq
  \Hom(\bZ^{(X)}\otimes L, F).\tag{76}
\end{equation}
Taking $L=\bZ^{(X)}$ and using
\[\bZ^{(X)}\otimes \bZ^{(Y)} \simeq\bZ^{(X\times Y)}\]
(\eqref{eq:104.17} page \ref{p:389}), \eqref{eq:106.76} gives
\hyperref[eq:106.75]{a')}, whereas \hyperref[eq:106.75]{b')} follows
via \eqref{eq:106.76} applied to both members, from associativity of the
operation $\otimes$. Formula \hyperref[eq:106.75]{c)} is the
particular case of \hyperref[eq:106.75]{a)} for $X=e$, in the same way
\hyperref[eq:106.75]{c')} follows from
\hyperref[eq:106.75]{a')}. Formula \hyperref[eq:106.75]{d)} follows
from \hyperref[eq:106.75]{c)}, \hyperref[eq:106.75]{b)} and the
relation
\[\bZ_\scrA\otimes L\simeq L.\]
The exactness property \hyperref[eq:106.75]{e')} is equivalent to the
similar exactness statement for the functors
\[L\mapsto\bHom(L,F)(X)\simeq\Hom(\bZ^{(X)}\otimes L,F),\]
for $X$ in \scrA, and thus reduces to \hyperref[eq:106.75]{e)} with
$L$ replaced by $\bZ^{(X)}\otimes L$.

I would like now to come back to the question of ``computation'' of
$\Hom(L,F)$ and $\bHom(L,F)$. We may for this end assume \scrA{} to be
described by a \emph{site} $A$ -- which, in case the ``topology'' on
$A$ defining the site structure is the chaotic one, brings us back to
the particular case $\scrA=\Ahat$ we are mainly interested in at
present. Accordingly, we'll consider the objects $F$ in $\scrA_\scrM$
as being functors
\begin{equation}
  \label{eq:106.77}
  F:A\op\to\scrM\tag{77}
\end{equation}
satisfying the standard exactness properties for sheaves (with respect
to the given site structure on $A$). In terms of \eqref{eq:106.71},
this is just the composition of the functor \eqref{eq:106.71} with the
canonical functor
\begin{equation}
  \label{eq:106.78}
  A\to\scrA=A^{\sim},\tag{78}
\end{equation}
associating to an object $a$ in $A$ the presheaf represented by it, in
the\pspage{415} most common case when this presheaf is a sheaf for any
choice of $a$, otherwise we take the sheaf associated to it. In the
first case (which we may reduce to if we prefer, by suitable choice of
the site $A$ for given topos \scrA) the functor \eqref{eq:106.78} is
fully faithful and moreover and embedding, therefore, we'll identify
an object $a$ in $A$ with the corresponding object in \scrA. Thus, the
description \eqref{eq:106.76} of $\bHom$ in terms of $\Hom$ may be
interpreted, from this point of view, as a formula with $X=a$ in $A$,
i.e., as describing the sheaf $\bHom(L,F)$ as a functor on
$A\op$. Accordingly, the question of describing the sheaf $\bHom(L,F)$
is reduced to the question of describing the objects $\Hom(L',F)$ in
\scrM, for $L'=\bZ^{(X)}\otimes L$. Thus, the main question here is to
give a ``computational'' description of the object $\Hom(L,F)$ in
\scrM, for $L$ in $\scrA\subab$ and $F$ in $\scrA_\scrM$, i.e., $F$
and $L$ being sheaves on $A$
\begin{equation}
  \label{eq:106.79}
  F:A\op\to\scrM,\quad
  L:A\op\to\Ab.\tag{79}
\end{equation}
The rule of the game here is to do so, using just
\hyperref[eq:106.75]{a)} in case of $X=a$ in $A$, and the exactness
property \hyperref[eq:106.75]{e)}.

It seems most convenient here to introduce again the additive envelope
$\Add(A)$ of the category $A$, which we'll assume to be small in what
follows, and the canonical additive functor
\begin{equation}
  \label{eq:106.80}
  \varepsilon\subab:\Add(A)\to\scrA\subab,\tag{80}
\end{equation}
extending the functor
\[a\mapsto\bZ^{(a)}:A\to\scrA\subab.\]
For a given $F$ \eqref{eq:106.77}, it follows from formula
\hyperref[eq:106.75]{(75 a))} that the composition
\[\widetilde F\circ\varepsilon\subab\op:\Add(A)\op
\xrightarrow{\varepsilon\subab\op} {\scrA\subab}\op
\xrightarrow{\widetilde F} \scrM\]
is just the canonical extension $\Add(F)$ of $F$ to $\Add(A)\op$,
whose value on the general object
\[x = \bigoplus_{i\in I} \Wh_A(a_i)\quad
\text{($I$ a finite indexing set)}\]
of $\Add(A)$ (where $\Wh_A(a)=\bZ^{(a)}$ as an object in
$\Add(A)\subset\Ahatab$) is just
\begin{equation}
  \label{eq:106.81}
  \Add(F)(x)=\prod_{i\in I}F(a_i).\tag{81}
\end{equation}
Now, it is easily checked that for any object $L$ in $\scrA\subab$,
i.e., any sheaf $L:A\op\to\Ab$, we have a canonical isomorphism in
$\scrA\subab$
\begin{equation}
  \label{eq:106.82}
  L\fromsim \varinjlim_{\text{$(x,u)$ in $\Add(A)_{/L}$}} \varepsilon\subab(x)\tag{82}
\end{equation}
(compare with the similar isomorphism on page \ref{p:403}). Using
\hyperref[eq:106.75]{(75~e)}, we\pspage{416} deduce from this the
expression
\begin{equation}
  \label{eq:106.83}
  \Hom(L,F)=\widetilde F(L)\simeq \varinjlim_{\text{$(x,u)$ in
      $\Add(A)_{/L}$}} \Add(F)(x),\tag{83}
\end{equation}
which in an evident way is functorial in $L$ for variable $L$.

This is about the best which can be done in general, it seems to me,
by way of ``computational'' expression of $\Hom(L,F)$ in terms of $F$
and $L$ given as in \eqref{eq:106.79}. Of course, the symbol
$\Add(A)_{/L}$ is relative to the canonical functor
$\varepsilon\subab$ \eqref{eq:106.80}, which is a full embedding in
case the site structure on $A$ is the chaotic one, i.e.,
$\scrA=\Ahat$. In computational terms, this category is rather
explicit, an object of the category is just a pair
\[(x,u)=\bigl((a_i)_{i\in I}, (u_i)_{i\in I}\bigr)\]
where $I$ is a finite indexing set, $(a_i)_{i\in I}$ a family of
objects of $A$, and for $i$ in $I$, $u_i$ is an element of $L(a_i)$ --
I'll leave to the reader the description of maps between such
objects. The value of $\Add(F)(x)$ is given by \eqref{eq:106.81}
above.

\begin{remark}
  The expression \eqref{eq:106.83} of $\Hom(L,F)=\widetilde F(L)$
  makes sense, provided only the additive category \scrM{} is stable
  under small inverse limits, without having to assume that \scrM{} be
  a pseudotopos. This makes us suspect that the functor
  \[\scrP\mapsto \scrP\circ\Wh :
  \bHom^!(\scrA\subab\op,\scrM)\to\scrA_\scrM\]
  is an equivalence (\eqref{eq:104.13} p.\ \ref{p:388}) without this
  extra assumption, provided \scrA{} is an actual topos (not only a
  pseudotopos as in loc.\ cit.). Indeed, we get a reasonable candidate
  for a quasi-inverse functor
  \[F\mapsto\widetilde
  F:\scrA_\scrM\to\bHom^!(\scrA\subab\op,\scrM).\]
  The only point still to check, with \eqref{eq:106.83} defining
  $\widetilde F$ for given $F$ in $\scrA_\scrM$, is that we get a
  functorial isomorphism
  \[\widetilde F(\bZ^{(a)}) \simeq F(a)\]
  for $a$ in $A$. In case $\scrA=\Ahat$, this follows from the fact
  that \eqref{eq:106.80} is fully faithful, hence $\Add(A)_{/L}$ for
  $L=\bZ^{(a)}$ admits a final object -- hence the limit
  \eqref{eq:106.83} is the value of $\Add(F)$ on the latter, namely
  $F(a)$.
\end{remark}

I feel the little program on the $\bHom$ and $\Hom$ operations, as
contemplated on page \ref{p:412}, is by now completed; all we've got
to do still is to dualize to get corresponding results for $*$ and
$\oast$. It's just a matter of essentially copying the formulaire
\eqref{eq:106.75}, which I'll do\pspage{417} for the sake of getting
more familiar with the more unusual operations $*$ and $\oast$. Now
of course, we'll have to restrict to the case $\scrA=\Ahat$, and use
the interpretation \eqref{eq:106.68} of the data $F$, $L'$ as functors
on $A\op$ and on $A$ with values in \scrM{} and \Ab{} respectively,
where now \scrM{} is an additive category stables under small
\emph{direct} limits. By duality, the ``sections'' or ``inverse
limits'' functor $\varprojlim_{A\op}$ (or ``cointegration'') is
replaced by the direct limit functor $\varinjlim_{A\op}$ (or
``integration''). With this in mind, we get the following
transcription of \eqref{eq:106.75}:
\begin{equation}
  \label{eq:106.84}
  \left\{%
    \renewcommand*{\arraystretch}{1.15}%
    \begin{array}{@{}r@{}rl@{}}
      \left\{\rule{0pt}{3.2ex}\right. &
      \begin{tabular}{@{}r@{}}
        a) \\ 
        a')
      \end{tabular} &
      \begin{array}{@{}l@{}}
        \smash{F*\bZ^{(b\op)}\simeq F(b)\quad\text{for any $b$ in $A$, hence
        $b\op$ in $B$}} \\
        \smash{F\oast\bZ^{(b\op)}\simeq(a\mapsto F(a\lor b)\simeq
        \varinjlim_{\text{$(x,\alpha)$ in $\preslice A{a\lor
        b}$}} F(x))}
      \end{array} \\
      \left\{\rule{0pt}{3.2ex}\right. &
      \begin{tabular}{@{}r@{}}
        b) \\
        b')
      \end{tabular} &
      \begin{array}{@{}l@{}}
        \smash{(F*L')*L'' \simeq F*(L'\otimes L'')} \\
        \smash{(F\oast L')*L'' \simeq F\oast(L'\otimes L'')}
      \end{array} \\
      \left\{\rule{0pt}{3.2ex}\right. &
      \begin{tabular}{@{}r@{}}
        c) \\
        c')
      \end{tabular} &
      \begin{array}{@{}l@{}}
        \smash{F*\bZ_\Bhat \simeq \varinjlim_{A\op} F} \\
        \smash{F \oast \bZ_\Bhat \simeq F}
      \end{array} \\
      & \text{d)} &\smash{F*L' \simeq \varinjlim_{A\op} F\oast L'} \\
      \left\{\rule{0pt}{6.2ex}\right. &
      \begin{tabular}{@{}r@{}}
        e) \\
        $\phantom{()}$ \\
        e') \\
        $\phantom{()}$
      \end{tabular} &
      \begin{tabular}{@{}l@{}}
        The functor $\smash{L'\mapsto F*L':\Bhatab\to\scrM}$ \\
        commutes to small direct limits \\
        Similar statement as e) for \\
        $\smash{L'\mapsto F\oast L':\Bhatab\to\AhatM}$.
      \end{tabular}
    \end{array}
  \right.\tag{84}
\end{equation}
\textbf{Comments.}\enspace This formulaire doesn't look wholly
symmetric to \eqref{eq:106.75}, due to the fact that we gave
\eqref{eq:106.75} in a somewhat more general context than topoi of the
type \Ahat{} only. This accounts for the letter $X$ or $Y$ in
\eqref{eq:106.75} (designating there an arbitrary object of \Ahat)
being replaced by a small letter $a$ or $b$ (designating objects in
$A$), which allows the dualization to be done. A slight trouble then
occurs when $A$ is not stable under binary products $a\times b$, these
products are only in \Ahat{} not in $A$, which accounts for the
slightly more complicated formula \hyperref[eq:106.84]{a')} of
\eqref{eq:106.84} in comparison to \hyperref[eq:106.75]{(75~a'))},
whose more explicit form, in the present context of data as in
\eqref{eq:106.68}, would be
\begin{equation}
  \label{eq:106.85}
  \bHom(\bZ^{(a)},F)\simeq\bigl(b\mapsto F(a\times
  b)\simeq\varprojlim_{\text{$(x,\alpha)$ in $A_{/a\times b}$}}
  F(x)\bigr).\tag{85} 
\end{equation}
Accordingly, the symbol $a\lor b$ (``sum'') in
\hyperref[eq:106.84]{(84~a'))} denotes the element $(a\op\times
b\op)\op$ of $(\Bhat)\op$ and can be identified with the \emph{sum} of
$a$ and $b$ in the category $A$\pspage{418} whenever the sum exists in
$A$. Accordingly, the category $\preslice A{a\lor b}$, dual to
$B_{/a\op\times b\op}$, can be described as
\begin{equation}
  \label{eq:106.86}
  \preslice A{a\lor b}=
  \begin{tabular}[t]{@{}l@{}}
    category of all triples $(x,u,v)$, with $x$ in $A$ and \\
    $u:a\to x$, $v:b\to x$ maps in $A$,
  \end{tabular}\tag{86}
\end{equation}
the maps in this category from $(x,u,v)$ to $(x',u',v')$ being just
maps $x\to x'$ ``compatible'' with the pairs $\alpha=(u,v)$ and
$\alpha'=(u',v')$ in the obvious way.

\begin{remarks}
  \namedlabel{rem:106.1}{1})\enspace
  An interesting particular case (although admittedly a little strange
  looking in our modelizing context!) is the one when $A$ is an
  additive category, hence stable under both binary sum and product
  operation, the two operations being canonically isomorphic, and
  written as $a\oplus b$. In this case, comparison of
  \eqref{eq:106.85} and \eqref{eq:106.86} shows that for a given
  object $a$ in $A$, hence $a\op$ in $B$, the operation
  $\Hom(\bZ^{(a)},{-})$ on \AhatM{} is canonically isomorphic to the
  operation ${-}\oast\bZ^{(a\op)}$. This immediately extends to a
  canonical isomorphism
  \begin{equation}
    \label{eq:106.87}
    \bHom(L,F)\simeq F\oast\check L \quad
    \text{for $L$ in $\Add(A)\subset\Ahatab$, % \\
      $F$ in \AhatM,}
    \tag{87}
  \end{equation}
  where we have denoted by
  \begin{equation}
    \label{eq:106.88}
    L\mapsto\check L: \Add(A)\op\toequ \Add(A\op)=\Add(B)\tag{88}
  \end{equation}
  the canonical antiequivalence between $\Add(A)$ and $\Add(B)$. In
  case $A$ is the final category, namely an additive category reduced
  to the zero object, and if we take moreover $\scrM=\Ab$,
  \eqref{eq:106.87} is the familiar formula of linear algebra, valid
  when $L$ is a free $\bZ$-module of finite type. It should be noted
  that $A$ being stable under binary products, it follows that
  $\Add(A)$ is stable under tensor products, and similarly for
  $\Add(B)$, and that the equivalence \eqref{eq:106.88} is compatible
  with tensor products. The relation \eqref{eq:106.87} is about the
  only relationship I could think of between the two types of
  operations upon a given category \AhatM.

  2)\enspace There are still two other, more trivial operations on a
  category \Ahat, of a similar nature to the two operations $\Hom$
  and $*$ considered so far. The more familiar one is componentwise
  tensor product
  \begin{equation}
    \label{eq:106.89}
    (L,F)\mapsto L\otimes F: \Ahatab\times\AhatM\to\AhatM,\tag{89}
  \end{equation}
  defined by
  \[{L\otimes F}\,(a) = L(a)\otimes F(a),\]
  where the second member denotes external tensor product of the
  abelian group $L(a)$ with the object $F(a)$ of \scrM{} (defined when
  \scrM{} is additive and stable under small direct limits). The
  other, deduced from \eqref{eq:106.89} by duality\pspage{419}
  \begin{equation}
    \label{eq:106.90}
    (L',F)\mapsto\bHom(L',F) : \Bhatab\times\AhatN\to\AhatN\tag{90}
  \end{equation}
  is defined when the additive category \scrN{} is stable under small
  inverse limits, and can be equally described as taking external
  $\Hom$'s componentwise
  \[\bHom(L',F)(a) = \Hom(L'(a),F(a)).\]
  These operations make sense too when \Ahat{} is replaced by an
  arbitrary topos \scrA, \Bhatab{} being replaced by the category of
  abelian cosheaves on \scrA. It doesn't seem worthwhile here to dwell
  on them, as they don't seem to be so relevant for the homology and
  cohomology formalism we want to develop in the next sections. I like
  to point out, though, that in the cohomology formalism of ringed
  topoi the tensor product operation \eqref{eq:106.89} and the derived
  operation $\overset{\mathrm L}{\otimes}$ on the relevant derived
  categories $\D_\bullet$ play an important role, and it is likely
  therefore that in a more extensive development of the homology and
  cohomology formalism within the context of topoi \Ahat{} and maps in
  \Cat, the same will hold for the dual operation \eqref{eq:106.90}
  too.

  The reader who may feel confused by the manifold use of the symbol
  $\bHom$ should notice that there is no possibility of confusion
  reasonably between \eqref{eq:106.90} and \eqref{eq:106.65} (p.\
  \ref{p:409}), as the argument $L'$ in \eqref{eq:106.90} is in
  \Bhatab, whereas the argument $L$ in \eqref{eq:106.65} is in an
  altogether different category \Ahatab. In the case when $A$ is the
  final category say, hence $A=B$, and a confusion might arise, the
  two operations turn out to be actually the same (up to canonical
  isomorphism). A similar remark applies to the fear of confusion
  between the kindred operations $\otimes$ and $\oast$. I daresay I
  devoted a considerable amount of attention on terminology and
  notation around the abelianization story -- and it does seem that a
  pretty coherent formalism is emerging indeed.
\end{remarks}

\bigbreak
\presectionfill\ondate{17.8.}\par

% 107
\hangsection{Review \texorpdfstring{\textup{(4)}}{(4)}: Case of
  general ground ring \texorpdfstring{$k$}{k}.}\label{sec:107}%
\phantomsection\addcontentsline{toc}{subsection}{\numberline
  {F)}Extension of ground ring from \texorpdfstring{$\bZ$ to $k$
  \textup($k$}{Z to k (k}-linearization\texorpdfstring{\textup)}{)}.}%
\textbf{\namedlabel{subsec:107.F}{F)}\enspace Extension of ground ring
  from $\bZ$ to $k$ ($k$-linearization).}\enspace%
I would like still to make a quick review of the main facts and
formulas of the last two sections, replacing throughout the ground
ring $\bZ$ by an arbitrary \emph{commutative} ring $k$, and additive
categories \scrM{} and additive functors between these, by
$k$-additive categories and $k$-additive functors. This will allow us
to check that the conceptual and notational set-up we got so far
extends smoothly to $k$-linearization.

Let's\pspage{420} recall that a \emph{$k$-additive category} \scrM{}
is an additive category endowed with the extra structure given by a
homomorphism of commutative rings
\begin{equation}
  \label{eq:107.91}
  k\to\End(\id_\scrM),\tag{91}
\end{equation}
where the second member denotes the (commutative) ring of all
endomorphisms of the identity functor of \scrM{} to itself. Defining
accordingly the notion of \emph{$k$-additive functor} between two
$k$-additive categories $\scrM$, $\scrM'$, we'll denote by
\begin{equation}
  \label{eq:107.92}
  \bHom_k(\scrM,\scrM')\subset\bHom(\scrM,\scrM')\tag{92}
\end{equation}
the full subcategory of $\bHom(\scrM,\scrM')$ made up with such
functors. Thus, we get a canonical fully faithful inclusion
\begin{equation}
  \label{eq:107.93}
  \bHom_k(\scrM,\scrM') \hookrightarrow \bHom_{\bZ}(\scrM,\scrM')
  \eqdef \Homadd(\scrM,\scrM').\tag{93}
\end{equation}
Wed defined accordingly the categories $\bHom_{k!}$, $\bHom_k^!$ as
full subcategories of \eqref{eq:107.92}, and the category
\[\bHom_k(\scrP,\scrQ;\scrM) \subset \bBiadd(\scrP,\scrQ;\scrM)\]
the full subcategory of $\bHom(\scrP\times\scrQ,\scrM)$ made up with
\emph{$k$-bilinear} functors, namely functors $k$-additive in each
argument (in case $k=\bZ$,\scrcomment{Actually, it was previously
  denoted by just $\bHom(\scrP,\scrQ;\scrM)$\ldots} this is the
category denoted previously by $\bBiadd$), and similarly for the
notations $\bHom_{k!!}$ and $\bHom_k^{!!}$.

It should be noted that for a given additive category \scrM, there is
a ``best'' choice for endowed it with a $k$-linear structure, in such
a way that any $k'$-linear structure just corresponds to ``ground ring
restriction'' with respect to suitable (well-defined) ring
homomorphism
\[k'\to k;\]
we just take the ``tautological'' linear structure with
\[k=\End(\id_\scrM),\]
and \eqref{eq:107.91} the identity.

If $A$ is any small category, we'll denote by
\begin{equation}
  \label{eq:107.94}
  \Ahatk = A\uphat_{k\textup{-Mod}} \simeq \bHom(A\op,\kMod)\tag{94}
\end{equation}
the category of objects in \Ahat{} endowed with a structure of
$k$-module, i.e., the category of presheaves on $A$ with values in the
category \kMod{} of $k$-modules (in the given basic universe
\scrU). This is of course a $k$-additive category, which for $k=\bZ$
reduces to the category\pspage{421} of additive presheaves on $A$:
\[ A\uphat_{\bZ} \eqdef \Ahatab.\]
We have, for a homomorphism of commutative rings
\[k\to k',\]
a corresponding functor between additive topoi
\begin{equation}
  \label{eq:107.95}
  \Ahatk \to A\uphat_{k'}, \quad
  F\mapsto F\otimes_k k' = (a\mapsto F(a)\otimes_k k'),\tag{95}
\end{equation}
by which we may interpret if we wish, in a rather evident way the
$k'$-linear topos $A\uphat_{k'}$ as deduced from the $k$-linear one
\Ahatk{} by ``ground ring extension'' $k\to k'$, namely as the
solution of a $2$-universal problem with respect to categories
$\bHom_{k!}(\Ahatk,\scrM)$, where \scrM{} is a $k'$-additive category
stable under small direct limits. The $k$-abelianization functor
\begin{equation}
  \label{eq:107.96}
  \Ahat\to\Ahatk, \quad X\mapsto k^{(X)}\quad\bigl(\;\simeq (a \mapsto
  k^{(X(a))})\bigr)\tag{96} 
\end{equation}
or $\Wh_{\Ahat,k}$, is defined as the composition
\[\Ahat\to A\uphat_{\bZ} = \Ahatab\to \Ahatk,\]
where the first functor is the familiar abelianization
$X\mapsto \bZ^{(X)}$, and the second is ground ring extension for
$\bZ\to k$. If \scrM{} is any $k$-additive category stable under small
direct limits, \eqref{eq:107.96} gives rise to a functor which is an
equivalence of categories $F\mapsto(X\mapsto F(k^{(X)}))$
\begin{equation}
  \label{eq:107.97}
  \bHom_{k!}(\Ahatk,\scrM)\toequ\bHom_!(\Ahat,\scrM) \quad (\;\toequ
  \bHom(A,\scrM)),\tag{97} 
\end{equation}
where the second equivalence is the familiar one of prop.\
\ref{prop:105.3} (p.\ \ref{p:394}), independent of any abelian
assumptions. Dually, we get an equivalence
\begin{equation}
  \label{eq:107.98}
  \bHom_k^!((\Ahatk)\op,\scrM)\toequ \bHom((\Ahat)\op,\scrM)
  \quad(\;\toequ \bHom(A\op,\scrM)),\tag{98}
\end{equation}
where \scrM{} is any $k$-additive category stable under small inverse
limits. From \eqref{eq:107.97} \eqref{eq:107.98} and replacing in
\eqref{eq:107.98} $A$ by the dual category $B=A\op$, and assuming the
$k$-additive category \scrM{} is stable under both types of small
limits, we get the duality equivalence
\begin{equation}
  \label{eq:107.99}
  \bHom_{k!}(\Ahatk,\scrM)\equ \bHom_k^!(\Bhatk,\scrM) \quad (\;\equ
  \bHom(A,\scrM)).\tag{99} 
\end{equation}
This may be viewed as giving two alternative descriptions, by the two
members of \eqref{eq:107.99}, of the category
\[\BhatM = \bHom(B\op=A,\scrM)\]
of \scrM-valued presheaves on $B$ (defined without any use of the
$k$-additive structure of \scrM). The left-hand side interpretation
\eqref{eq:107.99}, via \scrM-valued\pspage{422} $k$-additive cosheaves
on the $k$-additive topos \Ahatk, gives rise to the operations $*_k$
and $\oast_k$ of \Ahatk{} upon \BhatM{} (operations previously denoted
by $*$ and $\oast$ when $k=\bZ$ and no confusion would arise from
dropping subscripts), and similarly the interpretation by right-hand
side of \eqref{eq:107.99}, via \scrM-valued $k$-additive sheaves on
the $k$-additive topos \Bhatk, gives rise to the operations $\Hom_k$
and $\bHom_k$ of \Bhatk{} upon \BhatM. Replacing in this comment $A$
by $B$, hence \Bhat{} by \Ahat, namely in terms of operations upon the
category of \scrM-valued sheaves on the topos \Ahat{} (or \scrM-valued
presheaves on $A$), we get the mutually dual pair of operations
\begin{equation}
  \label{eq:107.100}
  \begin{aligned}
    (F,L')&\mapsto F*_kL' : \AhatM\times\Bhatk\to\scrM, \\
    (F,L')&\mapsto F\oast_kL' : \AhatM\times\Bhatk\to\AhatM
  \end{aligned}
  \tag{100}
\end{equation}
and
\begin{equation}
  \label{eq:107.101}
  \begin{aligned}
    (L,F)&\mapsto \Hom_k(L,F) : \Ahatk\times\AhatM\to\scrM, \\
    (L,F)&\mapsto \bHom_k(L,F) : \Ahatk\times\AhatM\to\AhatM.
  \end{aligned}
  \tag{101}
\end{equation}
The operations \eqref{eq:107.100} are ruled by formulaire
\eqref{eq:106.84} (with subscripts $k$ added), whereas the operations
\eqref{eq:107.101} are ruled by formulaire \eqref{eq:106.75} with
subscripts (see moreover for the latter comments on page \ref{p:417},
and formula \eqref{eq:106.85} for \hyperref[eq:106.75]{(75~a'))}; they
are valid provided the additive category is stable under small direct
resp.\ inverse limits. Moreover, we get a ``computational'' expression
of $\Hom_k(L,F)$ by a formula extending \eqref{eq:106.83} which we'll
still have to write down, and correspondingly for $F*_kL'$ (by a dual
formula, which we forgot to include in the previous section). To do
so, we have to introduce still
\begin{equation}
  \label{eq:107.102}
  \Add_k(A)\subset\Ahatk,\tag{102}
\end{equation}
the $k$-additive envelope of $A$, which may be described (beside by
the familiar $2$-universal property in the context of $k$-additive
categories and functors from $A$ into these) as the full subcategory
of \Ahatk{} generated by finite sums of objects of the type $k^{(a)}$
with $a$ in $A$ -- i.e., the general object of $\Add_k(A)$ may be
written
\[\bigoplus_{i\in I}k^{(a_i)},\]
where $(a_i)_{i\in I}$ is any finite family of objects of $A$. When
the finiteness condition on $I$ is dropped, we get a larger full
subcategory
\begin{equation}
  \label{eq:107.103}
  \Addinf_k(A)\subset\Ahatk,\tag{103}
\end{equation}
which may also be interpreted as ``the'' solution of the $2$-universal
problem of sending $A$ into categories which are $k$-additive and
moreover infinitely additive, i.e., stable under small direct
sums. Enlarging the subcategories \eqref{eq:107.102} and
\eqref{eq:107.103} of \Ahatk{} by adjoining all objects of
\Ahatk\pspage{423} isomorphic to direct factors of objects in the
considered subcategory, we get to (strictly) full subcategories of
\Ahatk{} containing the latter, which may be interpreted as being just
the subcategory $\Proj(\Ahatk)$ of \emph{projective} objects of
\Ahatk{} when starting with \eqref{eq:107.103}, and as the subcategory
$\UlProj(\Ahatk)$ of \emph{ultraprojective} objects, namely objects
projective and of finite presentation, when starting with
\eqref{eq:107.102}. These may be equally interpreted as the abstract
Karoubi envelops of the categories \eqref{eq:107.103} and
\eqref{eq:107.102}, deduced from these formally by adjoining images
(=coimages) of projectors (or equivalently, as $2$-universal solutions
of the $2$-universal problem of sending the given category
\eqref{eq:107.103} or \eqref{eq:107.102} into ``karoubian
categories'', namely categories stable under images (=coimages) of
projectors, with maps between these being functors commuting to those
images or coimages of projectors):
\begin{equation}
  \label{eq:107.104}
  \Proj(\Ahatk)\equeq\Kar(\Addinf_k(A)), \quad
  \UlProj(\Ahatk)\equeq\Kar(\Add_k(A)).\tag{104}
\end{equation}
Accordingly, these two categories may be equally described, directly
in terms of $A$, as the solutions of the two $2$-universal problems,
obtained from mapping $A$ into $k$-additive karoubian categories,
which in the first case (corresponding to $\Proj(\Ahatk)$) are
moreover assume infinitely additive.

To sum up the situation, we get in \Ahatk{} a diagram of four
remarkable full subcategories \eqref{eq:107.102}, \eqref{eq:107.103},
\eqref{eq:107.104}, which may be interpreted (as well as \Ahatk{}
itself) as the solutions of five corresponding ``$k$-additive''
$2$-universal problems, in terms of sending $A$ into $k$-additive
categories satisfying suitable extra exactness assumptions (namely
being karoubian for the two categories in \eqref{eq:107.104}, being
infinitely additive for the two categories $\Addinf_k(A)$ and its
Karoubi envelope $\Proj(\Ahatk)$, and being stable for small direct
limits in case of \Ahatk). Including equally the non-additive
categories $A$ and \Ahat{} and the functors $A\to\Add_k(A)$,
$\Ahatk\to\Ahat$, we get a seven term diagram of canonical functors
between categories of presheaves upon $A$:
\begin{equation}
  \label{eq:107.105}
  \left\{
    \begin{tabular}{@{}c@{}}
      \begin{tikzcd}[baseline=(O.base),sep=small]
        A\ar[d] & & \\
        \Add_k(A)\ar[r,hook]\ar[d,hook] &
        \UlProj(\Ahatk)\equeq\KarAdd_k(A) \ar[d,hook,shift right=3.5em] & \\
        \Addinf_k(A)\ar[r,hook] &
        \Proj(\Ahatk)\equeq\KarAddinf_k(A)\ar[r,hook] & \Ahatk\ar[d] \\
        & & |[alias=O]| \Ahat
      \end{tikzcd},
    \end{tabular}\right.
  \tag{105}
\end{equation}
where the five categories in the two intermediate lines are
$k$-additive as well as all functors between them in the diagram,
which are moreover fully faithful. For any $k$-additive\pspage{424}
category \scrM{} stable under small direct limits, taking cosheaves on
\Ahat{} with values in \scrM, and their restrictions to the six other
categories in the diagram \eqref{eq:107.105}, we get a transposed
seven term diagram as follows, part of which reduces to the four term
diagram \eqref{eq:105.58} (p.\ \ref{p:406}) in case $k=\bZ$:
\begin{widematter}
  \begin{equation}
  \label{eq:107.106}
  \left\{
    \begin{tabular}{@{}c@{}}
      \begin{tikzcd}[baseline=(O.base),sep=small]
        \bHom_!(\Ahat,\scrM) \ar[d,"\equ"'] & & \\
        \bHom_{k!}(\Ahatk,\scrM) \ar[r,"\equ"] &
        \bHomaddinfkar_k(\Proj(\Ahatk),\scrM) \ar[r,dash,"\equ"]
        \ar[d,"\equ"] &
        \bHomaddinf_k(\Addinf_k(A),\scrM) \ar[d,"\equ"] \\
        & \bHomaddkar_k(\UlProj(\Ahatk),\scrM) \ar[r,dash,"\equ"] &
        \bHomadd_k(\Add_k(A),\scrM) \ar[d,"\equ"] \\
        & & |[alias=O]| \bHom(A,\scrM)
      \end{tikzcd},
    \end{tabular}\right.
  \tag{106}
\end{equation}
\end{widematter}
where the meaning of the symbols used (such as index $k$, suffixes
``add'' or ``addinf'' and ``kar'') for qualifying $\bHom$ and denoting
various full subcategories of $\bHom$ categories, is clear from the
explanations given previously. Replacing $A$ by $B$ and
$\bHom_!(\Ahat,\scrM)$ by $\bHom^!(\Bhat,\scrM)$, we get a diagram
``dual'' to \eqref{eq:107.106} (containing the five-term diagram
\eqref{eq:106.62} (p.\ \ref{p:408}) in case $k=\bZ$), which we'll not
write out here, valid for any $k$-additive category \scrM{} stable
under small inverse limits. When \scrM{} is a $k$-additive category
stable under both types of small limits, then the last term of the
diagram \eqref{eq:107.106} is equal to the last term of the dual one,
hence a system of fourteen mutually equivalent categories (compare p.\
\ref{p:408}, when we considered ten among them only!), expressing as
many ways for interpreting the notion of an \scrM-valued copresheaf on
$A$, i.e., an object of $\bHom(A,\scrM)$ (which is one among the
fourteen\ldots).

Let's comment a little on the significance of the various five
$k$-additive categories appearing in \eqref{eq:107.105}. The largest
one \Ahatk{} is there precisely as the all-encompassing category of
$k$-additive presheaves, where to carry through all kinds of
$k$-linear constructions between presheaves on $A$. The significance
of the (second largest) subcategory $\Proj(\Ahatk)$, made up with all
projective objects of \Ahatk, comes mainly from homological algebra
and emphasis upon replacing objects of \Ahatk{} by projective
resolutions; these are chain complexes in $\Proj(\Ahatk)$, which may
be viewed as being defined (by any given object in \Ahatk) ``up to
chain homotopy''. More sweepingly still, we get from general
principles the canonical equivalence of categories
\begin{equation}
  \label{eq:107.star}
  \D^-(\Ahatk) \fromequ \mathrm K^-(\Proj(\Ahatk)),\tag{*}
\end{equation}
where $\D^-$ designates the ``derived category bounded from
above''\pspage{425} of a given \emph{abelian} category (defined in
terms of differential operators with degree $+1$, and
quasi-isomorphisms between complexes with degrees bounded from above),
whereas $\mathrm K^-$ designates localization of the category of
differential complexes with degrees bounded from above of a given
\emph{additive} category, localization being taken with respect to
homotopisms.

As any object of $\Proj(\Ahatk)$ is a direct factor of an object in
$\Addinf_k(A)$, and hence, any object in \Ahatk{} is isomorphic to a
quotient of an object in $\Addinf_k(A)$, it follows again from general
principles that the categories in \eqref{eq:107.star} are equally
equivalent to $\mathrm K^-(\Addinf_k(A))$, hence
\begin{equation}
  \label{eq:107.107}
  \D^-(\Ahatk) \fromequ \mathrm K^-(\Proj(\Ahatk)) \fromequ
  \mathrm K^-(\Addinf_k(A)).\tag{107}
\end{equation}
The advantage of $\Addinf_k(A)$ over $\Proj(\Ahatk)$ is that its
objects, and maps between objects, are more readily described in
computational terms, just working with small direct sums of objects of
the type $k^{(a)}$ (with $a$ in $A$), and corresponding matrices, with
entries in free $k$-modules $k^{(\Hom(a,b))}$. Thus, if we call
\emph{cointegrator} (with coefficients in $k$) \emph{for} $A$ any
projective resolution of the constant presheaf $k_A$ with value $k$,
and denote such object by $L_k^A$, we may view $L_k^A$ as an object
determined up to unique isomorphism, either in $\mathrm
K^-(\Proj(\Ahatk))$, or in $\mathrm K^-(\Addinf_k(A))$ -- and it is
the latter interpretation which looks the most convenient. Objects in
the first category, namely complexes with degrees bounded from above
and projective components, which happen to be in the first (i.e.,
components are in $\Addinf_k(A)$, i.e., are direct sums of objects of
the type $k^{(a)}$) may be called ``\emph{quasi-special}'' (extending
the terminology previously used for cointegrators and integrators, in
case $k=\bZ$). We'll call them \emph{special} if the components are
even in $\Add_k(A)$. The category $\Add_k(A)$ and its Karoubi envelope
$\UlProj(\Ahatk)$ may be viewed both as embodying \emph{finiteness
  conditions}, and similarly for the two corresponding $\mathrm K^-$
categories, which are of course equivalent:
\begin{equation}
  \label{eq:107.108}
  \mathrm K^-(\UlProj(\Ahatk)) \fromequ \mathrm K^-(\Add_k(A)),\tag{108}
\end{equation}
and presumably the canonical functor from \eqref{eq:107.108} to
\eqref{eq:107.107} is fully faithful, under suitable coherence
conditions at any rate\ldots

\bigbreak
\presectionfill\ondate{22.8.}\pspage{426}\par

% 108
\hangsection{Review \texorpdfstring{\textup{(5)}}{(5)}: Homology and
  cohomology \texorpdfstring{\textup(absolute case\textup)}{(absolute
    case)}.}\label{sec:108}%
Since last Monday, namely for about one week, I have been mainly taken
by a rather dense sequence of encounters and events, the center of
which has been the unexpected news of my granddaughter Ella's death at
the age of nine, by a so-called health accident. I resumed some
mathematical pondering last night. Today, I got a short letter from
Ronnie Brown, mainly with the announcement of the loss of his son
Gabriel, twenty years old, which occurred about the same time by a
climbing accident. It is a good thing that Ronnie felt like telling me
in a few words about this, while we have never yet seen each other and
our letters so far have been restricted to mathematics, with maybe
sometimes some personal comments about his or my own involvement in
mathematics. It is through these, surely, that a mutual sympathy has
come into being, not merely motivated by a common interest in
mathematics -- and this sympathy I feel has been the main force giving
life to our correspondence while mathematically speaking more than
once it has been rather a ``dialogue de
sourds''.\scrcomment{``dialogue of the deaf''} (This is due mainly to
my illiteracy homotopy in theory, and to my reluctance to get really
involved in any ``technical'' matters, until I am really forced to by
what I am just doing.)

I want now to go on with the overall review on ``abelianization'' and
its relation to the homology and cohomology formalism for small
categories, serving as models for homotopy types.

\addtocounter{subsection}{6}
\subsection{Homology and cohomology (absolute case).}
\label{subsec:108.G}
My aim is to give a perfectly dual treatment of cohomology and
homology, which is one main reason why I have to take as coefficients
for both, not merely usual abelian presheaves on a given small
category $A$, or sheaves of $k$-modules for a given ring $k$, but more
generally sheaves with values in any abelian category \scrM, stable
under small direct or inverse limits (according as to whether we are
interesting in taking homology, or cohomology invariants). It will
then turn out that homology of $A$ for \scrM-valued presheaves (or
complexes of such) is ``the same'' as cohomology of the dual category
$B$, with coefficients in the corresponding $\scrM\op$-valued ones.

As I am a lot more familiar with cohomology, it is by this I'll begin
again. Here, as in the case of an arbitrary topos \scrX, the
cohomology invariants $\mathrm H^i(\scrX,F)$ with values in an abelian
sheaf $F$ may be\pspage{427} viewed as being just the invariants
$\Ext^i(\bZ_\scrX,F)$ in the category of all abelian sheaves, where
$\bZ_\scrX$ is the constant sheaf on \scrX{} with value $\bZ$. The
similar fact holds when $F$ is any sheaf of modules over a sheaf of
rings $\scrO_\scrX$ on \scrX, with $\bZ_\scrX$ being replaced by
$\scrO_\scrX$ in the interpretation above:
\[\mathrm H^i(\scrX,F) \simeq \Ext^i_{\scrO_\scrX}(\scrO_\scrX, F),\]
which is often quite useful in the cohomology formalism. We are going
to restrict here to the case of a constant sheaf of rings, defined by
a fixed commutative ring $k$, which will allow us to play around as
announced with the duality relation between $A$ and $B=A\op$, provided
moreover we take more general coefficients still, namely (pre)sheaves
with value in a given $k$-additive category \scrM{} stable under the
relevant limits. (Presumably, the case of a locally constant
commutative sheaf of rings could be dealt with too, but we'll not dive
into this here!) Another important (and by now familiar?) conceptual
point is that, rather than the $\Ext^i$'s which give only partial
information, we are interested in the object they come from (as the
``cohomology objects''), namely the objects $\mathrm
R\Hom_\scrOX(L,F)$ in a suitable derived category. In the present
case when \scrX{} is the topos associated to the small category $A$,
hence the category of \scrOX-modules has sufficiently many projective
(namely direct sums of sheaves of the type $\scrO_\scrX^{(a)}$ with
$a$ in $A$), the $\mathrm R\Hom_\scrOX(L,F)$ may be computed, taking a
projective resolution $L_\bullet$ of $L$, by the formula
\[\mathrm R\Hom_\scrOX(L,F) \simeq \Hom_\scrOX^\bullet(L_\bullet,F)\]
(an isomorphism in $\D^+\kMod$ say), and similarly when replacing $L$
and $F$ by arguments $L_\bullet$ and $F^\bullet$ in $\D^-$ and $\D^+$
of the category of \scrOX-modules. As a result, we get a pairing,
computable here using projective resolutions of the argument
$L_\bullet$:
\begin{equation}
  \label{eq:108.star}
  (L_\bullet,F^\bullet)\mapsto\mathrm
  R\Hom_\scrOX(L_\bullet,F^\bullet) : \D^-(\scrOX)\times\D^+(\scrOX)
  \to \D^+\kMod,\tag{*}
\end{equation}
where $k$ is a commutative ring and \scrOX{} is endowed with a
structure of $k$-algebra. Using $\bHom_\scrOX$ and its total derived
functor, we get likewise
\begin{equation}
  \label{eq:108.starstar}
  (L_\bullet,F^\bullet)\mapsto\mathrm
  R\bHom_\scrOX(L_\bullet,F^\bullet) : \D^-(\scrOX)\times\D^+(\scrOX)
  \to \D^+(\scrOX),\tag{**}
\end{equation}
with
\[\mathrm R\bHom_\scrOX(L_\bullet,F^\bullet) \simeq
\bHom_\scrX^{\bullet\bullet}(\scrL_\bullet,F^\bullet),\]
where $\scrL_\bullet$ is a projective resolution of $L_\bullet$, and
$\bHom^{\bullet\bullet}$ stands for the simple complex associated to
the double complex obtained by taking\pspage{428} $\bHom$'s
componentwise (and we have the similar formula of course for the
$\Hom$'s and $\mathrm R\Hom$'s non-bold-faced).

Taking $L_\bullet=\scrOX$ (or any resolution of \scrOX), the $\mathrm
R\Hom$ invariant \eqref{eq:108.star} reduces to the total derived
functor $\mathrm R\Gamma$ of the sections functor
\begin{equation}
  \label{eq:108.i}
  \mathrm R\Hom_\scrOX(\scrOX,F^\bullet)\simeq\mathrm R\Gamma_\scrX(F^\bullet)\tag{i}
\end{equation}
(whereas $\mathrm R\bHom_\scrOX(\scrOX,F^\bullet)\simeq F^\bullet$ of
course), which in turn allows to give the following familiar
expression of $\mathrm R\Hom$ in terms of $\mathrm R\bHom$:
\begin{equation}
  \label{eq:108.ii}
  \mathrm R\Hom_\scrOX(L_\bullet,F^\bullet) \simeq \mathrm
  R\Gamma_\scrX(\mathrm R\bHom_\scrOX(L_\bullet,F^\bullet)),\tag{ii}
\end{equation}
coming from the similar isomorphism
$\Hom_\scrOX\simeq\Gamma_\scrX\bHom_\scrOX$. All this is standard
cohomology formalism, valid on an arbitrary ringed topos
$(\scrX,\scrOX)$, except for the possibility of computing $\RHom$ and
$\RbHom$ by taking \emph{projective resolutions of the first argument}
(rather than injective ones of the second), which is special to the
case when $\scrX=\Ahat$, to which we'll now restrict.

Let now $A$ be a fixed small category, $k$ a fixed commutative ring,
\scrM{} a $k$-additive category, stable under small inverse limits. We
want to define a total derived functor of the functor
\begin{equation}
  \label{eq:108.109}
  (L,F)\mapsto \Hom_k(L,F):\Ahatk \times \AhatM\to\scrM,\tag{109}
\end{equation}
which should be a functor
\begin{equation}
  \label{eq:108.110}
  (L_\bullet,F^\bullet)\mapsto \RHom_k(L_\bullet,F^\bullet) :
  \D^-(\Ahatk)\times \D^+(\AhatM) \to \D^+(\scrM),\tag{110}
\end{equation}
and similarly
\begin{equation}
  \label{eq:108.111}
  (L_\bullet,F^\bullet)\mapsto\RbHom_k(L_\bullet,F^\bullet) :
  \D^-(\Ahatk) \times \D^+(\AhatM) \to \D^+(\AhatM).\tag{111}
\end{equation}
For this, in order for $\D^+(\AhatM)$ to be defined, we better assume
\scrM{} to be an \emph{abelian} category, hence \AhatM{} is abelian
too. Of course, we'll write
\begin{equation}
  \label{eq:108.112}
  \begin{aligned}
    \Ext_k^i(L_\bullet,F^\bullet) &=
    \mathrm H^i(\RHom_k(L_\bullet,F^\bullet)), \\
    \bExt_k^i(L_\bullet,F^\bullet) &=
    \mathrm H^i(\RbHom_k(L_\bullet,F^\bullet)),
  \end{aligned}\tag{112}
\end{equation}
these global and local $\Ext^i$ may be viewed as ``external''
$\Ext^i$'s, as contrarily to the familiar case, the components of the
two arguments $L_\bullet$ and $K^\bullet$ are not in the same category
-- just as the $\Hom_k$ in \eqref{eq:108.109} and the corresponding
\begin{equation}
  \label{eq:108.109prime}
  \bHom_k:\Ahatk\times\AhatM\to\AhatM\tag{109'}
\end{equation}
has arguments in the two different categories \Ahatk{} and \AhatM.

As\pspage{429} we don't know about the existence of enough injective
in \AhatM, the only way for defining the pairings \eqref{eq:108.110},
\eqref{eq:108.111} is now by using projective resolutions of the first
argument, writing
\begin{equation}
  \label{eq:108.113}
  \begin{aligned}
    \RHom_k(L_\bullet,F^\bullet) &=
    \Hom_k^{\bullet\bullet}(\scrL_\bullet,F^\bullet), \\
    \RbHom_k(L_\bullet,F^\bullet) &=
    \bHom_k^{\bullet\bullet}(\scrL_\bullet,F^\bullet) 
  \end{aligned}\tag{113}
\end{equation}
where $\scrL_\bullet$ is a projective resolution of $L_\bullet$ in
\Ahatk. As the latter is defined up to chain homotopy, it follows that
for fixed $L_\bullet$ and $F^\bullet$, the second members of
\eqref{eq:108.113} are defined up to chain homotopy, i.e., they may be
viewed as objects in $\mathrm K^+(\scrM)$ and $\mathrm K^+(\AhatM)$
respectively. They are defined as such, even without assuming \scrM{}
to be abelian and hence $\D^+(\scrM)$ and $\D^+(\AhatM)$ to be
defined. When we make this assumption, in order to check that the
formulæ \eqref{eq:108.113} do define pairings as in \eqref{eq:108.110}
and \eqref{eq:108.111}, we still have to check that for a
quasi-isomorphism
\[F^\bullet\to (F')\bullet\]
in \AhatM, the corresponding maps between $\RHom_k$ and $\RbHom_k$ are
quasi-isomorphisms too. This will follow immediately, provided we
check that for fixed \emph{projective} $L$ in \Ahatk{} and variable
$F$ in \AhatM, the functors
\[F\mapsto\Hom_k(L,F)\quad\text{and}\quad
F\mapsto\bHom_k(L,F)\]
from \AhatM{} to \scrM{} and \AhatM{} respectively are exact. Now,
this is clear for $\Hom_k$ when $L$ is of the type $k^{(a)}$, hence
$\Hom(L,F)\simeq F(a)$, hence it follows when $L$ is a small direct
sum of objects $k^{(a_i)}$, hence
\[\Hom_k(L,F) = \prod_i F(a_i),\]
provided we make on \scrM{} the mild extra assumption that \emph{a
  small direct product of epimorphisms is again an epimorphism}. As
any projective object of \Ahatk{} is a direct factor of a small direct
sum as above, the exactness result we want then follows, hence the
looked-for pairing \eqref{eq:108.110}. The same argument will hold for
$\bHom_k$, provided we check exactness of the functors
\[F\mapsto \bHom_k(k^{(a)},F)=(b\mapsto F(a\times b)).\]
Here it seems we get into trouble when $A$ is not stable under binary
products -- in this case there is little chance that the functor above
be exact, even when restricting to the case $\scrM=\Ab_k \eqdef\kMod$,
hence $\AhatM=\Ahatk$ and $L$, $F$ have values in the same category
(namely presheaves of $k$-modules). This may seem strange, as we know
(and recalled above) that in this standard case there is no problem
for defining a\pspage{430} pairing \eqref{eq:108.111} $\RbHom_k$. The
point here is that, whereas a reasonable $\RbHom$ can be defined
indeed, it \emph{cannot} be computed in terms of a projective
resolution of the first argument as in \eqref{eq:108.113}; or
equivalently, that for projective $L$ it is not necessarily true that
\[\bExt_k^i(L,F)=0\quad\text{for $i>0$;}\]
this in turn relates to the observation that, contrarily to what
happens for the notion of injective sheaves of modules (on an
arbitrary topos), it is not true that the property for a sheaf of
modules to be projective is stable under localization (even for
a constant sheaf of rings $k$ on a topos \Ahat). Indeed, the
localization of $k^{(a)}$ with respect to $A_{/b}$ (with $a$ and $b$
in $A$) is $k^{(a')}$ where $a'$ is $a\times b$ viewed as an object in
$A_{/b}$, and for any sheaf of $k$-modules $F$ on $A_{/b}$ we have
\[\Ext_{A_{/a}}^i(k^{(a')},F) = \mathrm H^i(A_{/a'=a\times b},F),\]
which need not be zero for $i>0$. If it was, this would imply that
$a\times b$ is $k$-acyclic (rather, that its connected components
are), a rather strong property indeed when $a\times b$ is not in
$A$\ldots

Thus, when $A$ is not stable under binary products, it doesn't seem
that there exists a pairing \eqref{eq:108.111} as I expected, except
(possibly) when there are enough injectives in \AhatM{} -- a case I do
not wish to examine for the time being, as I am mainly interested now
in a formalism using projective resolutions instead of injective
ones. Anyhow, for the purpose of subsuming the cohomology functor
$\RGamma_A$ under the $\RHom_k$ formalism, by formula
\begin{equation}
  \label{eq:108.114}
  \RGamma_A(F^\bullet) = \RHom_k(k_A,F^\bullet) \simeq
  \Hom_k^{\bullet\bullet}(L_\bullet^A,F^\bullet),\tag{114}
\end{equation}
where
\begin{equation}
  \label{eq:108.115}
  L_\bullet^A\to k_A\tag{115}
\end{equation}
is a projective resolution of $k_A$, it is the pairing $\RHom_k$ and
not $\RbHom_k$ which is the relevant one. Let's recall that a
projective resolution \eqref{eq:108.115} is called a
\emph{cointegrator} (for the category $A$, with coefficients in $k$),
as by formula \eqref{eq:108.114} it allows indeed to express
``cointegration'' of any \scrM-valued presheaf or complex of such
presheaves (with degrees bounded from below).

Thus, for the time being we just got the pairing $\RHom_k$ in
\eqref{eq:108.110}, and the corresponding functor
\begin{equation}
  \label{eq:108.116}
  F^\bullet\mapsto \RGamma_A(F^\bullet) : \D^+(\AhatM) \to \D^+(\scrM),\tag{116}
\end{equation}
and\pspage{431} not the pairing $\RbHom_k$ in \eqref{eq:108.111}, and
hence no formula \eqref{eq:108.ii} (p.\ \ref{p:428}) relating the two
-- which makes me feel a little silly! I'll have to come back upon
this later. At present, let's dualize what we got, assuming now that
\scrN{} is a $k$-additive abelian category stable under small
\emph{direct} limits, and such that \emph{a small direct sum of
  monomorphisms in \scrN{} is again a monomorphism}. We then get a
pairing
\begin{equation}
  \label{eq:108.117}
  (F_\bullet,L'_\bullet) \mapsto F_\bullet \Last_k L'_\bullet :
  \D^-(\AhatN)\times\D^-(\Bhatk) \to \D^-(\scrN),\tag{117}
\end{equation}
defined by the formula (dual to \eqref{eq:108.112})
\begin{equation}
  \label{eq:108.118}
  F_\bullet \Last_k L'_\bullet \simeq F_\bullet *_k \scrL'_\bullet,\tag{118}
\end{equation}
where in the second member $\scrL'_\bullet$ is a projective resolution
of $L_\bullet$ in \Bhatk{} ($B=A\op$ being of course the dual category
of $A$), and the $*_k$ denotes the simple complex associated to the
double complex obtained by applying $*_k$ componentwise. Using the
composite equivalence
\begin{multline}
  \label{eq:108.119}
  F_\bullet\mapsto(F_\bullet)\op: (\D^-(\AhatN))\op\equeq
  \D^+((\AhatN)\op)\equeq \D^+(\BhatM),\\
  \text{ with $\scrM=\scrN\op$,}\tag{119}
\end{multline}
we get the tautological duality isomorphism
\begin{equation}
  \label{eq:108.120}
  \bigl(F_\bullet\Last_k L'_\bullet)\op \simeq \RHom_k(L'_\bullet,
  (F_\bullet)\op),\tag{120} 
\end{equation}
where the expression $\Last_k$ in the first member is relative to the
pair $(A,\scrN)$, whereas the expression $\RHom_k$ in the second is
relative to the dual pair $(B,\scrM)$. Symmetrically, we get
\begin{equation}
  \label{eq:108.120prime}
  \bigl(\RHom_k(L_\bullet,F^\bullet)\bigr)\op \simeq L_\bullet \Last_k
  (F_\bullet)\op,\tag{120'} 
\end{equation}
which is essentially the inverse isomorphism of \eqref{eq:108.120},
but for the pair $(B,\scrM)$ instead of $(A,\scrN)$.

We still should dualize the functor $\RGamma_A$ \eqref{eq:108.116}
(defined by \eqref{eq:108.114}), which we do, recalling that
$\Gamma_A$ is just the inverse limit functor $\varprojlim_{A\op}$,
which is dual to the direct limit functor $\varinjlim_{A\op}$, thus,
``integration'' of \scrN-valued presheaves is just (at least morally)
the total left derived functor of the latter, and may be denoted by
\begin{equation}
  \label{eq:108.121}
  \mathrm L{\varinjlim_{A\op}},\tag{121}
\end{equation}
while using for $\RGamma_A$ the equivalent notation, dual to
\eqref{eq:108.121}
\begin{equation}
  \label{eq:108.122}
  \mathrm R{\varprojlim_{A\op}} = \RGamma_A.\tag{122}
\end{equation}
I am not wholly happy, though, with the purely algebraic flavor of
these notations, not really suggestive of the manifold geometric
intuitions\pspage{432} surrounding the familiar homology and
cohomology notations $\mathrm H_\bullet$ and $\mathrm H^\bullet$. This
flavor is at least partially preserved, it seems to me, in the
notation $\RGamma_A$ (because of the geometric intuition tied with the
sections functor), whereas there is not yet a familiar geometric
notion of a ``cosections functor''. As we would like to have the
duality symmetry reflected as perfectly as possible in the notation, I
am going to use the notations
\begin{equation}
  \label{eq:108.123}
  \left\{
    \begin{aligned}
      \RH^\bullet(A,F^\bullet) &= \mathrm
      R{\varprojlim_{A\op}}(F^\bullet)
      \;(\;=\RGamma_A(F^\bullet))
      : \D^+(\AhatM)\to\scrM \\
      \LH_\bullet(A,F_\bullet) &= \mathrm
      L{\varinjlim_{A\op}}(F_\bullet) : \D^-(\AhatN)\to\scrN.
    \end{aligned}
  \right.\tag{123}
\end{equation}
With these notations, the duality isomorphisms
(\ref{eq:108.120},\ref{eq:108.120prime}) take the form (as announced):
\begin{equation}
  \label{eq:108.124}
  \left\{
    \begin{aligned}
      (\RH^\bullet(A,F^\bullet))\op &\simeq
      \LH_\bullet(B,(F^\bullet)\op) \\
      (\LH_\bullet(A,F_\bullet))\op &\simeq
      \RH^\bullet(B,(F_\bullet)\op)\quad,
    \end{aligned}
  \right.\tag{124}
\end{equation}
where the first members are defined in terms of cohomology resp.\
homology invariants with respect to the pair $(A,\scrM)$ resp.\
$(A,\scrN)$, whereas the second members denote homology resp.\
cohomology invariants with respect to the dual pairs $(B,\scrN)$
resp.\ $(B,\scrM)$.
\begin{remarks}
  This perfect symmetry, or rather essential identity, between
  ``homology'' or ``integration'' and ``cohomology'' or
  ``cointegration'', is obtained here at the price of working with
  presheaves with values in rather general abelian categories,
  subjected to some simple exactness properties. It should be
  remembered moreover that for the time being, $\RH^\bullet$ has not
  been defined as the total right derived functor of the sections of
  inverse limits functor, therefore the notations \eqref{eq:108.121}
  and \eqref{eq:108.122} are somewhat misleading. To feel really at
  ease, we should still work out conditions that ensure that \AhatM{}
  has enough injectives and that $\RHom_k$ can be defined also using
  such resolutions -- in which case we'll expect too to have a
  satisfactory formalism for the $\RbHom_k$ functor.
\end{remarks}

\bigbreak
\presectionfill\ondate{23.8.}\pspage{433}\par

% 109
\hangsection[Review (6): A further step in linearization: coalgebra
\dots]{Review \texorpdfstring{\textup{(6)}}{(6)}: A further step in
  linearization: coalgebra structures
  \texorpdfstring{$P\to P\otimes_k P$ in \Cat}{P->PkP in
    (Cat)}.}\label{sec:109}%
I still did a little scratchwork last night, about the question of
existence of enough injectives or projectives in a category
\[\AhatN = \bHom(A\op,\scrN),\]
where \scrN{} is a $k$-additive abelian category. Introducing the
small $k$-additive category
\[P=\Add_k(A),\]
and remembering the canonical equivalence
\[\AhatN=\bHom(A\op,\scrN)\equeq \bHom_k(P\op,\scrN),\]
the question just stated may be viewed as a particular case of the
same question for a category of the type
\begin{equation}
  \label{eq:109.125}
  \PampN \eqdef \bHom_k(P\op,\scrN),\tag{125}
\end{equation}
where now $P$ is \emph{any} small $k$-additive category. (Compare with
the reflections on pages \ref{p:403}, \ref{p:404}.) It is immediate
that in \PampN{} exist all types of (direct or inverse) limits
which exist in \scrN, and they are computed ``componentwise'' for each
argument $a$ in $P$ -- from this follows that if \scrN{} is abelian,
so is \PampN.
\addtocounter{propositionnum}{3}
\begin{propositionnum}\label{prop:109.4}
  Assume the $k$-additive category \scrN{} is stable under small
  direct limits, and is abelian, and that any object of \scrN{} is
  isomorphic to the quotient of a projective object. Then the same
  holds for \PampN. Assume moreover that any projective object $x$ of
  \scrN{} is $k$-flat, i.e., the functor
  \[U\mapsto U\otimes_k x : \AbOf_k\to\scrN\]
  is exact, i.e., transform monomorphisms into monomorphisms. Then for
  any projective object $F$ in \PampN, the functor
  \[L'\mapsto F *_k L' : Q\supamp\to\scrN\quad (\text{where
    $Q=P\op$})\]
  is exact, i.e.\ \textup(as it is known to commute to small direct
  limits\textup), it transforms monomorphisms into monomorphisms.
\end{propositionnum}
\begin{comments}
  Here, the operation $*_k$ (similar to a tensor product) is defined
  as in the non-additive set-up (with \PampN, $Q\supamp$ being
  replaced by \AhatN, \Bhatk) reviewed in section \ref{sec:107}, and
  follows from the canonical equivalence of categories
  \begin{equation}
    \label{eq:109.star}
    \PampN \fromequ \bHom_{k!}(Q\supamp, \scrN)\tag{*}
  \end{equation}
  (this\pspage{434} is formula \eqref{eq:105.starstar} of page
  \ref{p:403} with $P,\scrN$ replaced by $Q,\scrM$). It should be
  noted that the assumptions made in prop.\ \ref{prop:109.4} are the
  weakest possible for the conclusions to hold (for any $k$-additive
  small category $P$), as these conclusions, in case $P =$ final
  category, just reduce to the assumptions.
\end{comments}

Here is the outline of a proof of prop.\ \ref{prop:109.4}. Using only
stability of \scrN{} under small direct limits (besides
$k$-additivity) we define a canonical $k$-biadditive pairing
\begin{equation}
  \label{eq:109.126}
  P\supamp \times\scrN \to \PampN ,
  \quad
  (L,x)\mapsto L\otimes x\eqdef
  (a\mapsto L(a)\otimes_k x)\tag{126}
\end{equation}
(NB\enspace I recall that $P\supamp$ is defined as
\[P\supamp = \bHom_{\bZ}(P\op,\AbOf)\fromequ\bHom_k(P\op,\AbOf_k),\]
here we interpret an object of $P\supamp$ is a $k$-additive functor
\[L:P\op\to\AbOf_k\;(\;\eqdef\kMod)\quad\text{.)}\]
The relevant fact here for objects of \PampN{} of the type $L\otimes_k
x$ is
\begin{equation}
  \label{eq:109.127}
  \begin{aligned}
    \Hom_\PampN(L\otimes_kx,F) &\simeq \Hom_\scrN(x,\Hom_k(L,F)) \\
    &\simeq \Hom_{P\supamp}(L, \Hom(x,F)),
  \end{aligned}\tag{127}
\end{equation}
where in the second term,
\[\Hom_k(L,F)\in\Ob\scrN\]
is defined in a way dual to $F*_k L'$ (cf.\ comments above), using the
equivalence (dual to \eqref{eq:109.star} above)
\begin{equation}
  \label{eq:109.starprime}
  \PampN\simeq\bHom_k^!((P\supamp)\op,\scrN),\tag{*'}
\end{equation}
which is defined only, however, when \scrN{} is stable under small
inverse limits (hence the first isomorphism in \eqref{eq:109.127}
makes sense only under this extra assumption); on the other hand, in
the third term in \eqref{eq:109.127}
\[\Hom(x,F) \eqdef (a\mapsto\Hom_\scrN(x,F(a))\quad\text{in
  $P\supamp$,}\]
and the isomorphism between the first and third term in
\eqref{eq:109.127} makes sense and is defined without any extra
assumption on \scrN.

We leave to the reader to check \eqref{eq:109.127} (where one is
readily reduced to the case when $L$ is an object $a$ in $P$, using
the commutation of the three functors obtained
$(P\supamp)\op\to\AbOf_k$ with small inverse limits). It follows, when
\scrN{} is abelian:
\begin{equation}
  \label{eq:109.128}
  \text{$L$ projective in $P\supamp$, $x$ projective in \scrN}
  \Rightarrow
  \text{$L\otimes_kx$ proj.\ in \PampN.}\tag{128}
\end{equation}

Assume now that any object of \scrN{} is quotient of a projective
one, and let $F$ be any object in \PampN. Formula \eqref{eq:109.127}
for $L=a$ in $P$ reduces\pspage{435} to the down-to-earth formula
\begin{equation}
  \label{eq:109.127prime}
  \Hom_\PampN(a\otimes_kx,F)\simeq\Hom(x,F(a)).\tag{127'}
\end{equation}
Now, let for any $a$ in $P$
\[x_a\to F(a)\]
be an epimorphism in \scrN, with $x_a$ projective. From
\eqref{eq:109.127prime} we get a map
\[a\otimes_kx_a\to F\]
in \PampN, hence a map
\[\scrF = \bigoplus_{\text{$a$ in $P$}} a\otimes_kx_a \to F,\]
it is easily seen that this is epimorphic (because the maps $x_a\to
F(a)$ are), and \scrF{} is projective as a direct sum of projective
objects. This proves the first statement in prop.\
\ref{prop:109.4}. For the second statement, we'll use the formula
\begin{equation}
  \label{eq:109.129}
  (L\otimes_k) *_k L' \simeq (L *_k L') *_k x\tag{129}
\end{equation}
for $L$ in $P\supamp$, $L'$ in $Q\supamp$, and $x$ in \scrN{} -- for
the proof, we may reduce to the case when $L$ is in $P$, $L'$ in
$Q=P\op$, say $L=a$ and $L'=b\op$, in which case both members identify
with $\Hom_P(b,a)\otimes_k x$. To prove that for $F$ projective,
$L'\mapsto F*_kL'$ takes monomorphisms into monomorphisms, using that
$F$ is a direct factor of objects of the type $a\otimes_k x$ with $a$
in $P$ and $x$ in \scrN, we are reduced to the case $F=a\otimes_kx$,
in which case by \eqref{eq:109.129} the functor reduces to
\[L'\mapsto L'(a)\otimes_kx,\]
which is again exact by the assumption that any projective object in
\scrN{} (and hence $x$) is $k$-flat.
\begin{remark}
  It is not automatic that a projective object in a $k$-additive
  abelian category be $k$-flat -- take for instance $k=\bZ$ and
  $\scrN=\AbOf_{\bF_p}$, where $\bF_p$ is a finite prime field, then
  all objects in \scrN{} are projective, whereas only the zero objects
  are \bZ-flat.
\end{remark}

We leave to the reader to write down the dual statement of prop.\
\ref{prop:109.4}, concerning injectives in a category \PampM, where
now \scrM{} is a $k$-additive abelian category stable under small
inverse limits, and possessing sufficiently many injectives (hence the
same holds in \PampM), and assuming eventually that these injectives $x$
are ``$k$-coflat'', i.e.,
\[U\mapsto\Hom_k(U,x) : \AbOf_k\op\to\scrM\]
is exact (i.e., transforms monomorphisms in $\AbOf_k$ into
epimorphisms in\pspage{436} \scrM), which implies that for $F$
injective in $P\supamp$, the functor
\[L\mapsto\Hom_k(L,F) : P\supamp\to\scrM\]
is exact.

To sum up, we get the
\begin{corollarynum}\label{cor:109.prop4.1}
  Let $P$ be any small $k$-additive category, and let \scrM{} be a
  $k$-additive category which satisfies the following assumptions:
  \begin{enumerate}[label=\alph*),font=\normalfont]
  \item\label{it:109.cor1.a}
    \scrM{} is abelian, and stable under small inverse limits,
  \item\label{it:109.cor1.b}
    \scrM{} has ``sufficiently many injectives'',
  \item\label{it:109.cor1.c}
    injective objects of \scrM{} are $k$-coflat,
  \item\label{it:109.cor1.d}
    any product \textup(with small indexing family\textup) of
    epimorphisms in \scrM{} is an epimorphism.
  \end{enumerate}
  Consider the $k$-biadditive pairing
  \begin{equation}
    \label{eq:109.130}
    (L,F)\mapsto\Hom_k(L,F) : P\supamp \times\PampM\to\scrM.\tag{130}
  \end{equation}
  This pairing admits a total right derived functor
  \begin{equation}
    \label{eq:109.131}
    (L_\bullet,F^\bullet)\mapsto\RHom_k(L_\bullet,F^\bullet) :
    \D^-(P\supamp) \times \D^+(\PampM) \to \D^+(\scrM),\tag{131}
  \end{equation}
  which can be computed using either projective resolutions of
  $L_\bullet$, or injective resolutions of $F^\bullet$, or both
  simultaneously.
\end{corollarynum}

We have a dual statement, concerning the pairing
\begin{equation}
  \label{eq:109.130prime}
  (F,L')\mapsto F*_kL': \PampN\times Q\supamp \to\scrN,
  \quad\text{with $Q=P\op$,}\tag{130'}
\end{equation}
giving rise to a total left derived functor
\begin{equation}
  \label{eq:109.131prime}
  (F_\bullet,L_\bullet') \mapsto F_\bullet\Last_k L_\bullet' :
  \D^-(\PampN) \times \D^-(Q\supamp) \to \D^-(\scrN),\tag{131'}
\end{equation}
using projective resolutions of either $F_\bullet$, or $L_\bullet'$,
or both. Here, \scrN{} is a $k$-additive category satisfying the
properties dual to \ref{it:109.cor1.a} to \ref{it:109.cor1.d} above,
i.e., such that $\scrM=\scrN\op$ satisfies the properties stated in
the corollary. We have the evident duality relations between the two
kinds of operations $\RHom_k$ and $\Last_k$, embodied by the formulæ
\eqref{eq:108.120} and \eqref{eq:108.120prime} of page \ref{p:431},
where the categories \Ahatk, \AhatM, \AhatN, etc.\ are replaced by
$P\supamp$, \PampM, \PampN, etc.\ (where the etc.'s refer to
replacement of $A$ by $B=A\op$ and of $P$ by $Q=P\op$).

Next question is to extend the $\RHom_k$ formalism to a $\RbHom_k$
formalism (and similarly from $\Last_k$ to $\Loast_k$), as envisioned
yesterday. To do so, in the wholly $k$-additive set-up we are now
working in, we still\pspage{437} need (in case of $\RbHom_k$ an
(``interior'') tensor product structure on $P\supamp$ (and dually for
$\Loast$, requiring a tensor product structure on $Q\supamp$), so as
to give rise to a $k$-biadditive pairing
\begin{equation}
  \label{eq:109.132}
  (L,F)\mapsto\bHom_k(L,F): P\supamp\times\PampM\to\PampM,\tag{132}
\end{equation}
(and dually,
\begin{equation}
  \label{eq:109.132prime}
  (F,L')\mapsto F\oast_kL' : \PampN\times Q\supamp\to\PampN
  \text{ ),}\tag{132'}
\end{equation}
for which we want to take the total right derived functor $\RbHom_k$
(or dually, the total left derived functor $\Loast_k$). As we say
yesterday, though, taking projective resolutions of $L$ will not do
(except in very special cases, such as $P=\Add_k(A)$ with $A$ in
\Cat{} stable under binary products), because for $L$ projective in
$P\supamp$, the functor
\[F\mapsto\bHom_k(L,F):\PampM\to\PampM\]
has little chance to be exact. Taking injective resolutions of
$F_\bullet$, however, we expected \emph{would} do in ``reasonable''
cases -- here the question is whether for $F$ injective, the functor
\[L\mapsto\bHom_k(L,F) \eqdef (a\mapsto\Hom_k(L\otimes_ka,F)) :
(P\supamp)\op\to\scrM\]
is exact (where $L\otimes_ka$ denotes the given tensor product within
$P\supamp$), i.e., whether the functors (for $F$ in \PampM, $a$ in
$P$)
\begin{equation}
  \label{eq:109.starbis}
  L\mapsto\Hom_k(L\otimes_ka,F) : (P\supamp)\op\to\scrM\tag{*}
\end{equation}
transform monomorphisms of $P\supamp$ into epimorphisms of \scrM. When
\scrM{} satisfies the conditions dual to
\ref{it:109.cor1.a}\ref{it:109.cor1.b}\ref{it:109.cor1.c} in the
corollary, hence $L\mapsto\Hom_k(L,F)$ is exact when $F$ in \PampM{}
is injective, exactness of \eqref{eq:109.starbis} above will follow
from exactness of
\[L\mapsto L\otimes_ka : P\supamp\to P\supamp.\]
Now, this latter exactness holds in the case we are interesting in
mainly, when $P=\Add_k(A)$ and hence
$P\supamp\equeq\bHom(A\op,\AbOf_k)$ and when tensor product for
presheaves on $A$ is defined as usual, componentwise -- then the
objects of $\Add_k(A)=P$ correspond to $k$-flat presheaves, hence
tensor product by these is exact. Thus:
\begin{corollarynum}\label{cor:109.prop4.2}
  Assume \scrM{} satisfies the conditions of corollary
  \ref{cor:109.prop4.1} above, let $A$ be any small category and
  $P=\Add_k(A)$. Then the pairing
  \[\bHom_k:\Ahatk\times\AhatM\to\AhatM\]
  \textup(cf.\ \eqref{eq:108.109prime} page \ref{p:428}\textup) admits
  a total right derived functor $\RbHom_k$
  \eqref{eq:108.111},\pspage{438} which may be computed using
  injective resolutions of the second argument $F^\bullet$ in
  $\RbHom_k(L_\bullet,F^\bullet)$ \textup(but not, in general, by
  using projective resolutions of $L_\bullet$\textup), i.e.,
  \begin{equation}
    \label{eq:109.133}
    \RbHom_k(L_\bullet,F^\bullet) \simeq
    \bHom_k^{\bullet\bullet}(L_\bullet,\scrF^\bullet),\tag{133}
  \end{equation}
  where $\scrF^\bullet$ is an injective resolution of $F^\bullet$
  \textup(i.e., a complex in \AhatM{} with degrees bounded from below
  and injective components, endowed with a quasi-isomorphism
  $F^\bullet\simeq\scrF^\bullet$\textup).
\end{corollarynum}

Applying now $\RGamma_A$ to both members of \eqref{eq:109.133}, and
using the similar isomorphism for $\RHom_k$ (valid by cor.\
\ref{cor:109.prop4.2}, we get the familiar formula
\begin{equation}
  \label{eq:109.134}
  \RHom_k(L_\bullet,F^\bullet)\simeq\RH^\bullet(\RbHom_k(L_\bullet,F^\bullet)),\tag{134}
\end{equation}
(where in accordance with \eqref{eq:108.124}, we wrote $\RH^\bullet$
instead of $\RGamma_A$), where however $F^\bullet$ is now a complex of
presheaves with values in a $k$-additive category \scrM{} (satisfying
the conditions dual to \ref{it:109.cor1.a} to \ref{it:109.cor1.d} in
cor.\ \ref{cor:109.prop4.1}), not just a presheaf of $k$-modules.

Replacing \scrM{} by a category \scrN{} satisfying the assumptions of
cor.\ \ref{cor:109.prop4.1}, we get likewise a total left derived
functor
\begin{equation}
  \label{eq:109.135}
  (F_\bullet,L_\bullet')\mapsto F\Loast_k L_\bullet' : \D^-(\AhatN)
  \times \D^-(\Bhatk) \to \D^-(\AhatN),\tag{135}
\end{equation}
which can be defined using projective resolutions of the first
argument $F_\bullet$ (but not by using projective resolutions of
$L_\bullet'$), giving rise to the isomorphism (dual to
\eqref{eq:109.134})
\begin{equation}
  \label{eq:109.134prime}
  F_\bullet \Last_k L_\bullet' \simeq \LH_\bullet(F_\bullet \Loast_k
  L_\bullet'). \tag{134'}
\end{equation}
The duality relationship between $\RbHom_k$ and $\Loast_k$ can be
expressed by two obvious formulæ, similar to \eqref{eq:108.120} and
\eqref{eq:108.120prime} for $\RHom_k$ and $\Last_k$, which we leave to
the reader.

\begin{remarks}
  \namedlabel{rem:109.1}{1)}\enspace If we want to consider
  ``multiplicative structure'' in the purely ``$k$-linear'' set-up,
  where the data is a small $k$-additive category $P$, rather than a
  small category $A$ (giving rise to $P=\Add_k(A)$), and corresponding
  cohomology and homology operations $\RbHom_k$ and $\Loast_k$, not
  only $\RHom_k$ and $\Last_k$, the natural thing to do, it seems, is
  to introduce a ``\emph{diagonal map}''
  \begin{equation}
    \label{eq:109.starbisbis}
    P \to P\otimes_k P,\tag{*}
  \end{equation}
  where the tensor product in the second member can be defined in a
  rather evident way (as solution of the obvious $2$-universal problem
  in terms of $k$-biadditive functors on $P\times P$), which will give
  rise in the ``usual'' way to a tensor product operation in both
  $P\supamp$ and $Q\supamp$ (where\pspage{439} $Q=P\op$). We'll come
  back upon this later, I expect. This structure
  \eqref{eq:109.starbisbis} will be the $k$-linear analogon of the
  usual diagonal map
  \begin{equation}
    \label{eq:109.starstar}
    A\to A\times A\tag{**}
  \end{equation}
  for a small category $A$, giving rise by $k$-linearization to
  \[\Add_k(A)\to\Add_k(A\times A)\equeq \Add_k(A)\otimes_k\Add_k(A),\]
  namely a structure of type \eqref{eq:109.starbisbis}. It just
  occurred to me, through the reflections of these last days, that the
  ``coalgebra structure'' \eqref{eq:109.starbisbis} may well turn out
  (taking $k=\bZ$) to be the more sophisticated structure than a usual
  coalgebra structure (cf.\ p.\ \ref{p:339} \eqref{eq:94.27}), needed
  in order to grasp ``in linear terms'' the notion of a homotopy type,
  possibly under restrictions such as $1$-connectedness, as pondered
  about in section \ref{sec:94}. This looks at any rate a more
  ``natural'' object than the De~Rham complex with divided powers,
  referred to in loc.\ cit., and is more evidently adapted to our
  point of view of using small categories as models for homotopy
  types. The greater sophistication, in comparison to De~Rham type
  complexes, lies in this, that here the objects serving as models
  (whether small categories, or small additive categories endowed with
  a diagonal map) are objects in a \emph{$2$-category}, whereas
  De~Rham complexes and the like are just objects in ordinary
  categories, without any question of taking ``maps between
  maps''. This feature implies, ``as usual'' (or in duality rather to
  familiar situations with tensor product functors\ldots) that the
  (anti)commutative and associative \emph{axioms} familiar from linear
  algebra (in the case of usual $k$-algebras or $k$-coalgebras),
  should be replaced by commutativity and associativity \emph{data},
  namely given isomorphisms (not identities) between two natural
  functors
  \[P\to P\otimes_k P, \quad P\to P\otimes_k P \otimes_k P\]
  deduced from \eqref{eq:109.starbisbis}. The axioms now will be more
  sophisticated, they will express ``coherence conditions'' on these
  data -- one place maybe where this is developed somewhat, in the
  context of diagonal maps \eqref{eq:109.starbisbis}, might be
  Saavedra's thesis.\scrcomment{\textcite{Saavedra1972}; see also in
    particular \textcite{DeligneMilne1982,Deligne1990,Deligne2002}}
  (The more familiar case, when starting with a tensor product
  operation on a category, together with associativity and/or
  commutativity data, has been done with care by various
  mathematicians, including Mac~Lane, Bénabou, Mme~Sinh Hoang Xuan,
  and presumably it should be enough to ``reverse arrows'' in order to
  get ``the'' natural set of coherence axioms for a diagonal map
  \eqref{eq:109.starbisbis}). The ``intriguing feature'' with the
  would-be De~Rham models for homotopy types (cf.\ p.\ \ref{p:341},
  \ref{p:342}), namely that the latter make sense over any\pspage{440}
  commutative ground ring $k$, not only \bZ, with corresponding notion
  of ring extension $k\to k'$, carries over to structures of the type
  \eqref{eq:109.starbisbis}. Indeed, for any $k$-additive category
  $P$, it is easy to define a $k'$-additive category
  \[P\otimes_kk',\quad \
  \text{for given homomorphism $k\to k'$,}\]
  for instance as the solution of the obvious $2$-universal problem
  corresponding to mapping $P$ $k$-additively into $k'$-additive
  categories, or more evidently by taking the same objects as for $P$,
  but with
  \[\Hom_{P'}(a,b) = \Hom_P(a,b)\otimes_k k'.\]
  Thus, any ``coalgebra structure in \Cat'' \eqref{eq:109.starbisbis}
  over the ground ring $k$, gives rise to a similar structure over
  ground ring $k'$.

  Of course, among the relevant axioms for the diagonal functor
  \eqref{eq:109.starbisbis}, is the existence of unit objects in
  $P\supamp$ and $Q\supamp$, which may be viewed equally as
  $k$-additive functors (defined up to unique isomorphism)
  \[P\op\to\AbOf_k, \quad P\to\AbOf_k,\]
  playing the role I would think of ``augmentation'' and
  ``coaugmentation'' in the more familiar set-up of ordinary
  coalgebras. Denoting these objects by $k_P$ and $k_Q$ respectively
  (in analogy to the constant presheaves $k_A$ and $k_B$ on $A$ and
  $B$), $\RHom(k_P,{-})$ now allows expression of cohomology or
  cointegration, and ${-}\Last_k(k_Q)$ allows expression of homology
  or integration (for complexes in \PampM{} say). ``Constant
  coefficients'' on $P$, i.e., in $P\supamp$ may now be defined, as
  objects in $P\supamp$ of the type
  \[U\otimes_k k_P,\]
  where $U$ is in $\AbOf_k$, i.e., is any $k$-module, and hence we get
  homology and cohomology invariants with coefficients in any such $U$
  (or complexes of such), and surely too cup and cap products\ldots
  Also, quasi-isomorphisms of structure \eqref{eq:109.starbisbis}
  (with units) can now be defined in an evident way, hence a derived
  category which merits to be understood, when $k=\bZ$, in terms of
  the homotopy category \Hot. I wouldn't expect of course that for any
  small category $A$, the abelianization $\Add(A)$ together with its
  diagonal map allows to recover the homotopy type, unless $A$ is
  $1$-connected. As was the case visibly for De~Rham complexes, if we
  hope to recover general homotopy types (not only $1$-connected
  ones), we should work with slightly more sophisticated structures
  still,\pspage{441} involving a group (or better still, a groupoid)
  and an operation of it on a structure of type
  \eqref{eq:109.starbisbis} (embodying a universal covering\ldots).

  Here I am getting, though, into thin air again, and I don't expect
  I'll ponder much more in this direction and see what comes out. The
  striking fact, however, here, is that quite unexpectedly, we get
  further hold and food for this thin-air intuition (which came up
  first in relation to De~Rham structures with divided powers), that
  there may be a reasonable (and essentially just one such) notion of
  a ``homotopy type over the ground ring $k$'' for any commutative
  ring $k$, reducing for $k=\bZ$ to usual homotopy types, and giving
  rise to base change functors
  \[\HotOf(k)\to\HotOf(k')\]
  for any ring homomorphism $k\to k'$. And I wonder whether this might
  not come out in some very simplistic way, in the general spirit of
  our ``modelizing story'', without having to work out in full a
  description of homotopy types by such sophisticated models as
  De~Rham complexes with divided powers, or coalgebra structures in
  \Cat, and looking up maybe the relations between these. (How by all
  means hope to recover a De~Rham structure from a stupid structure
  \eqref{eq:109.starbisbis}???)
  
  \namedlabel{rem:109.2}{2)}\enspace The condition \ref{it:109.cor1.d}
  in corollary \ref{cor:109.prop4.1} is needed in order to ensure that
  a derived functor $\RHom_k(L_\bullet,F^\bullet)$ may be defined
  using \emph{projective} resolutions of $L_\bullet$, whereas
  conditions \ref{it:109.cor1.b}, \ref{it:109.cor1.c} ensure that a
  functor $\RHom_k$ may be defined using \emph{injective} resolutions
  of $F^\bullet$. It is a well-known standard fact of homological
  algebra that in case both methods work (namely here, when all four
  assumptions are satisfied) that the two methods yield the same
  result, which may equally be described by resolving simultaneously
  the two arguments. (NB\enspace condition \ref{it:109.cor1.a} is
  needed anyhow for $\Hom_k$ to be defined and for \PampM{} being an
  abelian category, which allows to define $\D^+(\PampM)$.) Our
  preference goes to the first method, which in case $P=\Add_k(A)$ and
  $L=k_A$, conduces to computation of cohomology
  $\RGamma_A(F^\bullet)$ in terms of a ``cointegrator'' $L_\bullet^A$
  on $A$. However, when it comes to introducing the variant
  $\RbHom_k$, this method breaks down, as we saw, it is the other one
  which works. We thus get a satisfactory formalism of $\RHom_k$ and
  $\RbHom_k$ (including formula \eqref{eq:109.134} relating them via
  $\RH^\bullet$) using only assumptions
  \ref{it:109.cor1.a}\ref{it:109.cor1.b}\ref{it:109.cor1.c}. 

  \namedlabel{rem:109.3}{3)}\enspace If we want to extend the
  $\RbHom_k$ formalism to the set-up when the data $A$ is replaced by
  a $k$-additive category $P$ endowed with a diagonal map as in remark
  \ref{rem:109.1}, the proof on page \ref{p:437} shows that what is
  needed\pspage{442} is exactness of the functor $L\mapsto L\otimes a$
  from $P\supamp$ to $P\supamp$, for any $a$ in $P$ -- which is a
  ``flatness'' condition on $a$. It is easily checked that his
  condition is satisfied provided $\Hom_P(b,a)$ is a flat $k$-module,
  for any $b$ in $P$ (more generally, an object $M$ in $P\supamp$ is
  flat for the tensor product structure in $P\supamp$, provided $M(b)$
  is flat for any $b$ in $P$). Thus, there will be a satisfactory
  $\RbHom_k$ theory provided the $k$-modules $\Hom_P(b,a)$, for $a,b$
  in $P$, are $k$-flat. It is immediately checked that this also means
  that any projective object $L$ in $P\supamp$ is ``$k$-flat'' (with
  respect to external tensor product $U\mapsto U\otimes_k L :
  \AbOf_k\to P\supamp$, as in prop.\ \ref{prop:109.4}), or
  equivalently still, that this holds when $L$ is any object $a$ in
  $P$. In case $P=\Add_k(A)$, when any object of $P$ is a finite sum
  of objects $k^{(x)}$ with $x$ in $A$, the modules $\Hom(b,a)$ for
  $b,a$ in $L$ are finite sums of modules of the type
  $\Hom(k^{(y)},k^{(x)})=k^{(\Hom(y,x))}$, and hence are projective,
  not only flat. It would seem that in the general case of $P$ endowed
  with a diagonal map, the ``natural'' assumption to make in order to
  have everything come out just as nicely as when $P$ comes from an
  $A$, is that the $k$-modules $\Hom_P(b,a)$ (for $a,b$ in $P$) should
  be projective, not only flat. Flatness however seems to be all that
  is needed in order to ensure that when $L$ in $P\supamp$ is
  projective (hence $L(a)$ is flat for any $a$ in $P$) and $F$ in
  \PampM{} is injective, then the objects $\Hom_k(L,F)$ and
  $\bHom_k(L,F)$ in \scrM{} and \PampM{} respectively are
  injective. This implies that for a $k$-additive functor
  \[u:\scrM\to\scrM'\]
  between categories \scrM, $\scrM'$ satisfying the assumptions of
  cor.\ \ref{cor:109.prop4.1}, and $u$ commuting moreover to small
  inverse limits (and hence to formation of $\Hom_k(L,F)$), we get a
  canonical isomorphism
  \begin{equation}
    \label{eq:109.136}
    \mathrm R u (\RHom_k(L_\bullet,F^\bullet)) \simeq
    \RHom_k(L_\bullet,\mathrm Ru^P(F^\bullet)),\tag{136}
  \end{equation}
  where
  \[u^P:\PampM\to P\supamp_{\scrM'}\]
  denotes the extension of $u$, and $\mathrm Ru$, $\mathrm Ru^P$ are
  the right derived functors. When $u$ is exact, we may replace
  $\mathrm Ru$, $\mathrm Ru^P$ by $u$, $u^P$ (applied componentwise to
  complexes, without any need to take an injective resolution
  first). There is a formula as \eqref{eq:109.136} with $\RHom_k$
  replaced by $\RbHom_k$, which I skip, as well as the dual formulas.
\end{remarks}

%%% Local Variables:
%%% mode: latex
%%% TeX-master: "main.tex"
%%% End:
